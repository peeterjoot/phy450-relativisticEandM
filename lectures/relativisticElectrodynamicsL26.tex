%
% Copyright � 2012 Peeter Joot.  All Rights Reserved.
% Licenced as described in the file LICENSE under the root directory of this GIT repository.
%

%\chapter{PHY450H1S.  Relativistic Electrodynamics Lecture 26 (Taught by Prof. Erich Poppitz).  Radiation reaction force for a dipole system}
%\chapter{Radiation reaction force for a dipole system}
\label{chap:relativisticElectrodynamicsL26}
%\blogpage{http://sites.google.com/site/peeterjoot/math2011/relativisticElectrodynamicsL26.pdf}
%\date{April 5, 2011}

\paragraph{Reading}

Covering chapter 8 \S 65 material from the text \citep{landau1980classical},
and \popcite{RelEMpp181-195.pdf}{lecture notes RelEMpp181-195.pdf}.
%: (182-189) [Tuesday, Mar. 29]; the EM potentials to order \((v/c)^2\) (190-193); the ``Darwin Lagrangian.  and Hamiltonian for a system of nonrelativistic charged particles to order \((v/c)^2\) and its many uses in physics (194-195) [Wednesday, Mar. 30]

Next week (last topic): attempt to go to the next order \((v/c)^3\) - radiation damping, the limitations of classical electrodynamics, and the relevant time/length/energy scales.

\section{Recap}

A system of N charged particles \(m_a, q_a ; a \in [1, N]\) closed system and nonrelativistic, \(v_a/c \ll 1\).  In this case we can incorporate EM effects in a Lagrangian \textunderline{ONLY} involving particles (EM field not a dynamical DOF).  In \textunderline{general case}, this works to \(O((v/c)^2)\), because at \(O((v/c))\) system radiation effects occur.

In a specific case, when

\begin{equation}\label{eqn:relativisticElectrodynamicsL26:10}
\frac{m_1}{q_1} = \frac{m_2}{q_2} = \frac{m_3}{q_3} = \cdots
\end{equation}

we can do that (meaning use a Lagrangian with particles only) to \(O((v/c)^4)\) because of specific symmetries in such a system.

The Lagrangian for our particle after the gauge transformation is

\begin{equation}\label{eqn:relativisticElectrodynamicsL26:30}
\LL_a = \inv{2} m_a \Bv_a^2 + \frac{m_a}{8} \frac{\Bv_a^4}{c^2}
-\sum_{b \ne a} \frac{q_a q_b}{\Abs{\Bx_a(t) - \Bx_b(t)}}
+\sum_b q_a q_b \frac{\Bv_a \cdot \Bv_b + (\Bn \cdot \Bv_a) (\Bn \cdot \Bv_b)}{2 c^2 \Abs{\Bx - \Bx_b} }.
\end{equation}

Next time we will probably get to the Lagrangian for the entire system.  It was hinted that this is called the Darwin Lagrangian (after Charles Darwin's grandson).

We find for whole system

\begin{equation}\label{eqn:relativisticElectrodynamicsL26:50}
\LL =  \sum_a \LL_a + \inv{2} \sum_a \LL_a (interaction)
\end{equation}

\begin{equation}\label{eqn:relativisticElectrodynamicsL26:70}
\LL = \inv{2} \sum_a m_a \Bv_a^2 + \sum_a \frac{m_a}{8} \frac{\Bv_a^4}{c^2}
-\sum_{ a < b} \frac{q_a q_b}{\Abs{\Bx_a(t) - \Bx_b(t)}}
+\sum_b q_a q_b \frac{\Bv_a \cdot \Bv_b + (\Bn \cdot \Bv_a) (\Bn \cdot \Bv_b)}{2 c^2 \Abs{\Bx - \Bx_b} }.
\end{equation}

This is the Darwin Lagrangian (also Charles).  The Darwin Hamiltonian, from \(H = \sum_a q_a p_a - \LL\), which toggles the sign on all but the first term, is

\begin{equation}\label{eqn:relativisticElectrodynamicsL26:90}
\begin{aligned}
H &= \sum_a \frac{p_a}{2 m_a} \Bv_a^2 - \sum_a \frac{p_a^4}{8 m_a^3 c^2}  \\
&+\sum_{ a < b} \frac{q_a q_b}{\Abs{\Bx_a(t) - \Bx_b(t)}}
- \sum_{ a < b } \frac{q_a q_b}{ 2 c^2 m_a m_b } \frac{\Bp_a \cdot \Bp_b + (\Bn_{a b} \cdot \Bp_a) (\Bn_{ a b } \cdot \Bp_b)}{\Abs{\Bx - \Bx_b} }.
\end{aligned}
\end{equation}

(note, this is also the result to be obtained in problem 2, \S 65 of the text.)

\section{Incorporating radiation effects as a friction term}
\index{radiation effects}

To \(O((v/c)^3)\) obvious problem due to radiation (system not closed).  We will incorporate radiation via a function term in the EOM

Again consider the dipole system

\begin{equation}\label{eqn:relativisticElectrodynamicsL26:110}
\begin{aligned}
m \ddot{z} &= -k z \\
\omega^2 &= \frac{k}{m}
\end{aligned}
\end{equation}

or
\begin{equation}\label{eqn:relativisticElectrodynamicsL26:130}
m \ddot{z} = -\omega^2 m z
\end{equation}

gives

\begin{equation}\label{eqn:relativisticElectrodynamicsL26:150}
\ddt{}\left( \frac{m}{2} \zdot^2 + \frac{ m \omega^2 }{2} z^2 \right) = 0
\end{equation}

(because there is no radiation).

The energy radiated per unit time averaged per period is

\begin{equation}\label{eqn:relativisticElectrodynamicsL26:170}
P = \frac{ 2 e^2 }{ 3 c^3} \expectation{ \ddot{z}^2 }
\end{equation}

We will modify the EOM

\begin{equation}\label{eqn:relativisticElectrodynamicsL26:190}
m \ddot{z} = -\omega^2 m z + f_{\text{radiation}}
\end{equation}

Employing an integration factor \(\dot{z}\) we have

\begin{equation}\label{eqn:relativisticElectrodynamicsL26:210}
m \ddot{z} \dot{z} = -\omega^2 m z \dot{z} + f_{\text{radiation}} \dot{z}
\end{equation}

or
\begin{equation}\label{eqn:relativisticElectrodynamicsL26:230}
\ddt{}
\left( m \dot{z}^2 + \omega^2 m z^2 \right)
=
f_{\text{radiation}} \dot{z}
\end{equation}

Observe that the last expression, force times velocity, has the form of power

\begin{equation}\label{eqn:relativisticElectrodynamicsL26:250}
m \frac{d^2 z}{dt^2} \frac{dz}{dt} = \ddt{} \left( \frac{m}{2} \left( \frac{dz}{dt} \right)^2 \right)
\end{equation}

So we can make an identification with the time rate of energy lost by the system due to radiation

\begin{equation}\label{eqn:relativisticElectrodynamicsL26:270}
\ddt{}
\left( m \dot{z}^2 + \omega^2 m z^2 \right)
\equiv \ddt{\calE}.
\end{equation}

Average over period both sides

\begin{equation}\label{eqn:relativisticElectrodynamicsL26:290}
\expectation{ \ddt{\calE} } =
\expectation{ f_{\text{radiation}} \dot{z} }
=
- \frac{2 e^2 }{3 c^3} \expectation{\ddot{z}^2}
\end{equation}

We demand this last equality, by requiring the energy change rate to equal that of the dipole power (but negative since it is a loss) that we previously calculated.

\paragraph{Claim:}

\begin{equation}\label{eqn:relativisticElectrodynamicsL26:310}
f_{\text{radiation}} = \frac{2 e^2 }{3 c^3} \dddot{z}
\end{equation}

\paragraph{Proof:}

We need to show

\begin{equation}\label{eqn:relativisticElectrodynamicsL26:330}
\expectation{ f_{\text{radiation}} }
= - \frac{2 e^2 }{3 c^3} \expectation{\ddot{z}^2}
\end{equation}

We have

\begin{equation}\label{eqn:relativisticElectrodynamicsL26:470}
\begin{aligned}
\frac{2 e^2}{3 c^3} \expectation{ \dddot{z} \dot{z} }
&= \frac{2 e^2}{3 c^3} \inv{T} \int_0^T dt \dddot{z} \dot{z} \\
&=
\frac{2 e^2}{3 c^3} \inv{T} \int_0^T dt \cancel{\ddt{} (\ddot{z} \dot{z}) }
-\frac{2 e^2}{3 c^3} \inv{T} \int_0^T dt (\ddot{z})^2
\end{aligned}
\end{equation}

We first used \((\ddot{z} \dot{z})' = \dddot{z} \dot{z} + (\ddot{z})^2\).  The first integral above is zero since the derivative of \(\ddot{z} \dot{z} = (-\omega^2 z_0 \sin\omega t)(\omega z_0 \cos\omega t) = -\omega^3 z_0^2 \sin(2 \omega t)/2\) is also periodic, and vanishes when integrated over the interval.

\begin{equation}\label{eqn:relativisticElectrodynamicsL26:350}
\frac{2 e^2}{3 c^3} \expectation{ \dddot{z} \dot{z} } =
-\frac{2 e^2}{3 c^3} \expectation{ (\ddot{z})^2 }
\end{equation}

We can therefore write

\begin{equation}\label{eqn:relativisticElectrodynamicsL26:370}
m \ddot{z} = -m \omega^2 z + \frac{2 e^2}{3 c^3} \dddot{z}
\end{equation}

Our ``frictional'' correction is the radiation reaction force proportional to the third derivative of the position.

Rearranging slightly, this is

\begin{equation}\label{eqn:relativisticElectrodynamicsL26:390}
\ddot{z} = - \omega^2 z + \frac{2}{3 c} \left( \frac{e^2}{m c^2} \right) \dddot{z}
 = - \omega^2 z + \frac{2}{3 c} \frac{r_e}{c} \dddot{z},
\end{equation}

where \(r_e \sim 10^{-13} \text{cm}\) is the ``classical radius'' of the electron.  In our frictional term we have \(r_e/c\), the time for light to cross the classical radius of the electron.

There are lots of problems with this.  One of the easiest is with \(\omega = 0\).  Then we have

\begin{equation}\label{eqn:relativisticElectrodynamicsL26:410}
\ddot{z} = \frac{2}{3} \frac{r_e}{c} \dddot{z}
\end{equation}

with solution

\begin{equation}\label{eqn:relativisticElectrodynamicsL26:430}
z \sim e^{\alpha t},
\end{equation}

where

\begin{equation}\label{eqn:relativisticElectrodynamicsL26:450}
\alpha \sim \frac{c}{r_e} \sim \inv{\tau_e}.
\end{equation}

This is a \textunderline{self accelerating system}!  Note that we can also get into this trouble with \(\omega \ne 0\), but those examples are harder to find (see: \citep{griffiths1999introduction}).

FIXME: borrow this text again to give that section a read.

The sensible point of view is that this third term (\(f_{\text{rad}}\)) should be taken seriously only if it is small compared to the first two terms.

% Makes more sense to incorporate this into L27 notes.
%\section{Obtaining the radiation reaction force by contining the expansion}
%
%We can also obtain this result more systematically by contining the expansion to the next order in \(v/c\).
