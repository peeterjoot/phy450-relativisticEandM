%
% Copyright � 2012 Peeter Joot.  All Rights Reserved.
% Licenced as described in the file LICENSE under the root directory of this GIT repository.
%

%\chapter{Energy Momentum Tensor}
\index{energy momentum tensor}
\label{chap:relativisticElectrodynamicsL22}
%\blogpage{http://sites.google.com/site/peeterjoot/math2011/relativisticElectrodynamicsL22.pdf}
%\date{Mar 23, 2011}
%
\paragraph{Reading}
%
Covering \S 32, \S 33 of chapter 4 in the text \citep{landau1980classical},
and \popcite{RelEMpp166-180.pdf}{lecture notes RelEMpp166-180.pdf}.
% pp. 169-172:} spacetime translation invariance of the EM field action and the conservation of the energy-momentum tensor (170-172); properties of the energy-momentum tensor (172.1); the meaning of its components: energy.
%
%\section{Disclaimer}
%
%I have no class notes for this lecture, as traffic conspired against me and I missed all but the last 5 minutes (a very frustrating drive!)  Here is my own walk through of the content that we must have covered, much of which I did as part of problem set 6 preparation.
%
\section{Total derivative of the Lagrangian density}
\index{Lagrangian density}

Rather cleverly, our Professor avoided the spacetime translation arguments of the text.  Inspired by an approach possible in classical mechanics to find that we have a conserved quantity derivable from a force law, he proceeds directly to taking the derivative of the Lagrangian density (see previous lecture notes for details building up to this).

I will proceed in exactly the same fashion.
%
\begin{equation}\label{eqn:relativisticElectrodynamicsL22:340}
\begin{aligned}
\partial_k \left( -\inv{16 \pi c} F_{i j} F^{i j} \right)
&= -\inv{8 \pi c} (\partial_k F_{i j} )F^{i j} \\
&= -\inv{8 \pi c} (\partial_k F_{i j} )F^{i j} \\
&= -\inv{8 \pi c} (\partial_k (\partial_i A_j - \partial_j A_i) )F^{i j} \\
&= -\inv{4 \pi c} (\partial_k \partial_i A_j )F^{i j} \\
&= -\inv{4 \pi c} (\partial_i \partial_k A_j )F^{i j} \\
&= -\inv{4 \pi c} (\partial_m \partial_k A_j )F^{m j} \\
&= -\inv{4 \pi c} \left( \partial_m ((\partial_k A_j )F^{m j}) - (\partial_m F^{m j}) \partial_k A_j \right) \\
&= -\inv{4 \pi c} \left( \partial_m ((\partial_k A_j )F^{m j}) - (\partial_m F^{m a}) \partial_k A_a \right) \\
&= -\inv{4 \pi c} \left( \partial_m ((\partial_k A_j )F^{m j}) - \left(\frac{4 \pi}{c} j^a \right) \partial_k A_a \right) \\
&= -\inv{4 \pi c} \partial_m ((\partial_k A_j )F^{m j}) + \left(\frac{1}{c^2} j^a \right) \partial_k A_a
\end{aligned}
\end{equation}
%
Multiplying through by \(c\) and renaming our derivative index using a delta function we have
%
\begin{equation}\label{eqn:relativisticElectrodynamicsL22:10}
\partial_k \left( -\inv{16 \pi } F_{i j} F^{i j} \right) =
\partial_m {\delta^{m}}_k \left( -\inv{16 \pi } F_{i j} F^{i j} \right)
= -\inv{4 \pi } \partial_m ((\partial_k A_j )F^{m j}) + \left(\frac{1}{c} j^a \right) \partial_k A_a
\end{equation}
%
We can now group the \(\partial_m\) terms for
%
\begin{equation}\label{eqn:relativisticElectrodynamicsL22:30}
\partial_m \left(
-\inv{4 \pi } (\partial_k A_j )F^{m j}
+ {\delta^{m}}_k \inv{16 \pi } F_{i j} F^{i j}
\right)
=
- \left(\frac{1}{c} j^a \right) \partial_k A_a
\end{equation}
%
Knowing the end goal, a quantity that is expressed in terms of \(F^{ij}\) let us raise the \(k\) indices, and any of the \(A_i\)'s that are along side of those
%
\begin{equation}\label{eqn:relativisticElectrodynamicsL22:50}
\partial_m \left(
-\inv{4 \pi } (\partial^k A^n )F^{m j} g_{n j}
+ g^{m k} \inv{16 \pi } F_{i j} F^{i j}
\right)
=
- \left(\frac{1}{c} j_a \right) \partial^k A^a.
\end{equation}
%
Next, we want to get rid of the explicit vector potential dependence
%
\begin{equation}\label{eqn:relativisticElectrodynamicsL22:360}
\begin{aligned}
\partial_m &\left( -\inv{4 \pi } (\partial^k A^n )F^{m j} g_{n j} \right) \\
&=
\partial_m \left( -\inv{4 \pi } (F^{k n} + \partial^n A^k )F^{m j} g_{n j} \right) \\
&=
\partial_m \left( -\inv{4 \pi } F^{k n} F^{m j} g_{n j}
- \inv{4 \pi} (\partial_m (\partial^n A^k )) F^{m j} g_{n j}
- \inv{4 \pi} (\partial^n A^k ) (\partial_m F^{m j}) g_{n j} \right) \\
&=
\partial_m \left( -\inv{4 \pi } F^{k n} F^{m j} g_{n j} \right)
- \inv{4 \pi} (\partial_m (\partial^n A^k )) F^{m j} g_{n j}
- (\partial^n A^k ) \inv{c} j_n \\
&=
\partial_m \left( -\inv{4 \pi } F^{k n} F^{m j} g_{n j} \right)
- \inv{4 \pi} (\partial_m \partial_j A^k ) F^{m j}
- (\partial^a A^k ) \inv{c} j_a.
\end{aligned}
\end{equation}
%
Since the operator \(F^{m j} \partial_m \partial_j\) is a product of symmetric and antisymmetric tensors (or operators), the middle term is zero, and we are left with
%
\begin{equation}\label{eqn:relativisticElectrodynamicsL22:70}
\partial_m \left(
-\inv{4 \pi } F^{k n} F^{m j} g_{n j}
+ g^{m k} \inv{16 \pi } F_{i j} F^{i j}
\right)
=
- \frac{1}{c} F^{k a} j_a
\end{equation}
%
This provides the desired conservation relationship
\boxedEquation{eqn:relativisticElectrodynamicsL22:90}{
\begin{aligned}
\partial_m T^{m k} &= - \inv{c} F^{k a} j_a \\
T^{m k} &=
\inv{4 \pi } \left(
-
F^{m j}
F^{k n}
g_{n j}
+ \frac{g^{m k}}{4} F_{i j} F^{i j} \right).
\end{aligned}
}
\section{Unpacking the tensor}
%
\paragraph{Energy term of the stress energy tensor}
\index{stress energy tensor!energy}
%
\begin{equation}\label{eqn:relativisticElectrodynamicsL22:380}
\begin{aligned}
T^{ 0 0 }
&=
-\inv{4 \pi} F^{ 0 j} {F^0}_j + \inv{16 \pi} F^{i j} F_{i j} \\
&=
-\inv{4 \pi} F^{ 0 \alpha} {F^0}_\alpha + \inv{16 \pi} \left(
F^{0 j} F_{0 j}
+F^{\alpha j} F_{\alpha j}
\right)
\\
&=
\inv{4 \pi} F^{ 0 \alpha} F^{0 \alpha} + \inv{16 \pi}
\left(
F^{0 \alpha} F_{0 \alpha}
+F^{\alpha 0} F_{\alpha 0}
+F^{\alpha \beta} F_{\alpha \beta}
\right)
\\
&=
\inv{4 \pi} \BE^2 + \inv{16 \pi} \left(
-2 \BE^2 +F^{\alpha \beta} F^{\alpha \beta}
\right).
\end{aligned}
\end{equation}
%
The spatially indexed field tensor components are
\begin{equation}\label{eqn:relativisticElectrodynamicsL22:400}
\begin{aligned}
F^{\alpha \beta}
&= \partial^\alpha A^\beta - \partial^\beta A^\alpha \\
&= -\partial_\alpha A^\beta + \partial_\beta A^\alpha \\
&= -\epsilon^{\sigma \alpha \beta} (\BB)^\sigma,
\end{aligned}
\end{equation}
so
%
\begin{equation}\label{eqn:relativisticElectrodynamicsL22:420}
\begin{aligned}
F^{\alpha \beta} F^{\alpha \beta}
&=
\epsilon^{\sigma \alpha \beta} (\BB)^\sigma
\epsilon^{\mu \alpha \beta} (\BB)^\mu \\
&= (2!) \delta^{\sigma \mu}
(\BB)^\sigma
(\BB)^\mu \\
&= 2 \BB^2
\end{aligned}
\end{equation}
%
A final bit of assembly gives us \(T^{0 0}\)
\boxedEquation{eqn:relativisticElectrodynamicsPS6:200}{
T^{ 0 0 } = \inv{8 \pi} ( \BE^2 + \BB^2 ) = \calE.
}
%
\paragraph{Momentum terms of the stress energy tensor}
\index{stress energy tensor!momentum}
For the spatial \(T^{k 0}\) components we have
%
\begin{equation}\label{eqn:relativisticElectrodynamicsL22:440}
\begin{aligned}
T^{\alpha 0}
&=
-\inv{4 \pi} F^{\alpha j} {F^0}_j + \inv{16 \pi} g^{\alpha 0} F^{i j} F_{i j} \\
&=
-\inv{4 \pi} F^{\alpha j} {F^0}_j \\
&=
-\inv{4 \pi}
\left(
F^{\alpha 0} {F^0}_0
+F^{\alpha \beta} {F^0}_\beta
\right) \\
&=
\inv{4 \pi} F^{\alpha \beta} F^{0 \beta} \\
&=
\inv{4 \pi} (-\epsilon^{\sigma \alpha \beta} (\BB)^\sigma) (-(\BE)^\beta) \\
&=
\inv{4 \pi} \epsilon^{\alpha \beta \sigma}
(\BE)^\beta
(\BB)^\sigma.
\end{aligned}
\end{equation}
%
So we have
\boxedEquation{eqn:relativisticElectrodynamicsPS6:220}{
T^{\alpha 0} = \inv{4 \pi} (\BE \cross \BB)^\alpha = \frac{\BS^\alpha}{c}.
}
%
\paragraph{Symmetry}
%
It is simple to show that \(T^{k m}\) is symmetric
%
\begin{equation}\label{eqn:relativisticElectrodynamicsL22:460}
\begin{aligned}
T^{m k}
&= -\inv{4\pi} F^{m j} {F^{k}}_j + \inv{16 \pi} g^{m k} F^{i j} F_{i j} \\
&= -\inv{4\pi} {F^{m}}_j F^{k j} + \inv{16 \pi} g^{k m} F^{i j} F_{i j} \\
&= T^{k m}
\end{aligned}
\end{equation}
%
\paragraph{Pressure and shear terms}
\index{pressure}
\index{shear}

Let us now expand \(T^{\beta \alpha}\), starting with the diagonal terms \(T^{\alpha\alpha}\).  Because this repeated index is not summed over, things get slightly irregular, so it is easier to drop the abstraction and just pick a specific \(\alpha\), say, \(\alpha = 1\).  Then we have
%
\begin{equation}\label{eqn:relativisticElectrodynamicsL22:480}
\begin{aligned}
T^{1 1}
&= \inv{4 \pi} \left( - F^{1 k} {F^1}_k - \inv{2} (\BB^2 - \BE^2) \right) \\
&= \inv{4 \pi} \left( - F^{1 0} F^{1 0} + F^{1 \alpha} F^{1 \alpha} - \inv{2} (\BB^2 - \BE^2) \right) \\
&= \inv{4 \pi} \left(
- E_x^2
+ F^{1 2} F^{1 2}
+ F^{1 3} F^{1 3}
 - \inv{2} (\BB^2 - \BE^2) \right).
\end{aligned}
\end{equation}
%
For the magnetic components above we have for example
%
\begin{equation}\label{eqn:relativisticElectrodynamicsL22:500}
\begin{aligned}
F^{1 2} F^{1 2}
&=
(\partial^1 A^2 - \partial^2 A^2) (\partial^1 A^2 - \partial^2 A^2) \\
&=
(\partial_1 A^2 - \partial_2 A^2) (\partial_1 A^2 - \partial_2 A^2) \\
&=
B_z^2
\end{aligned}
\end{equation}
%
So we have
%
\begin{equation}\label{eqn:relativisticElectrodynamicsPS6:240}
T^{1 1}
= \inv{4 \pi} \left(
- E_x^2 + B_y^2 + B_z^2
- \inv{2} (\BB^2 - \BE^2) \right)
\end{equation}
%
Or
%
\begin{equation}\label{eqn:relativisticElectrodynamicsPS6:260}
T^{1 1}
= \inv{8 \pi} \left(
- E_x^2 + E_y^2 + E_z^2
- B_x^2 + B_y^2 + B_z^2
\right).
\end{equation}
%
Clearly, the other diagonal terms follow the same pattern, and we can do a cyclic permutation of coordinates to find
%
\begin{equation}\label{eqn:relativisticElectrodynamicsPS6:280}
\begin{aligned}
T^{1 1} &= \inv{8 \pi} \left( - E_x^2 + E_y^2 + E_z^2 - B_x^2 + B_y^2 + B_z^2 \right) \\
T^{2 2} &= \inv{8 \pi} \left( - E_y^2 + E_z^2 + E_x^2 - B_y^2 + B_z^2 + B_x^2 \right) \\
T^{3 3} &= \inv{8 \pi} \left( - E_z^2 + E_x^2 + E_y^2 - B_z^2 + B_x^2 + B_y^2 \right)
\end{aligned}
\end{equation}
%
For the off diagonal terms, let us pick \(T^{1 2}\) and expand that.  We have
%
\begin{equation}\label{eqn:relativisticElectrodynamicsL22:520}
\begin{aligned}
T^{1 2}
&= \inv{4 \pi} \left( - F^{1 k} {F^2}_k - \inv{2} g^{1 2}(\BB^2 - \BE^2) \right) \\
&= \inv{4 \pi} \left( - F^{1 0} F^{2 0} + F^{1 \alpha} F^{2 \alpha} \right) \\
&= \inv{4 \pi} \biglr{ - E_x E_y
+ \mathLabelBox{F^{1 1}}{\(=0\)} F^{2 1}
+ F^{1 2} \mathLabelBox{F^{2 2}}{\(=0\)}
+ F^{1 3} F^{2 3}
} \\
&= \inv{4 \pi} \left( - E_x E_y + (-B_y) B_x \right).
\end{aligned}
\end{equation}
%
Again, with cyclic permutation of the coordinates we have
%
\begin{equation}\label{eqn:relativisticElectrodynamicsPS6:300}
\begin{aligned}
T^{1 2} &= -\inv{4 \pi} \left( E_x E_y + B_x B_y \right) \\
T^{2 3} &= -\inv{4 \pi} \left( E_y E_z + B_y B_z \right) \\
T^{3 1} &= -\inv{4 \pi} \left( E_z E_x + B_z B_x \right).
\end{aligned}
\end{equation}
%
In class these were all written in the compact notation
\boxedEquation{eqn:relativisticElectrodynamicsPS6:320}{
T^{\alpha \beta} = -\inv{4 \pi} \left(
E_\alpha E_\beta
+B_\alpha B_\beta
- \inv{2} \delta_{\alpha\beta} (\BE^2 + \BB^2) \right).
}
