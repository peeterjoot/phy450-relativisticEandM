%
% Copyright � 2012 Peeter Joot.  All Rights Reserved.
% Licenced as described in the file LICENSE under the root directory of this GIT repository.
%

%\chapter{Radiation reaction force continued, and limits of classical electrodynamics}
%\chapter{PHY450H1S.  Relativistic Electrodynamics Lecture 27 (Taught by Prof. Erich Poppitz).  Radiation reaction force continued, and limits of classical electrodynamics}
\label{chap:relativisticElectrodynamicsL27}
%\blogpage{http://sites.google.com/site/peeterjoot/math2011/relativisticElectrodynamicsL27.pdf}
%\date{April 6, 2011}
%
\paragraph{Reading}
%
Some of this, at least the second order expansion, was covered in chapter 8 \S 65 material from the text \citep{landau1980classical}.

Covering \popcite{RelEMpp181-195.pdf}{lecture notes RelEMpp181-195.pdf}.
% pp. 198.1-200}: (last topic): attempt to go to the next order \((v/c)^3\) - radiation damping, the limitations of classical electrodynamics, and the relevant time/length/energy scales.
%
\section{Radiation reaction force}
\index{radiation reaction force}

We previously obtained the radiation reaction force by adding a ``frictional'' force to the harmonic oscillator system.  Now its time to obtain this by continuing the expansion of the potentials to the next order in \(\Bv/c\).

Recall that our potentials are
%
\begin{equation}\label{eqn:relativisticElectrodynamicsL27:10}
\begin{aligned}
\phi(\Bx, t) &= \int d^3 \Bx' \frac{\rho\left(\Bx', t - \Abs{\Bx - \Bx'}/c\right)}{\Abs{\Bx - \Bx'}} \\
\BA(\Bx, t) &= \inv{c}\int d^3 \Bx' \frac{\Bj\left(\Bx', t - \Abs{\Bx - \Bx'}/c\right)}{\Abs{\Bx - \Bx'}}.
\end{aligned}
\end{equation}
%
We can expand in Taylor series about \(t\).  For the charge density this is
%
\begin{equation}\label{eqn:relativisticElectrodynamicsL27:30}
\begin{aligned}
\rho&\left(\Bx', t - \Abs{\Bx - \Bx'}/c\right) \\
&\approx
\rho(\Bx', t)
- \frac{\Abs{\Bx - \Bx'}}{c} \PD{t}{} \rho(\Bx', t)  \\
&+ \inv{2} \left(\frac{\Abs{\Bx - \Bx'}}{c}\right)^2 \PDSq{t}{} \rho(\Bx', t)
- \inv{6} \left(\frac{\Abs{\Bx - \Bx'}}{c}\right)^3 \frac{\partial^3}{\partial t^3} \rho(\Bx', t)
\end{aligned},
\end{equation}
%
so that our scalar potential to third order is
%
\begin{equation}\label{eqn:relativisticElectrodynamicsL27:330}
\begin{aligned}
\phi(\Bx, t)
&=
\int d^3 \Bx' \frac{\rho(\Bx', t) }{\Abs{\Bx - \Bx'}}
- \frac{\Abs{\Bx - \Bx'}}{c} \PD{t}{} \int d^3 \Bx' \frac{\rho(\Bx', t) }{\Abs{\Bx - \Bx'}} \\
&+ \inv{2} \left(\frac{\Abs{\Bx - \Bx'}}{c}\right)^2 \PDSq{t}{} \int d^3 \Bx' \frac{\rho(\Bx', t) }{\Abs{\Bx - \Bx'}}
- \inv{6} \left(\frac{\Abs{\Bx - \Bx'}}{c}\right)^3 \frac{\partial^3}{\partial t^3} \int d^3 \Bx' \frac{\rho(\Bx', t) }{\Abs{\Bx - \Bx'}} \\
&=
\int d^3 \Bx' \frac{\rho(\Bx', t) }{\Abs{\Bx - \Bx'}}
-
\cancel{\PD{t}{} \int d^3 \Bx' \frac{\rho(\Bx', t) }{\Abs{\Bx - \Bx'}}
\frac{\Abs{\Bx - \Bx'}}{c} }
\\
&+ \inv{2}
\PDSq{t}{} \int d^3 \Bx' \frac{\rho(\Bx', t) }{\Abs{\Bx - \Bx'}}
\left(\frac{\Abs{\Bx - \Bx'}}{c}\right)^2
- \inv{6}
\frac{\partial^3}{\partial t^3} \int d^3 \Bx' \frac{\rho(\Bx', t) }{\Abs{\Bx - \Bx'}}
\left(\frac{\Abs{\Bx - \Bx'}}{c}\right)^3  \\
&= \phi^{(0)} + \phi^{(2)} + \phi^{(3)}
\end{aligned}
\end{equation}
%
Expanding the vector potential in Taylor series to second order we have
%
\begin{equation}\label{eqn:relativisticElectrodynamicsL27:350}
\begin{aligned}
\BA(\Bx, t)
&=
\inv{c} \int d^3 \Bx' \frac{\Bj(\Bx', t) }{\Abs{\Bx - \Bx'}}
- \inv{c} \frac{\Abs{\Bx - \Bx'}}{c} \PD{t}{} \int d^3 \Bx' \frac{\Bj(\Bx', t) }{\Abs{\Bx - \Bx'}} \\
%&+ \inv{2 c} \left(\frac{\Abs{\Bx - \Bx'}}{c}\right)^2 \PDSq{t}{} \int d^3 \Bx' \frac{\Bj(\Bx', t) }{\Abs{\Bx - \Bx'}}
%- \inv{6 c} \left(\frac{\Abs{\Bx - \Bx'}}{c}\right)^3 \frac{\partial^3}{\partial t^3} \int d^3 \Bx' \frac{\Bj(\Bx', t) }{\Abs{\Bx - \Bx'}} \\
&=
\inv{c} \int d^3 \Bx' \frac{\Bj(\Bx', t) }{\Abs{\Bx - \Bx'}}
-
\inv{c^2} \PD{t}{} \int d^3 \Bx' \Bj(\Bx', t)
\\
%&+ \inv{2 c}
%\PDSq{t}{} \int d^3 \Bx' \frac{\Bj(\Bx', t) }{\Abs{\Bx - \Bx'}}
%\left(\frac{\Abs{\Bx - \Bx'}}{c}\right)^2
%- \inv{6 c}
%\frac{\partial^3}{\partial t^3} \int d^3 \Bx' \frac{\Bj(\Bx', t) }{\Abs{\Bx - \Bx'}}
%\left(\frac{\Abs{\Bx - \Bx'}}{c}\right)^3  \\
&= \BA^{(1)} + \BA^{(2)}
\end{aligned}
\end{equation}
%
We have already considered the effects of the \(\BA^{(1)}\) term, and now move on to \(\BA^{(2)}\).  We will write \(\phi^{(3)}\) as a total derivative
%
\begin{equation}\label{eqn:relativisticElectrodynamicsL27:50}
\phi^{(3)} = \inv{c} \PD{t}{} \left(
- \inv{6 c^2}
\frac{\partial^2}{\partial t^2} \int d^3 \Bx' \rho(\Bx', t)
\Abs{\Bx - \Bx'}^2
\right)
= \inv{c} \PD{t}{} f^{(2)}(\Bx, t)
\end{equation}
%
and gauge transform it away as we did with \(\phi^{(2)}\) previously.
%
\begin{equation}\label{eqn:relativisticElectrodynamicsL27:70}
\begin{aligned}
\phi^{(3)'} &= \phi^{(3)} - \inv{c} \PD{t}{f^{(2)}} = 0 \\
\BA^{(2)'} &= \BA^{(2)} + \spacegrad f^{(2)}
\end{aligned}
\end{equation}
%
\begin{equation}\label{eqn:relativisticElectrodynamicsL27:370}
\begin{aligned}
\BA^{(2)'}
&=
-
\inv{c^2} \PD{t}{} \int d^3 \Bx' \Bj(\Bx', t)
- \inv{6 c^2}
\frac{\partial^2}{\partial t^2} \int d^3 \Bx' \rho(\Bx', t)
\spacegrad_{\Bx} \Abs{\Bx - \Bx'}^2.
\end{aligned}
\end{equation}
%
Looking first at the first integral we can employ the trick of writing \(\Be_\alpha = \PDi{x^{\alpha'}}{\Bx'}\), and then employ integration by parts
%
\begin{equation}\label{eqn:relativisticElectrodynamicsL27:390}
\begin{aligned}
\int_V d^3 \Bx' \Bj(\Bx', t)
&=
\int_V d^3 \Bx'
\Be_\alpha j^\alpha (\Bx', t)  \\
&=
\int_V d^3 \Bx'
\PD{x^{\alpha'}}{\Bx'}
j^\alpha (\Bx', t)  \\
&=
\int_V d^3 \Bx'
\PD{x^{\alpha'}}{}
\left( \Bx' j^\alpha (\Bx', t) \right)
-\int_V d^3 \Bx'
\Bx' \PD{x^{\alpha'}}{} j^\alpha (\Bx', t) \\
&=
\int_{\partial V} d^2 \Bsigma \cdot
\left( \Bx' j^\alpha (\Bx', t) \right)
-\int d^3 \Bx'
\Bx' -\PD{t}{} \rho(\Bx', t) \\
&=
\PD{t}{} \int d^3 \Bx'
\Bx' \rho(\Bx', t).
\end{aligned}
\end{equation}
%
For the second integral, we have
%
\begin{equation}\label{eqn:relativisticElectrodynamicsL27:410}
\begin{aligned}
\spacegrad_{\Bx} \Abs{\Bx - \Bx'}^2
&=
\Be_\alpha \partial_\alpha
(x^\beta - x^{\beta'})
(x^\beta - x^{\beta'}) \\
&=
2 \Be_\alpha \delta_{\alpha \beta}
(x^\beta - x^{\beta'}) \\
&= 2 (\Bx - \Bx'),
\end{aligned}
\end{equation}
%
so our gauge transformed vector potential term is reduced to
%
\begin{equation}\label{eqn:relativisticElectrodynamicsL27:430}
\begin{aligned}
\BA^{(2)'}
&=
-\inv{c^2} \frac{\partial^2}{\partial t^2} \int d^3 \Bx' \rho(\Bx', t) \left(\Bx' + \inv{3}(\Bx - \Bx') \right) \\
&=
-\inv{c^2} \frac{\partial^2}{\partial t^2} \int d^3 \Bx' \rho(\Bx', t) \left(\inv{3} \Bx + \frac{2}{3}\Bx' \right).
\end{aligned}
\end{equation}
%
Now we wish to employ a discrete representation of the charge density
%
\begin{equation}\label{eqn:relativisticElectrodynamicsL27:90}
\rho(\Bx', t) = \sum_{b=1}^N q_b \delta^3(\Bx' - \Bx_b(t)).
\end{equation}
%
The second order vector potential becomes
\begin{equation}\label{eqn:relativisticElectrodynamicsL27:450}
\begin{aligned}
\BA^{(2)'}
&=
-\inv{c^2} \frac{\partial^2}{\partial t^2} \int d^3 \Bx' \left(\inv{3} \Bx + \frac{2}{3}\Bx' \right)
\sum_{b=1}^N q_b \delta^3(\Bx' - \Bx_b(t)) \\
&=
-\inv{c^2} \frac{\partial^2}{\partial t^2} \sum_{b=1}^N q_b \left( \cancel{\inv{3} \Bx} + \frac{2}{3}\Bx_b(t) \right) \\
&=
-\frac{2}{3 c^2}
\sum_{b=1}^N q_b \ddot{\Bx}_b(t) \\
&=
-\frac{2}{3 c^2} \frac{d^2}{dt^2}
\left( \sum_{b=1}^N q_b \Bx_b(t) \right).
\end{aligned}
\end{equation}
%
We end up with a dipole moment
%
\begin{equation}\label{eqn:relativisticElectrodynamicsL27:110}
\Bd(t) = \sum_{b=1}^N q_b \Bx_b(t)
\end{equation}
%
so we can write
%
\begin{equation}\label{eqn:relativisticElectrodynamicsL27:130}
\BA^{(2)'} = -\frac{2}{3 c^2} \ddot{\Bd}(t).
\end{equation}
%
Observe that there is no magnetic field due to this contribution since there is no explicit spatial dependence
%
\begin{equation}\label{eqn:relativisticElectrodynamicsL27:150}
\spacegrad \cross \BA^{(2)'} = 0
\end{equation}
%
we have also gauge transformed away the scalar potential contribution so have only the time derivative contribution to the electric field
%
\begin{equation}\label{eqn:relativisticElectrodynamicsL27:170}
\BE = -\inv{c} \PD{t}{\BA} - \cancel{\spacegrad \phi} = \frac{2}{3 c^2} \dddot{\Bd}(t).
\end{equation}
%
To \(O((v/c)^3)\) there is a homogeneous electric field felt by all particles, hence every particle feels a ``friction'' force
%
\begin{equation}\label{eqn:relativisticElectrodynamicsL27:190}
\Bf_{\text{rad}} = q \BE = \frac{2 q}{3 c^3} \dddot{\Bd}(t).
\end{equation}
%
\paragraph{Moral:} \(\Bf_{\text{rad}}\) arises in third order term \(O((v/c)^3)\) expansion and thus should not be given a weight as important as the two other terms.  i.e.  Its consequences are less.
%
\makeexample{Our dipole system}{example:relativisticElectrodynamicsL27:1}{
%
\begin{equation}\label{eqn:relativisticElectrodynamicsL27:470}
\begin{aligned}
m \ddot{z}
&= - m \omega^2 a + \frac{2 e^2}{3 c^3} \dddot{z} \\
&= - m \omega^2 a + \frac{2 m}{3 c} \frac{e^2}{m c^2} \dddot{z} \\
&= - m \omega^2 a + \frac{2 m}{3} \frac{r_e}{c} \dddot{z}.
\end{aligned}
\end{equation}
%
Here \(r_e \sim 10^{-13} \text{cm}\) is the classical radius of the electron.  For periodic motion
%
\begin{equation}\label{eqn:relativisticElectrodynamicsL27:490}
\begin{aligned}
z &\sim e^{i \omega t} z_0 \\
\ddot{z} &\sim \omega^2 z_0 \\
\dddot{z} &\sim \omega^3 z_0.
\end{aligned}
\end{equation}
%
The ratio of the last term to the inertial term is
%
\begin{equation}\label{eqn:relativisticElectrodynamicsL27:210}
\sim \frac{ \omega^3 m (r_e/c) z_0 }{ m \omega^2 z_0 } \sim \omega \frac{r_e}{c} \ll 1,
\end{equation}
%
so
%
\begin{equation}\label{eqn:relativisticElectrodynamicsL27:510}
\begin{aligned}
\omega
&\ll \frac{c}{r_e}  \\
&\sim \inv{\tau_e}  \\
&\sim \frac{ 10^{10} \text{cm}/\text{s}}{10^{-13} \text{cm}}  \\
&\sim 10^{23} \text{Hz}.
\end{aligned}
\end{equation}
%
So long as \(\omega \ll 10^{23} \text{Hz}\), this approximation is valid.
}
%
\section{Limits of classical electrodynamics}
%
What sort of energy is this?  At these frequencies QM effects come in
%
\begin{equation}\label{eqn:relativisticElectrodynamicsL27:230}
\Hbar \sim 10^{-33} \text{J} \cdot \text{s} \sim 10^{-15} \text{eV} \cdot \text{s}
\end{equation}
%
\begin{equation}\label{eqn:relativisticElectrodynamicsL27:250}
\Hbar \omega_{max} \sim
10^{-15} \text{eV} \cdot \text{s} \times 10^{23} \inv{\text{s}} \sim 10^8 \text{eV} \sim 100 \text{MeV}
\end{equation}
%
whereas the rest energy of the electron is
%
\begin{equation}\label{eqn:relativisticElectrodynamicsL27:270}
m_e c^2 \sim \inv{2} \text{MeV} \sim \text{MeV}.
\end{equation}
%
At these frequencies it is possible to create \(e^{+}\) and \(e^{-}\) pairs.  A theory where the number of particles (electrons and positrons) is NOT fixed anymore is required.  An estimate of this frequency, where these effects have to be considered is possible.

PICTURE: different length scales with frequency increasing to the left and length scales increasing to the right.

\begin{itemize}
\item \(10^{-13} \text{cm}\), \(r_e = e^2/m c^2\).  LHC exploration.
\item \(137 \times 10^{-13} \text{cm}\), \(\Hbar/m_e c \sim \lambda/2\pi\), the Compton wavelength of the electron.  QED and quantum field theory.
\item \((137)^2 \times 10^{-13}  \text{cm} \sim 10^{-10} \text{cm}\), Bohr radius.  QM, and classical electrodynamics.
\end{itemize}

here
%
\begin{equation}\label{eqn:relativisticElectrodynamicsL27:290}
\alpha = \frac{e^2}{4 \pi \epsilon_0 \Hbar c } = \inv{137},
\end{equation}
%
is the fine structure constant.

Similar to the distance scale restrictions, we have field strength restrictions.  A strong enough field (Electric) can start creating electron and positron pairs.  This occurs at about
%
\begin{equation}\label{eqn:relativisticElectrodynamicsL27:310}
e E \lambda/2\pi \sim 2 m_e c^2
\end{equation}
%
so the critical field strength is
%
\begin{equation}\label{eqn:relativisticElectrodynamicsL27:530}
\begin{aligned}
E_{\text{crit}}
&\sim \frac{m_e c^2 }{\lambda/2\pi e}  \\
&\sim \frac{m_e c^2 }{\Hbar e} m_e c  \\
&\sim \frac{m_e^2 c^3}{\Hbar e}
\end{aligned}
\end{equation}
%
\paragraph{Is this real?}
%
Yes, with a very heavy nucleus with some electrons stripped off, the field can be so strong that positron and electron pairs will be created.  This can be \textunderline{observed} in heavy ion collisions!
