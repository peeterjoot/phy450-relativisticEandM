%
% Copyright © 2012 Peeter Joot.  All Rights Reserved.
% Licenced as described in the file LICENSE under the root directory of this GIT repository.
%
%
\makeproblem{Force per unit area for a Infinite parallel plate capacitor.}{pr:relativisticElectrodynamicsT9:3}{
Find the forces per unit area \( \sigma_{\alpha\beta} \) for a Infinite parallel plate capacitor.
} % makeproblem
%
\makeanswer{pr:relativisticElectrodynamicsT9:3}{
%
\begin{equation}\label{eqn:relativisticElectrodynamicsT9:350}
\begin{aligned}
\BB &= 0  \\
\BE &= - \frac{\sigma}{\epsilon_0} \Be_z.
\end{aligned}
\end{equation}
%
FIXME: derive this.  Observe that we have no distance dependence in the field because it is an infinite plate.
%
\begin{equation}\label{eqn:relativisticElectrodynamicsT9:370}
\begin{aligned}
\sigma_{1 1} &= \left( - \inv{2} \delta^{1 1} \left( \frac{-\sigma}{\epsilon_0} \right)^2 \right) = - \frac{ \sigma^2}{ 2 \epsilon_0^2 } = \sigma_{22} \\
\sigma_{3 3} &= \left( (E^3)^2 - \inv{2} \BE^2 \right)  = - \inv{2} \BE^2 = - \sigma_{2 2}.
\end{aligned}
\end{equation}
%
Force per unit area is then
%
\begin{equation}\label{eqn:relativisticElectrodynamicsT9:900}
\begin{aligned}
f_\alpha
&= n_\beta \sigma_{\alpha \beta} \\
&= n_3 \sigma_{\alpha 3}.
\end{aligned}
\end{equation}
%
So
%
\begin{equation}\label{eqn:relativisticElectrodynamicsT9:390}
\begin{aligned}
f_1 &= 0 = f_2 \\
f_3 &= \sigma_{3 3} = -\frac{\sigma^2}{2 \epsilon_0^2}.
\end{aligned}
\end{equation}
%
\begin{equation}\label{eqn:relativisticElectrodynamicsT9:410}
\Bf = -\frac{\sigma^2}{2 \epsilon_0^2} \Be_z.
\end{equation}
} % makeanswer
