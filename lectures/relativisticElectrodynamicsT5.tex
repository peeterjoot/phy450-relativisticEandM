%
% Copyright � 2012 Peeter Joot.  All Rights Reserved.
% Licenced as described in the file LICENSE under the root directory of this GIT repository.
%

%\chapter{Angular momentum of EM fields}
\index{field!angular momentum}
\label{chap:relativisticElectrodynamicsT5}
%\blogpage{http://sites.google.com/site/peeterjoot/math2011/relativisticElectrodynamicsT5.pdf}
%\date{Mar 10, 2011}
%
\makeproblem{Angular momentum of EM fields}{pr:relativisticElectrodynamicsT5:0}{
%
(This was a worked problem covered in tutorial 5).

Long solenoid of radius \(R\), n turns per unit length, current \(I\).  Coaxial with with solenoid are two long cylindrical shells of length \(l\) and \((\text{radius},\text{charge})\) of \((a, Q)\), and \((b, -Q)\) respectively, where \(a < b\).

When current is gradually reduced what happens?

To determined this, compute the
\makesubproblem{}{pr:relativisticElectrodynamicsT5:1:a}
%
initial Magnetic field,
%
\makesubproblem{}{pr:relativisticElectrodynamicsT5:1:b}
%
initial Electric field,
%
\makesubproblem{}{pr:relativisticElectrodynamicsT5:1:c}
%
Poynting vector before the current changes,
%
\makesubproblem{}{pr:relativisticElectrodynamicsT5:1:d}
%
momentum density of the EM fields,
%
\makesubproblem{}{pr:relativisticElectrodynamicsT5:1:e}
%
induced electric field after the current is changed, and
%
\makesubproblem{}{pr:relativisticElectrodynamicsT5:1:f}
%
the torque and angular momentum induced by the fields.

} % makeproblem
%
\makeanswer{pr:relativisticElectrodynamicsT5:0}{
%
\makeSubAnswer{Initial Magnetic field}{pr:relativisticElectrodynamicsT5:1:a}
%
For the initial static conditions where we have only a (constant) magnetic field, the Maxwell-Ampere equation takes the form
%
\begin{equation}\label{eqn:relativisticElectrodynamicsT5:10}
\spacegrad \cross \BB = \frac{4 \pi}{c} \Bj
\end{equation}
%
\paragraph{On the name of this equation}.  In notes from one of the lectures I had this called Maxwell-Faraday equation, despite the fact that this is not the one that Maxwell made his displacement current addition.  Did the Professor call it that, or was this my addition?  In \citep{wiki:Faraday} Faraday's law is also called the Maxwell-Faraday equation.  \citep{fleisch2007ssg} calls this the Ampere-Maxwell equation, which makes more sense.
%
Put into integral form by integrating over an open surface we have
%
\begin{equation}\label{eqn:relativisticElectrodynamicsT5:30}
\int_A (\spacegrad \cross \BB) \cdot d\Ba = \frac{4 \pi}{c} \int_A \Bj \cdot d\Ba
\end{equation}

The current density passing through the surface is defined as the enclosed current, circulating around the bounding loop
%
\begin{equation}\label{eqn:relativisticElectrodynamicsT5:50}
I_{\text{enc}} = \int_A \Bj \cdot d\Ba.
\end{equation}

This is a sensible definition.  Consider a little bit of that current
%
\begin{equation}\label{eqn:relativisticElectrodynamicsT5:51}
dI_{\text{enc}} = \frac{dQ}{dV} \Bv \cdot d\Ba.
\end{equation}

If we consider the charge density volume \(dV = da dl\), where \(da = \vcap \cdot d\Ba\), we have
%
\begin{equation}\label{eqn:relativisticElectrodynamicsT5:52}
dI_{\text{enc}} = \frac{dQ}{dl} \frac{dl}{dt} = \frac{dQ}{dt}
\end{equation}

At least dimensionally, this is a sensible quantity to define.

Motivation aside, by Stokes Theorem, we can therefore write the circulation of the magnetic field in terms of this enclosed current
%
\begin{equation}\label{eqn:relativisticElectrodynamicsT5:70}
\int_{\partial A} \BB \cdot d\Bl = \frac{4 \pi}{c} I_{\text{enc}}
\end{equation}

Now consider separately the regions inside and outside the cylinder.  Inside we have
%
\begin{equation}\label{eqn:relativisticElectrodynamicsT5:71}
\int_{\partial A} B \cdot d \Bl = \frac{4 \pi I }{c} = 0,
\end{equation}

Outside of the cylinder we have the equivalent of \(n\) loops, each with current \(I\), so we have
%
\begin{equation}\label{eqn:relativisticElectrodynamicsT5:90}
\int \BB \cdot d\Bl = \frac{4 \pi n I L}{c} = B L.
\end{equation}

Our magnetic field is constant while \(I\) is constant, and in vector form this is
%
\begin{equation}\label{eqn:relativisticElectrodynamicsT5:110}
\BB = \frac{4 \pi n I}{c} \zcap
\end{equation}
%
\makeSubAnswer{Initial Electric field}{pr:relativisticElectrodynamicsT5:1:b}
%
How about the electric fields?

For \(r < a\), and \(r > b\) we have \(\BE = 0\) since there is no charge enclosed by any Gaussian surface that we choose.

Between \(a\) and \(b\) we have, for a Gaussian surface of height \(l\) (assuming that \(l \gg a\))
%
\begin{equation}\label{eqn:relativisticElectrodynamicsT5:130}
E (2 \pi r) l = 4 \pi (+Q),
\end{equation}

so we have
%
\begin{equation}\label{eqn:relativisticElectrodynamicsT5:150}
\BE = \frac{2 Q }{r l} \rcap.
\end{equation}
%
\makeSubAnswer{Poynting vector before the current changes}{pr:relativisticElectrodynamicsT5:1:c}
%
Our Poynting vector, the energy flux per unit time, is
%
\begin{equation}\label{eqn:relativisticElectrodynamicsT5:151}
\BS = \frac{c}{4 \pi} (\BE \cross \BB)
\end{equation}

This is non-zero only in the region both between the solenoid and the enclosing cylinder (radius \(b\)) since that is the only place where both \(\BE\) and \(\BB\) are non-zero.  That is
%
\begin{equation}\label{eqn:relativisticElectrodynamicsT5:371}
\begin{aligned}
\BS &= \frac{c}{4 \pi} (\BE \cross \BB) \\
&=
\frac{c}{4 \pi} \frac{2 Q }{r l} \frac{4 \pi n I}{c} \rcap \cross \zcap \\
&= -\frac{2 Q n I}{r l} \phicap
\end{aligned}
\end{equation}

(since \(\rcap \cross \phicap = \zcap\), so \(\zcap \cross \rcap = \phicap\) after cyclic permutation)
%
\paragraph{A motivational aside:  Momentum density}
\index{momentum density}

Suppose \(\Abs{\BE} = \Abs{\BB}\), then our Poynting vector is
\index{Poynting vector}
%
\begin{equation}\label{eqn:relativisticElectrodynamicsT5:170}
\BS = \frac{c}{4 \pi} \BE \cross \BB = \frac{ c \kcap}{4 \pi} \BE^2,
\end{equation}

but
%
\begin{equation}\label{eqn:relativisticElectrodynamicsT5:190}
\calE = \text{energy density} = \frac{\BE^2 + \BB^2}{8 \pi} = \frac{\BE^2}{4 \pi},
\end{equation}

so
%
\begin{equation}\label{eqn:relativisticElectrodynamicsT5:210}
\BS = c \kcap \calE = \Bv \calE.
\end{equation}

Now recall the between (relativistic) mechanical momentum \(\Bp = \gamma m \Bv\) and energy \(\calE = \gamma m c^2\)
%
\begin{equation}\label{eqn:relativisticElectrodynamicsT5:230}
\Bp = \frac{\Bv}{c^2} \calE.
\end{equation}

This justifies calling the quantity
%
\begin{equation}\label{eqn:relativisticElectrodynamicsT5:250}
\BP_{\text{EM}} = \frac{\BS}{c^2},
\end{equation}

the momentum density.
%
\makeSubAnswer{Momentum density of the EM fields}{pr:relativisticElectrodynamicsT5:1:d}
So we label our scaled Poynting vector the momentum density for the field
\begin{equation}\label{eqn:relativisticElectrodynamicsT5:270}
\BP_{\text{EM}} = -\frac{2 Q n I}{c^2 r l} \phicap,
\end{equation}
and can now compute an angular momentum density in the field between the solenoid and the outer cylinder prior to changing the currents
%
\begin{equation}\label{eqn:relativisticElectrodynamicsT5:391}
\begin{aligned}
\BL_{\text{EM}}
&= \Br \cross \BP_{\text{EM}} \\
&= r \rcap \cross \BP_{\text{EM}}.
\end{aligned}
\end{equation}
This gives us
\begin{equation}\label{eqn:relativisticElectrodynamicsT5:271}
\BL_{\text{EM}} = -\frac{2 Q n I}{c^2 l} \zcap = \text{constant}.
\end{equation}
Note that this is the angular momentum density in the region between the solenoid and the inner cylinder, between \(z = 0\) and \(z = l\).  Outside of this region, the angular momentum density is zero.
%
\makeSubAnswer{Induced electric field after the current is changed}{pr:relativisticElectrodynamicsT5:1:e}
When we turn off (or change) \(I\), some of the magnetic field \(\BB\) will be converted into electric field \(\BE\) according to Faraday's law
%
\begin{equation}\label{eqn:relativisticElectrodynamicsT5:290}
\spacegrad \cross \BE = - \inv{c} \PD{t}{\BB}.
\end{equation}

In integral form, utilizing an open surface, this is
%
\begin{equation}\label{eqn:relativisticElectrodynamicsT5:411}
\begin{aligned}
\int_A (\spacegrad \cross \Bl) \cdot \ncap dA
&=
\int_{\partial A} \BE \cdot d\Bl \\
&= - \inv{c} \int_A \PD{t}{\BB} \cdot d\BA \\
&= - \inv{c} \PD{t}{\Phi_B(t)},
\end{aligned}
\end{equation}

where we introduce the magnetic flux
%
\begin{equation}\label{eqn:relativisticElectrodynamicsT5:291}
\Phi_B(t) = \int_A \BB \cdot d\BA.
\end{equation}

We can utilizing a circular surface cutting directly across the cylinder perpendicular to \(\zcap\) of radius \(r\).  Recall that we have the magnetic field \eqnref{eqn:relativisticElectrodynamicsT5:110} only inside the solenoid.  So for \(r < R\) this flux is
%
\begin{equation}\label{eqn:relativisticElectrodynamicsT5:431}
\begin{aligned}
\Phi_B(t)
&= \int_A \BB \cdot d\BA \\
&= (\pi r^2) \frac{4 \pi n I(t)}{c}.
\end{aligned}
\end{equation}

For \(r > R\) only the portion of the surface with radius \(r \le R\) contributes to the flux
%
\begin{equation}\label{eqn:relativisticElectrodynamicsT5:451}
\begin{aligned}
\Phi_B(t)
&= \int_A \BB \cdot d\BA \\
&=
\lr{ \pi R^2 }
\frac{4 \pi n I(t)}{c}.
\end{aligned}
\end{equation}

We can now compute the circulation of the electric field
%
\begin{equation}\label{eqn:relativisticElectrodynamicsT5:292}
\int_{\partial A} \BE \cdot d\Bl = - \inv{c} \PD{t}{\Phi_B(t)},
\end{equation}

by taking the derivatives of the magnetic flux.  For \(r > R\) this is
%
\begin{equation}\label{eqn:relativisticElectrodynamicsT5:471}
\begin{aligned}
\int_{\partial A} \BE \cdot d\Bl
&= (2 \pi r) E \\
&=
-
\lr{ \pi R^2 }
\frac{4 \pi n \dot{I}(t)}{c^2}.
\end{aligned}
\end{equation}

This gives us the magnitude of the induced electric field
%
\begin{equation}\label{eqn:relativisticElectrodynamicsT5:491}
\begin{aligned}
E
&= -\lr{ \pi R^2 } \frac{4 \pi n \dot{I}(t)}{2 \pi r c^2} \\
&= -\frac{2 \pi R^2 n \dot{I}(t)}{r c^2}.
\end{aligned}
\end{equation}

Similarly for \(r < R\) we have
%
\begin{equation}\label{eqn:relativisticElectrodynamicsT5:293}
E = -\frac{2 \pi r n \dot{I}(t)}{c^2}
\end{equation}

Summarizing we have
%
\begin{equation}\label{eqn:relativisticElectrodynamicsT5:310}
\BE =
\left\{
\begin{array}{l l}
-\frac{2 \pi r n \dot{I}(t)}{c^2} \phicap 		& \mbox{For \(r < R\)} \\
-\frac{2 \pi R^2 n \dot{I}(t)}{r c^2} \phicap 		& \mbox{For \(r > R\)}
\end{array}
\right.
\end{equation}
%
\makeSubAnswer{Torque and angular momentum induced by the fields}{pr:relativisticElectrodynamicsT5:1:f}
%
Our torque \(\BN = \Br \cross \BF = d\BL/dt\) on the outer cylinder (radius \(b\)) that is induced by changing the current is
%
\begin{equation}\label{eqn:relativisticElectrodynamicsT5:511}
\begin{aligned}
\BN_b
&= (b \rcap) \cross (-Q \BE_{r = b}) \\
&= b Q \frac{2 \pi R^2 n \dot{I}(t)}{b c^2} \rcap \cross \phicap \\
&= \inv{c^2} 2 \pi R^2 n Q \dot{I} \zcap.
\end{aligned}
\end{equation}

This provides the induced angular momentum on the outer cylinder
%
\begin{equation}\label{eqn:relativisticElectrodynamicsT5:531}
\begin{aligned}
\BL_b
&= \int dt \BN_b = \frac{ 2 \pi n R^2 Q}{c^2} \int_I^0 \frac{dI}{dt} dt \\
&= -\frac{2 \pi n R^2 Q}{c^2} I.
\end{aligned}
\end{equation}

This is the angular momentum of \(b\) induced by changing the current or changing the magnetic field.

On the inner cylinder we have
%
\begin{equation}\label{eqn:relativisticElectrodynamicsT5:551}
\begin{aligned}
\BN_a
&= (a \rcap ) \cross (Q \BE_{r = a}) \\
&= a Q \left(- \frac{2 \pi}{c} n a \dot{I} \right) \rcap \cross \phicap \\
&= -\frac{2 \pi n a^2 Q \dot{I}}{c^2} \zcap.
\end{aligned}
\end{equation}

So our induced angular momentum on the inner cylinder is
%
\begin{equation}\label{eqn:relativisticElectrodynamicsT5:330}
\BL_a = \frac{2 \pi n a^2 Q I}{c^2} \zcap.
\end{equation}

The total angular momentum in the system has to be conserved, and we must have
%
\begin{equation}\label{eqn:relativisticElectrodynamicsT5:350}
\BL_a + \BL_b = -\frac{2 n I Q}{c^2} \pi \lr{ R^2 - a^2 } \zcap.
\end{equation}

At the end of the tutorial, this sum was equated with the field angular momentum density \(\BL_{\text{EM}}\), but this has different dimensions.  In fact, observe that the volume in which this angular momentum density is non-zero is the difference between the volume of the solenoid and the inner cylinder
%
\begin{equation}\label{eqn:relativisticElectrodynamicsT5:351}
V = \pi R^2 l - \pi a^2 l,
\end{equation}

so if we are to integrate the angular momentum density \eqnref{eqn:relativisticElectrodynamicsT5:271} over this region we have
%
\begin{equation}\label{eqn:relativisticElectrodynamicsT5:271b}
\int \BL_{\text{EM}} dV = -\frac{2 Q n I}{c^2} \pi \lr{ R^2 - a^2 } \zcap
\end{equation}

which does match with the sum of the mechanical angular momentum densities \eqnref{eqn:relativisticElectrodynamicsT5:350} as expected.
} % makeanswer
