%
% Copyright � 2012 Peeter Joot.  All Rights Reserved.
% Licenced as described in the file LICENSE under the root directory of this GIT repository.
%

%\chapter{Action for the field}
\index{action!field}
\label{chap:relativisticElectrodynamicsL12}
%\useCCL
%\blogpage{http://sites.google.com/site/peeterjoot/math2011/relativisticElectrodynamicsL12.pdf}
%\date{Feb 10, 2011}

\paragraph{Reading}

Covering chapter 3 material from the text \citep{landau1980classical}, and
\popcite{RelEMpp84-102.pdf}{lecture notes RelEMpp84-102.pdf}.
%: relativity, gauge invariance, and superposition principles and the action for the electromagnetic field coupled to charged particles (91-95); the 4-current and its physical interpretation (96-102), including a needed mathematical interlude on delta-functions of functions (98-100) [Wednesday, Feb. 8; Thursday, Feb. 10]

\section{Where we are}
%
\begin{equation}\label{eqn:relativisticElectrodynamicsL12:10}
F_{ij} = \partial_i A_j - \partial_j A_i
\end{equation}

We learned that one half of Maxwell's equations comes from the Bianchi identity
%
\begin{equation}\label{eqn:relativisticElectrodynamicsL12:30}
\epsilon^{ijkl} \partial_j F_{kl} = 0
\end{equation}

the other half (for vacuum) is
%
\begin{equation}\label{eqn:relativisticElectrodynamicsL12:50}
\partial_j F_{ji} = 0
\end{equation}

To get here we have to consider the action for the field.

\section{Generalizing the action to multiple particles}
\index{action}
\index{multiple particles}

We have learned that the action for a single particle is
%
\begin{equation}\label{eqn:relativisticElectrodynamicsL12:90}
\begin{aligned}
S
&= S_{\text{matter}} + S_{\text{interaction}} \\
&= -m c \int ds - \frac{e}{c} \int ds^i A_i
\end{aligned}
\end{equation}

This generalizes to more particles
%
\begin{equation}\label{eqn:relativisticElectrodynamicsL12:70}
S_{\text{``particles in field''}}
=
-
\sum_A
m_A c \int_{x^A(\tau)} ds
-
\sum_A
\frac{e_A}{c} \int dx^i_A A_i(x_A(\tau))
\end{equation}

\(A\) labels the particles, and \(x^A(\tau)\), \(\{x^A(\tau), A= 1 \cdots N\}\) is the worldline of particle \(A\).

