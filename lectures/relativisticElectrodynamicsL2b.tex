%
% Copyright © 2012 Peeter Joot.  All Rights Reserved.
% Licenced as described in the file LICENSE under the root directory of this GIT repository.
%

\index{spacetime}

We will need to develop some tools to work with these concepts in a concrete fashion.  It is convenient to combine space \R{3} and time \R{1} into a 4d ``spacetime''.  In \citep{landau1980classical} this is called fictitious spacetime for reasons that are not clear.  Points in this space are also called ``events'', or ``spacetime points'', or ``world point''.  The ``world line'' is the trajectory for a particle in spacetime.

PICTURE: \R{3} represented as a plane, and \(t\) up.  For every point we can plot an \(\Bx(t)\) in this combined space.

\section{Intervals for light like behaviour}
\index{light like}

Consider two frames, one moving along the x-axis at a (constant) rate not yet specified.

``events'' have coordinates \((t, \Bx)\) in \(O\) and \((t', \Bx')\) in \(O'\).  Because we now have to model the mathematics without a notion of simultaneity, we must now also introduce different time coordinates \(t\), and \(t'\) in the two frames.

Let us imagine that at at time \(t_1\) light is emitted at \(\Bx_1\), and at time \(t_2\) this light is absorbed.  Our space time events are then \((t_1, \Bx_1)\) and \((t_2, \Bx_2)\).  In the \(O\) frame, the light will go a distance \(c(t_2 - t_1)\).  This same distance can also be expressed as
%
\begin{equation}\label{eqn:relativisticElectrodynamicsL2:10}
\sqrt{ (\Bx_1 - \Bx_2)^2}.
\end{equation}

These are equal.  It is convenient to work without the square roots, so we write
%
\begin{equation}\label{eqn:relativisticElectrodynamicsL2:20}
(\Bx_1 - \Bx_2)^2 = c^2 (t_2 - t_1)^2
\end{equation}

Or
%
\begin{equation}\label{eqn:relativisticElectrodynamicsL2:30}
\begin{aligned}
c^2 &(t_2 - t_1)^2 - (\Bx_1 - \Bx_2)^2 = \\
&c^2 (t_2 - t_1)^2
- (x_1 - x_2)^2
- (y_1 - y_2)^2
- (z_1 - z_2)^2 = 0.
\end{aligned}
\end{equation}

We can repeat the same argument for the primed frame.  In this frame, at time \(t_1'\) light is emitted at \(\Bx_1'\), and at time \(t_2'\) this light is absorbed.  Our space time events in this frame are then \((t_1', \Bx_1')\) and \((t_2', \Bx_2')\).  As above, in this \(O'\) frame, the light will go a distance \(c(t_2' - t_1')\), with a similar Euclidean distance involving \(\Bx_1'\) and \(\Bx_2'\).  That is
%
\begin{equation}\label{eqn:relativisticElectrodynamicsL2:40}
\begin{aligned}
c^2 &(t_2' - t_1')^2 - (\Bx_1' - \Bx_2')^2 = \\
&c^2 (t_2' - t_1')^2
- (x_1' - x_2')^2
- (y_1' - y_2')^2
- (z_1' - z_2')^2 = 0.
\end{aligned}
\end{equation}

We get zero for this quantity in any inertial frame 1.  This quantity is found to be very important, and want to give this a label.  We call this the ``interval'', or the ``spacetime interval'', and write this as follows:
%
\begin{equation}\label{eqn:relativisticElectrodynamicsL2:50}
s_{12}^2 = c^2 (t_2 - t_1)^2 - (\Br_2 - \Br_1)^2
\end{equation}

This is a quantity calculated between any two spacetime points with coordinates \((t_2, \Br_2)\) and \((t_1, \Br_1)\) in some frame.

So far we have argued that \(c\) being the same in any two frames implies that spacetime events ``separated by a zero interval'' in one frame are ``separated by a zero interval'' in any other frame.

\section{Invariance of infinitesimal intervals}

For events that are infinitesimally close to each other.  i.e. \(t_2 - t_1\) and \(\Br_2 -\Br_1\) are small (infinitesimal), it is convenient to denote \(t_2 - t_1\) and \(\Br_2 - \Br_1\) by \(dt\) and \(d\Br\) respectively.  We can then define
%
\begin{equation}\label{eqn:relativisticElectrodynamicsL2:60}
ds_{12}^2 = c^2 dt^2 - d\Br^2,
\end{equation}
or
\begin{equation}\label{eqn:relativisticElectrodynamicsL2:60b}
ds= \sqrt{c^2 dt^2 - d\Br^2}.
\end{equation}
We will use this a lot.

We have learned that if \(s_{12} = 0\) in one frame, then \(s_{12}' = 0\) in any other frame.  We generally expect that there is a relation \(s_{12}' = F(s_12)\) between the intervals in two frames.  So far we have learned that \(F(0) = 0\).

Let us now consider the case where both of these intervals are infinitesimal.  Then we can write
%
\begin{equation}\label{eqn:relativisticElectrodynamicsL2:70}
ds_{12}' = F(ds_{12}) = F(0) + F'(0) ds_{12} + \cdots = F'(0) ds_{12} + \cdots.
\end{equation}

We will neglect terms \(O(ds_{12})^2\) and higher.  Thus equality of zero intervals between two frames implies that
%
\begin{equation}\label{eqn:relativisticElectrodynamicsL2:80}
ds_{12}' \sim ds_{12}.
\end{equation}

Now we must invoke an assumption (principle) of homogeneity of time and space and isotropy of space.  This interval should not depend on where these events take place, or on the time that the measurements were performed.  If this is the case then we conclude that the proportionality constant relating the two intervals is not a function of position or space.  We argue that this proportionality can then only be a function of the (absolute) relative speed between the frames.

We write this as
\begin{equation}\label{eqn:relativisticElectrodynamicsL2:90}
ds_{12}' = F(v_{12}) ds_{12}
\end{equation}

This argument can be turned around and we say that \(ds_{12} = \tilde{F}(v_{12}) ds_{12}'\).  Thus \(\tilde{F} = F\), because there is no distinction between \(O\) and \(O'\).  We want to conclude that
%
\begin{equation}\label{eqn:relativisticElectrodynamicsL2:100}
ds_{12} = F(v_{12}) ds_{12}' = F(v_{12}) \tilde{F}(v_{12}) ds_{12}
\end{equation}

and then conclude that \(F = \tilde{F} = 1\).  This argument is to be continued.  To complete this conclusion we will need to perform some additional math, once we cover finite intervals.
