%
% Copyright � 2012 Peeter Joot.  All Rights Reserved.
% Licenced as described in the file LICENSE under the root directory of this GIT repository.
%

%\chapter{Potentials at a distance from a localized current distribution.  EM fields due to dipole radiation}
\index{dipole radiation}
\label{chap:relativisticElectrodynamicsL20}
%\blogpage{http://sites.google.com/site/peeterjoot/math2011/relativisticElectrodynamicsL20.pdf}
%\date{Mar 15, 2011}
%
\paragraph{Reading}
%
Covering chapter 8 material from the text \citep{landau1980classical}, and
\popcite{RelEMpp147-165.pdf}{lecture notes RelEMpp147-165.pdf}.
% EM fields of a moving source (147-148+HW5); a particle at rest (148); a constant velocity particle (149-152); behavior of EM fields ``at infinity'' for a general-worldline source and radiation (152-153) [Tuesday, Mar. 15]; radiated power (154); fields in the ``wave zone'' and discussions of approximations made (155-159); EM fields due to electric dipole radiation (160-163); Poynting vector, angular distribution, and power of dipole radiation (164-165) [Wednesday, Mar. 16...]
%
\section{Multipole expansion of the fields}
\index{multipole expansion}
%
\begin{equation}\label{eqn:relativisticElectrodynamicsL20:10}
A^i(\Bx, t) = \inv{c} \int d^3 \Bx' j^i\left(\Bx', t - \frac{\Abs{\Bx - \Bx'} }{c}\right) \inv{\Abs{\Bx - \Bx'} }
\end{equation}
%
This integral is over the region of space where the sources \(j^i\) are non-vanishing, but this region is limited.  The value \(\Abs{\Bx'} \le l\), so we can expand the denominator in multipole expansion
%
\begin{equation}\label{eqn:relativisticElectrodynamicsL20:460}
\begin{aligned}
\inv{\Abs{\Bx - \Bx'} }
&=
\inv{\sqrt{(\Bx - \Bx')^2} } \\
&=
\inv{\sqrt{\Bx^2 + {\Bx'}^2 - 2 \Bx \cdot \Bx'} } \\
&=
\inv{\Abs{\Bx} } \inv{\sqrt{1 + \frac{{\Bx'}^2}{\Bx^2} - 2 \frac{\rcap}{\Abs{\Bx} } \cdot \Bx'} } \\
&\approx
\inv{\Abs{\Bx} } \inv{\sqrt{1 - 2 \frac{\rcap}{\Abs{\Bx} } \cdot \Bx'} } \\
&\approx
\inv{\Abs{\Bx} } \left(1 + \frac{\rcap}{\Abs{\Bx} } \cdot \Bx' \right).
\end{aligned}
\end{equation}
%
Neglecting all but the first order term in the expansion we have
\begin{equation}\label{eqn:relativisticElectrodynamicsL20:30}
\inv{\Abs{\Bx - \Bx'} }
\approx
\inv{\Abs{\Bx} } + \frac{\Bx}{\Abs{\Bx}^3} \cdot \Bx' .
\end{equation}
%
Similarly, for the retarded time we have
%
\begin{equation}\label{eqn:relativisticElectrodynamicsL20:480}
\begin{aligned}
t - \frac{\Abs{\Bx - \Bx'} }{c}
&\approx t - \frac{\Abs{\Bx} }{c} \left( 1 - \frac{\Bx \cdot \Bx'}{\Abs{\Bx}^2} \right) \\
&= t - \frac{\Abs{\Bx} }{c} + \frac{\Bx \cdot \Bx'}{c \Abs{\Bx} }
\end{aligned}
\end{equation}
%
We can now do a first order Taylor expansion of the current \(j^i\) about the retarded time
%
\begin{equation}\label{eqn:relativisticElectrodynamicsL20:50}
j^i\left(\Bx', t - \frac{\Abs{\Bx} }{c} + \frac{\Bx \cdot \Bx'}{c \Abs{\Bx} } + \cdots \right)
\approx
j^i\left(\Bx', t - \frac{\Abs{\Bx} }{c}\right) + \PD{t}{j^i} \left(\Bx, t - \frac{\Abs{\Bx} }{c}\right) \frac{\Bx \cdot \Bx'}{c \Abs{\Bx} }.
\end{equation}
%
To elucidate the physics, imagine that time dependence of the source is periodic with angular frequency \(\omega_0\).  For example:
%
\begin{equation}\label{eqn:relativisticElectrodynamicsL20:500}
j^i = A(\Bx) e^{-i \omega t}.
\end{equation}
%
Here we have
\begin{equation}\label{eqn:relativisticElectrodynamicsL20:70}
\PD{t}{j^i} = -i \omega_0 j^i,
\end{equation}
so, for the magnitude of the second term we have
%
\begin{equation}\label{eqn:relativisticElectrodynamicsL20:90}
\Abs{\PD{t}{j^i} \frac{\Bx \cdot \Bx'}{c \Abs{\Bx} }} = \omega_0 \Abs{j^i \frac{\Bx \cdot \Bx'}{c \Abs{\Bx} }}.
\end{equation}
Requiring second term much less than the first term means
%
\begin{equation}\label{eqn:relativisticElectrodynamicsL20:110}
\Abs{\omega_0 \frac{\Bx \cdot \Bx'}{c \Abs{\Bx} }} \ll 1.
\end{equation}
But recall
\begin{equation}\label{eqn:relativisticElectrodynamicsL20:130}
\Abs{\frac{\Bx \cdot \Bx'}{c \Abs{\Bx} }} \le l,
\end{equation}
so for our Taylor expansion to be valid we have the following constraints on the angular velocity and the position vectors for our charge and measurement position
%
\begin{equation}\label{eqn:relativisticElectrodynamicsL20:150}
\Abs{\omega_0 \frac{\Bx \cdot \Bx'}{c \Abs{\Bx} }} \le \frac{\omega_0 l}{c} \ll 1.
\end{equation}
%
This is a physical requirement size of the wavelength of the emitter (if the wavelength does not meet this requirement, this expansion does not work).  The connection to the wavelength can be observed by noting that we have
%
\begin{equation}\label{eqn:relativisticElectrodynamicsL20:520}
\begin{aligned}
\frac{\omega_0}{c} &= k \\
2 \pi k &= \inv{\lambda} \\
\implies \frac{\omega_0}{c} &\sim \inv{\lambda}.
\end{aligned}
\end{equation}
% as noted in class.  was not clear to me:
%\frac{\omega_0}{c} &= \frac{2 \pi}{c T} \sim \inv{\lambda}
%
\section{Putting the pieces together.  Potentials at a distance}
%
\paragraph{Moral:} We will utilize two expansions (we need two small parameters)
%
\begin{enumerate}
\item \(\Abs{\Bx} \gg l\)
\item \(\lambda \gg l\)
\end{enumerate}

Plugging into our current
%
\begin{equation}\label{eqn:relativisticElectrodynamicsL20:190}
A^i(\Bx, t)
\approx \inv{c} \int d^3 \Bx'
\left( j^i\left(\Bx', t - \frac{\Abs{\Bx} }{c}\right) + \PD{t}{j^i} \left(\Bx, t - \frac{\Abs{\Bx} }{c}\right) \frac{\Bx \cdot \Bx'}{c \Abs{\Bx} } \right)
\left( \inv{\Abs{\Bx} } + \frac{\Bx}{\Abs{\Bx}^3} \cdot \Bx' \right)
\end{equation}
%
\begin{equation}\label{eqn:relativisticElectrodynamicsL20:210}
\begin{aligned}
A^0(\Bx, t)
&\approx
\inv{\Abs{\Bx} } \int d^3 \Bx' \rho\left(\Bx', t - \frac{\Abs{\Bx} }{c}\right) \\
&+\frac{\Bx}{\Abs{\Bx}^3} \cdot \int d^3 \Bx' \Bx' \rho \left(\Bx', t - \frac{\Abs{\Bx} }{c}\right)
+\frac{\Bx}{c \Abs{\Bx}^2} \cdot \int d^3 \Bx' \Bx' \PD{t}{\rho}\left(\Bx', t - \frac{\Abs{\Bx} }{c}\right).
\end{aligned}
\end{equation}
%
The first term is the total charge evaluated at the retarded time.  In the second term (and in the third, where its derivative is taken) we have
%
\begin{equation}\label{eqn:relativisticElectrodynamicsL20:230}
\int d^3 \Bx' \Bx' \rho\left(\Bx', t - \frac{\Abs{\Bx} }{c}\right) = \Bd(t_r),
\end{equation}
%
which is the dipole moment evaluated at the retarded time \(t_r = t - \Abs{\Bx}/c\).  In the last term we can pull out the time derivative (because we are integrating over \(\Bx'\))
%
\begin{equation}\label{eqn:relativisticElectrodynamicsL20:540}
\begin{aligned}
\inv{\Abs{\Bx}^2} \Bx \cdot \int d^3 \Bx' \Bx' \PD{t}{} \rho\left(\Bx', t - \frac{\Abs{\Bx} }{c}\right)
&=
\inv{\Abs{\Bx}^2} \Bx \cdot \PD{t}{} \int d^3 \Bx' \Bx' \rho\left(\Bx', t - \frac{\Abs{\Bx} }{c}\right) \\
&=
\inv{\Abs{\Bx}^2} \Bx \cdot \PD{t}{}\Bd \left(t - \frac{\Abs{\Bx} }{c}\right)
\end{aligned}
\end{equation}
%
For the spatial components of the current lets just keep the first term
%
\begin{equation}\label{eqn:relativisticElectrodynamicsL20:560}
\begin{aligned}
A^\alpha(\Bx, t)
&\approx
\inv{ c \Abs{\Bx} } \int d^3 \Bx' j^\alpha\left(\Bx', t - \frac{\Abs{\Bx} }{c}\right) \\
&=
\inv{ c \Abs{\Bx} } \int d^3 \Bx' (\spacegrad_{\Bx'} x^\alpha) \cdot \Bj\left(\Bx', t - \frac{\Abs{\Bx} }{c}\right)  \\
&=
\inv{ c \Abs{\Bx} } \int d^3 \Bx'
\left(
\spacegrad \cdot \left( {x'}^\alpha \Bj \left(\Bx', t - \frac{\Abs{\Bx} }{c}\right) \right)
- {x'}^\alpha \spacegrad_{\Bx'} \cdot \Bj\left(\Bx', t - \frac{\Abs{\Bx} }{c}\right)
\right) \\
&=
\inv{ c \Abs{\Bx} } \oint_{S^2_\infty} d^2 \Bsigma \cdot {x'}^\alpha \Bj\left(\Bx', t - \frac{\Abs{\Bx} }{c}\right)
+\inv{ c \Abs{\Bx} } \int d^3 \Bx' {x'}^\alpha \PD{t}{}\rho\left(\Bx', t - \frac{\Abs{\Bx} }{c}\right)
\end{aligned}
\end{equation}
%
There is two tricks used here.  One was writing the unit vector \(\Be_\alpha = \spacegrad x^\alpha\).  The other was use of the continuity equation \(\PDi{t}{\rho} + \spacegrad \cdot \Bj = 0\).  This first trick was mentioned as one of the few tricks of physics that will often be repeated since there are not many good ones.

With the first term vanishing on the boundary (since \(j^i\) is localized), and pulling the time derivatives out of the integral, we can summarize the dipole potentials as
\boxedEquation{eqn:relativisticElectrodynamicsL20:250}{
\begin{aligned}
A^0(\Bx, t) &= \frac{Q\left(t - \frac{\Abs{\Bx} }{c}\right)}{\Abs{\Bx} } + \frac{\Bx \cdot \Bd\left(t - \frac{\Abs{\Bx} }{c}\right)}{\Abs{\Bx}^3} + \frac{\Bx \cdot \dot{\Bd}\left(t - \frac{\Abs{\Bx} }{c}\right)}{c \Abs{\Bx}^2} \\
\BA(\Bx, t) &= \inv{c \Abs{\Bx} } \dot{\Bd}\left(t - \frac{\Abs{\Bx} }{c}\right).
\end{aligned}
}
%
\makeexample{Electric dipole radiation}{example:relativisticElectrodynamicsL20:1}{
%
PICTURE: two closely separated oppositely charges, wiggling along the line connecting them (on the z-axis).  \(-q\) at rest, while \(+q\) oscillates.
%
\begin{equation}\label{eqn:relativisticElectrodynamicsL20:270}
z_+(t) = z_0 + a \sin\omega t.
\end{equation}
%
Since we have put the \(-q\) charge at the origin, it has no contribution to the dipole moment, and we have
%
\begin{equation}\label{eqn:relativisticElectrodynamicsL20:290}
\Bd(t) = \Be_3 q (z_0 + a \sin\omega t).
\end{equation}
%
Thus
%
\begin{equation}\label{eqn:relativisticElectrodynamicsL20:310}
\begin{aligned}
A^0(\Bx, t) &= \inv{\Abs{\Bx}^3} \Bx \cdot \Bd\left(t - \frac{\Abs{\Bx} }{c}\right) + \inv{c \Abs{\Bx}^2} \Bx \cdot \dot{\Bd}\left(t - \frac{\Abs{\Bx} }{c}\right) \\
\BA(\Bx, t) &= \frac{\dot{\Bd}\left(t - \frac{\Abs{\Bx} }{c}\right)}{c \Abs{\Bx} }
\end{aligned}
\end{equation}
%
so with \(t_r = t - \Abs{\Bx}/c\), and \(z = \Bx \cdot \Be_3\) in the dipole dot product, we have
%
\begin{equation}\label{eqn:relativisticElectrodynamicsL20:400}
\begin{aligned}
A^0(\Bx, t) &=
\frac{z q}{\Abs{\Bx}^3} ( z_0 + a \sin(\omega t_r) ) + \frac{z q}{c \Abs{\Bx}^2} a \omega \cos(\omega t_r) \\
\BA(\Bx, t) &= \inv{c \Abs{\Bx} } \Be_3 q a \omega \cos(\omega t_r)
\end{aligned}
\end{equation}
%
These hold provided \(\Abs{\Bx} \gg (z_0, a)\) and \(\omega l/c \ll 1\).  Recall that \(\omega \lambda = c/2\pi\), which has dimensions of velocity.

FIXME: think through and justify \(\omega l = v\).

Observe that \(\omega l \sim v\) so this is a requirement that our charged positive particle is moving with \(\Abs{\Bv}/c \ll 1\).

Now we will take derivatives.  The first term of the scalar potential will be ignored since the \(1/\Abs{\Bx}^2\) is non-radiative.
%
\begin{equation}\label{eqn:relativisticElectrodynamicsL20:580}
\begin{aligned}
\BE
&= -\spacegrad A^0 - \inv{c} \PD{t}{\BA} \\
&= -\frac{z a \omega q}{\Abs{\Bx}^2 c} (-\omega \sin(\omega t_r)) \left( - \inv{c} \spacegrad \Abs{\Bx} \right)
- \inv{c^2 \Abs{\Bx} } \Be_3 q a \omega^2 (-\sin(\omega t_r)).
\end{aligned}
\end{equation}
%
We have used \(\spacegrad t_r = -\spacegrad \Abs{\Bx}/c\), and \(\spacegrad \Abs{\Bx} = \rcap\), and \(\partial_t t_r = 1\).
%
\begin{equation}\label{eqn:relativisticElectrodynamicsL20:420}
\BE = \frac{ q a \omega^2 }{c^2 \Abs{\Bx} } \sin(\omega t_r) \left( \Be_3 - \frac{z}{\Abs{\Bx} } \rcap \right)
\end{equation}
%
So,
%
\begin{equation}\label{eqn:relativisticElectrodynamicsL20:440}
\Abs{\BS} \sim \omega^4
\end{equation}
%
The power is proportional to \(\omega^4\).  Higher frequency radiation has more power : this is why the sky is blue!  It all comes from the fact that the electric field is proportional to the squared acceleration (\(\sim \omega^2\)).
}
