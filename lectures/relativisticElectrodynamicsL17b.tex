%
% Copyright © 2012 Peeter Joot.  All Rights Reserved.
% Licenced as described in the file LICENSE under the root directory of this GIT repository.
%
\section{Solving Maxwell's equation.}
\index{Maxwell's equation!solving}
\index{Lienard-Wiechert potentials}

Our equations are
%
\begin{equation}\label{eqn:relativisticElectrodynamicsL17b:210}
\begin{aligned}
\epsilon^{i j k l} \partial_j F_{k l} &= 0 \\
\partial_i F^{i k} &= \frac{4 \pi}{c} j^k,
\end{aligned}
\end{equation}
%
where we assume that \(j^k(\Bx, t)\) is a given.  Our task is to find \(F^{i k}\), the \((\BE, \BB)\) fields.

Proceed by finding \(A^i\).  First, as usual when \(F_{i j} = \partial_i A_j - \partial_j A_i\).  The Bianchi identity is satisfied so we focus on the current equation.

In terms of potentials
%
\begin{equation}\label{eqn:relativisticElectrodynamicsL17b:230}
\partial_i (\partial^i A^k - \partial^k A^i) = \frac{ 4 \pi}{c} j^k.
\end{equation}
%
or
%
\begin{equation}\label{eqn:relativisticElectrodynamicsL17b:250}
\partial_i \partial^i A^k - \partial^k (\partial_i A^i) = \frac{ 4 \pi}{c} j^k.
\end{equation}
%
We want to work in the Lorentz gauge \(\partial_i A^i = 0\).  This is justified by the simplicity of the remaining problem
%
\begin{equation}\label{eqn:relativisticElectrodynamicsL17b:270}
\partial_i \partial^i A^k = \frac{4 \pi}{c} j^k.
\end{equation}
%
Write
%
\begin{equation}\label{eqn:relativisticElectrodynamicsL17b:290}
\partial_i \partial^i = \inv{c^2} \PDSq{t}{} - \Delta = \square,
\end{equation}
where
\begin{equation}\label{eqn:relativisticElectrodynamicsL17b:291}
\Delta = \PDSq{x}{} + \PDSq{y}{} + \PDSq{z}{}.
\end{equation}
This \(\square\) is the d'Alembert operator (``d'Alembertian'').
Our equation is
%
\begin{equation}\label{eqn:relativisticElectrodynamicsL17b:310}
\square A^k = \frac{4 \pi}{c} j^k,
\end{equation}
(in the Lorentz gauge)

If we learn how to solve (**), then we have learned all.

Method: Green's function's

In electrostatics where \(j^0 = 0\), \(A^0 \ne 0\) only, we have
%
\begin{equation}\label{eqn:relativisticElectrodynamicsL17b:330}
\Delta A^0 = -4 \pi \rho.
\end{equation}
Solution
%
\begin{equation}\label{eqn:relativisticElectrodynamicsL17b:350}
\Delta_{\Bx} G(\Bx - \Bx') = \delta^3( \Bx - \Bx').
\end{equation}
%
PICTURE: (a small box)
%
\begin{equation}\label{eqn:relativisticElectrodynamicsL17b:370}
\rho(\Bx') d^3 \Bx',
\end{equation}
acting through distance \(\Abs{\Bx - \Bx'}\), acting at point \(\Bx\).  With
\begin{dmath}\label{eqn:relativisticElectrodynamicsL17b:371}
G(\Bx, \Bx') = -1/4 \pi\Abs{\Bx - \Bx'},
\end{dmath}
we have
%
\begin{equation}\label{eqn:relativisticElectrodynamicsL17b:410}
\begin{aligned}
\int d^3 \Bx' \Delta_{\Bx} G(\Bx - \Bx') \rho(\Bx')
&= \int d^3 \Bx' \delta^3( \Bx - \Bx') \rho(\Bx') \\
&= \rho(\Bx).
\end{aligned}
\end{equation}
Since \(G\) is deemed a linear operator, we have \(\Delta_\Bx G = G \Delta_\Bx\), we find
%
\begin{equation}\label{eqn:relativisticElectrodynamicsL17b:430}
\begin{aligned}
\rho(\Bx)
&=
\int d^3 \Bx' \Delta_{\Bx} G(\Bx - \Bx') 4 \pi \rho(\Bx') \\
&=
\int d^3 \Bx' \inv{\Abs{\Bx - \Bx'}} \rho(\Bx').
\end{aligned}
\end{equation}
We end up finding that
%
\begin{equation}\label{eqn:relativisticElectrodynamicsL17b:390}
\phi(\Bx) = \int \frac{\rho(\Bx')}{\Abs{\Bx - \Bx'}} d^3 \Bx',
\end{equation}
thus solving the problem.  We wish next to do this for the Maxwell equation \eqnref{eqn:relativisticElectrodynamicsL17b:310}.

The Green's function method is effective, but I can not help but consider it somewhat of a cheat, since one has to through higher powers know what the Green's function is.  In the electrostatics case, at least we can work from the potential function and take its Laplacian to find that this is equivalent (thus implicitly solving for the Green's function at the same time).  It will be interesting to see how we do this for the forced d'Alembertian equation.
