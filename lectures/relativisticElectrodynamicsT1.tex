%
% Copyright � 2012 Peeter Joot.  All Rights Reserved.
% Licenced as described in the file LICENSE under the root directory of this GIT repository.
%

%\chapter{Four vectors and a worked flux density problem}
\index{flux density}
\label{chap:relativisticElectrodynamicsT1}
%\blogpage{http://sites.google.com/site/peeterjoot/math2011/relativisticElectrodynamicsT1.pdf}
%\date{Jan 20, 2011}

\makeproblem{Photon Energy flux in other frames}{pr:relativisticElectrodynamicsT1:1}{

In a source's rest frame \(S\) emits radiation isotropically with a frequency \(\omega\) with number flux \(f(\text{photons}/\text{cm}^2 s)\).  Moves along x'-axis with speed \(V\) in an observer frame (\(O\)).  What does the energy flux in \(O\) look like?
} % makeproblem

\makeanswer{pr:relativisticElectrodynamicsT1:1}{
 % makeanswer

Simon (our TA) blasted through a problem from Hartle \citep{hartle2003gravity}, section 5.17 (all the while apologizing for going so slow).  It took me a while to work through my notes to come up with something that was comprehensible to me.

At one point he asked if anybody was completely lost.  Nobody said yes, but given the class title, I had the urge to say ``No, just relatively lost''.

% moved intro to four vectors into chapter with that content.

We will work in momentum space, where we have

\begin{equation}\label{eqn:relativisticElectrodynamicsT1:130}
\begin{aligned}
p^i &= (p^0, \Bp) = \left( \frac{E}{c}, \Bp\right) \\
p^2 &= \frac{E^2}{c^2} -\Bp^2 \\
\Bp &= \Hbar \Bk \\
E &= \Hbar \omega \\
p^i &= \Hbar k^i \\
k^i &= \left(\frac{\omega}{c}, \Bk\right)
\end{aligned}
\end{equation}

We set up the \(x'\)-axis to be the direction of motion, and we call \(\alpha\) the angle from it, or the azimuthal angle.  The wavevector, \(\Bk\), is the direction the wave travels. Therefore, if we want to find the angle the radiation makes to the direction of motion, you need the projection of the wavevector onto the \(x\)-axis, or \(k^1/\Abs{\Bk}\). In other words, imagine a piece of radiation emitted in a certain direction, the angle it makes with the \(x'\)-axis is the cosine of the projection on the \(x'\)-axis over the magnitude.

This azimuthal angle in the unprimed frame is

\begin{equation}\label{eqn:relativisticElectrodynamicsT1:140}
\cos \alpha = \frac{k^1}{\Abs{\Bk}} = \frac{k^1}{\omega/c},
\end{equation}

In the observer's reference frame (the primed coordinates), the source is moving in the \(+x\) direction, and therefore, we must boost in the \(-x\) from the source's frame, or \(-\beta\).  Transforming out wave four vector in the same fashion as regular mechanical position and momentum four vectors, we have for the observer

\begin{equation}\label{eqn:relativisticElectrodynamicsT1:140b}
\cos \alpha' = \frac{{k^1}'}{\omega'/c} = \frac{\gamma (k^1 + \beta \omega/c)}{\gamma(\omega/c + \beta k^1)}
\end{equation}

%Also note that we have the primed frame moving negatively along the x-axis, instead of the usual positive origin shift.  The question is vague enough to allow this since it only requires motion.

\paragraph{check 1}

as \(\beta \rightarrow 1\) (ie: our primed frame velocity approaches the speed of light relative to the rest frame), \(\cos \alpha' \rightarrow 1\), \(\alpha' = 0\).  The surface gets more and more compressed.

In the original reference frame the radiation was isotropic.  In the new frame how does it change with respect to the angle?  This is really a question to find this number flux rate

\begin{equation}\label{eqn:relativisticElectrodynamicsT1:150}
f'(\alpha') = ?
\end{equation}

In our rest frame the total number of photons traveling through the surface in a given interval of time is

\begin{equation}\label{eqn:relativisticElectrodynamicsT1:160}
\begin{aligned}
N &= \int d\Omega dt f(\alpha) = \int d \phi \sin \alpha d\alpha = -2 \pi \int d(\cos\alpha) dt f(\alpha) \\
\end{aligned}
\end{equation}

Here we utilize the spherical solid angle \(d\Omega = \sin \alpha d\alpha d\phi = - d(\cos\alpha) d\phi\), and integrate \(\phi\) over the \([0, 2\pi]\) interval.  We also have to assume that our number flux density is not a function of horizontal angle \(\phi\) in the rest frame.

In the moving frame we similarly have
\begin{equation}\label{eqn:relativisticElectrodynamicsT1:160b}
\begin{aligned}
N' &= -2 \pi \int d(\cos\alpha') dt' f'(\alpha'),
\end{aligned}
\end{equation}

and we again have had to assume that our transformed number flux density is not a function of the horizontal angle \(\phi\).  This seems like a reasonable move since \({k^2}' = k^2\) and \({k^3}' = k^3\) as they are perpendicular to the boost direction.

\begin{equation}\label{eqn:relativisticElectrodynamicsT1:170}
f'(\alpha') = \frac{d(\cos\alpha)}{d(\cos\alpha')} \left( \frac{dt}{dt'} \right) f(\alpha)
\end{equation}


Now, utilizing a conservation of mass argument, we can argue that \(N = N'\).  Regardless of the motion of the frame, the same number of particles move through the surface.  Taking ratios, and examining an infinitesimal time interval, and the associated flux through a small patch, we have

\begin{equation}\label{eqn:relativisticElectrodynamicsT1:180}
\left( \frac{d(\cos\alpha)}{d(\cos\alpha')} \right) = \left( \frac{d(\cos\alpha')}{d(\cos\alpha)} \right)^{-1} = \gamma^2 ( 1 + \beta \cos\alpha)^2
\end{equation}

Part of the statement above was a do-it-yourself.  First recall that \(c t' = \gamma ( c t + \beta x )\), so \(dt/dt'\) evaluated at \(x=0\) is \(1/\gamma\).

The rest is messier.  We can calculate the \(d(\cos)\) values in the ratio above using \eqnref{eqn:relativisticElectrodynamicsT1:140}.  For example, for \(d(\cos(\alpha))\) we have

\begin{equation}\label{eqn:relativisticElectrodynamicsT1:270}
\begin{aligned}
d(\cos\alpha)
&= d \left( \frac{k^1}{\omega/c} \right) \\
&= dk^1 \inv{\omega/c} - c \inv{\omega^2} d\omega.
\end{aligned}
\end{equation}

If one does the same thing for \(d(\cos\alpha')\), after a whole whack of messy algebra one finds that the differential terms and a whole lot more mystically cancels, leaving just

\begin{equation}\label{eqn:relativisticElectrodynamicsT1:171}
\frac{d\cos\alpha'}{d\cos\alpha} = \frac{\omega^2/c^2}{(\omega/c + \beta k^1)^2} (1 - \beta^2)
\end{equation}

A bit more reduction with reference back to \eqnref{eqn:relativisticElectrodynamicsT1:140b} verifies \eqnref{eqn:relativisticElectrodynamicsT1:180}.

Also note that again from \eqnref{eqn:relativisticElectrodynamicsT1:140b} we have

\begin{equation}\label{eqn:relativisticElectrodynamicsT1:190a}
\cos\alpha' = \frac{\cos\alpha + \beta}{1 + \beta \cos\alpha}
\end{equation}

and rearranging this for \(\cos\alpha'\) gives us
\begin{equation}\label{eqn:relativisticElectrodynamicsT1:190}
\cos\alpha = \frac{\cos\alpha' - \beta}{1 - \beta \cos\alpha'},
\end{equation}

which we can sum to find that

\begin{equation}\label{eqn:relativisticElectrodynamicsT1:190b}
1 + \beta \cos\alpha = \inv{\gamma^2 (1 - \beta \cos \alpha') },
\end{equation}

so putting all the pieces together we have

\begin{equation}\label{eqn:relativisticElectrodynamicsT1:200}
f'(\alpha') = \inv{\gamma^3} \frac{f(\alpha)}{(1-\beta \cos\alpha')^2}
\end{equation}

The question asks for the energy flux density.  We get this by multiplying the number density by the frequency of the light in question.  This is, as a function of the polar angle, in each of the frames.

\begin{equation}\label{eqn:relativisticElectrodynamicsT1:210}
\begin{aligned}
L(\alpha) &= \Hbar \omega(\alpha) f(\alpha) = \Hbar \omega f \\
L'(\alpha') &= \Hbar \omega'(\alpha') f'(\alpha') = \Hbar \omega' f'
\end{aligned}
\end{equation}

But we have
\begin{equation}\label{eqn:relativisticElectrodynamicsT1:220}
\omega'(\alpha')/c = \gamma( \omega/c + \beta k^1 ) = \gamma \omega/c ( 1 + \beta \cos\alpha )
\end{equation}

Aside, \(\beta << 1\),

\begin{equation}\label{eqn:relativisticElectrodynamicsT1:230}
\omega' = \omega ( 1 + \beta \cos\alpha) + O(\beta^2) = \omega + \delta \omega
\end{equation}

\begin{equation}\label{eqn:relativisticElectrodynamicsT1:240}
\begin{aligned}
\delta \omega &= \beta, \alpha = 0 		\qquad \text{blue shift} \\
\delta \omega &= -\beta, \alpha = \pi 		\qquad \text{red shift}
\end{aligned}
\end{equation}

The energy flux density in the unprimed observer frame is now found to be

\begin{equation}\label{eqn:relativisticElectrodynamicsT1:241}
L'(\alpha') = \frac{L/\gamma}{(\gamma (1 - \beta \cos\alpha'))^3}
\end{equation}

And the forward backward ratio is

\begin{equation}\label{eqn:relativisticElectrodynamicsT1:250}
L'(0)/L'(\pi) = {\left( \frac{ 1 + \beta }{1-\beta} \right)}^3,
\end{equation}

allowing us to conclude that the forward radiation is bigger than the backwards radiation (and much bigger when the motion approaches the speed of light).
}
