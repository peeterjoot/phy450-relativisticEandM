%
% Copyright � 2012 Peeter Joot.  All Rights Reserved.
% Licenced as described in the file LICENSE under the root directory of this GIT repository.
%

%\chapter{Four vectors and tensors}
\index{four vector}
\index{tensor}
\label{chap:relativisticElectrodynamicsL6}
%\blogpage{http://sites.google.com/site/peeterjoot/math2011/relativisticElectrodynamicsL6.pdf}
%\date{Jan 25, 2011}
%
\paragraph{Reading}
%
Covering chapter 1 material from the text \citep{landau1980classical}, and
\popcite{RelEM27-44.pdf}{lecture notes RelEM27-44.pdf}.
%: nonrelativistic limit of boosts (33); number of parameters of Lorentz transformations (34-35); introducing four-vectors, the metric tensor, the invariant ``dot-product and SO(1,3) (36-40); the Poincare group (41); the convenience of ``upper'' and ``lower''indices (42-43); tensors (44)
%
\section{The Special Orthogonal group (for Euclidean space)}
\index{special orthogonal group!Euclidean}

Lorentz transformations are like ``rotations'' for \((t, x, y, z)\) that preserve \((ct)^2 - x^2 - y^2 - z^2\).  There are 6 continuous parameters:

\begin{itemize}
\item 3 rotations in \(x,y,z\) space
\item 3 ``boosts'' in \(x\) or \(y\) or \(z\).
\end{itemize}

For rotations of space we talk about a group of transformations of 3D Euclidean space, and call this the \(S0(3)\) group.  Here \(S\) is for Special, \(O\) for Orthogonal, and \(3\) for the dimensions.

For a transformed vector in 3D space we write
%
\begin{equation}\label{eqn:relativisticElectrodynamicsL6:10}
\begin{bmatrix}
x \\
y \\
z
\end{bmatrix}
\rightarrow
\begin{bmatrix}
x \\
y \\
z
\end{bmatrix}' = O
\begin{bmatrix}
x \\
y \\
z
\end{bmatrix}.
\end{equation}

Here \(O\) is an orthogonal \(3 \times 3\) matrix, and has the property
%
\begin{equation}\label{eqn:relativisticElectrodynamicsL6:11}
O^T O = \BOne.
\end{equation}

Taking determinants, we have
%
\begin{equation}\label{eqn:relativisticElectrodynamicsL6:12}
\det{ O^T } \det{ O} = 1,
\end{equation}

and since \(\det{O^\T} = \det{ O }\), we have
%
\begin{equation}\label{eqn:relativisticElectrodynamicsL6:13}
(\det{O})^2 = 1,
\end{equation}
so our determinant must be
\begin{equation}\label{eqn:relativisticElectrodynamicsL6:14}
\det O = \pm 1.
\end{equation}

We work with the positive case only, avoiding the transformations that include reflections.

The Unitary condition \(O^\T O = 1\) is an indication that the inner product is preserved.  Observe that in matrix form we can write the inner product
%
\begin{equation}\label{eqn:relativisticElectrodynamicsL6:15}
\Br_1 \cdot \Br_2 =
\begin{bmatrix}
x_1 & y_1 & z_1
\end{bmatrix}
\begin{bmatrix}
x_1 \\
y_2 \\
x_3 \\
\end{bmatrix}.
\end{equation}

For a transformed vector \(X' = O X\), we have \({X'}^\T = X^\T O^\T\), and
%
\begin{equation}\label{eqn:relativisticElectrodynamicsL6:16}
X' \cdot X' = (X^\T O^\T) (O X) = X^\T (O^\T O) X = X^T X = X \cdot X
\end{equation}
%
\section{The Special Orthogonal group (for spacetime)}
\index{special orthogonal group!spacetime}

This generalizes to Lorentz boosts!  There are two differences
\index{Lorentz boost}

\begin{enumerate}
\item Lorentz transforms should be \(4 \times 4\) not \(3 \times 3\) and act in \((ct, x, y, z)\), and NOT \((x,y,z)\).
\item They should leave invariant NOT \(\Br_1 \cdot \Br_2\), but \(c2 t_2 t_1 - \Br_2 \cdot \Br_1\).
\end{enumerate}

Do not get confused that I demanded \(c^2 t_2 t_1 - \Br_2 \cdot \Br_1 = \text{invariant}\) rather than \(c^2 (t_2 - t_1)^2 - (\Br_2 - \Br_1)^2 = \text{invariant}\).  Expansion of this (squared) interval, provides just this four vector dot product and its invariance condition
%
\begin{equation}\label{eqn:relativisticElectrodynamicsL6:260}
\begin{aligned}
\text{invariant}
&=
c^2 (t_2 - t_1)^2 - (\Br_2 - \Br_1)^2 \\
&=
(c^2 t_2^2 - \Br_2^2) + (c^2 t_2^2 - \Br_2^2)
- 2 c^2 t_2 t_1 + 2 \Br_1 \cdot \Br_2.
\end{aligned}
\end{equation}

Observe that we have the sum of two invariants plus our new cross term, so this cross term, (-2 times our dot product to be defined), must also be an invariant.
%
\paragraph{Introduce the four vector}
\index{four vector}
%
\begin{equation}\label{eqn:relativisticElectrodynamicsL6:280}
\begin{aligned}
x^0 &= ct \\
x^1 &= x \\
x^2 &= y \\
x^3 &= z
\end{aligned}
\end{equation}

Or \((x^0, x^1, x^2, x^3) = \{ x^i, i = 0,1,2,3 \}\).

We will also write
%
\begin{equation}\label{eqn:relativisticElectrodynamicsL6:300}
\begin{aligned}
x^i &= (ct, \Br) \\
\tilde{x}^i &= (c\tilde{t}, \tilde{\Br})
\end{aligned}
\end{equation}

Our inner product is
%
\begin{equation}\label{eqn:relativisticElectrodynamicsL6:20}
c^2 t \tilde{t} - \Br \cdot \tilde{\Br}
\end{equation}

Introduce the \(4 \times 4\) matrix
% used double bar abs (norm) here
\begin{equation}\label{eqn:relativisticElectrodynamicsL6:30}
\Norm{g_{ij}} =
\begin{bmatrix}
1 & 0 & 0 & 0 \\
0 & -1 & 0 & 0 \\
0 & 0 & -1 & 0 \\
0 & 0 & 0 & -1 \\
\end{bmatrix}
\end{equation}

This is called the Minkowski spacetime metric.

Then
%
\begin{equation}\label{eqn:relativisticElectrodynamicsL6:320}
\begin{aligned}
c^2 t \tilde{t} - \Br \cdot \tilde{\Br}
&\equiv \sum_{i, j = 0}^3 \tilde{x}^i g_{ij} x^j \\
&= \sum_{i, j = 0}^3 \tilde{x}^i g_{ij} x^j \\
&
\tilde{x}^0 x^0
-\tilde{x}^1 x^1
-\tilde{x}^2 x^2
-\tilde{x}^3 x^3
\end{aligned}
\end{equation}
%
\paragraph{Einstein summation convention}.  Whenever indices are repeated that are assumed to be summed over.
\index{Einstein summation convention}

We also write
%
\begin{equation}\label{eqn:relativisticElectrodynamicsL6:40}
X =
\begin{bmatrix}
x^0 \\
x^1 \\
x^2 \\
x^3 \\
\end{bmatrix}
\end{equation}
\begin{equation}\label{eqn:relativisticElectrodynamicsL6:50}
\tilde{X} =
\begin{bmatrix}
\tilde{x}^0 \\
\tilde{x}^1 \\
\tilde{x}^2 \\
\tilde{x}^3 \\
\end{bmatrix}
\end{equation}
%
\begin{equation}\label{eqn:relativisticElectrodynamicsL6:60}
G =
\begin{bmatrix}
1 & 0 & 0 & 0 \\
0 & -1 & 0 & 0 \\
0 & 0 & -1 & 0 \\
0 & 0 & 0 & -1 \\
\end{bmatrix}
\end{equation}

Our inner product
%
\begin{equation}\label{eqn:relativisticElectrodynamicsL6:340}
\begin{aligned}
c^2 t \tilde{t} - \tilde{\Br} \cdot \Br
&= \tilde{X}^\T G X \\
&=
\begin{bmatrix}
\tilde{x}^0 & \tilde{x}^1 & \tilde{x}^2 & \tilde{x}^3
\end{bmatrix}
\begin{bmatrix}
1 & 0 & 0 & 0 \\
0 & -1 & 0 & 0 \\
0 & 0 & -1 & 0 \\
0 & 0 & 0 & -1 \\
\end{bmatrix}
\begin{bmatrix}
\tilde{x}^0 \\
\tilde{x}^1 \\
\tilde{x}^2 \\
\tilde{x}^3 \\
\end{bmatrix}
\end{aligned}
\end{equation}

Under Lorentz boosts, we have
%
\begin{equation}\label{eqn:relativisticElectrodynamicsL6:70}
X = \hat{O} X',
\end{equation}

where
%
\begin{equation}\label{eqn:relativisticElectrodynamicsL6:80}
\hat{O} =
\begin{bmatrix}
\gamma & - \gamma v_x/c  & 0 & 0 \\
- \gamma v_x/c & \gamma  & 0 & 0 \\
0 & 0 & 1 & 0 \\
0 & 0 & 0 & 1
\end{bmatrix}
\end{equation}

(for x-direction boosts)
%
\begin{equation}\label{eqn:relativisticElectrodynamicsL6:90}
\begin{aligned}
\tilde{X} &= \hat{O} {\tilde{X}}' \\
{\tilde{X}}^\T &= {{\tilde{X'}}}^\T {\hat{O}}^\T
\end{aligned}
\end{equation}

But \(\hat{O}\) must be such that \(\tilde{X}^\T G X\) is invariant.  i.e.
%
\begin{equation}\label{eqn:relativisticElectrodynamicsL6:100}
\tilde{X}^\T G X = {\tilde{X'}}^\T (\hat{O}^\T G \hat{O}) X' = {X'}^\T (G) X' \qquad \mbox{\(\forall X'\) and \(\tilde{X}'\)}
\end{equation}

This implies
\boxedEquation{eqn:relativisticElectrodynamicsL6:110}{
\hat{O}^\T G \hat{O} = G.
}
Such \(\hat{O}\)'s are called ``pseudo-orthogonal''.

Lorentz transformations are represented by the set of all \(4 \times 4\) pseudo-orthogonal matrices.
In symbols
%
\begin{equation}\label{eqn:relativisticElectrodynamicsL6:120}
\hat{O}^T G \hat{O} = G.
\end{equation}
Just as before we can take the determinant of both sides.  Doing so we have
%
\begin{equation}\label{eqn:relativisticElectrodynamicsL6:130}
\det(\hat{O}^T G \hat{O}) = \det(\hat{O}^T) \det(G) \det(\hat{O}) = \det(G)
\end{equation}
The \(\det(G)\) terms cancel, and since \(\det(\hat{O}^T) = \det(\hat{O})\), this leaves us with \((\det(\hat{O}))^2 = 1\), or
%
\begin{equation}\label{eqn:relativisticElectrodynamicsL6:140}
\det(\hat{O}) = \pm 1.
\end{equation}
We take the \(\det 0 = +1\) case only, so that the transformations do not change orientation (no reflection in space or time).  This set of transformation forms the group
%
\begin{equation*}
SO(1,3)
\end{equation*}

Special orthogonal, one time, 3 space dimensions.  Note that when the \(-1\) determinant is also allowed the group is called the \(O(1,3)\) set of transformations.

Einstein relativity can be defined as the ``laws of physics that leave four vectors invariant in the
%
\begin{equation*}
SO(1,3) \times T^4
\end{equation*}

symmetry group.

Here \(T^4\) is the group of translations in spacetime with 4 continuous parameters.   The complete group of transformations that form the group of relativistic physics has \(10 = 3 + 3 + 4\) continuous parameters.

This group is called the Poincare group of symmetry transforms.
%
\section{Lower index notation}
\index{lower indexes}

Our inner product is written
%
\begin{equation}\label{eqn:relativisticElectrodynamicsL6:150}
\tilde{x}^i g_{ij} x^j
\end{equation}

but this is very cumbersome.  The convenient way to write this is instead
%
\begin{equation}\label{eqn:relativisticElectrodynamicsL6:160}
\tilde{x}^i g_{ij} x^j = \tilde{x}_j x^j = \tilde{x}^i x_i
\end{equation}

where
%
\begin{equation}\label{eqn:relativisticElectrodynamicsL6:170}
x_i = g_{ij} x^j = g_{ji} x^j
\end{equation}

Note: A check that we should always be able to make.  Indexes that are not summed over should be conserved.  So in the above we have a free \(i\) on the LHS, and should have a non-summed \(i\) index on the RHS too (also lower matching lower, or upper matching upper).

Non-matched indices are bad in the same sort of sense that an expression like
%
\begin{equation}\label{eqn:relativisticElectrodynamicsL6:180}
\Br = 1
\end{equation}

is not well defined (assuming a vector space \(\Br\) and not a multivector Clifford algebra that is;)

Expanded out explicitly (noting that all off diagonal terms of the metric tensor are zero):
%
\begin{equation}\label{eqn:relativisticElectrodynamicsL6:360}
\begin{aligned}
x_0 &= g_{0 0} x^0 = ct  \\
x_1 &= g_{1 j} x^j = g_{11} x^1 = -x^1 \\
x_2 &= g_{2 j} x^j = g_{22} x^2 = -x^2 \\
x_3 &= g_{3 j} x^j = g_{33} x^3 = -x^3
\end{aligned}
\end{equation}

We would not have objects of the form
%
\begin{equation}\label{eqn:relativisticElectrodynamicsL6:190}
x^i x^i = (ct)^2 + \Br^2
\end{equation}

for example.  This is not a Lorentz invariant quantity.
%
\paragraph{Lorentz scalar example:} \(\tilde{x}^i x_i\)
\paragraph{Lorentz vector example:} \(x^i\)
%
This last is also called a rank-1 tensor.

Lorentz rank-2 tensors: ex: \(g_{ij}\)

or other 2-index objects.

Why in the world would we ever want to consider two index objects.  We are not just trying to be hard on ourselves.  Recall from classical mechanics that we have a two index object, the inertial tensor.

In mechanics, for a rigid body we had the energy
%
\begin{equation}\label{eqn:relativisticElectrodynamicsL6:200}
T = \sum_{ij = 1}^3 \Omega_i I_{ij} \Omega_j
\end{equation}

The inertial tensor was this object
%
\begin{equation}\label{eqn:relativisticElectrodynamicsL6:210}
I_{ij} = \sum_{a = 1}^N m_a \left(\delta_{ij} \Br_a^2 - r_{a_i} r_{a_j} \right)
\end{equation}

or for a continuous body
%
\begin{equation}\label{eqn:relativisticElectrodynamicsL6:220}
I_{ij} = \int \rho(\Br) \left(\delta_{ij} \Br^2 - r_{i} r_{j} \right)
\end{equation}

In electrostatics we have the quadrupole tensor, ... and we have other such objects all over physics.

Note that the energy \(T\) of the body above cannot depend on the coordinate system in use.  This is a general property of tensors.  These are object that transform as products of vectors, as \(I_{ij}\) does.

We call \(I_{ij}\) a rank-2 3-tensor.  rank-2 because there are two indices, and 3 because the indices range from \(1\) to \(3\).

The point is that tensors have the property that the transformed tensors transform as
%
\begin{equation}\label{eqn:relativisticElectrodynamicsL6:230}
I_{ij}' = \sum_{l, m = 1,2,3} O_{il} O_{jm} I_{lm}
\end{equation}

%FIXME: show this based on the definition above of \(I_{ij}\).
Another example: the completely antisymmetric rank 3, 3-tensor
%
\begin{equation}\label{eqn:relativisticElectrodynamicsL6:240}
\epsilon_{ijk}
\end{equation}

