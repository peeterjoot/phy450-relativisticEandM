%
% Copyright � 2012 Peeter Joot.  All Rights Reserved.
% Licenced as described in the file LICENSE under the root directory of this GIT repository.
%

%\chapter{Action and relativistic dynamics}
\index{action}
\label{chap:relativisticElectrodynamicsL7}
%\blogpage{http://sites.google.com/site/peeterjoot/math2011/relativisticElectrodynamicsL7.pdf}
%\date{Jan 26, 2011}
%
\paragraph{Reading}
%
Covering chapter 2 material from the text \citep{landau1980classical}, and
\popcite{RelEMpp52-56.pdf}{lecture notes RelEMpp52-56.pdf}.
%,
%%: equation of motion, symmetries, and conserved quantities (energy-momentum 4 vector) from relativistic particle action [Wednesday, Jan. 26, Tuesday, Feb. 1]
%and \popcite{RelEMp53.1.pdf}{RelEMp53.1.pdf}.
%, containing some additional notes completing an argument on page 53.
%
\section{The relativity principle.}
\index{relativity principle}

The relativity principle implies that the EOM should be expressed in 4-vector form, just like Newton's EOM are expressed in 3-vector form
%
\begin{equation}\label{eqn:relativisticElectrodynamicsL7:10}
m \ddot{\Br} = \Bf.
\end{equation}
%
Observe that in coordinate form this is
\begin{equation}\label{eqn:relativisticElectrodynamicsL7:20}
m \ddot{r}^i = f^i, \qquad i = 1,2,3.
\end{equation}
or for a rotated frame \(O'\)
\begin{equation}\label{eqn:relativisticElectrodynamicsL7:30}
m \ddot{r'}^i = {f'}^i, \qquad i = 1,2,3.
\end{equation}
We need 4-velocity and 4-acceleration generalizations of this.

Later we will study action and Lagrangian, and then relativity will require that the action be a Lorentz scalar.  The analogy for a Newtonian point particle is a scalar under rotations.
%
\paragraph{Four vector velocity}
\index{velocity!four vector}
%
\paragraph{Definition:} Velocity s the rate of change of position in \((ct, \Bx)\)-space.  Position means specifying both \(ct\) and \(\Bx\) for a point in spacetime.
%
PICTURE: \(x^0 = ct\) axis up, and \(x^1, x^2, x^3\) axis over, with worldline \(x = x(\tau)\).  Here \(\tau\) is a parameter for the worldline, and provides a mapping for the curve in spacetime.

PICTURE: 3D vectors, \(\Br(t)\), \(\Br(t + \Delta t)\), and the difference vector \(\Br(t + \Delta t) - \Br(t)\).

We write
%
\begin{equation}\label{eqn:relativisticElectrodynamicsL7:40}
\Bv(t) \equiv \lim_{\Delta t \rightarrow 0} \frac{\Br(t + \Delta t) - \Br(t)}{ \Delta t}.
\end{equation}
%
For four vectors we will parametrize the worldline by its ``length'', with \(O\) taken from some arbitrary point on it.  We can also take \(\tau\) to be the proper time, and the only difference will be the factor of \(c\) (which becomes especially easy with the choice \(c=1\) that is avoided in this class).
%
\begin{equation}\label{eqn:relativisticElectrodynamicsL7:50}
\frac{x^i(\tau + \Delta \tau) - x^i(\tau)}{\Delta \tau}.
\end{equation}
%
We will take the limit
%
\begin{equation}\label{eqn:relativisticElectrodynamicsL7:60}
\frac{dx^i}{d\tau} =
\lim_{\Delta \tau \rightarrow 0}
\frac{x^i(\tau + \Delta \tau) - x^i(\tau)}{\Delta \tau}.
\end{equation}
%
and then define a dimensionless ``proper velocity''
%
\begin{equation}\label{eqn:relativisticElectrodynamicsL7:61}
u^i \equiv \inv{c} \frac{dx^i}{d\tau} = \frac{dx^i}{ds}.
\end{equation}
%
This is a nice quantity, we are dividing a vector by a Lorentz scalar, and thus get a four vector as a result (i.e. the result transforms as a four vector).

PICTURE: small fragment of a worldline with constant slope over the infinitesimal interval.  \(dx^0\) up and \(dx^1\) over.
%
%\begin{equation}\label{eqn:relativisticElectrodynamicsL7:70}
%u^i \equiv \frac{dx^i}{ds}.
%\end{equation}
%
\begin{equation}\label{eqn:relativisticElectrodynamicsL7:290}
\begin{aligned}
ds^2
&= (dx^0)^2 - (dx^1)^2 \\
&= c^2 \left( (dt)^2 - \inv{c^2} (dx^1)^2 \right) \\
&= c^2 (dt)^2 \left( 1 - \inv{c^2} \frac{dx^1}{dt^2} \right).
\end{aligned}
\end{equation}
%
Or
%
\begin{equation}\label{eqn:relativisticElectrodynamicsL7:90}
\begin{aligned}
ds = c dt \sqrt{1 - \inv{c^2} \frac{dx^1}{dt^2} }.
\end{aligned}
\end{equation}
%
NOTE: Prof admits pulling a fast one, since he has aligned the worldline along the \(x^1\) axis, however this is always possible by rotating the coordinate system.
%
\begin{equation}\label{eqn:relativisticElectrodynamicsL7:310}
\begin{aligned}
u^0
&= \frac{dx^0}{ds} \\
&= \frac{c dt}{ c dt \sqrt{ 1 - \Bv^2/c^2} } \\
&= \frac{1}{ \sqrt{ 1 - \Bv^2/c^2} } \\
&= \gamma.
\end{aligned}
\end{equation}
%
\begin{equation}\label{eqn:relativisticElectrodynamicsL7:330}
\begin{aligned}
u^1
&= \frac{dx^1}{ds} \\
&= \frac{dx^1 }{ c dt \sqrt{ 1 - \Bv^2/c^2} } \\
&= \frac{v^1/c}{ \sqrt{ 1 - \Bv^2/c^2} } \\
&= \gamma \frac{v^1}{c}.
\end{aligned}
\end{equation}
%
Similarly
\begin{equation}\label{eqn:relativisticElectrodynamicsL7:350}
\begin{aligned}
u^2 &= \gamma \frac{v^2}{c} \\
u^3 &= \gamma \frac{v^2}{c}.
\end{aligned}
\end{equation}
%
We have now unpacked the four velocity, and have
%
\begin{equation}\label{eqn:relativisticElectrodynamicsL7:100}
u^i = \left( \gamma, \frac{\Bv}{c} \gamma \right).
\end{equation}
\paragraph{Length of the four velocity vector}
Recall that this length is
%
\begin{equation}\label{eqn:relativisticElectrodynamicsL7:370}
\begin{aligned}
u^i g_{ij} u^j
&= u^i u_i  \\
&= u_i u^i  \\
&= (u^0)^2 - (u_i)^2 \\
&= \gamma^2 - \gamma^2 \frac{\Bv}{c} \cdot \frac{\Bv}{c} \\
&= \gamma^2 \left(1 - \frac{\Bv^2}{c^2} \right).
\end{aligned}
\end{equation}
%
The four velocity in physics is
\begin{equation}\label{eqn:relativisticElectrodynamicsL7:110}
u^i = \left( \gamma, \frac{\Bv}{c} \gamma \right).
\end{equation}
%
but in mathematics the meaning of \(u^i u_i = 1\) means that this quantity is the unit tangent vector to the worldline.
%
\paragraph{Four acceleration}
\index{four acceleration}

In Newtonian physics we have
%
\begin{equation}\label{eqn:relativisticElectrodynamicsL7:111}
\Ba = \frac{d\Bv}{dt}.
\end{equation}
%
Our relativistic mapping of this, with \(v \rightarrow u^i\) and \(t \rightarrow s\), gives
%
\begin{equation}\label{eqn:relativisticElectrodynamicsL7:120}
w^i = \frac{d u^i}{ds}.
\end{equation}
%
Geometrically \(w^i\) is the normal to the worldline.  This follows from \(u^i g_{ij} u^j = 1\), so
%
\begin{equation}\label{eqn:relativisticElectrodynamicsL7:390}
\begin{aligned}
\frac{d}{ds} \left( u^i g_{ij} u^j \right)
&=
\frac{d u^i}{ds} g_{ij} u^j
+u^i g_{ij} \frac{d u^j}{ds} \\
&=
\frac{d u^i}{ds} g_{ij} u^j
+u^j
\mathLabelBox
[
   labelstyle={xshift=2cm},
   linestyle={out=270,in=90, latex-}
]
{g_{ji}}{\(= g_{ij}\)} \frac{d u^i}{ds} \\
&=
\frac{d u^i}{ds} g_{ij} u^j
+u^j g_{ji} \frac{d u^i}{ds} \\
&=
2 \frac{d u^i}{ds} g_{ij} u^j.
\end{aligned}
\end{equation}
%
Note that we have utilized the fact above that the dummy summation indices can be swapped (or changed to anything else we feel inclined to use).

The conclusion is that the dot product of the acceleration and the velocity is zero
%
\begin{equation}\label{eqn:relativisticElectrodynamicsL7:130}
w_i u^i = 0.
\end{equation}
%
\section{Relativistic action.}
\index{action!relativistic}
%
\begin{equation}\label{eqn:relativisticElectrodynamicsL7:140}
S_{ab} = ?.
\end{equation}
%
What is the action for a worldline from \(a \rightarrow b\).
\index{worldline}

We want something that has velocity dependence (\(u^i\) not \(\Bv\)), but that is Lorentz invariant and has only first derivatives.

The relativistic length is the simplest so we could form
%
\begin{equation}\label{eqn:relativisticElectrodynamicsL7:150}
\int ds u^i u_i.
\end{equation}
%
but that is not interesting since \(u^i u_i = 1\).  We could form
%
\begin{equation}\label{eqn:relativisticElectrodynamicsL7:160}
\int ds u^i \frac{u_i}{ds} = \int ds w^i u_i.
\end{equation}
%
but then this is just zero.

We could form something like
%
\begin{equation}\label{eqn:relativisticElectrodynamicsL7:170}
\int ds \frac{w^i}{ds} u_i.
\end{equation}
This is non zero and non-constant, but evaluating the EOM for such an action would produce a result that has higher than second order derivatives.
We are left with
\begin{equation}\label{eqn:relativisticElectrodynamicsL7:180}
S_{ab} = \text{constant} \int_a^b ds.
\end{equation}
%
To fix this constant we note that if we want to minimize the action over the infinitesimal interval, then we need a minus sign.  Since the Lagrangian has dimensions of energy, and the dimensions of energy times time are momentum, our action must then have dimensions of momentum.  So one possible constant that fixes up our dimensions is \(mc\).  Construct an action with the following form
%
\begin{equation}\label{eqn:relativisticElectrodynamicsL7:190}
S_{ab} = - m c\int_a^b ds,
\end{equation}
does the job we want.  Here ``m'' is a characteristic of the particle, which \textunderline{is a Lorentz scalar}.  It also happens to have dimensions of mass.  With \(ds = c dt \sqrt{1 - \Bv^2/c^2}\), we have
%
\begin{equation}\label{eqn:relativisticElectrodynamicsL7:200}
S_{ab} = - m c^2 \int_{t_a}^{t_b} dt \sqrt{ 1 - \inv{c^2} \left( \frac{d \Bx(t) }{dt} \right)^2 }.
\end{equation}
%
Now everything looks like it was in classical mechanics.
%
\begin{equation}\label{eqn:relativisticElectrodynamicsL7:210}
S_{ab} = \int_{t_a}^{t_b} \LL(\dot{\Bx}(t)) dt.
\end{equation}
\begin{equation}\label{eqn:relativisticElectrodynamicsL7:220}
\LL(\dot{\Bx}(t)) = -m c^2.
\end{equation}
%
Now find the extremum of \(S\).  That problem is really to compute the variation in the action that results from varying the coordinates around the stationary point, and equate that variation to zero to find the extremum
%
\begin{equation}\label{eqn:relativisticElectrodynamicsL7:230}
\delta S = S[\Bx(t) + \delta \Bx(t)] - S[ \Bx(t) ] = 0.
\end{equation}
%
The usual condition is imposed where we have zero variation of the coordinates at the boundaries of the action integral
%
\begin{equation}\label{eqn:relativisticElectrodynamicsL7:231}
0 = \delta \Bx(t_a) = \delta \Bx(t_b).
\end{equation}
%
Returning to our action we have
%
\begin{equation}\label{eqn:relativisticElectrodynamicsL7:240}
\frac{d}{dt} \PD{\dot{\Bx}}{\LL} = \PD{\Bx}{\LL} = 0.
\end{equation}
%
This last is zero because it is a free particle with no position dependence.
%
\begin{equation}\label{eqn:relativisticElectrodynamicsL7:410}
\begin{aligned}
0
&= -m c^2 \frac{d}{dt} \PD{\dot{\Bx}}{} \sqrt{ 1 - \dot{\Bx}^2 } \\
&= -m c^2 \frac{d}{dt} \frac{- \dot{\Bx}}{\sqrt{ 1 - \dot{\Bx}^2 } } \\
&= m c^2 \frac{d}{dt} \gamma \dot\Bx.
\end{aligned}
\end{equation}
%
So we have
%
\begin{equation}\label{eqn:relativisticElectrodynamicsL7:241}
\frac{d}{dt} (\gamma \dot{\Bx}) = 0.
\end{equation}
%
By evaluating this, we can eventually show that we can construct a four vector equation.  Doing this we have
%
\begin{equation}\label{eqn:relativisticElectrodynamicsL7:430}
\begin{aligned}
\frac{d}{dt} (\gamma \Bv)
&=
\frac{d}{dt} \left( \left(1 - \Bv^2/c^2\right)^{-1/2} \Bv \right) \\
&=
-2 (-1/2) \Bv (\Bv \cdot \dot{\Bv})/c^2 \left(1 - \Bv^2/c^2\right)^{-3/2} + \left(1 - \Bv^2/c^2\right)^{-1/2} \dot{\Bv} \\
&=
\gamma \left( \frac{\Bv (\Bv \cdot \dot{\Bv}) }{ c^2 - \Bv^2 } + \dot{\Bv} \right),
\end{aligned}
\end{equation}
or
\begin{equation}\label{eqn:relativisticElectrodynamicsL7:242}
\frac{\Bv (\Bv \cdot \dot{\Bv}) }{ c^2 - \Bv^2 } + \dot{\Bv} = 0.
\end{equation}
Clearly \(\dot{\Bv} = 0\) is a solution, but is it the only solution?
Dotting this with \(\Bv\) we have
\begin{equation}\label{eqn:relativisticElectrodynamicsL7:450}
\begin{aligned}
0
&= \frac{\Bv^2 (\Bv \cdot \dot{\Bv}) }{ c^2 - \Bv^2 } + \dot{\Bv} \cdot \Bv  \\
&= (\Bv \cdot \dot{\Bv}) \left( 1 + \frac{\Bv^2}{c^2 - \Bv^2} \right) \\
&= (\Bv \cdot \dot{\Bv}) \frac{c^2}{c^2 - \Bv^2}.
\end{aligned}
\end{equation}
%
This implies that \(\dot{\Bv} = 0\) (a contraction) or that \(\Bv \cdot \dot{\Bv} = 0\).  To examine the perpendicularity question, let us take cross products.  This gives
%
\begin{equation}\label{eqn:relativisticElectrodynamicsL7:243}
0 =
\frac{(\Bv \cross \Bv) (\Bv \cdot \dot{\Bv}) }{ c^2 - \Bv^2 } + \dot{\Bv} \cross \Bv.
\end{equation}
%
We have found that \(\Bv \cdot \dot{\Bv} = 0\) and \(\Bv \cross \dot{\Bv} = 0\).  This can only mean that \(\dot{\Bv} = 0\), contradicting the assumption that is non-zero.  We conclude that \(\dot{\Bv} = 0\) is the only solution to \eqnref{eqn:relativisticElectrodynamicsL7:242}.
%
\section{Next time.}
%
We want to finish up and show how this results in a four velocity equation.  We have
%
\begin{equation}\label{eqn:relativisticElectrodynamicsL7:250}
\frac{d}{dt} ( \gamma \Bv) = 0.
\end{equation}
%
which is
%
\begin{equation}\label{eqn:relativisticElectrodynamicsL7:260}
\frac{d}{dt} ( u^\alpha ) = 0, \qquad \mbox{for \(u^\alpha = u^1, u^2, u^3\)}.
\end{equation}
%
eventually, we will show that we also have
%
\begin{equation}\label{eqn:relativisticElectrodynamicsL7:270}
\frac{d}{dt} ( u^i ) = 0.
\end{equation}
