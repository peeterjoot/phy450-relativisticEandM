%
% Copyright � 2012 Peeter Joot.  All Rights Reserved.
% Licenced as described in the file LICENSE under the root directory of this GIT repository.
%

%\chapter{Lorentz force equation energy term, and four vector formulation of the Lorentz force equation}
\index{Lorentz force}
\index{Lorentz force!four vector form}
\label{chap:relativisticElectrodynamicsL10}
%\blogpage{http://sites.google.com/site/peeterjoot/math2011/relativisticElectrodynamicsL10.pdf}
%\date{Feb 8, 2011}

\paragraph{Reading}

Covering chapter 3 material from the text \citep{landau1980classical}, and
\popcite{RelEMpp74-83.pdf}{lecture notes RelEMpp74-83.pdf}.
%: gauge transformations in 3-vector language (74); energy of a relativistic particle in EM field (75); variational principle and equation of motion in 4-vector form (76-77); the field strength tensor (78-80); the fourth equation of motion (81)

\section{What is the significance to the gauge invariance of the action?}
\index{gauge invariance}
\index{action}

We had argued that under a gauge transformation

\begin{equation}\label{eqn:relativisticElectrodynamicsL10:10}
A_i \rightarrow A_i + \PD{x^i}{\chi},
\end{equation}

the action for a particle changes by a boundary term

\begin{equation}\label{eqn:relativisticElectrodynamicsL10:30}
- \frac{e}{c} ( \chi(x_b) - \chi(x_a) ).
\end{equation}

Because \(S\) changes by a boundary term only, variation problem is not affected.  The extremal trajectories are then the same, hence the EOM are the same.

\paragraph{A less high brow demonstration}

With our four potential split into space and time components
\begin{equation}\label{eqn:relativisticElectrodynamicsL10:50}
A^i = (\phi, \BA),
\end{equation}

the lower index representation of the same vector is

\begin{equation}\label{eqn:relativisticElectrodynamicsL10:70}
A_i = (\phi, -\BA).
\end{equation}

Our gauge transformation is then

\begin{equation}\label{eqn:relativisticElectrodynamicsL10:90}
\begin{aligned}
A_0 &\rightarrow A_0 + \PD{x^0}{\chi} \\
-\BA &\rightarrow -\BA + \PD{\Bx}{\chi}
\end{aligned}
\end{equation}

or
\begin{equation}\label{eqn:relativisticElectrodynamicsL10:110}
\begin{aligned}
\phi &\rightarrow \phi + \inv{c}\PD{t}{\chi} \\
\BA &\rightarrow \BA - \spacegrad \chi.
\end{aligned}
\end{equation}

Now observe how the electric and magnetic fields are transformed

\begin{equation}\label{eqn:relativisticElectrodynamicsL10:410}
\begin{aligned}
\BE
&= - \spacegrad \phi - \inv{c} \PD{t}{\BA} \\
&\rightarrow
- \spacegrad \left( \phi + \inv{c}\PD{t}{\chi} \right) - \inv{c}\PD{t}{} \left( \BA - \spacegrad \chi \right) \\
\end{aligned}
\end{equation}

Sufficient continuity of \(\chi\) is assumed, allowing commutation of the space and time derivatives, and we are left with just \(\BE\)

For the magnetic field we have

\begin{equation}\label{eqn:relativisticElectrodynamicsL10:430}
\begin{aligned}
\BB
&= \spacegrad \cross \BA  \\
&\rightarrow
\spacegrad \cross (\BA  - \spacegrad \chi) \\
\end{aligned}
\end{equation}

Again with continuity assumptions, \(\spacegrad \cross (\spacegrad \chi) = 0\), and we are left with just \(\BB\).  The electromagnetic fields (as opposed to potentials) do not change under gauge transformations.

We conclude that the \(\{A_i\}\) description is hugely redundant, but despite that, local \(\LL\) and \(H\) can only be written in terms of the potentials \(A_i\).

\paragraph{Energy term of the Lorentz force.  Three vector approach}

With the Lagrangian for the particle given by

\begin{equation}\label{eqn:relativisticElectrodynamicsL10:130}
\LL = - mc^2 \InvGamma + \frac{e}{c} \BA \cdot \Bv - e \phi,
\end{equation}

we define the energy as

\begin{equation}\label{eqn:relativisticElectrodynamicsL10:150}
\calE = \Bv \cdot \PD{\Bv}{\LL} - \LL
\end{equation}

This is not necessarily a conserved quantity, but we define it as the energy anyways (we do not really have a Hamiltonian when the fields are time dependent).  Associated with this quantity is the general relationship

\begin{equation}\label{eqn:relativisticElectrodynamicsL10:170}
\ddt{\calE} = -\PD{t}{\LL},
\end{equation}

and when the Lagrangian is invariant with respect to time translation the energy \(\calE\) will be a conserved quantity (and also the Hamiltonian).

Our canonical momentum is
\begin{equation}\label{eqn:relativisticElectrodynamicsL10:190}
\PD{\Bv}{\LL} = \gamma m \Bv + \frac{e}{c} \BA
\end{equation}

So our energy is
\begin{equation}\label{eqn:relativisticElectrodynamicsL10:450}
\begin{aligned}
\calE = \gamma m \Bv^2 + \frac{e}{c} \BA \cdot \Bv - \left( - mc^2 \InvGamma + \frac{e}{c} \BA \cdot \Bv - e \phi \right).
\end{aligned}
\end{equation}

Or
\begin{equation}\label{eqn:relativisticElectrodynamicsL10:210}
\calE = \mathLabelBox{\frac{m c^2}{\InvGamma}}{\((\conj)\)} + e \phi.
\end{equation}

The contribution of \((\conj)\) to the energy \(\calE\) comes from the free (kinetic) particle portion of the Lagrangian \(\LL = -m c^2 \InvGamma\), and we identify the remainder as a potential energy

\begin{equation}\label{eqn:relativisticElectrodynamicsL10:230}
\calE = \frac{m c^2}{\InvGamma} + \mathLabelBox{e \phi}{"potential"}.
\end{equation}

For the kinetic portion we can also show that we have
\begin{equation}\label{eqn:relativisticElectrodynamicsL10:250}
\frac{d}{dt} \calE_{\text{kinetic}}
=
\frac{m c^2}{\InvGamma}
= e \BE \cdot \Bv.
\end{equation}

To show this observe that we have

\begin{equation}\label{eqn:relativisticElectrodynamicsL10:470}
\begin{aligned}
\frac{d}{dt} \calE_{\text{kinetic}}
&= m c^2 \frac{d\gamma}{dt} \\
&= m c^2 \frac{d}{dt} \inv{\InvGamma} \\
&= m c^2 \frac{\frac{\Bv}{c^2} \cdot \frac{d\Bv}{dt}}{\left(1 - \frac{\Bv^2}{c^2}\right)^{3/2}} \\
&= \frac{m \gamma \Bv \cdot \frac{d\Bv}{dt}}{1 - \frac{\Bv^2}{c^2}}
\end{aligned}
\end{equation}

We also have

\begin{equation}\label{eqn:relativisticElectrodynamicsL10:490}
\begin{aligned}
\Bv \cdot \ddt{\Bp}
&= \Bv \cdot \ddt{} \frac{m \Bv}{\InvGamma} \\
&= m\Bv^2 \ddt{\gamma} + m \gamma \Bv \cdot \ddt{\Bv} \\
&= m\Bv^2 \ddt{\gamma} + m c^2 \ddt{\gamma} \left( 1 - \frac{\Bv^2}{c^2} \right) \\
&= m c^2 \ddt{\gamma}.
\end{aligned}
\end{equation}

Utilizing the Lorentz force equation, we have

\begin{equation}\label{eqn:relativisticElectrodynamicsL10:270}
\Bv \cdot \ddt{\Bp} = e \left( \BE + \frac{\Bv}{c} \cross \BB \right) \cdot \Bv = e \BE \cdot \Bv
\end{equation}

and are able to assemble the above, and find that we have
\begin{equation}\label{eqn:relativisticElectrodynamicsL10:290}
\ddt{(m c^2 \gamma)} = e \BE \cdot \Bv
\end{equation}

\section{Four vector Lorentz force}
\index{four vector}
\index{Lorentz force}

Using \(ds = \sqrt{ dx^i dx_i } \) our action can be rewritten

\begin{equation}\label{eqn:relativisticElectrodynamicsL10:510}
\begin{aligned}
S
&= \int \left( -m c ds - \frac{e}{c} u^i A_i ds \right) \\
&= \int \left( -m c ds - \frac{e}{c} dx^i A_i \right) \\
&= \int \left( -m c \sqrt{ dx^i dx_i} - \frac{e}{c} dx^i A_i \right) \\
\end{aligned}
\end{equation}

\(x^i(\tau)\) is a worldline \(x^i(0) = a^i\), \(x^i(1) = b^i\),

We want \(\delta S = S[ x + \delta x ] - S[ x ] = 0\) (to linear order in \(\delta x\))

The variation of our proper length is
\begin{equation}\label{eqn:relativisticElectrodynamicsL10:530}
\begin{aligned}
\delta ds
&=
\delta \sqrt{ dx^i dx_i } \\
&= \inv{ 2 \sqrt{ dx^i dx_i }} \delta (dx^j dx_j)
\end{aligned}
\end{equation}

Observe that for the numerator we have
\begin{equation}\label{eqn:relativisticElectrodynamicsL10:550}
\begin{aligned}
\delta (dx^j dx_j)
&= \delta ( dx^j g_{jk} dx^k ) \\
&= \delta ( dx^j ) g_{jk} dx^k + dx^j g_{jk} \delta ( dx^k ) \\
&= \delta ( dx^j ) g_{jk} dx^k + dx^k g_{kj} \delta ( dx^j ) \\
&= 2 \delta ( dx^j ) g_{jk} dx^k \\
&= 2 \delta ( dx^j ) dx_j
\end{aligned}
\end{equation}

\paragraph{TIP:} If this goes too quick, or there is any disbelief, write these all out explicitly as \(dx^j dx_j = dx^0 dx_0 + dx^1 dx_1 + dx^2 dx_2 + dx^3 dx_3\) and compute it that way.

For the four vector potential our variation is

\begin{equation}\label{eqn:relativisticElectrodynamicsL10:310}
\delta A_i = A_i(x + \delta x) - A_i = \PD{x^j}{A_i} \delta x^j = \partial_j A_i \delta x^j
\end{equation}

(i.e. By chain rule)

Completing the proper length variations above we have

\begin{equation}\label{eqn:relativisticElectrodynamicsL10:570}
\begin{aligned}
\delta \sqrt{ dx^i dx_i }
&= \inv{ \sqrt{ dx^i dx_i }} \delta (dx^j) dx_j \\
&= \delta (dx^j) \dds{x_j}  \\
&= \delta (dx^j) u_j \\
&= d \delta x^j u_j
\end{aligned}
\end{equation}

We are now ready to assemble results and do the integration by parts

\begin{equation}\label{eqn:relativisticElectrodynamicsL10:590}
\begin{aligned}
\delta S
&= \int \left(
-m c d (\delta x^j) u_j
- \frac{e}{c} d (\delta x^i) A_i
- \frac{e}{c} dx^i \partial_j A_i \delta x^j
\right) \\
&=
{\left.
\left( -m c (\delta x^j) u_j - \frac{e}{c} (\delta x^i) A_i \right)
\right\vert}_a^b
+\int \left(
m c \delta x^j d u_j
+ \frac{e}{c} (\delta x^i) d A_i
- \frac{e}{c} dx^i \partial_j A_i \delta x^j
\right) \\
\end{aligned}
\end{equation}

Our variation at the endpoints is zero \(\evalbar{\delta x^i}{a} = \evalbar{\delta x^i}{b} = 0\), killing the non-integral terms

\begin{equation}\label{eqn:relativisticElectrodynamicsL10:610}
\begin{aligned}
\delta S
&=
\int
\delta x^j
\left(
m c d u_j
+ \frac{e}{c} d A_j
- \frac{e}{c} dx^i \partial_j A_i
\right).
\end{aligned}
\end{equation}

Observe that our differential can also be expanded by chain rule

\begin{equation}\label{eqn:relativisticElectrodynamicsL10:330}
d A_j = \PD{x^i}{A_j} dx^i = \partial_i A_j dx^i,
\end{equation}

which simplifies the variation further

\begin{equation}\label{eqn:relativisticElectrodynamicsL10:630}
\begin{aligned}
\delta S
&=
\int
\delta x^j
\left(
m c d u_j
+ \frac{e}{c} dx^i ( \partial_i A_j - \partial_j A_i )
\right) \\
&=
\int
\delta x^j ds
\left(
m c \frac{d u_j}{ds}
+ \frac{e}{c} u^i ( \partial_i A_j - \partial_j A_i )
\right) \\
\end{aligned}
\end{equation}

Since this is true for all variations \(\delta x^j\), which is arbitrary, the interior part is zero everywhere in the trajectory.  The antisymmetric portion, a rank 2 4-tensor is called the electromagnetic field strength tensor, and written

\boxedEquation{eqn:relativisticElectrodynamicsL10:350}{
F_{ij} = \partial_i A_j - \partial_j A_i.
}

In matrix form this is

\begin{equation}\label{eqn:relativisticElectrodynamicsL10:370}
\Norm{ F_{ij} } =
\begin{bmatrix}
0 & E_x & E_y & E_z \\
-E_x & 0 & -B_z & B_y \\
-E_y & B_z & 0 & -B_x \\
-E_z & -B_y & B_x & 0
\end{bmatrix}.
\end{equation}

In terms of the field strength tensor our Lorentz force equation takes the form

\boxedEquation{eqn:relativisticElectrodynamicsL10:390}{
\dds{(m c u_i)} = \frac{e}{c} F_{ij} u^j.
}
