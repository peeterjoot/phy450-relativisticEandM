%
% Copyright � 2012 Peeter Joot.  All Rights Reserved.
% Licenced as described in the file LICENSE under the root directory of this GIT repository.
%

%\chapter{Green's function solution to Maxwell's equation.  Lienard-Wiechert potentials}
\index{Lienard-Wiechert potentials}
\label{chap:relativisticElectrodynamicsL18}
%\blogpage{http://sites.google.com/site/peeterjoot/math2011/relativisticElectrodynamicsL18.pdf}
%\date{Mar 9, 2011}
%
\paragraph{Reading}
%
Covering chapter 8 material from the text \citep{landau1980classical}, and
\popcite{RelEMpp136-146.pdf}{lecture notes RelEMpp136-146.pdf}.
%: continued reminder of electrostatic Green's function (136); the retarded Green's function of the d'Alembert operator: derivation and properties (137-140); the solution of the d'Alembert equation with a source: retarded potentials (141-142); retarded time.

\section{Solving the forced wave equation}
\index{wave equation!forced}

See the notes for a complex variables and Fourier transform method of deriving the Green's function.  In class, we will just pull it out of a magic hat.  We wish to solve
%
\begin{equation}\label{eqn:relativisticElectrodynamicsL18:10}
\square A^k = \partial_i \partial^i A^k = \frac{4 \pi}{c} j^k
\end{equation}

(with a \(\partial_i A^i = 0\) gauge choice).

Our Green's method utilizes
%
\begin{equation}\label{eqn:relativisticElectrodynamicsL18:30}
\square_{(\Bx, t)} G(\Bx - \Bx', t - t') = \delta^3( \Bx - \Bx') \delta( t - t')
\end{equation}

If we know such a function, our solution is simple to obtain
%
\begin{equation}\label{eqn:relativisticElectrodynamicsL18:50}
A^k(\Bx, t)
= \int d^3 \Bx' dt' \frac{4 \pi}{c} j^k(\Bx', t') G(\Bx - \Bx', t - t').
\end{equation}
Proof:
\begin{equation}\label{eqn:relativisticElectrodynamicsL18:740}
\begin{aligned}
\square_{(\Bx, t)} A^k(\Bx, t)
&=
\int d^3 \Bx' dt' \frac{4 \pi}{c} j^k(\Bx', t')
\square_{(\Bx, t)}
G(\Bx - \Bx', t - t') \\
&=
\int d^3 \Bx' dt' \frac{4 \pi}{c} j^k(\Bx', t')
\delta^3( \Bx - \Bx') \delta( t - t') \\
&=
\frac{4 \pi}{c} j^k(\Bx, t)
\end{aligned}
\end{equation}

Claim:
%
\begin{equation}\label{eqn:relativisticElectrodynamicsL18:70}
G(\Bx, t) = \frac{\delta(t - \Abs{\Bx}/c)}{4 \pi \Abs{\Bx} }
\end{equation}

This is the retarded Green's function of the operator \(\square\), where
%
\begin{equation}\label{eqn:relativisticElectrodynamicsL18:90}
\square G(\Bx, t) = \delta^3(\Bx) \delta(t)
\end{equation}

\section{Elaborating on the wave equation Green's function}
\index{Green's function}

The Green's function \eqnref{eqn:relativisticElectrodynamicsL18:520} is a distribution that is non-zero only on the future lightcone.  Observe that for \(t < 0\) we have
%
\begin{equation}\label{eqn:relativisticElectrodynamicsL18:940}
\begin{aligned}
\delta\left(t - \frac{\Abs{\Bx}}{c}\right)
&=
\delta\left(-\Abs{t} - \frac{\Abs{\Bx}}{c}\right) \\
&= 0.
\end{aligned}
\end{equation}

We say that \(G\) is supported only on the future light cone.  At \(\Bx = 0\), only the contributions for \(t > 0\) matter.  Note that in the ``old days'', Green's functions used to be called influence functions, a name that works particularly well in this case.  We have other Green's functions for the d'Alembertian.  The one above is called the retarded Green's functions and we also have an advanced Green's function.  Writing \(+\) for advanced and \(-\) for retarded these are
%
\begin{equation}\label{eqn:relativisticElectrodynamicsL18:600}
G_{\pm} =
\frac{\delta\left(t \pm \frac{\Abs{\Bx}}{c}\right)}{4 \pi \Abs{\Bx}}
\end{equation}

There are also causal and non-causal variations that will not be of interest for this course.

This arms us now to solve any problem in the Lorentz gauge
%
\begin{equation}\label{eqn:relativisticElectrodynamicsL18:620}
A^k(\Bx, t) = \inv{c} \int d^3 \Bx' dt'
\frac{\delta\left(t - t' - \frac{\Abs{\Bx -\Bx'}}{c}\right)}{4 \pi \Abs{\Bx - \Bx'}}
j^k(\Bx', t')
+\text{An arbitrary collection of EM waves.}
\end{equation}

The additional EM waves are the possible contributions from the homogeneous equation.

Since \(\delta(t - t' - \Abs{\Bx -\Bx'}/c)\) is non-zero only when \(t' = t - \Abs{\Bx -\Bx'}/c)\), the non-homogeneous parts of \eqnref{eqn:relativisticElectrodynamicsL18:620} reduce to
%
\begin{equation}\label{eqn:relativisticElectrodynamicsL18:625}
A^k(\Bx, t) = \inv{c} \int d^3 \Bx'
\frac{j^k(\Bx', t - \Abs{\Bx - \Bx'}/c)}{4 \pi \Abs{\Bx - \Bx'}}.
\end{equation}

Our potentials at time \(t\) and spatial position \(\Bx\) are completely specified in terms of the sums of the currents acting at the retarded time \(t - \Abs{\Bx - \Bx'}/c\).  The field can only depend on the charge and current distribution in the past.  Specifically, it can only depend on the charge and current distribution on the past light cone of the spacetime point at which we measure the field.

%
\makeexample{The Green's function, a charged particle moving on a worldline}{example:relativisticElectrodynamicsL18:1}{
%
\begin{equation}\label{eqn:relativisticElectrodynamicsL18:150}
(c t, \Bx_c(t))
\end{equation}

(\(c\) for classical)

For this particle
%
\begin{equation}\label{eqn:relativisticElectrodynamicsL18:170}
\begin{aligned}
\rho(\Bx, t) &= e \delta^3(\Bx - \Bx_c(t)) \\
\Bj(\Bx, t) &= e \dot{\Bx}_c(t) \delta^3(\Bx - \Bx_c(t))
\end{aligned}
\end{equation}
%
\begin{equation}\label{eqn:relativisticElectrodynamicsL18:960}
\begin{aligned}
\begin{bmatrix}
A^0(\Bx, t) \\
\BA(\Bx, t)
\end{bmatrix}
&=
\inv{c}
\int d^3 \Bx' dt'
\frac{ \delta( t - t' - \Abs{\Bx - \Bx'}/c }{\Abs{\Bx - \Bx'}}
\begin{bmatrix}
c e \\
e \dot{\Bx}_c(t)
\end{bmatrix}
\delta^3(\Bx - \Bx_c(t)) \\
&=
\int_{-\infty}^\infty
\frac{ \delta( t - t' - \Abs{\Bx - \Bx_c(t')}/c }{\Abs{\Bx_c(t') - \Bx}}
\begin{bmatrix}
e \\
e \frac{\dot{\Bx}_c(t)}{c}
\end{bmatrix}
\end{aligned}
\end{equation}

PICTURE: light cones, and curved worldline.  Pick an arbitrary point \((\Bx_0, t_0)\), and draw the past light cone, looking at where this intersects with the trajectory

For the arbitrary point \((\Bx_0, t_0)\) we see that this point and the retarded time \((\Bx_c(t_r), t_r)\) obey the relation
%
\begin{equation}\label{eqn:relativisticElectrodynamicsL18:190}
c (t_0 - t_r) = \Abs{\Bx_0 - \Bx_c(t_r)}
\end{equation}

This retarded time is unique.  There is only one such intersection.
\index{retarded time}

Our job is to calculate
%
\begin{equation}\label{eqn:relativisticElectrodynamicsL18:210}
\int_{-\infty}^\infty \delta(f(x)) g(x) = \frac{g(x_\conj)}{\Abs{f'(x_\conj)}}
\end{equation}

where \(f(x_\conj) = 0\).
%
\begin{equation}\label{eqn:relativisticElectrodynamicsL18:230}
f(t') = t - t' - \Abs{\Bx - \Bx_c(t')}/c
\end{equation}
%
\begin{equation}\label{eqn:relativisticElectrodynamicsL18:980}
\begin{aligned}
\PD{t'}{f}
&= -1 - \inv{c} \PD{t'}{} \sqrt{ (\Bx - \Bx_c(t')) \cdot (\Bx - \Bx_c(t')) } \\
&= -1 + \inv{c} \frac{(\Bx - \Bx_c(t')) \cdot \Bv_c(t')}{\Abs{\Bx - \Bx_c(t')}}
\end{aligned}
\end{equation}

This is with
%
\begin{equation}\label{eqn:relativisticElectrodynamicsL18:700}
\Bv_c = \PD{t'}{\Bx_c}.
\end{equation}

Putting things back together, the potentials due to a moving charge are
%
\begin{equation}\label{eqn:relativisticElectrodynamicsL18:1000}
\begin{aligned}
\begin{bmatrix}
A^0(\Bx, t) \\
\BA(\Bx, t)
\end{bmatrix}
&=
e \inv{\Abs{\Bx_c(t_r) - \Bx}}
\begin{bmatrix}
1 \\
\frac{\Bv_c}{c}
\end{bmatrix}
\inv{\Abs{
-1 + \inv{c} \frac{(\Bx - \Bx_c(t_r)) \cdot \Bv_c(t_r)}{\Abs{\Bx - \Bx_c(t_r)}}
}
} \\
&=
e
%\inv{\Abs{\Bx_c(t_r) - \Bx}}
\begin{bmatrix}
1 \\
\frac{\Bv_c}{c}
\end{bmatrix}
\inv{\Abs{
\Abs{\Bx - \Bx_c(t_r)} - (\Bx - \Bx_c(t_r)) \cdot \Bv_c(t_r)/c
} }
\end{aligned}
\end{equation}

Provided \(\Abs{\Bx - \Bx_c} > (\Bx - \Bx_c(t_r)) \cdot \Bv_c(t_r)/c\), we have the Lienard-Wiechert potentials.
%
\begin{equation}\label{eqn:relativisticElectrodynamicsL18:720}
\begin{bmatrix}
A^0(\Bx, t) \\
\BA(\Bx, t)
\end{bmatrix}
=
e
%\inv{\Abs{\Bx_c(t_r) - \Bx}}
\begin{bmatrix}
1 \\
\frac{\Bv_c}{c}
\end{bmatrix}
\inv{
\Abs{\Bx - \Bx_c} - (\Bx - \Bx_c(t_r)) \cdot \Bv_c(t_r)/c }
\end{equation}

FIXME: What provides the previous inequality required to get to this final point?
}
