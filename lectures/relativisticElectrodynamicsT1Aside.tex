%
% Copyright © 2012 Peeter Joot.  All Rights Reserved.
% Licenced as described in the file LICENSE under the root directory of this GIT repository.
%

Simon (our TA) stated \cref{eqn:relativisticElectrodynamicsT1:130} without justification.  Here's a little justification for the frequency four vector.

We know some of it from the QM context, and if we have been reading ahead know a bit of this from our text \citep{landau1980classical} (the energy momentum four vector relationships).  Let us go back to the classical electromagnetism and recall what we know about the relation of frequency and wave numbers for continuous fields.  We want solutions to Maxwell's equation in vacuum and can show that such solution also implies that our fields obey a wave equation
%
\begin{equation}\label{eqn:relativisticElectrodynamicsT1:131}
\inv{c^2} \frac{\partial^2 \Psi}{\partial t^2} - \spacegrad^2 \Psi = 0,
\end{equation}

where \(\Psi\) is one of \(\BE\) or \(\BB\) or any component of either of these.  There are other constraints imposed on the solutions by Maxwell's equations, but the electric and magnetic field components must obey \eqnref{eqn:relativisticElectrodynamicsT1:131} in addition to those constraints.

A Fourier transform trial solution of the form
%
\begin{equation}\label{eqn:relativisticElectrodynamicsT1:132}
\Psi = (2 \pi)^{-3/2} \int \tilde{\Psi}(\Bk, 0) e^{i (\omega t \pm \Bk \cdot \Bx) } d^3 \Bk.
\end{equation}

can be applied to the wave equation, producing the constraint
%
\begin{equation}\label{eqn:relativisticElectrodynamicsT1:133}
\inv{c^2} (i \omega)^2 \Psi - (\pm i \Bk)^2 \Psi = 0.
\end{equation}

So even in the continuous field domain (no QM), we have a relationship between frequency and wave number.  We see that this also happens to have the form of a lightlike spacetime interval
%
\begin{equation}\label{eqn:relativisticElectrodynamicsT1:134}
\frac{\omega^2}{c^2} - \Bk^2 = 0.
\end{equation}

Also recall that the photoelectric effect imposes an experimental constraint on photon energy, where we have
%
\begin{equation}\label{eqn:relativisticElectrodynamicsT1:135}
E = h \nu = \frac{h}{2\pi} 2 \pi \nu = \Hbar \omega
\end{equation}

Therefore if we impose a mechanics like \(P = (E/c, \Bp) \) relativistic energy-momentum relationship on light, it then makes sense to form a nilpotent (lightlike) four vector for our photon energy.  This combines our special relativistic expectations, with the constraints on the fields imposed by classical electromagnetism.  We can then write for the photon four momentum
%
\begin{equation}\label{eqn:relativisticElectrodynamicsT1:136}
P = \left( \frac{\Hbar \omega}{c}, \Hbar k \right)
\end{equation}

