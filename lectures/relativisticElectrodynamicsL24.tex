%
% Copyright � 2012 Peeter Joot.  All Rights Reserved.
% Licenced as described in the file LICENSE under the root directory of this GIT repository.
%

%\chapter{PHY450H1S.  Relativistic Electrodynamics Lecture 24 (Taught by Prof. Erich Poppitz).  Non-relativistic electrostatic Lagrangian}
%\chapter{Non-relativistic electrostatic Lagrangian}
\index{Lagrangian!electrostatic}
\label{chap:relativisticElectrodynamicsL24}
%\blogpage{http://sites.google.com/site/peeterjoot/math2011/relativisticElectrodynamicsL24.pdf}
%\date{Mar 30, 2011}

\paragraph{Reading}

Covering chapter 5 \S 37, and chapter 8 \S 65 material from the text \citep{landau1980classical}, and
\popcite{RelEMpp181-195.pdf}{lecture notes RelEMpp181-195.pdf}.
%: the Lagrangian for a system of non relativistic charged particles to zeroth order in \((v/c)\): electrostatic energy of a system of charges and .mass renormalization.

\section{A closed system of charged particles}

Consider a closed system of charged particles \((m_a, q_a)\) and imagine there is a frame where they are non-relativistic \(v_a/c \ll 1\).  In this case we can describe the dynamics using a Lagrangian only for particles.  i.e.
%FIXME: language?  I will just small parameter \(v/c\).
%
\begin{equation}\label{eqn:relativisticElectrodynamicsL24:50}
\LL = \LL( \Bx_1, \cdots, \Bx_N, \Bv_1, \cdots, \Bv_N)
\end{equation}

If we work t order \((v/c)^2\).

If we try to go to \(O((v/c)^3\), it is difficult to only use \(\LL\) for particles.

This can be inferred from
%
\begin{equation}\label{eqn:relativisticElectrodynamicsL24:70}
P = \frac{2}{3} \frac{e^2}{c^3} \Abs{\ddot{\Bd}}^2
\end{equation}

because at this order, due to radiation effects, we need to include EM field as dynamical.

\section{Start simple}

Start with a system of (non-relativistic) free particles
%
\begin{equation}\label{eqn:relativisticElectrodynamicsL24:430}
\begin{aligned}
S
&= \sum_a - m_a c \int_{\text{a-th particle worldline}} ds  \\
&= \sum_a -m_a c^2 \int_{t_1}^{t_2} dt \sqrt{ 1 - \Bv_a^2/c^2} \\
&\approx \sum_a -m_a c^2 \int_{t_1}^{t_2} dt \left( 1 - \inv{2} \frac{\Bv^2}{c^2} - \inv{8} \frac{\Bv_a^4}{c^4}
\right) \\
&= \sum_a \int_{t_1}^{t_2} dt \left( \cancel{-m_a c^2} + \inv{2} m_a \Bv^2 + \inv{8} m_a \Bv_a^2 \frac{\Bv_a^2}{c^2} \right) \\
\end{aligned}
\end{equation}

So in the non-relativistic limit, after dropping the constant term that does not effect the dynamics, our Lagrangian is
%
\begin{equation}\label{eqn:relativisticElectrodynamicsL24:90}
\LL(\Bx_a, \Bv_a) = \inv{2} \sum_a m_a \Bv_a^2 + \inv{8} \frac{m_a \Bv_a^4}{c^2}
\end{equation}

The first term is \(O((v/c)^0)\) where the second is \(O((v/c)^2)\).

Next include the fact that particles are charged.
%
\begin{equation}\label{eqn:relativisticElectrodynamicsL24:110}
\LL_{\text{interaction}} = \sum_a \left( \cancel{q_a \frac{\Bv_a}{c} \cdot \BA(\Bx_a, t)} - q_a \phi(\Bx_a, t) \right)
\end{equation}

Here, working to \(O((v/c)^0)\), where we consider the particles moving so slowly that we have only a Coulomb potential \(\phi\), not \(\BA\).

HERE: these are NOT 'EXTERNAL' potentials.  They are caused by all the charged particles.
%
\begin{equation}\label{eqn:relativisticElectrodynamicsL24:130}
\partial_i F^{i l} = \frac{4 \pi}{c} j^l = 4 \pi \rho
\end{equation}

For \(l = \alpha\) we have have \(4 \pi \rho \Bv/c\), but we will not do this today (tomorrow).

To leading order in \(v/c\), particles only created Coulomb fields and they only ``feel'' Coulomb fields.  Hence to \(O((v/c)^0)\), we have
%
\begin{equation}\label{eqn:relativisticElectrodynamicsL24:150}
\LL = \sum_a \frac{m_a \Bv_a^2}{2} - q_a \phi(\Bx_a, t)
\end{equation}

What is the \(\phi(\Bx_a, t)\), the Coulomb field created by all the particles.

\paragraph{How to find?}
%
\begin{equation}\label{eqn:relativisticElectrodynamicsL24:170}
\partial_i F^{i 0} = \frac{4 \pi}{c} = 4 \pi \rho,
\end{equation}
or
%
\begin{equation}\label{eqn:relativisticElectrodynamicsL24:190}
\spacegrad \cdot \BE = 4 \pi \rho = - \spacegrad^2 \phi,
\end{equation}
where
\begin{equation}\label{eqn:relativisticElectrodynamicsL24:210}
\rho(\Bx, t) = \sum_a q_a \delta^3 (\Bx - \Bx_a(t))
\end{equation}

This is a Poisson equation
%
\begin{equation}\label{eqn:relativisticElectrodynamicsL24:230}
\Delta \phi(\Bx) = \sum_a q_a 4 \pi \delta^3(\Bx - \Bx_a)
\end{equation}

(where the time dependence has been suppressed).  This has solution
%
\begin{equation}\label{eqn:relativisticElectrodynamicsL24:10}
\phi(\Bx, t) = \sum_b \frac{q_b}{\Abs{\Bx - \Bx_b(t)}}
\end{equation}

This is the sum of instantaneous Coulomb potentials of all particles at the point of interest.  Hence, it appears that \(\phi(\Bx_a, t)\) should be evaluated in \eqnref{eqn:relativisticElectrodynamicsL24:10} at \(\Bx_a\)?

However \eqnref{eqn:relativisticElectrodynamicsL24:10} becomes infinite due to contributions of the a-th particle itself.  Solution to this is to drop the term, but let us discuss this first.

Let us talk about the electrostatic energy of our system of particles.
%
\begin{equation}\label{eqn:relativisticElectrodynamicsL24:450}
\begin{aligned}
\calE
&= \inv{8 \pi} \int d^3 \Bx \left(\BE^2 + \cancel{\BB^2} \right) \\
&= \inv{8 \pi} \int d^3 \Bx \BE \cdot (-\spacegrad \phi) \\
&= \inv{8 \pi} \int d^3 \Bx \left( \spacegrad \cdot (\BE \phi) - \phi \spacegrad \cdot \BE \right) \\
&= -\inv{8 \pi} \oint d^2 \Bsigma \cdot \BE \phi + \inv{8 \pi} \int d^3 \Bx \phi \spacegrad \cdot \BE  \\
\end{aligned}
\end{equation}

The first term is zero since \(\BE \phi\) for a localized system of charges \(\sim 1/r^3\) or higher as \(V \rightarrow \infty\).

In the second term
%
\begin{equation}\label{eqn:relativisticElectrodynamicsL24:250}
\spacegrad \cdot \BE =
4 \pi \sum_a q_a \delta^3(\Bx - \Bx_a(t))
\end{equation}

So we have
%
\begin{equation}\label{eqn:relativisticElectrodynamicsL24:270}
\sum_a \inv{2} \int d^3 \Bx q_a \delta^3(\Bx - \Bx_a) \phi(\Bx)
\end{equation}

for
%
\begin{equation}\label{eqn:relativisticElectrodynamicsL24:20}
\calE
= \inv{2} \sum_a q_a \phi(\Bx_a)
\end{equation}

Now substitute \eqnref{eqn:relativisticElectrodynamicsL24:10} into \eqnref{eqn:relativisticElectrodynamicsL24:20} for
%
\begin{equation}\label{eqn:relativisticElectrodynamicsL24:290}
\calE = \inv{2} \sum_a \frac{q_a^2}{\Abs{\Bx - \Bx_a}} + \inv{2} \sum_{a \ne b} \frac{q_a q_b}{\Abs{\Bx_a - \Bx_b}}
\end{equation}

or
%
\begin{equation}\label{eqn:relativisticElectrodynamicsL24:310}
\calE = \inv{2} \sum_a \frac{q_a^2}{\Abs{\Bx - \Bx_a}} + \sum_{a < b} \frac{q_a q_b}{\Abs{\Bx_a - \Bx_b}}
\end{equation}

The first term is the sum of the electrostatic self energies of all particles.  The source of this infinite self energy is in assuming a \textunderline{point like nature} of the particle.  i.e.  We modeled the charge using a delta function instead of using a continuous charge distribution.

Recall that if you have a charged sphere of radius \(r\)

PICTURE: total charge \(q\), radius \(r\), our electrostatic energy is
%
\begin{equation}\label{eqn:relativisticElectrodynamicsL24:330}
\calE \sim \frac{q^2}{r}
\end{equation}

Stipulate that rest energy \(m_e c^2\) is all of electrostatic origin \(\sim e^2/r_e\) we get that
%
\begin{equation}\label{eqn:relativisticElectrodynamicsL24:350}
r_e \sim \frac{e^2}{m_e c^2}
\end{equation}

This is called the classical radius of the electron, and is of a very small scale \(10^{-13} \text{cm}\).

As a matter of fact the applicability of classical electrodynamics breaks down much sooner than this scale since quantum effects start kicking in.

Our Lagrangian is now
%
\begin{equation}\label{eqn:relativisticElectrodynamicsL24:370}
\LL_a = \inv{2} m_a \Bv_a^2 - q_a \phi(\Bx_a, t)
\end{equation}

where \(\phi\) is the electrostatic potential due to all \textunderline{other} particles, so we have
%
\begin{equation}\label{eqn:relativisticElectrodynamicsL24:390}
\LL_a = \inv{2} m_a \Bv_a^2 - \inv{2} \sum_{a \ne b} \frac{q_a q_b }{\Abs{\Bx_a - \Bx_b}}
\end{equation}

and for the system
%
\begin{equation}\label{eqn:relativisticElectrodynamicsL24:410}
\LL = \inv{2} \sum_a m_a \Bv_a^2 - \sum_{a < b} \frac{q_a q_b }{\Abs{\Bx_a - \Bx_b}}
\end{equation}

This is THE Lagrangian for electrodynamics in the non-relativistic case, starting with the relativistic action.

\section{What is next?}

We continue to the next order of \(v/c\) tomorrow.
