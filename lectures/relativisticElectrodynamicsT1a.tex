%
% Copyright © 2012 Peeter Joot.  All Rights Reserved.
% Licenced as described in the file LICENSE under the root directory of this GIT repository.
%
\section{Introducing four vectors}
\index{four vector}

(From tutorial 1)

A 3-vector:
%
\begin{equation}\label{eqn:relativisticElectrodynamicsT1:10}
\begin{aligned}
\Ba &= (a_x, a_y, a_z) = (a^1, a^2, a^3) \\
\Bb &= (b_x, b_y, b_z) = (b^1, b^2, b^3).
\end{aligned}
\end{equation}
%
For this we have the dot product
\begin{equation}\label{eqn:relativisticElectrodynamicsT1:20}
\Ba \cdot \Bb = \sum_{\alpha=1}^3 a^\alpha b^\alpha
\end{equation}
%
Greek letters in this course (opposite to everybody else in the world, because of Landau and Lifshitz) run from 1 to 3, whereas roman letters run through the set \(\{0,1,2,3\}\).

We want to put space and time on an equal footing and form the composite quantity (four vector)
\begin{equation}\label{eqn:relativisticElectrodynamicsT1:40}
x^i = (ct, \Br) = (x^0, x^1, x^2, x^3),
\end{equation}
where
\begin{equation}\label{eqn:relativisticElectrodynamicsT1:80}
\begin{aligned}
x^0 &= ct \\
x^1 &= x \\
x^2 &= y \\
x^3 &= z.
\end{aligned}
\end{equation}
%
It will also be convenient to drop indices when referring to all the components of a four vector and we will use lower or upper case non-bold letters to represent such four vectors.  For example
%
\begin{equation}\label{eqn:relativisticElectrodynamicsT1:81}
X = (ct, \Br),
\end{equation}
or
\begin{equation}\label{eqn:relativisticElectrodynamicsT1:82}
u = \gamma \left(1, \Bv/c \right).
\end{equation}
%
Three vectors will be represented as letters with over arrows \(\vec{a}\) or (in text) bold face \(\Ba\).

Recall that the squared spacetime interval between two events \(X_1\) and \(X_2\) is defined as
%
\begin{equation}\label{eqn:relativisticElectrodynamicsT1:60}
{S_{X_1, X_2}}^2 = (ct_1 - c t_2)^2 - (\Bx_1 - \Bx_2)^2.
\end{equation}
%
In particular, with one of these zero, we have an operator which takes a single four vector and spits out a scalar, measuring a ``distance'' from the origin
%
\begin{equation}\label{eqn:relativisticElectrodynamicsT1:30}
s^2 = (ct)^2 - \Br^2.
\end{equation}
%
This motivates the introduction of a dot product for our four vector space.
%
\begin{equation}\label{eqn:relativisticElectrodynamicsT1:50}
X \cdot X = (ct)^2 - \Br^2 = (x^0)^2 - \sum_{\alpha=1}^3 (x^\alpha)^2
\end{equation}
%
Utilizing the spacetime dot product of \eqnref{eqn:relativisticElectrodynamicsT1:50} we have for the dot product of the difference between two events
%
\begin{equation}\label{eqn:relativisticElectrodynamicsT1a:140}
\begin{aligned}
(X - Y) \cdot (X - Y)
&=
(x^0 - y^0)^2 - \sum_{\alpha =1}^3 (x^\alpha - y^\alpha)^2 \\
&=
X \cdot X + Y \cdot Y - 2 x^0 y^0 + 2 \sum_{\alpha =1}^3 x^\alpha y^\alpha.
\end{aligned}
\end{equation}
%
From this, assuming our dot product \eqnref{eqn:relativisticElectrodynamicsT1:50} is both linear and symmetric, we have for any pair of spacetime events
%
\begin{equation}\label{eqn:relativisticElectrodynamicsT1:55}
X \cdot Y = x^0 y^0 - \sum_{\alpha =1}^3 x^\alpha y^\alpha.
\end{equation}
%
How do our four vectors transform?  This is really just a notational issue, since this has already been discussed.  In this new notation we have
%
\begin{equation}\label{eqn:relativisticElectrodynamicsT1:90}
\begin{aligned}
{x^0}' &= ct' = \gamma ( ct - \beta x) = \gamma ( x^0 - \beta x^1 ) \\
{x^1}' &= x' = \gamma ( x - \beta ct ) = \gamma ( x^1 - \beta x^0 ) \\
{x^2}' &= x^2 \\
{x^3}' &= x^3
\end{aligned}
\end{equation}
%
where \(\beta = V/c\), and \(\gamma^{-2} = 1 - \beta^2\).

In order to put some structure to this, it can be helpful to express this dot product as a quadratic form.  We write
%
\begin{equation}\label{eqn:relativisticElectrodynamicsT1:100}
\begin{aligned}
A \cdot B =
\begin{bmatrix}
a^0 & \Ba^\T
\end{bmatrix}
\begin{bmatrix}
1 & 0 & 0 & 0 \\
0 & -1 & 0 & 0 \\
0 & 0 & -1 & 0 \\
0 & 0 & 0 & -1
\end{bmatrix}
\begin{bmatrix}
b^0 \\
\Bb
\end{bmatrix}
= A^\T G B.
\end{aligned}
\end{equation}
%
We can write our Lorentz boost as a matrix
%
\begin{equation}\label{eqn:relativisticElectrodynamicsT1:110}
\begin{bmatrix}
\gamma & -\beta \gamma & 0 & 0 \\
-\beta \gamma & \gamma & 0 & 0 \\
0 & 0 & 1 & 0 \\
0 & 0 & 0 & 1
\end{bmatrix}
\end{equation}
%
so that the dot product between two transformed four vectors takes the form
%
\begin{equation}\label{eqn:relativisticElectrodynamicsT1:120}
A' \cdot B' = A^\T O^\T G O B
\end{equation}
