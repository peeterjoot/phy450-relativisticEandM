%
% Copyright � 2012 Peeter Joot.  All Rights Reserved.
% Licenced as described in the file LICENSE under the root directory of this GIT repository.
%

%\chapter{Two worked problems}
\label{chap:relativisticElectrodynamicsT2}
%\blogpage{http://sites.google.com/site/peeterjoot/math2011/relativisticElectrodynamicsT2.pdf}
%\date{Jan 27, 2011}
%
%\section{What we will discuss}
%
%\begin{itemize}
%\item 4-vectors: position, velocity, acceleration
%\item non-inertial observers
%\end{itemize}

%\chapter{Non-inertial (local) observers}

This was from the second tutorial.
%
\section{Basis construction.}
%
Observations are made of either the three-vector, or the time like components of four-vectors, since these are the quantities that we can measure from our local observer frame.  This is something that can be viewed in an approximate sense as being inertial, provided that we ignore the earth's rotation, the rotation around the solar system, the rotation of the solar system in the galaxy, the rotation of the galaxy in the local cluster, and so forth.  Provided none of these are changing too fast relative to our measurements, we can make the inertial approximation.

Example.  If we want to measure energy, it is the timelike component of the momentum.
%
\begin{equation}\label{eqn:relativisticElectrodynamicsT2:500}
E = c p^0.
\end{equation}
%
PICTURE:  Let us imagine a moving worldline in three dimensions.  We can setup a frame and associated basis along the worldline of the particle, as well as a frame and basis for the stationary observer.

In class Simon used notation like \(\{ e_{\hat{o}}^i \}\), and \(\{ e_{\hat{a}}^i \}\), but also used \(e_{\hat{0}}^i\), \(e_{\hat{1}}^i\), \(e_{\hat{2}}^i\), \(e_{\hat{3}}^i\).  It was fairly clear by the context what was meant, but lets avoid any more than one index at a time, and write \(\{ f_i \}\) for the frame moving along the worldline, and \(\{ e_i \}\) for the stationary frame.
%
\paragraph{Constructing a basis along the worldline}
\index{worldline}

For any timelike four-vector worldline we have a four-vector velocity of magnitude \(c\), so we are free to define a timelike basis vector for our moving frame as
%
\begin{equation}\label{eqn:relativisticElectrodynamicsT2:510}
f_0 = u.
\end{equation}
%
going back to the first problem for \(u^i\) we have
%
\begin{equation}\label{eqn:relativisticElectrodynamicsT2:599b}
f_0 = ( \cosh( a c t ), \sinh( a c t), 0, 0 ).
\end{equation}
We are free to pick spatial unit vectors perpendicular to this, so for the \(y\) and \(z\) components it is natural to use
\begin{equation}\label{eqn:relativisticElectrodynamicsT2:599a}
\begin{aligned}
f_2 &= ( 0, 0, 1, 0 ) \\
f_3 &= ( 0, 0, 0, 1 ).
\end{aligned}
\end{equation}
%
We need one more, that is perpendicular to each of the above.  By inspection one can pick
%
\begin{equation}\label{eqn:relativisticElectrodynamicsT2:599}
f_1 = ( \sinh( a c t ), \cosh( a c t), 0, 0).
\end{equation}
%
Did Simon use any other principle to define this last beastie?  I missed it if he did.  I see that this happens to be the unit vector proportional to \(x^i\).
%
\paragraph{Consider the stationary observer}
%
For a stationary observer, our worldline and four velocity respectively, for some constant \(\Bx_0\) is
%
\begin{equation}\label{eqn:relativisticElectrodynamicsT2:600}
\begin{aligned}
X &= ( ct, \Bx_0 ) \\
\frac{dX}{ds} &= \inv{c} \frac{dX}{d\tau} = ( 1, \Bzero).
\end{aligned}
\end{equation}
%
Our time like unit vector is very simple
%
\begin{equation}\label{eqn:relativisticElectrodynamicsT2:610}
e_0 = \frac{dX}{ds} = ( 1, \Bzero ).
\end{equation}
%
For the spatial unit vectors we have many choices.  One would be aligned from the origin to the position vector
%
\begin{equation}\label{eqn:relativisticElectrodynamicsT2:620}
e_1 = \left( 0, \frac{\Bx}{\Abs{\Bx}} \right),
\end{equation}
%
with \(e_2\) and \(e_3\) oriented in any pair of mutually perpendicular spatial directions.  Another option would be simply pick a \(e_\alpha\) for each of the normal Euclidean basis directions
%
\begin{equation}\label{eqn:relativisticElectrodynamicsT2:630}
\begin{aligned}
e_1 &= ( 0, 1, 0, 0 ) \\
e_2 &= ( 0, 0, 1, 0 ) \\
e_3 &= ( 0, 0, 0, 1 ).
\end{aligned}
\end{equation}
%
Observe, that we have (no sum) \(e_\alpha \cdot e_\alpha = -1\) (and \(e_0 \cdot e_0 = 1\)).
%
\paragraph{Consider an inertial observer}
%
Now consider a slightly more complex case, where an observer is moving with some constant velocity \(\BV = c \Bbeta\).  Our worldline is
%
\begin{equation}\label{eqn:relativisticElectrodynamicsT2:700}
X = ( ct, \Bx_0 + \Bbeta c t) .
\end{equation}
%
Let us calculate the four velocity.  We have
%
\begin{equation}\label{eqn:relativisticElectrodynamicsT2:710}
\frac{dX}{dt} = c ( 1, \Bbeta ).
\end{equation}
%
From this our proper time is
%
\begin{equation}\label{eqn:relativisticElectrodynamicsT2:720}
\tau = \inv{c} \int_0^t c \sqrt{ (1, \Bbeta)^2 } dt = \sqrt{1 - \Bbeta^2} t.
\end{equation}
%
Our worldline and four-velocity, parametrized in terms of proper time, with \(\gamma = (1 - \Bbeta^2)^{-1/2}\), are then
%
\begin{equation}\label{eqn:relativisticElectrodynamicsT2:730}
\begin{aligned}
X &= ( \gamma c\tau, \Bx_0 + \gamma \Bbeta c \tau) \\
u &= \gamma ( 1, \Bbeta ).
\end{aligned}
\end{equation}
%
For this system, let us label the basis \(\{h_k\}\).  From above our time like unit vector is
%
\begin{equation}\label{eqn:relativisticElectrodynamicsT2:740}
h_0 = \gamma ( 1, \Bbeta ).
\end{equation}
%
We observe that this has the desired time like property, \((h_0)^2 = 1 > 0\).

Now, let us try Gram-Schmidt, subtracting the projection of \(h_0\) on \(e_1\) from \(e_1\) and see what we get.  Our projection is
%
\begin{equation}\label{eqn:relativisticElectrodynamicsT2:2220}
\begin{aligned}
\Proj_{h_0}(e_1)
&= \frac{e_1 \cdot h_0}{h_0 \cdot h_0} h_0 \\
&= (0, 1, 0, 0) \cdot \gamma (1, \Bbeta) \gamma (1, \Bbeta) \\
&= -\gamma^2 \beta_x (1, \Bbeta).
\end{aligned}
\end{equation}
%
We should have a space like vector normal to \(h_0\) once we take the Gram-Schmidt difference
%
\begin{equation}\label{eqn:relativisticElectrodynamicsT2:750}
e_1 - \frac{e_1 \cdot h_0}{h_0 \cdot h_0} h_0 =
(0, 1, 0, 0) + \gamma^2 \beta_x (1, \Bbeta).
\end{equation}
%
Let us compute the norm of this vector and verify that it is space like.  We should also verify that it is normal to \(h_0\) as expected.  For the norm we have
%
\begin{equation}\label{eqn:relativisticElectrodynamicsT2:2240}
\begin{aligned}
-1 + \beta_x^2 + 2 \beta_x \gamma^2 (0, 1, 0, 0) \cdot (1, \Bbeta)
&=
-1 + \beta_x^2 + 2 \beta_x \gamma^2 (-\beta_x) \\
&=
\beta_x^2 ( 1 - 2 \gamma^2 ) -1 \\
&=
\beta_x^2 \frac{1 - \Bbeta^2 - 2 }{1 - \Bbeta^2} -1 \\
&=
-\beta_x^2 \frac{1 + \Bbeta^2}{1 - \Bbeta^2} -1.
\end{aligned}
\end{equation}
%
This is less than zero as we expect for a spacelike vector.  Good.  Our second spacelike unit vector is thus
%
\begin{equation}\label{eqn:relativisticElectrodynamicsT2:760}
h_1 = \left( \beta_x^2 \frac{1 + \Bbeta^2}{1 - \Bbeta^2} +1 \right)^{-1/2} \left(
(0, 1, 0, 0) + \gamma^2 \beta_x (1, \Bbeta)
\right).
\end{equation}
%
Let us verify that these two computed spacetime basis vectors are normal.  Their dot product is proportional to
%
\begin{equation}\label{eqn:relativisticElectrodynamicsT2:2260}
\begin{aligned}
((0, 1, 0, 0) + \gamma^2 \beta_x (1, \Bbeta)) \cdot (1, \Bbeta)
&= -\beta_x + \gamma^2 \beta_x ( 1 - \Bbeta^2 ) \\
&= -\beta_x + \beta_x \\
&= 0 \qedmarker.
\end{aligned}
\end{equation}
%
We could continue this, continuing the Gram-Schmidt iteration using \(e_2\) and \(e_3\) for the remainder of the initial spanning set.

Doing so, we would have
%
\begin{equation}\label{eqn:relativisticElectrodynamicsT2:770}
h_2 \sim e_2 - \frac{e_2 \cdot h_1}{h_1 \cdot h_1} h_1 - \frac{e_2 \cdot h_0}{h_0 \cdot h_0} h_0.
\end{equation}
%
After scaling so that \(h_2 \cdot h_2 = -1\), we would then have
%
\begin{equation}\label{eqn:relativisticElectrodynamicsT2:780a}
h_3 \sim
e_3
- \frac{e_3 \cdot h_2}{h_2 \cdot h_2} h_2
- \frac{e_3 \cdot h_1}{h_1 \cdot h_1} h_1
- \frac{e_3 \cdot h_0}{h_0 \cdot h_0} h_0.
\end{equation}
%
\paragraph{Projections and the reciprocal basis}
\index{reciprocal basis}

Recall that for Euclidean space, when we had orthonormal vectors, we could simplify the Gram-Schmidt procedure from
%
\begin{equation}\label{eqn:relativisticElectrodynamicsT2:770b}
e_{k+1} \sim f_{k+1} - \sum_{i=0}^k \frac{f_{k+1} \cdot e_i}{e_i \cdot e_i} e_i,
\end{equation}
%
to
%
\begin{equation}\label{eqn:relativisticElectrodynamicsT2:770c}
e_{k+1} \sim f_{k+1} - \sum_{i=0}^k \left( f_{k+1} \cdot e_i \right) e_i.
\end{equation}
%
However, for our non-Euclidean space, we cannot do this.  This suggests a nice intuitive motivation for the reciprocal basis.  We can define, for any normalized basis \(\{f^i\}\) in our Minkowski space (no sum)
%
\begin{equation}\label{eqn:relativisticElectrodynamicsT2:775}
e^i = \frac{e_i}{e_i \cdot e_i}.
\end{equation}
%
Now our Gram-Schmidt iteration becomes
%
\begin{equation}\label{eqn:relativisticElectrodynamicsT2:770e}
e_{k+1} \sim f_{k+1} - \sum_{i=0}^k \left( f_{k+1} \cdot e_i \right) e^i,
\end{equation}
%
and we identify, for a four vector \(b\), the projection onto the chosen basis vector, as (no sum)
%
\begin{equation}\label{eqn:relativisticElectrodynamicsT2:780b}
\Proj_{e^i}(b) = (b \cdot e_i) e^i.
\end{equation}
%
In particular, we have for the resolution of identity (now with summation implied again)
%
\begin{equation}\label{eqn:relativisticElectrodynamicsT2:790}
b = (b \cdot e_i) e^i.
\end{equation}
%
This is nice and it allows us to work with four vectors in their entirety, instead of in coordinates.  We have
%
\begin{equation}\label{eqn:relativisticElectrodynamicsT2:800}
x = x^i e_i = x_i e^i,
\end{equation}
%
where
%
\begin{equation}\label{eqn:relativisticElectrodynamicsT2:810}
\begin{aligned}
x^i &= x \cdot e^i \\
x_i &= x \cdot e_i.
\end{aligned}
\end{equation}
%
Also note that \(e^\alpha = -e_\alpha\) and \(e^0 = e_0\), just as the coordinates themselves vary sign with index raising and lowering dependent on whether they are time like or space like.

We have seen that the representation of the basis can be chosen to depend on the observer, and for the stationary observer, we had simply
%
\begin{equation}\label{eqn:relativisticElectrodynamicsT2:820}
\begin{aligned}
e_0 &= (1, 0, 0, 0) \\
e_1 &= (0, 1, 0, 0) \\
e_2 &= (0, 0, 1, 0) \\
e_3 &= (0, 0, 0, 1),
\end{aligned}
\end{equation}
%
with a reciprocal basis \(e^i \cdot e_j = {\delta^i}_j\)
%
\begin{equation}\label{eqn:relativisticElectrodynamicsT2:830}
\begin{aligned}
e^0 &= (1, 0, 0, 0) \\
e^1 &= -(0, 1, 0, 0) \\
e^2 &= -(0, 0, 1, 0) \\
e^3 &= -(0, 0, 0, 1).
\end{aligned}
\end{equation}
%
\paragraph{An alternate basis for the inertial frame}
\index{inertial frame}

Given the same \(h_0\) as defined above for the inertial frame, let us define an alternate \(h_1\), subtracting the timelike component from the worldline of the particle itself.  Let
%
\begin{equation}\label{eqn:relativisticElectrodynamicsT2:2280}
\begin{aligned}
X &= ( \gamma c \tau, \Bx_0 + \gamma \Bbeta c \tau ) \\
h_0 &= \gamma ( 1, \Bbeta ) \\
Y & = X - (X \cdot h_0) h^0.
\end{aligned}
\end{equation}
%
The dot product above is
%
\begin{equation}\label{eqn:relativisticElectrodynamicsT2:2300}
\begin{aligned}
X \cdot h_0
&=
( \gamma c \tau, \Bx_0 + \gamma \Bbeta c \tau ) \cdot \gamma (1, \Bbeta) \\
&=
\gamma^2 c \tau - \gamma (\Bbeta \cdot \Bx_0) - \gamma^2 \Bbeta^2 c \tau \\
&=
\gamma^2 c \tau ( 1 - \Bbeta^2) - \gamma (\Bbeta \cdot \Bx_0) \\
&=
c \tau - \gamma (\Bbeta \cdot \Bx_0).
\end{aligned}
\end{equation}
%
Our rejection of \(h_0\) from \(X\) is then
%
\begin{equation}\label{eqn:relativisticElectrodynamicsT2:2320}
\begin{aligned}
Y
&= ( \gamma c \tau, \Bx_0 + \gamma \Bbeta c \tau ) - (c \tau - \gamma (\Bbeta \cdot \Bx_0)) \gamma( 1, \Bbeta) \\
&= ( \gamma^2 (\Bbeta \cdot \Bx_0), \Bx_0 + \gamma \Bbeta c \tau - c \tau \gamma \Bbeta + \gamma^2 (\Bbeta \cdot \Bx_0) \Bbeta ) \\
&= ( \gamma^2 (\Bbeta \cdot \Bx_0), \Bx_0 + \gamma^2 (\Bbeta \cdot \Bx_0) \Bbeta ) \\
&= \gamma^2 (\Bbeta \cdot \Bx_0)(1, \Bbeta) + (0, \Bx_0).
\end{aligned}
\end{equation}
%
We can verify that this is spacelike by computing the square
%
\begin{equation}\label{eqn:relativisticElectrodynamicsT2:2340}
\begin{aligned}
Y^2
&= \gamma^2 (\Bbeta \cdot \Bx_0)^2 - \Bx_0^2 + 2 \gamma^2 (\Bbeta \cdot \Bx_0) (1, \Bbeta) \cdot (0, \Bx_0) \\
&= \gamma^2 (\Bbeta \cdot \Bx_0)^2 - \Bx_0^2 - 2 \gamma^2 (\Bbeta \cdot \Bx_0)^2 \\
&= -\gamma^2 (\Bbeta \cdot \Bx_0)^2 - \Bx_0^2 \\
&< 0.
\end{aligned}
\end{equation}
%
A final normalization of this yields
%
\begin{equation}\label{eqn:relativisticElectrodynamicsT2:900}
h_1 = (\gamma^1 (\Bbeta \cdot \Bx_0)^2 + \Bx_0^2)^{-1/2} \left( \gamma^2 (\Bbeta \cdot \Bx_0)(1, \Bbeta) + (0, \Bx_0) \right).
\end{equation}
%
It is easy enough to verify that we have \(h_1 \cdot h_0 = 0\) as desired.
%
\paragraph{A followup note on the worldline basis}
\index{basis!worldline}

Note that we can construct the spatial vector \(f^1\) in \eqnref{eqn:relativisticElectrodynamicsT2:599} systematically without use of any sort of intuition.  We get this by Gram-Schmidt directly
%
\begin{equation}\label{eqn:relativisticElectrodynamicsT2:2360}
\begin{aligned}
f_1
&\sim e_1 - (e_1 \cdot e_0) e^0 - (\cancel{e_1 \cdot e_2}) e_2 - (\cancel{e_1 \cdot e_3}) e_3 \\
&= (0, 1, 0, 0) - (0, 1, 0, 0) \cdot ( \cosh(ac\tau), \sinh(ac\tau), 0, 0) e_0 \\
&= (0, 1, 0, 0) + \sinh(ac\tau) ( \cosh(ac\tau), \sinh(ac\tau), 0, 0) \\
&= (\sinh(ac\tau) \cosh(ac\tau), 1 + \sinh^2(ac\tau), 0, 0 ) \\
&= (\sinh(ac\tau) \cosh(ac\tau), \cosh^2(ac\tau), 0, 0 ) \\
&\sim (\sinh(ac\tau), \cosh(ac\tau), 0, 0 ) \qedmarker.
\end{aligned}
\end{equation}
%
It is also noteworthy to observe that we have \(f_i \cdot f_j = 0, i \ne j\), and \(f_0 \cdot f_0 = 1\) and \(f_\alpha \cdot f_\alpha = -1\), as desired.
%
\paragraph{Relating the Lorentz transformation and coordinate transformations}
%
We are familiar now with the tensor form of the Lorentz transformation.  This takes coordinates to coordinates
%
\begin{equation}\label{eqn:relativisticElectrodynamicsT2:901}
{x'}^i = {L_j}^i x^j.
\end{equation}
%
Specifying just the coordinates and not the basis associated with the coordinates leaves out some valuable seeming information.  For instance, is the basis associated with the pre and post transformed coordinates the same?

For example, suppose that our basis for the primed coordinates is \(\{f_i\}\), construction of the four vector (in its entirety) out of its coordinates and this basis requires the sum
%
\begin{equation}\label{eqn:relativisticElectrodynamicsT2:2380}
\begin{aligned}
X
&= {x'}^i f_i \\
&= ({L_j}^i f_i) x^j.
\end{aligned}
\end{equation}
%
This interior sum \({L_j}^i f_i\) is a linear combination of the primed basis vectors, but we see that these are in fact a set of vectors, and can be considered the basis for the unprimed coordinates.  We could for example write
%
\begin{equation}\label{eqn:relativisticElectrodynamicsT2:910}
e_i = {L_j}^i f_i.
\end{equation}
%
With such a description, our Lorentz transformation becomes just a mechanism to map vectors in one basis into another.  To make this clear, let us work in the opposite order, and suppose that we have a pair of bases \(\{e_i\}\) and \(\{f_i\}\).  For any vector \(X\) we can calculate the coordinates utilizing the reciprocal frame.
%
\begin{equation}\label{eqn:relativisticElectrodynamicsT2:920}
X = (X \cdot e^i) e_i = (X \cdot f^j) f_j.
\end{equation}
%
Writing
%
\begin{equation}\label{eqn:relativisticElectrodynamicsT2:930}
\begin{aligned}
x^i &= X \cdot e^i \\
{x'}^i &= X \cdot f^i.
\end{aligned}
\end{equation}
This is
\begin{equation}\label{eqn:relativisticElectrodynamicsT2:940}
{x'}^k f_k = x^j e_j.
\end{equation}
Dotting with \(f^i\) we have
%
\begin{equation}\label{eqn:relativisticElectrodynamicsT2:950}
{x'}^i = x^j (e_j \cdot f^i).
\end{equation}
%
In this form we see explicitly that the Lorentz transformation is in fact the ``direction cosines'' associated with a change of basis.  Specifically, we can write
%
\begin{equation}\label{eqn:relativisticElectrodynamicsT2:960}
{L_j}^i = e_j \cdot f^i.
\end{equation}
%
I like this as a way to view the Lorentz transformation, since the explicit inclusion of the basis sets involved makes the geometry clear.
\makeexample{A coordinate calculation example.}{example:relativisticElectrodynamicsT2:1}{
I have gone to the effort of calculating some basis representations in a lot more detail than we covered in the tutorial, and explore some of the ideas further.  This seemed important to get a feel for what we were discussing, and to see how the pieces fit together.

Let us do one more simple example, where we look at the coordinates of a four vector in the coordinate system where the time like direction is the proper velocity, and also eliminate the \(y\) and \(z\) coordinates from the mix to simplify it further.  For such a system we have only two choices for our spatial basis vector (we can alter the sign).

For our spacetime point, consider the worldline for a particle moving at a constant velocity.  That is
%
\begin{equation}\label{eqn:relativisticElectrodynamicsT2:970}
X = (ct, p_0 + \beta c t).
\end{equation}
%
As before our proper time is
%
\begin{equation}\label{eqn:relativisticElectrodynamicsT2:980}
\tau = \sqrt{1 - \beta^2} t,
\end{equation}
%
allowing us to re-parametrize the worldline, and have a proper time parametrized velocity
%
\begin{equation}\label{eqn:relativisticElectrodynamicsT2:970b}
\begin{aligned}
X &= (\gamma c \tau, p_0 + \beta \gamma c \tau) \\
u &= \gamma (1, \beta).
\end{aligned}
\end{equation}
%
Let us utilize the standard basis for the stationary frame, and denote this \(\{e_i\}\)
%
\begin{equation}\label{eqn:relativisticElectrodynamicsT2:981}
\begin{aligned}
e_0 &= (1, 0) \\
e_1 &= (0, 1).
\end{aligned}
\end{equation}
%
and calculate a basis \(\{f_i\}\) for which \(f_0 = u\) is the time like direction.  By Gram-Schmidt, our space like basis vector is
%
\begin{equation}\label{eqn:relativisticElectrodynamicsT2:2400}
\begin{aligned}
f_1
&\sim e_1 - (e_1 \cdot f_0) f^0 \\
&= (0,1) - (0, 1) \cdot \gamma (1,\beta) \gamma (1, \beta) \\
&= (0,1) - \gamma^2 (-\beta) (1, \beta) \\
&= (\gamma^2 \beta, 1 + \beta^2 \gamma^2 ) \\
&= \left(\gamma^2 \beta, \inv{1-\beta^2}(1 -\beta^2 + \beta^2) \right) \\
&= \left(\gamma^2 \beta, \gamma^2 \right) \\
&\sim - \gamma (\beta, 1).
\end{aligned}
\end{equation}
%
The negative sign here is a bit of sneaky move and chosen only after calculating the coordinates of the vector in this frame, so that at speed \(\beta = 0\), the coordinates in frames \(\{e^i\}\) and \(\{f^i\}\) are the same.  Our basis is then
%
\begin{equation}\label{eqn:relativisticElectrodynamicsT2:982}
\begin{aligned}
f_0 &= \gamma(1, \beta) \\
f_1 &= -\gamma(\beta, 1).
\end{aligned}
\end{equation}
%
One can quickly verify that \(f_0 \cdot f_0 = 1, f_1 \cdot f_1 = -1\), and \(f_0 \cdot f_1 = 0\).  Our reciprocal frame, defined so that \(f^i \cdot f_j = {\delta^i}_j\) is
%
\begin{equation}\label{eqn:relativisticElectrodynamicsT2:982b}
\begin{aligned}
f^0 &= \gamma(1, \beta) \\
f^1 &= \gamma(\beta, 1).
\end{aligned}
\end{equation}
%
With this basis our coordinate representation is
%
\begin{equation}\label{eqn:relativisticElectrodynamicsT2:983}
X = \mathLabelBox{(X \cdot f^0)}{\(x^0\)} f_0 + \mathLabelBox{(X \cdot f^1)}{\(x^1\)} f_1,
\end{equation}
and we calculate our coordinates to be
\begin{equation}\label{eqn:relativisticElectrodynamicsT2:984}
\begin{aligned}
x^0 &= c \tau - \gamma p_0 \beta \\
x^1 &= \gamma p_0.
\end{aligned}
\end{equation}
As a check one can verify that \(X = x^0 f_0 + x^1 f_1\) as expected.  So we see that in a frame for which the proper velocity is the time like basis vector, our particle is at rest (moving only in time).

Some interesting information can be extracted after making the coordinate calculation.  It is interesting to note that the position \(x^1 = \gamma p_0\) equals \(p_0\) when \(\beta = 0\).  When the particle is observed at rest in one frame, it remains at rest in the frame for which its proper velocity is the time like direction (the particle's rest frame).  Furthermore, when the particle is observed moving, the position in the particles rest frame is always greater than the observed position \(x_0 \gamma \ge x_0\).  In other words, the particle's position appears closer to the origin in the observer's frame than it is in the rest frame (it is position is contracted).

Also see that the rest frame time matches the observer frame time when the particle is observed at rest (\(\beta = 0\)).  The time in the rest frame is always less than the time in the observer frame and by increasing \(beta\) we can shift the initial time position of the particle in its rest frame as far backwards as we like.  Similarly, if the particle is observed moving backwards in the observer frame, the initial time position of the particle in the rest frame can be pushed as far forward in time as we like.
}
%
\paragraph{An initially confusing aspect of the given non-inertial worldline}
For the worldline
%
\begin{equation}\label{eqn:relativisticElectrodynamicsT2:2000}
X = \inv{a} ( \sinh( a c \tau ), \cosh( a c \tau) ),
\end{equation}
we calculated
\begin{equation}\label{eqn:relativisticElectrodynamicsT2:2010}
\begin{aligned}
u &= ( \cosh( a c \tau ), \sinh( a c \tau) ) \\
f_0 = u &= ( \cosh( a c \tau ), \sinh( a c \tau) ) \\
f_1 &= ( \sinh( a c \tau ), \cosh( a c \tau) ).
\end{aligned}
\end{equation}
%
The curious thing about this basis is that when one calculates the rest frame coordinates
%
\begin{equation}\label{eqn:relativisticElectrodynamicsT2:2020}
\begin{aligned}
x^0 &= X \cdot f^0 = 0 \\
x^1 &= X \cdot f^1 = \inv{a},
\end{aligned}
\end{equation}
%
the timelike coordinate is zero uniformly?  We can verify easily that the position four vector is recovered as expected from \(X = x^0 f_0 + x^1 f_1\), but it still seems irregular that we have no timelike coordinate?

Oh!  I see.  This is a spacelike four vector.  Look at the length
%
\begin{equation}\label{eqn:relativisticElectrodynamicsT2:2030}
X^2 = \inv{a^2} ( \sinh^2 ( a c \tau) - \cosh^2 (a c \tau) ) = -\inv{a^2} < 0.
\end{equation}
%
Because it is spacelike in one frame, it can only be (just) spacelike in its rest frame.

%By calculating this coordinate, we also see that a choice of
%
%\begin{equation}\label{eqn:relativisticElectrodynamicsT2:2040}
%f_1 = -( \sinh( a c \tau ), \cosh( a c \tau) ),
%\end{equation}
%
%would have been a better one.  Then our particle's coordinate in the rest frame would be \(1/a\) at \(t=0\).  With the initial choice of the basis vector \(f^1\), it is coordinate ends up being the inverse of its position at \(t=0\).
%
\section{Split of energy and momentum (VERY ROUGH NOTES).}
%
\paragraph{Disclaimer.} At the very end of the tutorial Simon jotted some very quick notes, and I have included what I got of those without editing below.  I have yet to go through these and make something coherent of them.
%
In a coordinate representation, the timelike component of our momentum was obtained by extracting the first coordinate
%
\begin{equation}\label{eqn:relativisticElectrodynamicsT2:1000}
p^0 = (p^0, p^1, p^2, p^3) \cdot (1, 0, 0, 0).
\end{equation}
%
This was (after scaling) was our energy term \(E = c p^0\), and we can extract this in the observer frame by dotting with our observer frame timelike basis vector \(e^0\)
%
\begin{equation}\label{eqn:relativisticElectrodynamicsT2:1001}
E_{\text{observer}} = c p \cdot e^0 \equiv c p^0.
\end{equation}
%
In the observers reference frame, where \(u^i = ( 1, 0, 0, 0)\), and \(p^i = m c u^i\), we have
%
\begin{equation}\label{eqn:relativisticElectrodynamicsT2:1002}
{p}^i = (m c, 0, 0, 0).
\end{equation}
%
\begin{equation}\label{eqn:relativisticElectrodynamicsT2:1003}
{u'}^i_{\text{observed}} = \gamma (( 1, v/c , 0, 0).
\end{equation}
%
\begin{equation}\label{eqn:relativisticElectrodynamicsT2:1004}
u^i_{\text{observer}} = (1, 0, 0, 0).
\end{equation}
%
\begin{equation}\label{eqn:relativisticElectrodynamicsT2:1005}
{u'}^i_{\text{observer}} =
\begin{bmatrix}
\gamma & \gamma v/c  & 0 & 0 \\
\gamma v/c  & \gamma  & 0 & 0 \\
0 & 0 & 0 & 0 \\
0 & 0 & 0 & 0
\end{bmatrix}
{u}^i.
\end{equation}
%
\begin{equation}\label{eqn:relativisticElectrodynamicsT2:1006}
p^0 = \gamma m c.
\end{equation}
%
\section{Frequency of light from a distant star (AGAIN VERY ROUGH NOTES).}
%
Suppose we have a star far away.  What is the frequency of the light emitted
%
\begin{equation}\label{eqn:relativisticElectrodynamicsT2:1007}
\hat{\omega} = \omega e^{- a c \tau }.
\end{equation}
%
FIXME: derive.

where \(\omega\) is the emitted frequency.

FIXME: This implied an elapsed time before the star would no longer be visible?
