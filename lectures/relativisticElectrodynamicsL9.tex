%
% Copyright � 2012 Peeter Joot.  All Rights Reserved.
% Licenced as described in the file LICENSE under the root directory of this GIT repository.
%

%\chapter{Dynamics in a vector field}
\label{chap:relativisticElectrodynamicsL9}
%\blogpage{http://sites.google.com/site/peeterjoot/math2011/relativisticElectrodynamicsL9.pdf}
%\date{Feb 2, 2011}
%
\paragraph{Reading}
%
Covering chapter 2 material from the text \citep{landau1980classical}, and
\popcite{RelEMpp56.1-73.pdf}{lecture notes RelEMpp56.1-73.pdf}.
%: comments on mass, energy, momentum, and massless particles (56.1-58); particles in external fields: Lorentz scalar field (59-62); reminder of a vector field under spatial rotations (63) and a Lorentz vector field (64-65) [Tuesday, Feb. 1]; the action for a relativistic particle in an external 4-vector field (65-66); the equation of motion of a relativistic particle in an external electromagnetic (4-vector) field (67,68,73) [Wednesday, Feb. 2]; mathematical interlude: (69-72): on 3x3 antisymmetric matrices, 3-vectors, and totally antisymmetric 3-index tensor - please read by yourselves, preferably by Wed., Feb. 2 class! (this is important, we will also soon need the 4-dimensional generalization)
%
\section{More on the action.}
\index{action!external 4-scalar field}

Action for a relativistic particle in an external 4-scalar field
%
\begin{equation}\label{eqn:relativisticElectrodynamicsL9:700}
S = -m c \int ds - g \int ds \phi(x).
\end{equation}
%
Unfortunately we have no 4-vector scalar fields (at least for particles that are long lived and stable).

PICTURE: 3-vector field, some arrows in various directions.

PICTURE: A vector \(\BA\) in an \(x,y\) frame, and a rotated (counterclockwise by angle \(\alpha\)) \(x', y'\) frame with the components in each shown pictorially.

We have
%
\begin{equation}\label{eqn:relativisticElectrodynamicsL9:710}
\begin{aligned}
A_x'(x', y') &= \cos\alpha A_x(x,y) + \sin\alpha A_y(x,y) \\
A_y'(x', y') &= -\sin\alpha A_x(x,y) + \cos\alpha A_y(x,y).
\end{aligned}
\end{equation}
%
\begin{equation}\label{eqn:relativisticElectrodynamicsL9:20}
\begin{bmatrix}
A_x'(x', y') \\
A_y'(x', y')
\end{bmatrix}
=
\begin{bmatrix}
\cos\alpha A_x(x,y) & \sin\alpha A_y(x,y) \\
-\sin\alpha A_x(x,y) & \cos\alpha A_y(x,y)
\end{bmatrix}
\begin{bmatrix}
A_x(x, y) \\
A_y(x, y)
\end{bmatrix}.
\end{equation}
%
More generally we have
%
\begin{equation}\label{eqn:relativisticElectrodynamicsL9:40}
\begin{bmatrix}
A_x'(x', y', z') \\
A_y'(x', y', z') \\
A_z'(x', y', z')
\end{bmatrix}
=
\hat{O}
\begin{bmatrix}
A_x(x, y, z) \\
A_y(x, y, z) \\
A_z(x, y, z)
\end{bmatrix}.
\end{equation}
%
Here \(\hat{O}\) is an \(SO(3)\) matrix rotating \(x \rightarrow x'\)
%
\begin{equation}\label{eqn:relativisticElectrodynamicsL9:60}
\BA(\Bx) \cdot \By = \BA'(\Bx') \cdot \By'.
\end{equation}
%
\begin{equation}\label{eqn:relativisticElectrodynamicsL9:80}
\BA \cdot \BB = \text{invariant}.
\end{equation}
%
A four vector field is \(A^i(x)\), with \(x = x^i, i = 0,1,2,3\) and we would write
%
\begin{equation}\label{eqn:relativisticElectrodynamicsL9:100}
\begin{bmatrix}
(x^0)' \\
(x^1)' \\
(x^2)' \\
(x^3)'
\end{bmatrix}
=
\hat{O}
\begin{bmatrix}
x^0 \\
x^1 \\
x^2 \\
x^3
\end{bmatrix}.
\end{equation}
%
Now \(\hat{O}\) is an \(SO(1,3)\) matrix.    Our four vector field is then
%
\begin{equation}\label{eqn:relativisticElectrodynamicsL9:120}
\begin{bmatrix}
(A^0)' \\
(A^1)' \\
(A^2)' \\
(A^3)'
\end{bmatrix}
=
\hat{O}
\begin{bmatrix}
A^0 \\
A^1 \\
A^2 \\
A^3
\end{bmatrix}.
\end{equation}
%
We have
%
\begin{equation}\label{eqn:relativisticElectrodynamicsL9:140}
A^i g_{ij} x^i = \text{invariant} = {A'}^i g_{ij} {x'}^i.
\end{equation}
%
From electrodynamics we know that we have a scalar field, the electrostatic potential, and a vector field

What is a plausible action?

How about
%
\begin{equation}\label{eqn:relativisticElectrodynamicsL9:160}
\int ds x^i g_{ij} A^j.
\end{equation}
%
This is not translation invariant.
%
\begin{equation}\label{eqn:relativisticElectrodynamicsL9:180}
\int ds x^i g_{ij} A^j.
\end{equation}
%
Next simplest is
%
\begin{equation}\label{eqn:relativisticElectrodynamicsL9:200}
\int ds u^i g_{ij} A^j.
\end{equation}
%
Could also do
%
\begin{equation}\label{eqn:relativisticElectrodynamicsL9:220}
\int ds A^i g_{ij} A^j.
\end{equation}
%
but it turns out that this is not gauge invariant (to be defined and discussed in detail).
%
\paragraph{An aside.  Dimensions of proper velocity.}
\index{proper velocity!dimensions}

Note that the convention for this course is to write
%
\begin{equation}\label{eqn:relativisticElectrodynamicsL9:240}
u^i = \left( \gamma, \gamma \frac{\Bv}{c} \right) = \frac{dx^i}{ds}.
\end{equation}
%
Where \(u^i\) is dimensionless (\(u^i u_i = 1\)).  Some authors use
%
\begin{equation}\label{eqn:relativisticElectrodynamicsL9:260}
u^i = \left( \gamma c, \gamma \Bv \right) = \frac{dx^i}{d\tau},
\end{equation}
%
where \(u^i u_i = c^2\), and \(u^i\) has dimensions of velocity.
%
\paragraph{Return to the problem}
%
The simplest action for a four vector field \(A^i\) is then
%
\begin{equation}\label{eqn:relativisticElectrodynamicsL9:280}
S = - m c \int ds - \frac{e}{c} \int ds u^i A_i.
\end{equation}
%
(Recall that \(u^i A_i = u^i g_{ij} A^j\)).

In this action \(e\) is nothing but a Lorentz scalar, a property of the particle that describes how it ``couples'' (or ``feels'') the electrodynamics field.

Similarly \(mc\) is a Lorentz scalar which is a property of the particle (inertia).

It turns out that all the electric charges in nature are quantized, and there are some deep reasons (in magnetic monopoles exist) for this.

Another reason for charge quantization apparently has to do with gauge invariance and associated compact groups.  Poppitz is amusing himself a bit here, hinting at some stuff that we can eventually learn.

Returning to our discussion, we have
%
\begin{equation}\label{eqn:relativisticElectrodynamicsL9:300}
S = - m c \int ds - \frac{e}{c} \int ds u^i g_{ij} A^j.
\end{equation}
%
with the electrodynamics four vector potential
%
\begin{equation}\label{eqn:relativisticElectrodynamicsL9:320}
\begin{aligned}
A^i &= (\phi, \BA) \\
u^i &= \left(\gamma, \gamma \frac{\Bv}{c} \right) \\
u^i g_{ij} A^j &= \gamma \phi - \gamma \frac{\Bv \cdot \BA}{c}.
\end{aligned}
\end{equation}
%
\begin{equation}\label{eqn:relativisticElectrodynamicsL9:730}
\begin{aligned}
S
&= - m c^2 \int dt \sqrt{1 - \frac{\Bv^2}{c^2}} - \frac{e}{c} \int c dt \sqrt{1 - \frac{\Bv^2}{c^2}} \left( \gamma \phi - \gamma \frac{\Bv}{c} \cdot \BA \right) \\
&= \int dt \left(
- m c^2 \sqrt{1 - \frac{\Bv^2}{c^2}} - e \phi(\Bx, t) + \frac{e}{c} \Bv \cdot \BA(\Bx, t)
\right).
\end{aligned}
\end{equation}
%
\begin{equation}\label{eqn:relativisticElectrodynamicsL9:340}
\PD{\Bv}{\LL} = \frac{m c^2}{\sqrt{1 - \frac{\Bv^2}{c^2}}} \frac{\Bv}{c^2} + \frac{e}{c} \BA(\Bx, t).
\end{equation}
%
\begin{equation}\label{eqn:relativisticElectrodynamicsL9:360}
\frac{d}{dt} \PD{\Bv}{\LL} = m \frac{d}{dt} (\gamma \Bv) + \frac{e}{c} \PD{t}{\BA} + \frac{e}{c} \PD{x^\alpha}{\BA} v^\alpha.
\end{equation}
%
Here \(\alpha,\beta = 1,2,3\) and are summed over.

For the other half of the Euler-Lagrange equations we have
%
\begin{equation}\label{eqn:relativisticElectrodynamicsL9:380}
\PD{x^\alpha}{\LL} = - e \PD{x^\alpha}{\phi} + \frac{e}{c} v^\beta \PD{x^\alpha}{A^\beta}.
\end{equation}
%
Equating these, and switching to coordinates for \eqnref{eqn:relativisticElectrodynamicsL9:360}, we have
%
\begin{equation}\label{eqn:relativisticElectrodynamicsL9:381}
m \frac{d}{dt} (\gamma v^\alpha) + \frac{e}{c} \PD{t}{A^\alpha} + \frac{e}{c} \PD{x^\beta}{A^\alpha} v^\beta
= - e \PD{x^\alpha}{\phi} + \frac{e}{c} v^\beta \PD{x^\alpha}{A^\beta}.
\end{equation}
%
A final rearrangement yields
%
\begin{equation}\label{eqn:relativisticElectrodynamicsL9:400}
\frac{d}{dt} m \gamma v^\alpha = e \mathLabelBox{\left( - \inv{c} \PD{t}{A^\alpha} - \PD{x^\alpha}{\phi} \right)}{\(E^\alpha\)} + \frac{e}{c} v^\beta \left( \PD{x^\alpha}{A^\beta} - \PD{x^\beta}{A^\alpha} \right).
\end{equation}
%
We can identity the second term with the magnetic field but first have to introduce antisymmetric matrices.
%
\section{antisymmetric matrices.}
\index{antisymmetric}
%
\begin{equation}\label{eqn:relativisticElectrodynamicsL9:750}
\begin{aligned}
M_{\mu\nu}
&= \PD{x^\mu}{A^\nu} - \PD{x^\nu}{A^\mu} \\
&= \epsilon_{\mu\nu\lambda} B_\lambda,
\end{aligned}
\end{equation}
%
where
%
\begin{equation}\label{eqn:relativisticElectrodynamicsL9:440}
\epsilon_{\mu\nu\lambda} =
\begin{array}{l l}
0 & \quad \mbox{if any two indices coincide} \\
1 & \quad \mbox{for even permutations of \(\mu\nu\lambda\)} \\
-1 & \quad \mbox{for odd permutations of \(\mu\nu\lambda\)}
\end{array}.
\end{equation}
%
Example:
%
\begin{equation}\label{eqn:relativisticElectrodynamicsL9:770}
\begin{aligned}
\epsilon_{123} &= 1 \\
\epsilon_{213} &= -1 \\
\epsilon_{231} &= 1.
\end{aligned}
\end{equation}
%
We can show that
%
\begin{equation}\label{eqn:relativisticElectrodynamicsL9:420}
B_\lambda = \inv{2} \epsilon_{\lambda\mu\nu} M_{\mu\nu}.
\end{equation}
%
\begin{equation}\label{eqn:relativisticElectrodynamicsL9:790}
\begin{aligned}
B_1
&= \inv{2} ( \epsilon_{123} M_{23} + \epsilon_{132} M_{32})  \\
&= \inv{2} ( M_{23} - M_{32})  \\
&= \partial_2 A_3 - \partial_3 A_2.
\end{aligned}
\end{equation}
%
Using
%
\begin{equation}\label{eqn:relativisticElectrodynamicsL9:430}
\epsilon_{\mu\nu\alpha} \epsilon_{\sigma\kappa\alpha} =
\delta_{\mu\sigma} \delta_{\nu\kappa} - \delta_{\nu\sigma} \delta_{\mu\kappa},
\end{equation}
%
we can verify the identity \eqnref{eqn:relativisticElectrodynamicsL9:420} by expanding
%
\begin{equation}\label{eqn:relativisticElectrodynamicsL9:810}
\begin{aligned}
\epsilon_{\mu\nu\lambda} B_\lambda
&=
\inv{2} \epsilon_{\mu\nu\lambda} \epsilon_{\lambda\alpha\beta} M_{\alpha\beta} \\
&=
\inv{2} (
\delta_{\mu\alpha} \delta_{\nu\beta} - \delta_{\nu\alpha} \delta_{\mu\beta}
)
M_{\alpha\beta} \\
&=
\inv{2} (M_{\mu\nu} - M_{\nu\mu}) \\
&=
M_{\mu\nu}.
\end{aligned}
\end{equation}
%
Returning to the action evaluation we have
%
\begin{equation}\label{eqn:relativisticElectrodynamicsL9:460}
\frac{d}{dt} ( m \gamma v^\alpha ) = e E^\alpha + \frac{e}{c} \epsilon_{\alpha\beta\gamma} v^\beta B_\gamma,
\end{equation}
but
\begin{equation}\label{eqn:relativisticElectrodynamicsL9:480}
\epsilon_{\alpha\beta\gamma} B_\gamma = (\Bv \cross \BB)_\alpha,
\end{equation}
so
%
\begin{equation}\label{eqn:relativisticElectrodynamicsL9:500}
\frac{d}{dt} ( m \gamma \Bv ) = e \BE + \frac{e}{c} \Bv \cross \BB,
\end{equation}
or
%
\begin{equation}\label{eqn:relativisticElectrodynamicsL9:520}
\frac{d}{dt} ( \Bp ) = e \left( \BE + \frac{\Bv}{c} \cross \BB \right).
\end{equation}
\paragraph{What is the energy component of the Lorentz force equation}
I asked this, not because I do not know (I could answer this myself from \(dp/d\tau = F \cdot v/c\), in the geometric algebra formalism, but I was curious if he had a way of determining this from what we have derived so far (intuitively I had expect this to be possible).  Answer was:

Observe that this is almost a relativistic equation, but we are not going to get to the full equation yet.  The energy component can be obtained from
%
\begin{equation}\label{eqn:relativisticElectrodynamicsL9:540}
\frac{du^0}{ds} = e F^{0j} u_j.
\end{equation}
%
Since the full equation is
%
\begin{equation}\label{eqn:relativisticElectrodynamicsL9:560}
\frac{du^i}{ds} = e F^{ij} u_j.
\end{equation}
%
``take with a grain of salt, may be off by sign, or factors of \(c\)''.

Also curious is that he claimed the energy component of this equation was not very important.  Why would that be?
%
\section{Gauge transformations.}
\index{gauge transformation}

Claim
%
\begin{equation}\label{eqn:relativisticElectrodynamicsL9:580}
S_{\text{interaction}} = - \frac{e}{c} \int ds u^i A_i.
\end{equation}
%
changes by boundary terms only under

``gauge transformation'' :
%
\begin{equation}\label{eqn:relativisticElectrodynamicsL9:600}
A_i = A_i' + \PD{x^i}{\chi}.
\end{equation}
%
where \(\chi\) is a Lorentz scalar.  This \(\PDi{x^i}{}\) is the four gradient.  Let us see this

Therefore the equations of motion are the same in an external \(A^i\) and \({A'}^i\).

Recall that the \(\BE\) and \(\BB\) fields do not change under such transformations.  Let us see how the action transforms
%
\begin{equation}\label{eqn:relativisticElectrodynamicsL9:830}
\begin{aligned}
S
&= - \frac{e}{c} \int ds u^i A_i  \\
&= - \frac{e}{c} \int ds u^i \left( {A'}_i + \PD{x^i}{\chi} \right) \\
&= - \frac{e}{c} \int ds u^i {A'}_i  - \frac{e}{c} \int ds \frac{dx^i}{ds} \PD{x^i}{\chi}.
\end{aligned}
\end{equation}
%
Observe that this last bit is just a chain rule expansion
%
\begin{equation}\label{eqn:relativisticElectrodynamicsL9:850}
\begin{aligned}
\frac{d}{ds} \chi(x^0, x^1, x^2, x^3)
&=
\PD{x^0}{\chi}\frac{dx^0}{ds} +
\PD{x^1}{\chi}\frac{dx^1}{ds} +
\PD{x^2}{\chi}\frac{dx^2}{ds} +
\PD{x^3}{\chi}\frac{dx^3}{ds} \\
&=
\PD{x^i}{\chi} \frac{dx^i}{ds},
\end{aligned}
\end{equation}
so we have
\begin{equation}\label{eqn:relativisticElectrodynamicsL9:610}
S
= - \frac{e}{c} \int ds u^i {A'}_i - \frac{e}{c} \int ds \frac{d \chi}{ds}.
\end{equation}
%
This allows the line integral to be evaluated, and we find that it only depends on the end points of the interval
%
\begin{equation}\label{eqn:relativisticElectrodynamicsL9:620}
S = - \frac{e}{c} \int ds u^i {A'}_i - \frac{e}{c} ( \chi(x_b) - \chi(x_a) ),
\end{equation}
%
which completes the proof of the claim that this gauge transformation results in an action difference that only depends on the end points of the interval.
%
\paragraph{Gauge invariance of \texorpdfstring{\(A \cdot A\)}{A squared} action}
\index{gauge invariance}

Now that we know what gauge invariance means, let us look at the portion of the potential action \eqnref{eqn:relativisticElectrodynamicsL9:220} discarded because it was not gauge invariant.  Under gauge transformation this becomes
%
\begin{equation}\label{eqn:relativisticElectrodynamicsL9:870}
\begin{aligned}
\int ds {A'}^i {A'}_i
&=
\int ds \left({A}_i + \PD{x^i}{\chi}\right) \left(A^i + \PD{x_i}{\chi}\right) \\
&=
\int ds {A}^i A_i
+ A^i \PD{x^i}{\chi}
+ A_i \PD{x_i}{\chi}
+ \PD{x^i}{\chi} \PD{x_i}{\chi} \\
&=
\int ds {A}^i A_i
+ 2 A^i \PD{x^i}{\chi}
+ \PD{x^i}{\chi} \PD{x_i}{\chi}.
\end{aligned}
\end{equation}
%
Without the proper velocity term we do not have a way to simply re-pack the chain rule expansion and eliminate the last two terms as we did with the Lorentz force action.
