%
% Copyright � 2012 Peeter Joot.  All Rights Reserved.
% Licenced as described in the file LICENSE under the root directory of this GIT repository.
%

%\chapter{PHY450H1S.  Relativistic Electrodynamics Lecture 25 (Taught by Prof. Erich Poppitz).  Second order interaction, the Darwin Lagrangian}
%\chapter{Second order interaction, the Darwin Lagrangian}
\index{Lagrangian!Darwin}
\index{Darwin!Lagrangian}
\label{chap:relativisticElectrodynamicsL25}
%\blogpage{http://sites.google.com/site/peeterjoot/math2011/relativisticElectrodynamicsL25.pdf}
%\date{Mar 31, 2011}
%
\paragraph{Reading}
%
Covering chapter 8 \S 65 material from the text \citep{landau1980classical}, and
\popcite{RelEMpp181-195.pdf}{lecture notes RelEMpp181-195.pdf}.
%: (182-189) [Tuesday, Mar. 29]; the EM potentials to order \((v/c)^2\) (190-193); the ``Darwin Lagrangian.  and Hamiltonian for a system of non-relativistic charged particles to order \((v/c)^2\) and its many uses in physics (194-195) [Wednesday, Mar. 30]
%Next week (last topic): attempt to go to the next order \((v/c)^3\) - radiation damping, the limitations of classical electrodynamics, and the relevant time/length/energy scales.
%
\section{Recap.}
%
Last time we started with our relativistic Lagrangian for a single particle
%
\begin{equation}\label{eqn:relativisticElectrodynamicsL25:470}
\LL_a = - m c^2 \sqrt{ 1 - \frac{\Bv_a^2}{c^2} } - \frac{q_a}{c} \frac{dx^i}{dt} A_i.
\end{equation}
%
and found that to the first order in \(v/c\) we had
%
\begin{equation}\label{eqn:relativisticElectrodynamicsL25:10}
\LL_a = \inv{2} m_a \Bv_a^2 - q_a \phi(\Bx_a, t).
\end{equation}
%
Here the potential was approximated by Taylor expansion to contain just
%
\begin{equation}\label{eqn:relativisticElectrodynamicsL25:30}
\phi(\Bx_a, t) = \inv{2} \sum_{a \ne b} \frac{q_b }{\Abs{\Bx_a - \Bx_b}}
+ \frac{q_a}{``\Bx_a - \Bx_a''}.
\end{equation}
%
The second term is something that no sane person would write, and represents the infinite electrostatic self energy of a charge.  This is infinite because we have assumed (by virtue of using a delta function for the current and charge distribution) that the charge is pointlike.  The ``solution'' to this problem was to omit this self energy term completely, essentially treating the charge of the electron as distributed.  We avoid looking specifically where it is located.

The logic here is that this does not affect the motion (i.e. The Euler Lagrange equations) for the particle, provided it is viewed from afar, with distances \(\gg\) \textunderline{size of particle}.

We made an estimate of the scale for which our Lagrangian does not apply.  Namely
%
\begin{equation}\label{eqn:relativisticElectrodynamicsL25:50}
\frac{e^2}{r_e} \sim m_e c^2,
\end{equation}
%
so we were able to conclude that the ``classical radius of the electron'', something that does not really exist, was of the scale
%
\begin{equation}\label{eqn:relativisticElectrodynamicsL25:70}
r_e \sim \frac{e^2 }{m_e c^2} \sim 10^{-13} \text{cm}.
\end{equation}
%
(We do see this quantity arise in physics, but it is not a radius in the classical sense).

If this estimate was right, we would calculate that classical EM is value at \(r \gg r_e \sim 10^{-13} \text{cm}\).  In reality, classical electrodynamics breaks down at much larger distances.

NOTE: LHC is probing \(\sim 10^{-16} \text{cm}\).

Our strategy here is to focus on the structure that can be observed.  We do not have a way to probe to the small scale distances where the structure of the electron is relevant, so our description avoids that small range.

FIXME: I can not honestly say that I grasp the logic used to drop this self energy term.  This was compared to the concept of mass renormalization from Quantum field theory, where if I recall correctly, certain infinities were avoided by carefully avoiding points of singularity where there was nothing observable.  This is definitely something to revisit.  If this shows up even in classical electrodynamics, it is going to be even harder to understand later with the complexity of Quantum field theory tossed into the mix.
%
\section{Moving on to the next order in \texorpdfstring{\((v/c)\).}{v over c}}
%
Recall that we dropped terms from the original Lagrangian, which was
%
\begin{equation}\label{eqn:relativisticElectrodynamicsL25:90}
\LL_a = - m_a c^2 \sqrt{ 1 - \frac{\Bv_a^2}{c^2}} - q_a \phi(\Bx_a, t) + q_a \frac{\Bv_a}{c} \cdot \BA(\Bx_a, t).
\end{equation}
%
We expanded the square root previously keeping only the first order term in \((v/c)^2\).  Now we will do one more.  Recall that our fractional binomial series expansion is
%
\begin{equation}\label{eqn:relativisticElectrodynamicsL25:91}
(1 + x)^n = 1
+ \frac{n}{1!} x +
+ \frac{n(n-1)}{2!} x^2 +
+ \frac{n(n-1)(n-2)}{3!} x^3 + \cdots.
\end{equation}
%
so the square root in the Lagrangian expands as
%
\begin{equation}\label{eqn:relativisticElectrodynamicsL25:490}
\begin{aligned}
- m_a &c^2 \sqrt{ 1 - \frac{\Bv_a^2}{c^2}} \\
&=
- m_a c^2 \left(1
+ \frac{1}{2^1 1!}
\left(- \frac{\Bv_a^2}{c^2}  \right)
+ \frac{1(-1)}{2^2 2!}
\left(- \frac{\Bv_a^2}{c^2}  \right)^2
+ \frac{1(-1)(-3)}{2^3 3!}
\left(- \frac{\Bv_a^2}{c^2}  \right)^3 + \cdots
\right) \\
&=
- m_a c^2 + m_a \frac{\Bv_a^2}{2} + m_a \frac{\Bv_a^4}{8 c^2} + \cdots.
\end{aligned}
\end{equation}
%
Thus to the next order the single particle Lagrangian is
%
\begin{equation}\label{eqn:relativisticElectrodynamicsL25:110}
\LL_a = \inv{2} m_a \Bv_a^2 + \frac{m_a}{8} \frac{\Bv_a^4}{c^2} - q_a \phi(\Bx_a, t) + q_a \frac{\Bv_a}{c} \cdot \BA(\Bx_a, t).
\end{equation}
%
\paragraph{Goal:} Calculate \(\phi(\Bx_a)\), \(\BA(\Bx_a)\) due to all other particles in a \(\Bv/c\) expansion.
%
We write
%
\begin{equation}\label{eqn:relativisticElectrodynamicsL25:130}
\phi(\Bx_a, t) =
\phi^{(0)}(\Bx_a, t)
+\phi^{(1)}(\Bx_a, t)
+\phi^{(2)}(\Bx_a, t).
\end{equation}
%
Last time we found that the zeroth order term in this approximation was
%
\begin{equation}\label{eqn:relativisticElectrodynamicsL25:150}
\phi^{(0)}(\Bx_a, t) = \sum_{b \ne a} \frac{q_b}{\Abs{\Bx_a(t) - \Bx_b(t)}},
\end{equation}
%
and we wish to calculate the next term in the expansion.

We also want to a first order approximation of the vector potential
%
\begin{equation}\label{eqn:relativisticElectrodynamicsL25:170}
\BA(\Bx_a, t) =
\cancel{\BA^{(0)}(\Bx_a, t)}
+\BA^{(1)}(\Bx_a, t)
+\cancel{\BA^{(2)}(\Bx_a, t)}.
\end{equation}
%
There is no zero order term and we do not need the second order term (today).

Because
%
\begin{equation}\label{eqn:relativisticElectrodynamicsL25:190}
\square \BA \sim \frac{\rho \Bv}{c}.
\end{equation}
%
We know the charge and current distributions
%
\begin{equation}\label{eqn:relativisticElectrodynamicsL25:210}
\phi(\Bx, t) = \int d^3 \Bx \frac{\rho\left(\Bx', t - \Abs{\Bx - \Bx'}/c\right)}{\Abs{\Bx - \Bx'}}.
\end{equation}
%
\begin{equation}\label{eqn:relativisticElectrodynamicsL25:230}
\begin{aligned}
\rho(\Bx, t) &= \sum_b q_b \delta^3 (\Bx - \Bx_b(t)) \\
\Bj(\Bx, t) &= \sum_b q_b \Bv_b(t) \delta^3 (\Bx - \Bx_b(t)).
\end{aligned}
\end{equation}
%
We will use the fact that particles have \(v \ll c\).  The typical time where the charge distribution will change significantly is of order \(\frac{r_{ab}}{v} \gg \frac{r_{ab}}{c}\).  (Here \(r_ab/c\) is the time that it takes light to cross the interval, whereas \(r_ab/v\) is the time that it takes the particle to do the same).

In other words, in time \(\Abs{\Bx - \Bx'}/c \sim r_{ab}/c\), \(\rho\) will not change much.
%
\begin{equation}\label{eqn:relativisticElectrodynamicsL25:250}
\rho\left(\Bx', t - \Abs{\Bx - \Bx'}/c\right) \approx \rho(\Bx', t)
- \frac{\Abs{\Bx - \Bx'}}{c} \PD{t}{} \rho(\Bx', t) + \inv{2} \left(\frac{\Abs{\Bx - \Bx'}}{c}\right)^2 \PDSq{t}{} \rho(\Bx', t).
\end{equation}
%
\begin{equation}\label{eqn:relativisticElectrodynamicsL25:270}
\begin{aligned}
\phi(\Bx, t)
&= \int d^3 \Bx' \frac{\rho(\Bx', t)}{\Abs{\Bx - \Bx'}} - \PD{t}{} \int d^3 \Bx' \inv{c} \rho(\Bx', t) \\
&\qquad +
\inv{2 c^2} \int d^3 \Bx \Abs{\Bx - \Bx'} \PDSq{t}{} \rho(\Bx', t).
\end{aligned}
\end{equation}
%
The second integral is the total charge \(\times 1/c\), and does not change in time.  So to first order our charge density is
%
\begin{equation}\label{eqn:relativisticElectrodynamicsL25:290}
\rho\left(\Bx', t - \Abs{\Bx - \Bx'}/c\right) \approx \rho(\Bx', t) = \sum_b \frac{q_b}{\Abs{\Bx - \Bx_b(t)}}.
\end{equation}
%
%FIXME: do we want to second order here?

How about \(\BA\)?
%
\begin{equation}\label{eqn:relativisticElectrodynamicsL25:310}
\BA(\Bx_a, t) =
\cancel{\BA^{(0)}(\Bx_a, t)}
+\BA^{(1)}(\Bx_a, t)
+\cancel{\BA^{(2)}(\Bx_a, t)}.
\end{equation}
%
\begin{equation}\label{eqn:relativisticElectrodynamicsL25:510}
\begin{aligned}
A^{(1)}
&= \inv{c} \int d^3 \Bx' \inv{\Abs{\Bx - \Bx'}} \Bj\left(\Bx', t - \Abs{\Bx - \Bx'}/c\right)  \\
&\approx \inv{c} \int d^3 \Bx' \inv{\Abs{\Bx - \Bx'}} \Bj(\Bx', t).
\end{aligned}
\end{equation}
%
Ah, this shows why it was written that there is no second order term.  Because \(\Bj \sim \Bv_a\), we necessarily have \(\Bv_a/c\) dependence even in the zeroth order expansion about \(t=0\) in our retarded time expansion of \(\BA(\Bx', t_r)\).

Assembling all the results, we have
%
\begin{equation}\label{eqn:relativisticElectrodynamicsL25:330}
\LL_a = \inv{2} m_a \Bv_a^2 + \frac{m_a}{8} \frac{\Bv_a^4}{c^2} - q_a \phi^{(0)}(\Bx_a, t) - q_a \phi^{(2)}(\Bx_a, t) + q_a \frac{\Bv_a}{c} \cdot \BA^{(1)}(\Bx_a, t).
\end{equation}
%
\begin{equation}\label{eqn:relativisticElectrodynamicsL25:530}
\begin{aligned}
\phi^{(2)}(\Bx, t)
&= \PD{t}{} \left( \inv{2 c^2} \PD{t}{} \int d^3 \Bx' \Abs{\Bx - \Bx'} \rho(\Bx', t) \right) \\
&= \PD{t}{} \left( \inv{2 c^2} \PD{t}{} \int d^3 \Bx' \Abs{\Bx - \Bx'} \sum_b q_b \delta^3(\Bx - \Bx_b(t)) \right) \\
&= \PD{t}{} \left( \inv{2 c^2} \PD{t}{} \sum_b q_b \Abs{\Bx - \Bx_b(t)} \right).
\end{aligned}
\end{equation}
%
And
%
\begin{equation}\label{eqn:relativisticElectrodynamicsL25:550}
\begin{aligned}
\BA^{(1)}(\Bx, t)
&= \inv{c} \int d^3 \Bx' \inv{\Abs{\Bx - \Bx'}} \Bj(\Bx, t) \\
&= \inv{c} \int d^3 \Bx' \inv{\Abs{\Bx - \Bx'}} \sum_b q_b \Bv_b \delta^3(\Bx -\Bx_b) \\
&= \inv{c} \sum_b q_b \Bv_b \inv{\Abs{\Bx -\Bx_b}}.
\end{aligned}
\end{equation}
%
Recall that \(\phi^{(0)}\) was given by \eqnref{eqn:relativisticElectrodynamicsL25:150}.
%
\section{A gauge transformation to simplify things.}
\index{gauge transformation}
%
\paragraph{Remember:} Gauge transformation
\index{gauge transformation}
%
\begin{equation}\label{eqn:relativisticElectrodynamicsL25:350}
\begin{aligned}
\phi'(\Bx, t) &= \phi(\Bx, t) - \inv{c} \PD{t}{f(\Bx, t)} \\
\BA'(\Bx, t) &= \BA(\Bx, t) + \spacegrad f(\Bx, t).
\end{aligned}
\end{equation}
%
This will not change the physics.  Take
%
\begin{equation}\label{eqn:relativisticElectrodynamicsL25:370}
f(\Bx, t) = \sum_b \frac{q_b}{2 c} \PD{t}{} \Abs{\Bx - \Bx_b(t)}.
\end{equation}
%
Then
%
\begin{equation}\label{eqn:relativisticElectrodynamicsL25:390}
{\phi'}^{(2)} = 0.
\end{equation}
%
\begin{equation}\label{eqn:relativisticElectrodynamicsL25:410}
{\BA'}^{(1)}(\Bx, t) = \inv{c} \sum_b \frac{q_b \Bv_b}{\Abs{\Bx - \Bx_b}} + \spacegrad \sum_b \frac{q_b}{2 c} \PD{t}{} \Abs{\Bx - \Bx_b}.
\end{equation}
%
Inverting the order of time and space derivatives we find
%
\begin{equation}\label{eqn:relativisticElectrodynamicsL25:570}
\begin{aligned}
\spacegrad \PD{t}{} \Abs{\Bx - \Bx_b(t)}
&=
\PD{t}{} \spacegrad \Abs{\Bx - \Bx_b(t)} \\
&=
\PD{t}{} e_\alpha \partial_\alpha ((x^\beta - x_b^\beta(t))^2)^{1/2}
 \\
&=
\PD{t}{} e_\alpha
\frac{(x^\beta - x_b^\beta(t)) \partial_\alpha (x^\beta - x_b^\beta(t))}{\Abs{\Bx -\Bx_b(t)}}
 \\
&=
\PD{t}{} e_\alpha
\frac{(x^\beta - x_b^\beta(t)) \delta_\alpha^\beta}{\Abs{\Bx -\Bx_b(t)}}
 \\
&=
\PD{t}{}
\frac{\Bx - \Bx_b(t)}{\Abs{\Bx -\Bx_b(t)}}.
\end{aligned}
\end{equation}
%
Let us write
%
\begin{equation}\label{eqn:relativisticElectrodynamicsL25:430}
\Bn \equiv \frac{\Bx - \Bx_b(t)}{\Abs{\Bx - \Bx_b}},
\end{equation}
%
for the unit vector in the direction pointing from \(\Bx_b\) to \(\Bx\).  Evaluating the time derivative, we have
%
\begin{equation}\label{eqn:relativisticElectrodynamicsL25:590}
\begin{aligned}
\dot{\Bn}
&=
\frac{- \Bv_b(t)}{\Abs{\Bx -\Bx_b(t)}}
+(\Bx - \Bx_b(t)) \PD{t}{} \inv{\Abs{\Bx -\Bx_b(t)}} \\
&=
\frac{- \Bv_b(t)}{\Abs{\Bx -\Bx_b(t)}}
+(\Bx - \Bx_b(t))
\left(
-\frac{1}{\cancel{2}}
\right) \frac{ \cancel{2} (x^\alpha - x^\alpha_b(t) (-v^\alpha_b(t))}{\Abs{\Bx - \Bx_b(t)}^3} \\
&=
\frac{- \Bv_b(t)}{\Abs{\Bx -\Bx_b(t)}}
+ \frac{\Bn (\Bn \cdot \Bv_b)}{\Abs{\Bx - \Bx_b(t)}}.
\end{aligned}
\end{equation}
%
Assembling all the results we have
%
\begin{equation}\label{eqn:relativisticElectrodynamicsL25:450}
{\BA'}^{(1)}(\Bx, t) = \sum_b q_b \frac{\Bv_b + \Bn (\Bn \cdot \Bv_b)}{2 c \Abs{\Bx - \Bx_b} },
\end{equation}
%
and the Lagrangian for our particle after the gauge transformation is
%
\begin{equation}\label{eqn:relativisticElectrodynamicsL25:330b}
\LL_a = \inv{2} m_a \Bv_a^2 + \frac{m_a}{8} \frac{\Bv_a^4}{c^2}
-\sum_{b \ne a} \frac{q_a q_b}{\Abs{\Bx_a(t) - \Bx_b(t)}}
+\sum_b q_a q_b \frac{\Bv_a \cdot \Bv_b + (\Bn \cdot \Bv_a) (\Bn \cdot \Bv_b)}{2 c^2 \Abs{\Bx - \Bx_b} }.
\end{equation}
%
Next time we will probably get to the Lagrangian for the entire system.  It was hinted that this is called the Darwin Lagrangian (after Charles Darwin's grandson).
