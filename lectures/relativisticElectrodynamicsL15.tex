%
% Copyright � 2012 Peeter Joot.  All Rights Reserved.
% Licenced as described in the file LICENSE under the root directory of this GIT repository.
%

%\chapter{Fourier solution of Maxwell's vacuum wave equation in the Coulomb gauge}
\index{wave equation!Coulomb gauge}
\label{chap:relativisticElectrodynamicsL15}
%\blogpage{http://sites.google.com/site/peeterjoot/math2011/relativisticElectrodynamicsL15.pdf}
%\date{Mar 1, 2011}

\paragraph{Reading}

Covering chapter 6 material from the text \citep{landau1980classical}, and
\popcite{RelEMpp114-127.pdf}{lecture notes RelEMpp114-127.pdf}.
%: reminder on wave equations (115); reminder on Fourier series and integral (115-117); Fourier expansion of the EM potential in Coulomb gauge and equation of motion for the spatial Fourier components (118-119); the general solution of Maxwell's equations in vacuum (120-121) [Tuesday, Mar. 1]

\section{Review of wave equation results obtained}
\index{wave equation}

Maxwell's equations in vacuum lead to Coulomb gauge and the Lorentz gauge.

\paragraph{Coulomb gauge}
\index{Coulomb gauge}
%
\begin{equation}\label{eqn:relativisticElectrodynamicsL15:10}
\begin{aligned}
A^0 &= 0 \\
\spacegrad \cdot \BA &= 0 \\
\left( \inv{c^2} \PDSq{t}{} - \Delta \right) \BA &= 0
\end{aligned}
\end{equation}

\paragraph{Lorentz gauge}
\index{Lorentz gauge}
%
\begin{equation}\label{eqn:relativisticElectrodynamicsL15:30}
\begin{aligned}
\partial_i A^i &= 0 \\
\left( \inv{c^2} \PDSq{t}{} - \Delta \right) A^i &= 0
\end{aligned}
\end{equation}

Note that \(\partial_i A^i = 0\) is invariant under gauge transformations
%
\begin{equation}\label{eqn:relativisticElectrodynamicsL15:50}
A^i \rightarrow A^i + \partial^i \chi
\end{equation}

where
%
\begin{equation}\label{eqn:relativisticElectrodynamicsL15:70}
\partial_i \partial^i \chi = 0,
\end{equation}

So if one uses the Lorentz gauge, this has to be fixed.

However, in both cases we have
%
\begin{equation}\label{eqn:relativisticElectrodynamicsL15:90}
\left( \inv{c^2} \PDSq{t}{} - \Delta \right) f = 0
\end{equation}

where
%
\begin{equation}\label{eqn:relativisticElectrodynamicsL15:110}
\inv{c^2} \PDSq{t}{} - \Delta
\end{equation}

is the wave operator.

Consider
%
\begin{equation}\label{eqn:relativisticElectrodynamicsL15:130}
\Delta = \PDSq{x}{}
\end{equation}

where we are looking for a solution that is independent of \(y, z\).  Recall that the general solution for this equation has the form
%
\begin{equation}\label{eqn:relativisticElectrodynamicsL15:150}
f(t, x) =
F_1 \left(t - \frac{x}{c}\right)
+F_2 \left(t + \frac{x}{c}\right)
\end{equation}

PICTURE: superposition of two waves with \(F_1\) moving along the x-axis in the positive direction, and \(F_2\) in the negative x direction.

It is notable that the text derives \eqnref{eqn:relativisticElectrodynamicsL15:150} in a particularly slick way.  It is still black magic, since one has to know the solution to find it, but very very cool.

\section{Review of Fourier methods}

It is often convenient to impose periodic boundary conditions
%
\begin{equation}\label{eqn:relativisticElectrodynamicsL15:170}
\BA(\Bx + \Be_i L) = \BA(\Bx), i = 1,2,3
\end{equation}

\paragraph{In one dimension}
%
\begin{equation}\label{eqn:relativisticElectrodynamicsL15:190}
f(x + L) = f(x)
\end{equation}
%
\begin{equation}\label{eqn:relativisticElectrodynamicsL15:210}
f(x) = \sum_{n=-\infty}^\infty e^{i \frac{2 \pi n}{L} x} \tilde{f}_n
\end{equation}

When \(f(x)\) is real we also have
%
\begin{equation}\label{eqn:relativisticElectrodynamicsL15:230}
f^\conj(x) = \sum_{n = -\infty}^\infty e^{-i \frac{2 \pi n}{L} x} (\tilde{f}_n)^\conj
\end{equation}

which implies
%
\begin{equation}\label{eqn:relativisticElectrodynamicsL15:250}
{\tilde{f}^\conj}_{n} = \tilde{f}_{-n}.
\end{equation}

We introduce a wave number
%
\begin{equation}\label{eqn:relativisticElectrodynamicsL15:270}
k_n = \frac{2 \pi n}{L},
\end{equation}

allowing a slightly simpler expression of the Fourier decomposition
%
\begin{equation}\label{eqn:relativisticElectrodynamicsL15:290}
f(x) = \sum_{n=-\infty}^\infty e^{i k_n x} \tilde{f}_{k_n}.
\end{equation}

The inverse transform is obtained by integration over some length \(L\) interval
%
\begin{equation}\label{eqn:relativisticElectrodynamicsL15:310}
\tilde{f}_{k_n} = \inv{L} \int_{-L/2}^{L/2} dx e^{-i k_n x} f(x)
\end{equation}

\paragraph{Verify:}

We should be able to recover the Fourier coefficient by utilizing the above
%
\begin{equation}\label{eqn:relativisticElectrodynamicsL15:880}
\begin{aligned}
\inv{L} \int_{-L/2}^{L/2} dx e^{-i k_n x} \sum_{m=-\infty}^\infty e^{i k_m x} \tilde{f}_{k_m} \\
&= \sum_{m = -\infty}^\infty \tilde{f}_{k_m} \delta_{mn} = \tilde{f}_{k_n},
\end{aligned}
\end{equation}

where we use the easily verifiable fact that
%
\begin{equation}\label{eqn:relativisticElectrodynamicsL15:800}
\inv{L} \int_{-L/2}^{L/2} dx e^{i (k_m - k_n) x} =
\begin{array}{l l}
0 & \quad \mbox{if \(m \ne n\)} \\
1 & \quad \mbox{if \(m = n\)} \\
\end{array}.
\end{equation}

It is conventional to absorb \(\tilde{f}_{k_n} = \tilde{f}(k_n)\) for
%
\begin{equation}\label{eqn:relativisticElectrodynamicsL15:330}
\begin{aligned}
f(x) &= \inv{L} \sum_n \tilde{f}(k_n) e^{i k_n x} \\
\tilde{f}(k_n) &= \int_{-L/2}^{L/2} dx f(x) e^{-i k_n x}
\end{aligned}
\end{equation}

To take \(L \rightarrow \infty\) notice
%
\begin{equation}\label{eqn:relativisticElectrodynamicsL15:350}
k_n = \frac{2 \pi}{L} n
\end{equation}

when \(n\) changes by \(\Delta n = 1\), \(k_n\) changes by \(\Delta k_n = \frac{2 \pi}{L} \Delta n\)

Using this
%
\begin{equation}\label{eqn:relativisticElectrodynamicsL15:370}
f(x) = \inv{2\pi} \sum_n \left( \frac{2\pi}{L} \Delta n \right) \tilde{f}(k_n) e^{i k_n x}
\end{equation}

With \(L \rightarrow \infty\), and \(\Delta k_n \rightarrow 0\)
%
\begin{equation}\label{eqn:relativisticElectrodynamicsL15:390}
\begin{aligned}
f(x) &= \int_{-\infty}^\infty \frac{dk}{2\pi} \tilde{f}(k) e^{i k x} \\
\tilde{f}(k) &= \int_{-\infty}^\infty dx f(x) e^{-i k x}
\end{aligned}
\end{equation}

\paragraph{Verify:}

A loose verification of the inversion relationship (the most important bit) is possible by substitution
%
\begin{equation}\label{eqn:relativisticElectrodynamicsL15:900}
\begin{aligned}
\int \frac{dk}{2\pi} e^{i k x} \tilde{f}(k)
&=
\iint \frac{dk}{2\pi} e^{i k x} dx' f(x') e^{-i k x'} \\
&=
\int dx' f(x') \inv{2\pi} \int dk e^{i k (x - x')}
\end{aligned}
\end{equation}

Now we employ the old physics ploy where we identify
%
\begin{equation}\label{eqn:relativisticElectrodynamicsL15:820}
\inv{2\pi} \int dk e^{i k (x - x')} = \delta(x - x').
\end{equation}

With that we see that we recover the function \(f(x)\) above as desired.

\paragraph{In three dimensions}
%
\begin{equation}\label{eqn:relativisticElectrodynamicsL15:470}
\begin{aligned}
\BA(\Bx, t) &= \int
\frac{d^3 \Bk}{(2\pi)^3}
\tilde{\BA}(\Bk, t) e^{i \Bk \cdot \Bx} \\
\tilde{\BA}(\Bx, t) &= \int d^3 \Bx \BA(\Bx, t) e^{-i \Bk \cdot \Bx}
\end{aligned}
\end{equation}

\paragraph{Application to the wave equation}
\index{wave equation}
%
\begin{equation}\label{eqn:relativisticElectrodynamicsL15:920}
\begin{aligned}
0 &=
\left( \inv{c^2} \PDSq{t}{} - \Delta \right) \BA(\Bx, t) \\
&=
\left( \inv{c^2} \PDSq{t}{} - \Delta \right)
\int
\frac{d^3 \Bk}{(2\pi)^3}
\tilde{\BA}(\Bk, t) e^{i \Bk \cdot \Bx} \\
&=
\int
\frac{d^3 \Bk}{(2\pi)^3}
\left(
\inv{c^2} \partial_{tt} \tilde{\BA}(\Bk, t) + \Bk^2 \BA(\Bk, t)
\right)
e^{i \Bk \cdot \Bx}
\end{aligned}
\end{equation}

Now operate with \(\int d^3 \Bx e^{-i \Bp \cdot \Bx }\)
%
\begin{equation}\label{eqn:relativisticElectrodynamicsL15:940}
\begin{aligned}
0 &=
\int d^3 \Bx e^{-i \Bp \cdot \Bx }
\int
\frac{d^3 \Bk}{(2\pi)^3}
\left(
\inv{c^2} \partial_{tt} \tilde{\BA}(\Bk, t) + \Bk^2 \BA(\Bk, t)
\right)
e^{i \Bk \cdot \Bx}  \\
&=
\int
d^3 \Bk
\delta^3(\Bp -\Bk)
\left(
\inv{c^2} \partial_{tt} \tilde{\BA}(\Bk, t) + \Bk^2 \BA(\Bk, t)
\right)
\end{aligned}
\end{equation}

Since this is true for all \(\Bp\) we have
%
\begin{equation}\label{eqn:relativisticElectrodynamicsL15:490}
\partial_{tt} \tilde{\BA}(\Bp, t) = -c^2 \Bp^2 \tilde{\BA}(\Bp, t)
\end{equation}

For every value of momentum we have a harmonic oscillator!
%
\begin{equation}\label{eqn:relativisticElectrodynamicsL15:510}
\ddot{x} = -\omega^2 x
\end{equation}

Fourier modes of EM potential in vacuum obey
%
\begin{equation}\label{eqn:relativisticElectrodynamicsL15:530}
\partial_{tt} \tilde{\BA}(\Bk, t) = -c^2 \Bk^2 \tilde{\BA}(\Bk, t)
\end{equation}

Because we are operating in the Coulomb gauge we must also have zero divergence.  Let us see how that translates to our Fourier representation


implies
%
\begin{equation}\label{eqn:relativisticElectrodynamicsL15:960}
\begin{aligned}
0 &= \spacegrad \cdot \BA(\Bx, t) \\
&= \int \frac{d^3 \Bk }{(2 \pi)^3} \spacegrad \cdot \left( e^{i \Bk \cdot \Bx} \cdot \tilde{\BA}(\Bk, t) \right)
\end{aligned}
\end{equation}

The chain rule for the divergence in this case takes the form
%
\begin{equation}\label{eqn:relativisticElectrodynamicsL15:840}
\spacegrad \cdot (\phi \BB) = (\spacegrad \phi) \cdot \BB + \phi \spacegrad \cdot \BB.
\end{equation}
Since our vector function \(\tilde{\BA}\) is not a function of spatial coordinates we have
\begin{equation}\label{eqn:relativisticElectrodynamicsL15:570}
0 = \int \frac{d^3 \Bk }{(2 \pi)^3} e^{i \Bk \cdot \Bx} (i \Bk \cdot \tilde{\BA}(\Bk, t)).
\end{equation}
This has two immediate consequences.  The first is that our momentum space potential is perpendicular to the wave number vector at all points in momentum space, and the second gives us a conjugate relation (substitute \(\Bk \rightarrow -\Bk'\) after taking conjugates for that one)
%
\begin{equation}\label{eqn:relativisticElectrodynamicsL15:590}
\begin{aligned}
\Bk \cdot \tilde{\BA}(\Bk, t) &= 0 \\
\tilde{\BA}(-\Bk, t) &= \tilde{\BA}^\conj(\Bk, t).
\end{aligned}
\end{equation}
%
\begin{equation}\label{eqn:relativisticElectrodynamicsL15:610}
\BA(\Bx, t) = \int
\frac{d^3 \Bk}{(2\pi)^3}
e^{i \Bk \cdot \Bx} \left( \inv{2} \tilde{\BA}(\Bk, t) + \inv{2} \tilde{\BA}^\conj(- \Bk, t) \right)
\end{equation}

Since out system is essentially a harmonic oscillator at each point in momentum space
%
\begin{equation}\label{eqn:relativisticElectrodynamicsL15:630}
\begin{aligned}
\partial_{tt} \tilde{\BA}(\Bk, t) &= - \omega_k^2 \tilde{\BA}(\Bk, t) \\
\omega_k^2 &= c^2 \Bk^2
\end{aligned}
\end{equation}

our general solution is of the form
%
\begin{equation}\label{eqn:relativisticElectrodynamicsL15:650}
\begin{aligned}
\tilde{\BA}(\Bk, t) &= e^{i \omega_k t} \Ba_{+}(\Bk) +e^{-i \omega_k t} \Ba_{-}(\Bk) \\
\tilde{\BA}^\conj(\Bk, t) &= e^{-i \omega_k t} \Ba_{+}^\conj(\Bk) +e^{i \omega_k t} \Ba_{-}^\conj(\Bk)
\end{aligned}
\end{equation}
%
\begin{equation}\label{eqn:relativisticElectrodynamicsL15:670}
\BA(\Bx, t)
= \int \frac{d^3 \Bk}{(2 \pi)^3} e^{i \Bk \cdot \Bx}
\inv{2} \left(
e^{i \omega_k t} (\Ba_{+}(\Bk) + \Ba_{-}^\conj(-\Bk))
+e^{-i \omega_k t} (\Ba_{-}(\Bk) + \Ba_{+}^\conj(-\Bk))
\right)
\end{equation}

Define
%
\begin{equation}\label{eqn:relativisticElectrodynamicsL15:690}
\Bbeta(\Bk) \equiv \inv{2} (\Ba_{-}(\Bk) + \Ba_{+}^\conj(-\Bk) )
\end{equation}

so that
%
\begin{equation}\label{eqn:relativisticElectrodynamicsL15:710}
\Bbeta(-\Bk) = \inv{2} (\Ba_{+}^\conj(\Bk) + \Ba_{-}(-\Bk))
\end{equation}

Our solution now takes the form
%
\begin{equation}\label{eqn:relativisticElectrodynamicsL15:730}
\BA(\Bx, t) = \int \frac{d^3\Bk}{(2 \pi)^3} \left(
e^{i (\Bk \cdot \Bx + \omega_k t)} \Bbeta^\conj(-\Bk)
+e^{i (\Bk \cdot \Bx - \omega_k t)} \Bbeta(\Bk)
\right)
\end{equation}

\paragraph{Claim:}

This is now manifestly real.  To see this, consider the first term with \(\Bk = -\Bk'\), noting that \(\int_{-\infty}^\infty dk = \int_{\infty}^\infty -dk' = \int_{-\infty}^\infty dk' \) with \(dk = -dk'\)
%
\begin{equation}\label{eqn:relativisticElectrodynamicsL15:860}
\int \frac{d^3\Bk'}{(2 \pi)^3} e^{i (-\Bk' \cdot \Bx + \omega_k t)} \Bbeta^\conj(\Bk')
\end{equation}

Dropping primes this is the conjugate of the second term.

\paragraph{Claim:}

We have \(\Bk \cdot \Bbeta(\Bk)  = 0\).

Since we have \(\Bk \cdot \tilde{\BA}(\Bk, t) = 0\), \eqnref{eqn:relativisticElectrodynamicsL15:650} implies that we have \(\Bk \cdot \Ba_{\pm}(\Bk) = 0\).  With each of these vector integration constants being perpendicular to \(\Bk\) at that point in momentum space, so must be the linear combination of these constants \(\Bbeta(\Bk)\).
