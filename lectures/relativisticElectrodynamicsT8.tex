%
% Copyright � 2012 Peeter Joot.  All Rights Reserved.
% Licenced as described in the file LICENSE under the root directory of this GIT repository.
%

\label{chap:relativisticElectrodynamicsT8}
%\blogpage{http://sites.google.com/site/peeterjoot/math2011/relativisticElectrodynamicsT8.pdf}
%\date{Mar 23, 2011}
%
\section{Review}
%
Recall for the electric dipole we started with a system like
%
\begin{equation}\label{eqn:relativisticElectrodynamicsT8:9}
\begin{aligned}
z_{+} &= 0 \\
z_{-} &= \Be_3( z_0 + a \sin(\omega t))
\end{aligned}
\end{equation}
%
(we did it with the opposite polarity)
%
\begin{equation}\label{eqn:relativisticElectrodynamicsT8:10}
\begin{aligned}
\BE &= \frac{q a \omega^2}{c^2} \sin\omega t_o \sin\theta \inv{\Abs{\Bx}} (- \thetacap ) = \inv{c^2 \Abs{\Bx}} (\ddot{\Bd}(t_r) \cross \rcap ) \cross \rcap  \\
\BB &= -\frac{q a \omega^2}{c^2} \sin\omega t_o \sin\theta \inv{\Abs{\Bx}} (- \phicap ) = \rcap \cross \BE.
\end{aligned}
\end{equation}
%
This was after the multipole expansion (\(\lambda \gg l\)).

Physical analogy: a high and low frequency wave interacting.  The low frequency wave becomes the envelope, and does not really ``see'' the dynamics of the high frequency wave.

We also figured out the Poynting vector was
%
\begin{equation}\label{eqn:relativisticElectrodynamicsT8:30}
\BS = \frac{c}{4\pi} \BE \cross \BB = \rcap \frac{ \sin^2 \theta \Abs{\ddot{\Bd}(t_r) }^2 }{ 4 \pi c^3 \Abs{\Bx}^2},
\end{equation}
%
and our Power was
%
\begin{equation}\label{eqn:relativisticElectrodynamicsT8:50}
\text{Power}(R) = \oint_{{S_R}^2} d^2 \Bsigma \cdot \expectation{\BS} = \frac{ q^2 a^2 \omega^4 }{3 c^3}.
\end{equation}
%
\section{Magnetic dipole}
\index{magnetic dipole}

PICTURE: positively oriented current \(I\) circulating around the normal \(\Bm\) at radius \(b\) in the x-y plane.  We have

(from third year)
%
\begin{equation}\label{eqn:relativisticElectrodynamicsT8:70}
\Abs{\Bm} = I \pi b^2.
\end{equation}
%
With the magnetic moment directed upwards along the z-axis
%
\begin{equation}\label{eqn:relativisticElectrodynamicsT8:71}
\Bm = I \pi b^2 \Be_3,
\end{equation}
%
where we have a frequency dependence in the current
%
\begin{equation}\label{eqn:relativisticElectrodynamicsT8:72}
I = I_o \sin(\omega t).
\end{equation}
%
With no static charge distribution we have zero scalar potential
%
\begin{equation}\label{eqn:relativisticElectrodynamicsT8:90}
\rho = 0 \implies A^0 = 0.
\end{equation}
%
Our first moments approximation of the vector potential was
\begin{equation}\label{eqn:relativisticElectrodynamicsT8:110}
A^\alpha(\Bx, t) \approx \inv{c \Abs{\Bx}} \int d^3 \Bx' j^\alpha(\Bx', t) + O(\text{higher moments}).
\end{equation}
%
Now we use our new trick introducing a \(1 = 1\) to rewrite the current
%
\begin{equation}\label{eqn:relativisticElectrodynamicsT8:130}
\left( \PD{{x'}^\beta}{{x'}^\alpha} \right) j^\beta = {\delta^\alpha}_\beta j^\beta = j^\alpha,
\end{equation}
%
or equivalently
%
\begin{equation}\label{eqn:relativisticElectrodynamicsT8:135}
\spacegrad x^\alpha = \Be_\alpha.
\end{equation}
%
Carrying out the trickery we have
%
\begin{equation}\label{eqn:relativisticElectrodynamicsT8:370}
\begin{aligned}
A^\alpha
&= \inv{c\Abs{\Bx}} \int d^3 \Bx' ( \spacegrad' {x'}^\alpha ) \cdot \BJ(\Bx', t_r) \\
&= \inv{c\Abs{\Bx}} \int d^3 \Bx' ( \partial_{\beta'} {x'}^\alpha ) j^\beta (\Bx', t_r) \\
&= \inv{c\Abs{\Bx}} \int d^3 \Bx' ( \partial_{\beta'} ({x'}^\alpha j^\beta(\Bx', t_r)) - {x'}^\alpha (\mathLabelBox{\spacegrad' \cdot \BJ(\Bx', t_r)}{\(=-\partial_0 \rho = 0\)}) ) \\
&= \inv{c\Abs{\Bx}} \int d^3 \Bx' \spacegrad' \cdot ({x'}^\alpha \BJ) \\
&= \oint_{{S_R}^2} d^2 \Bsigma \cdot ({x'}^\alpha \BJ) \\
&= 0.
\end{aligned}
\end{equation}
%
We see that the first order approximation is insufficient to calculate the vector potential for the magnetic dipole system, and that we have
%
\begin{equation}\label{eqn:relativisticElectrodynamicsT8:170}
A^\alpha = 0 + \text{higher moments}
\end{equation}
%
Looking back to what we would done in class, we would also dropped this term of the vector potential, using the same arguments.  What we had left was
%
\begin{equation}\label{eqn:relativisticElectrodynamicsT8:171}
\BA(\Bx, t) = \inv{c \Abs{\Bx} } \dot{\Bd}\left(t - \frac{\Abs{\Bx} }{c}\right)
= \inv{ c \Abs{\Bx} } \int d^3 \Bx' {x'}^\alpha \PD{t}{}\rho\left(\Bx', t - \frac{\Abs{\Bx} }{c}\right),
\end{equation}
%
but that additional term is also zero in this magnetic dipole system since we have no static charge distribution.

There are two options to resolve this

\begin{enumerate}
\item calculate \(\BA\) using higher order moments \(\lambda \gg b\).  Go to next order in \(b/\lambda\).

This is complicated!

\item Use EM dualities (the slick way!)
\end{enumerate}

Recall that Maxwell's equations are
%
\begin{equation}\label{eqn:relativisticElectrodynamicsT8:190}
\begin{aligned}
\spacegrad \cdot \BE &= 4 \pi \rho \\
\spacegrad \cdot \BB &= 0 \\
\spacegrad \cross \BE &= -\inv{c} \PD{t}{\BB} \\
\spacegrad \cross \BB &= \inv{c} \PD{t}{\BE} + 4 \pi \BJ
\end{aligned}
\end{equation}
%
If \(j^i = 0\), then taking \(\BE \rightarrow \BB\) and \(\BB \rightarrow \BE\) we get the same equations.

Introduce dual charges \(\rho_m\) and \(\BJ_m\)
%
\begin{equation}\label{eqn:relativisticElectrodynamicsT8:210}
\begin{aligned}
\spacegrad \cdot \BE &= 4 \pi \rho_e \\
\spacegrad \cdot \BB &= 4 \pi \rho_m \\
\spacegrad \cross \BE &= -\inv{c} \PD{t}{\BB} + 4 \pi \BJ_m\\
\spacegrad \cross \BB &= \inv{c} \PD{t}{\BE} + 4 \pi \BJ_e
\end{aligned}
\end{equation}
%
Duality \(\BE \rightarrow \BB\) provided \(\rho_e \rightarrow \rho_m\) and \(\BJ_e \rightarrow \BJ_m\), or
%
\begin{equation}\label{eqn:relativisticElectrodynamicsT8:230}
\begin{aligned}
F^{i j} &\rightarrow \tilde{F}^{i j} = \epsilon^{i j k l} F_{k l} \\
j^{k} &\rightarrow \tilde{j}^k
\end{aligned}
\end{equation}
%
With radiation : the duality transformation takes the electric dipole moment to the magnetic dipole moment \(\Bd \rightarrow \Bm\).
%
\begin{equation}\label{eqn:relativisticElectrodynamicsT8:250}
\begin{aligned}
\BB &= -\inv{c^2 \Abs{\Bx}} (\ddot{\Bm} \cross \rcap) \cross \rcap \\
\BE &= \rcap \cross \BB
\end{aligned}
\end{equation}
%
with
%
\begin{equation}\label{eqn:relativisticElectrodynamicsT8:270}
\text{Power} \sim \expectation{\Abs{\ddot{\Bm}^2}}
\end{equation}
%
\begin{equation}\label{eqn:relativisticElectrodynamicsT8:290}
\expectation{\Abs{\ddot{\Bm}^2}}
= \inv{2} (I_o \pi b^2 \omega^2)^2
\end{equation}
%
where
%
\begin{equation}\label{eqn:relativisticElectrodynamicsT8:310}
I_o = \dot{q} = \omega q
\end{equation}
%
So the power of the magnetic dipole is
%
\begin{equation}\label{eqn:relativisticElectrodynamicsT8:330}
P_m(R) = \frac{b^4 q^2 \pi^2 \omega^6}{3 c^5}
\end{equation}
%
Taking ratios of the magnetic and electric power we find
%
\begin{equation}\label{eqn:relativisticElectrodynamicsT8:390}
\begin{aligned}
\frac{P_m}{E_m}
&= \frac{b^4 q^2 \pi^2 \omega^6}{b^2 q^2 \omega^4 c^2}  \\
&\sim \frac{b^2 \omega^2}{c^2} \\
&= \left(\frac{b \omega}{c}\right)^2 \\
&= \left(\frac{b }{\lambda}\right)^2
\end{aligned}
\end{equation}
%
This difference in power shows the second order moment dependence, in the \(\lambda \gg b\) approximations.

FIXME: go back and review the ``third year'' content and see where the magnetic dipole moment came from.  That is the key to this argument, since we need to see how this ends up equivalent to a pair of charges in the electric field case.

