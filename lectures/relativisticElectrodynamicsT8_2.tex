%
% Copyright © 2012 Peeter Joot.  All Rights Reserved.
% Licenced as described in the file LICENSE under the root directory of this GIT repository.
%
%
%\section{Midterm solution discussion}
%
In the last part of the tutorial, the bonus question from the tutorial was covered.  This was to determine the Yukawa potential from the differential equation that we found in the earlier part of the problem.

I took a couple notes about this on paper, but do not intend to write them up.  Everything proceeded exactly as I would have expected them to for solving the problem (I barely finished the midterm as is, so I did not have a chance to try it).  Take Fourier transforms and then evaluate the inverse Fourier integral.  This is exactly what we can do for the Coulomb potential, but actually easier since we do not have to introduce anything to offset the poles (and we recover the Coulomb potential in the \(M \rightarrow 0\) case).

There was one notable point in this Yukawa potential derivation, which was not obvious to me immediately
%
\begin{equation}\label{eqn:relativisticElectrodynamicsT8:350}
\tilde{\rho}(\Bk) = \int d^3 \Bx e^{-i \Bk \cdot \Bx} \rho(\Bx) = 1.
\end{equation}
%
However, the Fourier transform equal to unity followed straight from the definition of the potential, which was a delta function
%
\begin{equation}\label{eqn:relativisticElectrodynamicsT8:340}
\rho(x) = \int ds \delta^4(x - x(\tau)).
\end{equation}
