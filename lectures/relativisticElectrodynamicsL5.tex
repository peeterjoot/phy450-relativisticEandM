%
% Copyright � 2012 Peeter Joot.  All Rights Reserved.
% Licenced as described in the file LICENSE under the root directory of this GIT repository.
%

%\chapter{Proper time, length contraction, time dialation, causality}
\index{proper time}
\index{length contraction}
\index{time dilation}
\index{causality}
\label{chap:relativisticElectrodynamicsL5}
%\blogpage{http://sites.google.com/site/peeterjoot/math2011/relativisticElectrodynamicsL5.pdf}
%\date{Jan 19, 2011}
%
\paragraph{Reading}
%
Covering chapter 1 material from the text \citep{landau1980classical}, and
\popcite{RelEM27-44.pdf}{lecture notes RelEM27-44.pdf}.
%: Using Minkowski diagram to see the perils of superluminal propagation (32.3); nonrelativistic limit of boosts (33); number of parameters of Lorentz transformations (34-35); introducing four-vectors, the metric tensor, the invariant ``dot-product and SO(1,3) (36-40); the Poincare group (41); the convenience of ``upper'' and ``lower''indices (42-43); tensors (44)
%
\section{More on proper time.}
\index{proper time}

PICTURE:1: worldline with small interval.

Considering a small interval somewhere on the worldline trajectory, we have
%
\begin{equation}\label{eqn:relativisticElectrodynamicsL5:10}
ds^2 = c^2 dt^2 - dx^2 = c^2 {dt'}^2,
\end{equation}
%
where \(dt'\) is the proper time elapsed in a frame moving with velocity \(v\), and \(dt\) is the time elapsed in a stationary frame.

We have
%
\begin{equation}\label{eqn:relativisticElectrodynamicsL5:20}
dt' = dt \sqrt{ 1 - (dx/dt)^2/c^2 } = dt \sqrt{ 1 - v^2/c^2 }.
\end{equation}
%
PICTURE:2: particle at rest.

For the particle at rest
%
\begin{equation}\label{eqn:relativisticElectrodynamicsL5:30}
c \tau_{21}^{\text{stationary}} = c ( t_2 - t_1 ) = \int_1^2 ds = \int_1^2 c dt.
\end{equation}
%
PICTURE:3: particle with motion.

``length'' of 1-2 ``curved'' worldline
%
\begin{equation}\label{eqn:relativisticElectrodynamicsL5:90}
\begin{aligned}
\int_1^2 ds'
&= \int_1^2 c dt' \\
&= \int_1^2 c dt \sqrt{ 1 - (d\Bv/dt)^2 },
\end{aligned}
\end{equation}
%
where in this case \([1,2]\) denotes the range of a line integral over the worldline.  We see that the multiplier of dt for any point along the curve is smaller than \(1\), so that the length along a straight line is longest (i.e. for the particle at rest).

We have argued that if 1,2 occur at the same place, the spacetime length of a straight line between them is the longest.  This remains the time  for all 1,2 timelike separated.

LOTS OF DISCUSSION.  See \href{https://www.physics.utoronto.ca/~poppitz/poppitz/PHY450_files/pp.24.1-24.4.pdf}{new posted notes for details}.

Back to page 18 of the notes.

We have argued that \(ds_{12} = {ds'}_{12} \implies s_{12} = {s'}_{12}\) for infinitesimal 1,2 even if not infinitesimal.

The idea is to represent the interval between twill not close 1,2 as a sum over small \(ds\)'s.

P6: \(x = x_2 t /t_2\) straight line through origin, with \(t \in [0, t_2]\).

P7: zoomed on part of this line.
%
\begin{equation}\label{eqn:relativisticElectrodynamicsL5:110}
\begin{aligned}
ds^2
&= c^2 dt^2 - dx^2 \\
&= c^2 dt^2 - \left(\frac{x_2}{t_2}\right)^2 dt^2 \\
&= c^2 dt^2 \left( 1 - \inv{c^2} \left(\frac{x_2}{t_2}\right)^2 \right),
\end{aligned}
\end{equation}
or
\begin{equation}\label{eqn:relativisticElectrodynamicsL5:40}
\int_0^1 ds
= c \int_0^{t_2} dt \sqrt{ 1 - \inv{c^2} \left(\frac{x_2}{t_2}\right)^2 }.
\end{equation}
%
In another frame just replace \(t \rightarrow t'\) and \(x_2 \rightarrow x_2'\)
%
\begin{equation}\label{eqn:relativisticElectrodynamicsL5:50}
\int_0^1 ds
= c \int_0^{t_2'} dt \sqrt{1 - \inv{c^2} \left(\frac{x_2'}{t_2'}\right)^2 }.
\end{equation}
%
\section{Length contraction.}
\index{length contraction}
Consider \(O\) and \(O'\) with \(O'\) moving in \(x\) with speed \(v_x > 0\).  Here we have
%
\begin{equation}\label{eqn:relativisticElectrodynamicsL5:60}
\begin{aligned}
x' &= \gamma \left( x - \frac{v_x}{c} ct \right) \\
c t' &= \gamma \left( ct - \frac{v_x}{c} x \right).
\end{aligned}
\end{equation}
%
PICTURE: spacetime diagram with \(ct'\) at angle \(\alpha\), where \(\tan \alpha = v_x/c\).

Two points \((x_A,0)\), \((x_B,0)\), with rest length measured as \(L = x_B - x_A\).  From the diagram \(c(t_B - t_A) = \tan\alpha L\), and from \eqnref{eqn:relativisticElectrodynamicsL5:60} we have
%
\begin{equation}\label{eqn:relativisticElectrodynamicsL5:70}
\begin{aligned}
x_A' &= \gamma \left( x_A - \frac{v_x}{c} c t_A \right) \\
x_B' &= \gamma \left( x_B - \frac{v_x}{c} c t_B \right),
\end{aligned}
\end{equation}
so that
\begin{equation}\label{eqn:relativisticElectrodynamicsL5:130}
\begin{aligned}
L' &= x_B' - x_A' \\
&= \gamma \left( (x_B - x_A) - \frac{v_x}{c} c (t_B -t_A) \right) \\
&= \gamma \left( L - \frac{v_x}{c} \tan \alpha L \right) \\
&= \gamma \left( L - \frac{v_x^2}{c^2} L \right) \\
&= \gamma L \left( 1 - \frac{v_x^2}{c^2} \right) \\
&= L \sqrt{ 1 - \frac{v_x^2}{c^2} }.
\end{aligned}
\end{equation}
%
\section{Superluminal speed and causality.}
\index{superluminal}
\index{causality}

If Einstein's relativity holds, superluminal motion is a ``no-no''.  Imagine that some ``tachyons'' exist that can instantaneously transmit stuff between observers.

PICTURE9: two guys with resting worldlines showing.

Can send info back to \(A\) before \(A\) sends to \(B\).  Superluminal propagation allows sending information not yet available.  Can show this for finite superluminal velocities (but hard) as well as infinite velocity superluminal speeds.  We see that time ordering can not be changed for events separated by time like separation.  Events separated by spacelike separation cannot be ca usually connected.
