%
% Copyright © 2012 Peeter Joot.  All Rights Reserved.
% Licenced as described in the file LICENSE under the root directory of this GIT repository.
%
\section{Dynamics}
\index{dynamics}
\index{action!particle}
\index{relativistic dynamics}

In Newtonian dynamics we have
%
\begin{equation}\label{eqn:relativisticElectrodynamicsL6:250}
m \ddot{\Br} = \Bf
\end{equation}

An equation of motion should be expressed in terms of vectors.  This equation is written in a way that shows that the law of physics is independent of the choice of coordinates.  We can do this in the context of tensor algebra as well.  Ironically, this will require us to explicitly work with the coordinate representation, but this work will be augmented by the fact that we require our tensors to transform in specific ways.

In Newtonian mechanics we can look to symmetries and the invariance of the action with respect to those symmetries to express the equations of motion.  Our symmetries in Newtonian mechanics leave the action invariant with respect to spatial translation and with respect to rotation.

We want to express relativistic dynamics in a similar way, and will have to express the action as a Lorentz scalar.  We are going to impose the symmetries of the Poincare group to determine the relativistic laws of dynamics, and the next task will be to consider the possibilities for our relativistic action, and see what that action implies for dynamics in a relativistic context.
