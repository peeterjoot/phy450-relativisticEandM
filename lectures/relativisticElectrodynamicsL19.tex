%
% Copyright � 2012 Peeter Joot.  All Rights Reserved.
% Licenced as described in the file LICENSE under the root directory of this GIT repository.
%

%\chapter{PHY450H1S.  Relativistic Electrodynamics Lecture 19 (Taught by Prof. Erich Poppitz).  Lienard-Wiechert potentials}
\index{Lienard-Wiechert potentials}
%\chapter{Lienard-Wiechert potentials}
\label{chap:relativisticElectrodynamicsL19}
%\blogpage{http://sites.google.com/site/peeterjoot/math2011/relativisticElectrodynamicsL19.pdf}
%\date{Mar 15, 2011}
%
\paragraph{Reading}
%
Covering chapter 8 material from the text \citep{landau1980classical}, and
\popcite{RelEMpp136-146.pdf}{lecture notes RelEMpp136-146.pdf}.
%: the Lienard-Wiechert potentials (143-146) [Wednesday, Mar. 9...]
%
\section{Fields from the Lienard-Wiechert potentials}
\index{Lienard-Wiechert potential}

(We finished off with the scalar and vector potentials in class, but I have put those notes with the previous lecture).

To find \(\BE\) and \(\BB\) need

\(\PD{t}{t_r}\), and \(\spacegrad t_r(\Bx, t)\)

where
%
\begin{equation}\label{eqn:relativisticElectrodynamicsL19:10}
t - t_r = \Abs{\Bx - \Bx_c(t_r) }.
\end{equation}
%
implicit definition of \(t_r(\Bx, t)\)

In HW5 you will show
%
\begin{equation}\label{eqn:relativisticElectrodynamicsL19:30}
\PD{t}{t_r} = \frac{\Abs{\Bx - \Bx_c(t_r) }}{
\Abs{\Bx - \Bx_c(t_r) } - \frac{\Bv_c }{c} \cdot (\Bx - \Bx_c(t_r))}.
\end{equation}
%
\begin{equation}\label{eqn:relativisticElectrodynamicsL19:50}
\spacegrad t_r = \inv{c} \frac{\Bx - \Bx_c(t_r) }{
\Abs{\Bx - \Bx_c(t_r) } - \frac{\Bv_c }{c} \cdot (\Bx - \Bx_c(t_r))},
\end{equation}
and then use this to show that the electric and magnetic fields due to a moving charge are
%
\begin{equation}\label{eqn:relativisticElectrodynamicsL19:70}
\begin{aligned}
\BE(\Bx, t)
&= \frac{e R}{ (\BR \cdot \Bu)^3 } \left( (c^2 - \Bv_c^2) \Bu + \BR \cross (\Bu \cross \Ba_c) \right) \\
&= \frac{\BR}{R} \cross \BE \\
\Bu &= c \frac{\BR}{R} - \Bv_c,
\end{aligned}
\end{equation}
where everything is evaluated at the retarded time \(t_r = t - \Abs{\Bx - \Bx_c(t_r)}/c\).
This looks quite a bit different than what we find in \S 63 (63.8) in the text, but a little bit of expansion shows they are the same.
%
\section{Check.  Particle at rest}
With
%
\begin{equation}\label{eqn:relativisticElectrodynamicsL19:420}
\begin{aligned}
\Bx_c &= \Bx_0 \\
X_c^k &= (ct, \Bx_0) \\
\Abs{\Bx - \Bx_c(t_r)} &= c(t - t_r).
\end{aligned}
\end{equation}
%
\imageFigure{../figures/phy450-relativisticEandM/particleAtRestTrCalc}{Retarded time for particle at rest}{fig:particleAtRestTrCalc}{0.4}

As illustrated in \cref{fig:particleAtRestTrCalc} the retarded position is
%
\begin{equation}\label{eqn:relativisticElectrodynamicsL19:100}
\Bx_c(t_r) = \Bx_0,
\end{equation}
for
\begin{equation}\label{eqn:relativisticElectrodynamicsL19:120}
\Bu = \frac{\Bx - \Bx_0}{\Abs{\Bx - \Bx_0}} c,
\end{equation}
%
and
%
\begin{equation}\label{eqn:relativisticElectrodynamicsL19:140}
\BE = e \frac{\cancel{ \Abs{\Bx - \Bx_0}} }{ (c \Abs{\Bx - \Bx_0})^3 } c^3 \frac{\Bx - \Bx_0}{\cancel{\Abs{\Bx - \Bx_0}}},
\end{equation}
%
which is Coulomb's law
%
\begin{equation}\label{eqn:relativisticElectrodynamicsL19:160}
\BE = e \frac{\Bx - \Bx_0}{\Abs{\Bx - \Bx_0}^3}.
\end{equation}
%
\section{Check.  Particle moving with constant velocity}
%
This was also computed in full in homework 5.  The end result was
%
\begin{equation}\label{eqn:relativisticElectrodynamicsL19:180}
\BE =
e
\frac{\Bx - \Bv t}{\Abs{\Bx - \Bv t}^3}
\frac{1 -\Bbeta^2}{ \left(1 - \frac{(\Bx \cross \Bbeta)^2}{\Abs{\Bx - \Bv t}^2} \right)^{3/2} }.
\end{equation}
%
Writing
%
\begin{equation}\label{eqn:relativisticElectrodynamicsL19:440}
\begin{aligned}
\frac{\Bx \cross \Bbeta}{\Abs{\Bx - \Bv t}}
&=
\inv{c} \frac{(\Bx - \Bv t) \cross \Bv}{\Abs{\Bx - \Bv t}}  \\
&=
\frac{\Abs{\Bv}}{c} \frac{(\Bx - \Bv t) \cross \Bv}{\Abs{\Bx - \Bv t} \Abs{\Bv}}.
\end{aligned}
\end{equation}
%
We can introduce an angular dependence between the charge's translated position and its velocity
%
\begin{equation}\label{eqn:relativisticElectrodynamicsL19:200}
\sin^2 \theta = \Abs{ \frac{\Bv \cross (\Bx - \Bv t)}{\Abs{\Bv} \Abs{\Bx - \Bv t}} }^2,
\end{equation}
%
and write the field as
%
\begin{equation}\label{eqn:relativisticElectrodynamicsL19:220}
\BE =
\mathLabelBox{e \frac{\Bx - \Bv t}{\Abs{\Bx - \Bv t}^3}}{(\(\conj\))}
\frac{1 -\Bbeta^2}{ \left(1 - \frac{\Bv^2}{c^2} \sin^2 \theta \right)^{3/2} }.
\end{equation}
%
Observe that \((\conj) = \text{Coulomb's law measured from the instantaneous position of the charge}\).

The electric field \(\BE\) has a time dependence, strongest when perpendicular to the instantaneous position when \(\theta = \pi/2\), since the denominator is smallest (\(\BE\) largest) when \(\Bv/c\) is not small.  This is strongly \(\theta\) dependent.

Compare
%
\begin{equation}\label{eqn:relativisticElectrodynamicsL19:460}
\begin{aligned}
\frac{\Abs{\BE(\theta = \pi/2)} - \Abs{\BE(\theta = \pi/2 + \Delta \theta)} }
{
\Abs{\BE(\theta = \pi/2)}
}
&\approx
\frac{
\inv{(1 - \Bv^2/c^2)^{3/2}}
 - \inv{(1 - \Bv^2/c^2(1 - (\Delta \theta)^2))^{3/2}}
}
{
\inv{(1 - \Bv^2/c^2)^{3/2}}
} \\
&=
1 -
\left(
\frac{1 - \Bv^2/c^2}{
1 - \Bv^2/c^2 + \Bv^2/c^2(\Delta \theta)^2
}
\right)^{3/2} \\
&=
1 -
\left(
\frac{1}{
1 + \Bv^2/c^2 \frac{(\Delta \theta)^2}{1 - \Bv^2/c^2}
}
\right)^{3/2}.
\end{aligned}
\end{equation}
%
Here we used
%
\begin{equation}\label{eqn:relativisticElectrodynamicsL19:240}
\sin(\theta + \pi/2) = \frac{e^{i (\theta + \pi/2)} - e^{-i(\theta + \pi/2)}}{2i} = \cos\theta.
\end{equation}
%
and
%
\begin{equation}\label{eqn:relativisticElectrodynamicsL19:260}
\cos^2 \Delta \theta \approx \left( 1 - \frac{(\Delta \theta)^2}{2} \right)^2 \approx 1 - (\Delta \theta)^2.
\end{equation}
%
FIXME: he writes:
%
\begin{equation}\label{eqn:relativisticElectrodynamicsL19:280}
\Delta \theta \le \sqrt{1 - \frac{\Bv^2}{c^2}}.
\end{equation}
%
I do not see where that comes from.

PICTURE: Various \(\BE\)'s up, and \(\Bv\) perpendicular to that, strongest when charge is moving fast.
%
\section{Back to extracting physics from the Lienard-Wiechert field equations}
%
Imagine that we have a localized particle motion with
%
\begin{equation}\label{eqn:relativisticElectrodynamicsL19:300}
\Abs{\Bx_c(t_r)} < l.
\end{equation}
%
The velocity vector
%
\begin{equation}\label{eqn:relativisticElectrodynamicsL19:320}
\Bu = c \frac{\Bx - \Bx_c(t_r)}{\Abs{\Bx - \Bx_c}}.
\end{equation}
%
does not grow as distance from the source, so from \eqnref{eqn:relativisticElectrodynamicsL19:70}, we have for \(\Abs{\Bx} \gg l\)
%
\begin{equation}\label{eqn:relativisticElectrodynamicsL19:340}
\BB, \BE \sim \inv{\Abs{\Bx}^2}(\cdots) + \inv{\Bx}(\text{acceleration term}).
\end{equation}
%
The acceleration term will dominate at large distances from the source.  Our Poynting magnitude is
%
\begin{equation}\label{eqn:relativisticElectrodynamicsL19:360}
\Abs{\BS} \sim \Abs{\BE \cross \BB} \sim \inv{\Bx^2} (\text{acceleration})^2.
\end{equation}
%
We can ask about
%
\begin{equation}\label{eqn:relativisticElectrodynamicsL19:380}
\oint d^2 \Bsigma \cdot \BS \sim R^2 \inv{R^2} (\text{acceleration})^2 \sim (\text{acceleration})^2.
\end{equation}
%
In the limit, for the radiation of EM waves
%
\begin{equation}\label{eqn:relativisticElectrodynamicsL19:400}
\lim_{R\rightarrow \infty} \oint d^2 \Bsigma \cdot \BS \ne 0.
\end{equation}
%
The energy flux through a sphere of radius \(R\) is called the radiated power.
