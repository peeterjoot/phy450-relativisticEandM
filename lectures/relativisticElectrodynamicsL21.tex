%
% Copyright � 2012 Peeter Joot.  All Rights Reserved.
% Licenced as described in the file LICENSE under the root directory of this GIT repository.
%

%\chapter{More on EM fields due to dipole radiation}
\label{chap:relativisticElectrodynamicsL21}
%\blogpage{http://sites.google.com/site/peeterjoot/math2011/relativisticElectrodynamicsL21.pdf}
%\date{Mar 22, 2011}

\paragraph{Reading}

Covering chapter 8 material from the text \citep{landau1980classical}, and
\popcite{RelEMpp147-165.pdf}{lecture notes RelEMpp147-165.pdf}.
% radiated power (154); fields in the ``wave zone'' and discussions of approximations made (155-159); EM fields due to electric dipole radiation (160-163); Poynting vector, angular distribution, and power of dipole radiation (164-165) [Wednesday, Mar. 16...]

\section{Where we left off}

For a localized charge distribution, we would arrived at expressions for the scalar and vector potentials far from the point where the charges and currents were localized.  This was then used to consider the specific case of a dipole system where one of the charges had a sinusoidal oscillation.  The charge positions for the negative and positive charges respectively were

\begin{equation}\label{eqn:relativisticElectrodynamicsL21:10}
\begin{aligned}
z_{-} &= 0 \\
z_{+} &= \Be_3( z_0 + a \sin(\omega t)) ,
\end{aligned}
\end{equation}

so that our dipole moment \(\Bd = \int \rho(\Bx') \Bx'\) is

\begin{equation}\label{eqn:relativisticElectrodynamicsL21:30}
\Bd = \Be_3 q (z_0 + a \sin(\omega t)).
\end{equation}

The scalar potential, to first order in a number of Taylor expansions at our point far from the source, evaluated at the retarded time \(t_r = t - \Abs{\Bx}/c\), was found to be

\begin{equation}\label{eqn:relativisticElectrodynamicsL21:35}
A^0(\Bx, t) = \frac{z q}{\Abs{\Bx}^3} ( z_0 + a \sin(\omega t_r) ) + \frac{z q}{c \Abs{\Bx}^2} a \omega \cos(\omega t_r),
\end{equation}

and our vector potential, also with the same approximations, was
\begin{equation}\label{eqn:relativisticElectrodynamicsL21:40}
\BA(\Bx, t) = \inv{c \Abs{\Bx} } \Be_3 q a \omega \cos(\omega t_r).
\end{equation}

We found that the electric field (neglecting any non-radiation terms that died off as inverse square in the distance) was

\begin{equation}\label{eqn:relativisticElectrodynamicsL21:50}
\BE = \frac{a \omega^2 q}{c^2 \Abs{\Bx}} \sin( \omega (t - \Abs{\Bx}/c) ) \left( \Be_3 - \rcap \frac{z}{\Abs{\Bx}} \right).
\end{equation}

\section{Direct computation of the magnetic radiation field}

Taking the curl of the vector potential \eqnref{eqn:relativisticElectrodynamicsL21:50} for the magnetic field, we will neglect the contribution from the \(1/\Abs{\Bx}\) since that will be inverse square, and die off too quickly far from the source

\begin{equation}\label{eqn:relativisticElectrodynamicsL21:320}
\begin{aligned}
\BB
&= \spacegrad \cross \BA \\
&= \spacegrad \cross \inv{c \Abs{\Bx} } \Be_3 q a \omega \cos(\omega (t - \Abs{\Bx}/c)) \\
&\approx - \frac{q a \omega}{c \Abs{\Bx} } \Be_3 \cross \spacegrad \cos(\omega (t - \Abs{\Bx}/c)) \\
&= - \frac{q a \omega}{c \Abs{\Bx} } \left( -\frac{\omega}{c} \right)
(\Be_3 \cross \spacegrad \Abs{\Bx}) \sin(\omega (t - \Abs{\Bx}/c)),
\end{aligned}
\end{equation}

which is

\begin{equation}\label{eqn:relativisticElectrodynamicsL21:60}
\BB = \frac{q a \omega^2}{c^2 \Abs{\Bx} } (\Be_3 \cross \rcap) \sin(\omega (t - \Abs{\Bx}/c)).
\end{equation}

Comparing to \eqnref{eqn:relativisticElectrodynamicsL21:50}, we see that this equals \(\rcap \cross \BE\) as expected.

\section{An aside: A tidier form for the electric dipole field}
\index{electric dipole field}

We can rewrite the electric field \eqnref{eqn:relativisticElectrodynamicsL21:50} in terms of the retarded time dipole

\begin{equation}\label{eqn:relativisticElectrodynamicsL21:80}
\BE = \inv{c^2 \Abs{\Bx}} \Bigl( -\ddot{\Bd}(t_r) + \rcap ( \ddot{\Bd}(t_r) \cdot \rcap ) \Bigr),
\end{equation}

where

\begin{equation}\label{eqn:relativisticElectrodynamicsL21:100}
\ddot{\Bd}(t) = - q a \omega^2 \sin(\omega t) \Be_3
\end{equation}

Then using the vector identity

\begin{equation}\label{eqn:relativisticElectrodynamicsL21:120}
(\BA \cross \rcap ) \cross \rcap = -\BA + (\rcap \cdot \BA) \rcap,
\end{equation}

we have for the fields

\boxedEquation{eqn:relativisticElectrodynamicsL21:140}{
\begin{aligned}
\BE &= \inv{c^2 \Abs{\Bx}} (\ddot{\Bd}(t_r) \cross \rcap) \cross \rcap \\
\BB &= \rcap \cross \BE.
\end{aligned}
}


\section{Calculating the energy flux}
\index{energy flux}

Our Poynting vector, the energy flux, is
\index{Poynting vector}

\begin{equation}\label{eqn:relativisticElectrodynamicsL21:160}
\BS = \frac{c}{4 \pi} \BE \cross \BB =
\frac{c}{4 \pi}
\left( \frac{q a \omega^2}{c^2 \Abs{\Bx} } \right)^2
\sin^2(\omega (t - \Abs{\Bx}/c))
\left( \Be_3 - \rcap \frac{z}{\Abs{\Bx}} \right) \cross (\rcap \cross \Be_3).
\end{equation}

Expanding just the cross terms we have

\begin{equation}\label{eqn:relativisticElectrodynamicsL21:340}
\begin{aligned}
\left( \Be_3 - \rcap \frac{z}{\Abs{\Bx}} \right) \cross (\rcap \cross \Be_3)
&=
-(\rcap \cross \Be_3) \cross \Be_3 - \frac{z}{\Abs{\Bx}} (\Be_3 \cross \rcap) \cross \rcap \\
&=
-(-\rcap + \Be_3(\Be_3 \cdot \rcap) ) - \frac{z}{\Abs{\Bx}} (-\Be_3 + \rcap (\rcap \cdot \Be_3)) \\
&=
\rcap - \cancel{\Be_3(\Be_3 \cdot \rcap)} + \frac{z}{\Abs{\Bx}} (\cancel{\Be_3} - \rcap (\rcap \cdot \Be_3)) \\
&=
\rcap( 1 - (\rcap \cdot \Be_3)^2 ).
\end{aligned}
\end{equation}

Note that we have utilized \(\rcap \cdot \Be_3 = z/\Abs{\Bx}\) to do the cancellations above, and for the final grouping.  Since \(\rcap \cdot \Be_3 = \cos\theta\), the direction cosine of the unit radial vector with the z-axis, we have for the direction of the Poynting vector

\begin{equation}\label{eqn:relativisticElectrodynamicsL21:360}
\begin{aligned}
\rcap( 1 - (\rcap \cdot \Be_3)^2 )
&= \rcap (1 - \cos^2\theta) \\
&= \rcap \sin^2\theta.
\end{aligned}
\end{equation}

Our Poynting vector is found to be directed radially outwards, and is

\begin{equation}\label{eqn:relativisticElectrodynamicsL21:180}
\BS =
\frac{c}{4 \pi}
\left( \frac{q a \omega^2}{c^2 \Abs{\Bx} } \right)^2
\sin^2(\omega (t - \Abs{\Bx}/c)) \sin^2\theta \rcap.
\end{equation}

The intensity is constant along the curves

\begin{equation}\label{eqn:relativisticElectrodynamicsL21:200}
\Abs{\sin\theta} \sim r
\end{equation}

PICTURE: dipole lobes diagram with \(\Bd\) up along the z axis, and \(\rcap\) pointing in an arbitrary direction.

FIXME: understand how this lobes picture comes from our result above.

PICTURE: field diagram along spherical north-south great circles, and the electric field \(\BE\) along what looks like it is the \(\thetacap\) direction, and \(\BB\) along what appear to be the \(\phicap\) direction, and \(\BS\) pointing radially out.

\paragraph{Utilizing the spherical unit vectors to express the field directions}

In class we see the picture showing these spherical unit vector directions.  We can see this algebraically as well.  Recall that we have for our unit vectors

\begin{equation}\label{eqn:relativisticElectrodynamicsL21:260}
\begin{aligned}
\rcap &= \Be_1 \sin\theta \cos\phi + \Be_2 \sin\theta \sin\phi + \Be_3 \cos\theta \\
\phicap &= \sin\theta ( \Be_2 \cos\phi - \Be_1 \sin\phi ) \\
\thetacap &= \cos\theta ( \Be_1 \cos\phi + \Be_2 \sin\phi ) - \Be_3 \sin\theta,
\end{aligned}
\end{equation}

with the volume element orientation governed by cyclic permutations of

\begin{equation}\label{eqn:relativisticElectrodynamicsL21:280}
\rcap \cross \thetacap = \phicap.
\end{equation}

We can now express the direction of the magnetic field in terms of the spherical unit vectors

\begin{equation}\label{eqn:relativisticElectrodynamicsL21:380}
\begin{aligned}
\Be_3 \cross \rcap
&=
\Be_3 \cross (\Be_1 \sin\theta \cos\phi + \Be_2 \sin\theta \sin\phi + \Be_3 \cos\theta ) \\
&=
\Be_3 \cross (\Be_1 \sin\theta \cos\phi + \Be_2 \sin\theta \sin\phi ) \\
&=
\Be_2 \sin\theta \cos\phi - \Be_1 \sin\theta \sin\phi  \\
&=
\sin\theta ( \Be_2 \cos\phi - \Be_1 \sin\phi ) \\
&=
\sin\theta \phicap.
\end{aligned}
\end{equation}

The direction of the electric field was in the direction of \((\ddot{\Bd} \cross \rcap) \cross \rcap\) where \(\Bd\) was directed along the z-axis.  This is then

\begin{equation}\label{eqn:relativisticElectrodynamicsL21:400}
\begin{aligned}
(\Be_3 \cross \rcap) \cross \rcap
&=
-\sin\theta \phicap \cross \rcap \\
&=
-\sin\theta \thetacap
\end{aligned}
\end{equation}

\boxedEquation{eqn:relativisticElectrodynamicsL21:140b}{
\begin{aligned}
\BE &= \frac{ q a \omega^2 }{c^2 \Abs{\Bx}} \sin(\omega t_r) \sin\theta \thetacap \\
\BB &= -\frac{ q a \omega^2 }{c^2 \Abs{\Bx}} \sin(\omega t_r) \sin\theta \phicap \\
\BS &= \left( \frac{ q a \omega^2 }{c^2 \Abs{\Bx}} \right)^2 \sin^2(\omega t_r) \sin^2\theta \rcap
\end{aligned}
}

\section{Calculating the power}
\index{power}

Integrating \(\BS\) over a spherical surface, we can calculate the power

FIXME: remind myself why Power is an appropriate label for this integral.

This is

\begin{equation}\label{eqn:relativisticElectrodynamicsL21:420}
\begin{aligned}
P(r, t)
&= \oint d^2 \Bsigma \cdot \BS \\
&= \int \cancel{r^2} \sin\theta d\theta d\phi
\frac{c}{4 \pi}
\left( \frac{q a \omega^2}{c^2 \cancel{\Abs{\Bx}} } \right)^2
\sin^2(\omega (t - \Abs{\Bx}/c)) \sin^2\theta \\
&=
\frac{q^2 a^2 \omega^4}{2 c^3 }
\sin^2(\omega (t - r/c))
\mathLabelBox{\int \sin^3\theta d\theta}{\(=4/3\)}
\end{aligned}
\end{equation}

\begin{equation}\label{eqn:relativisticElectrodynamicsL21:220}
P(r, t) = \frac{2}{3} \frac{q^2 a^2 \omega^4}{c^3} \sin^2(\omega (t - r/c)) =
\frac{q^2 a^2 \omega^4}{3 c^3} (1 - \cos(2 \omega (t - r/c))
\end{equation}

Averaging over a period kills off the cosine term

\begin{equation}\label{eqn:relativisticElectrodynamicsL21:240}
\expectation{P(r, t)} = \frac{\omega}{2 \pi} \int_0^{2 \pi/\omega} dt P(t) = \frac{q^2 a^2 \omega^4}{3 c^3} = \frac{2}{3 c^3} \expectation{ \ddot{d}(t_r) }
\end{equation}

and we once again see that higher frequencies radiate more power (i.e. why the sky is blue).

\section{Types of radiation}
\index{radiation}

We have seen now radiation from localized current distributions, and called that electric dipole radiation.  There are many other sources of electrodynamic radiation, of which here are a couple.

\begin{itemize}
\item Magnetic dipole radiation.

This will be covered more in more depth in the tutorial.  Picture of a positive circulating current \(I = I_o \sin \omega t\) given, and a magnetic dipole moment \(\Bmu = \pi b^2 I \Be_3\).

This sort of current loop is a source of magnetic dipole radiation.

\item Cyclotron radiation.

This is the label for acceleration induced radiation (at high velocities) by particles moving in a uniform magnetic field.

PICTURE: circular orbit with speed \(v = \omega r\).  The particle trajectories are

\begin{equation}\label{eqn:relativisticElectrodynamicsL21:300}
\begin{aligned}
x &= r \cos \omega t \\
y &= r \sin \omega t
\end{aligned}
\end{equation}

This problem can be treated as two electric dipoles out of phase by 90 degrees.

PICTURE: 4 lobe dipole picture, with two perpendicular dipole moment arrows.  Resulting superposition sort of smeared together.

\end{itemize}

