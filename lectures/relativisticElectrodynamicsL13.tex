%
% Copyright � 2012 Peeter Joot.  All Rights Reserved.
% Licenced as described in the file LICENSE under the root directory of this GIT repository.
%

%\chapter{Variational principle for the field}
\index{field!variational principle}
\label{chap:relativisticElectrodynamicsL13}
%\blogpage{http://sites.google.com/site/peeterjoot/math2011/relativisticElectrodynamicsL13.pdf}
%\date{Feb 15, 2011}
%
\paragraph{Reading}
%
Covering chapter 4 material from the text \citep{landau1980classical}, and
\popcite{RelEMpp103-113.pdf}{lecture notes RelEMpp103-113.pdf}.
%: variational principle for the electromagnetic field and the relevant boundary conditions (103-105); the second set of Maxwell's equations from the variational principle (106-108); Maxwell's equations in vacuum and the wave equation in the non-relativistic Coulomb gauge (109-111)
%
\section{Review.  Our action.}
%
\begin{equation}\label{eqn:relativisticElectrodynamicsL13:630}
\begin{aligned}
S
&= S_{\text{particles}} + S_{\text{interaction}} + S_{\text{EM field}} \\
&= \sum_A \int_{x_A^i(\tau)} ds ( -m_A c )
- \sum_A
\frac{e_A}{c}
\int dx_A^i A_i(x_A)
- \inv{16 \pi c} \int d^4 x F^{ij } F_{ij}.
\end{aligned}
\end{equation}
%
Our dynamics variables are
%
\begin{equation}\label{eqn:relativisticElectrodynamicsL13:30}
\left\{
\begin{array}{l l}
x_A^i(\tau) & \quad \mbox{\(A = 1, \cdots, N\)} \\
A^i(x) & \quad \mbox{\(A = 1, \cdots, N\)}
\end{array}
\right.
\end{equation}
%
We saw that the interaction term could also be written in terms of a delta function current, with
\begin{equation}\label{eqn:relativisticElectrodynamicsL13:50}
S_{\text{interaction}}
= -\inv{c^2} \int d^4x j^i(x) A_i(x),
\end{equation}
and
\begin{equation}\label{eqn:relativisticElectrodynamicsL13:70}
j^i(x) = \sum_A c e_A \int_{x(\tau)} dx_A^i \delta^4( x - x_A(\tau)).
\end{equation}
%
Variation with respect to \(x_A^i(\tau)\) gave us
%
\begin{equation}\label{eqn:relativisticElectrodynamicsL13:90}
m c \dds{u^i_A} = \frac{e}{c} u_A^j F_{ij}.
\end{equation}
%
Note that it is easy to get the sign mixed up here.  With our \((+,-,-,-)\) metric tensor, if the second index is the summation index, we have a positive sign.

Only the \(S_{\text{particles}}\) and \(S_{\text{interaction}}\) depend on \(x_A^i(\tau)\).
%
\section{The field action variation.}
\index{action!field}
%
\paragraph{Today:} We will find the EOM for \(A^i(x)\).  The dynamical degrees of freedom are \(A^i(\Bx,t)\)
%
\begin{equation}\label{eqn:relativisticElectrodynamicsL13:110}
S[A^i(\Bx, t)] = -\inv{16 \pi c} \int d^4x F_{ij}F^{ij} - \inv{c^2} \int d^4 x A^i j_i.
\end{equation}
%
Here \(j^i\) are treated as ``sources''.

We demand that
%
\begin{equation}\label{eqn:relativisticElectrodynamicsL13:130}
\delta S = S[ A^i(\Bx, t) + \delta A^i(\Bx, t)] - S[ A^i(\Bx, t) ] = 0 + O(\delta A)^2.
\end{equation}
%
We need to impose two conditions.
\begin{itemize}
\item At spatial \(\infty\), i.e. at \(\Abs{\Bx} \rightarrow \infty, \forall t\), we will impose the condition
%
\begin{equation}\label{eqn:relativisticElectrodynamicsL13:150}
\evalbar{A^i(\Bx, t)}{\Abs{\Bx} \rightarrow \infty} \rightarrow 0.
\end{equation}
%
This is sensible, because fields are created by charges, and charges are assumed to be localized in a bounded region.  The field outside charges will \(\rightarrow 0\) at \(\Abs{\Bx} \rightarrow \infty\).  Later we will treat the integration range as finite, and bounded, then later allow the boundary to go to infinity.

\item at \(t = -T\) and \(t = T\) we will imagine that the values of \(A^i(\Bx, \pm T)\) are fixed.

This is analogous to \(x(t_i) = x_1\) and \(x(t_f) = x_2\) in particle mechanics.

Since \(A^i(\Bx, \pm T)\) is given, and equivalent to the initial and final field configurations, our extremes at the boundary is zero
%
\begin{equation}\label{eqn:relativisticElectrodynamicsL13:170}
\delta A^i(\Bx, \pm T) = 0.
\end{equation}
%
\end{itemize}

PICTURE: a cylinder in spacetime, with an attempt to depict the boundary.
%
\section{Computing the variation.}
%
\begin{equation}\label{eqn:relativisticElectrodynamicsL13:190}
\delta S[A^i(\Bx, t)]
= -\inv{16 \pi c} \int d^4 x \delta (F_{ij}F^{ij}) - \inv{c^2} \int d^4 x \delta(A^i) j_i.
\end{equation}
%
Looking first at the variation of just the \(F^2\) bit we have
%
\begin{equation}\label{eqn:relativisticElectrodynamicsL13:650}
\begin{aligned}
\delta (F_{ij}F^{ij})
&=
\delta(F_{ij}) F^{ij} + F_{ij} \delta(F^{ij}) \\
&=
2 \delta(F^{ij}) F_{ij} \\
&=
2 \delta(\partial^i A^j - \partial^j A^i) F_{ij} \\
&=
2 \delta(\partial^i A^j) F_{ij} - 2 \delta(\partial^j A^i) F_{ij} \\
&=
2 \delta(\partial^i A^j) F_{ij} - 2 \delta(\partial^i A^j) F_{ji} \\
&=
4 \delta(\partial^i A^j) F_{ij} \\
&=
4 F_{ij} \partial^i \delta(A^j).
\end{aligned}
\end{equation}
Our variation is now reduced to
\begin{equation}\label{eqn:relativisticElectrodynamicsL13:670}
\begin{aligned}
\delta S[A^i(\Bx, t)]
&= -\inv{4 \pi c} \int d^4 x F_{ij} \partial^i \delta(A^j) - \inv{c^2} \int d^4 x j^i \delta(A_i) \\
&= -\inv{4 \pi c} \int d^4 x F^{ij} \PD{x^i}{} \delta(A_j) - \inv{c^2} \int d^4 x j^i \delta(A_i).
\end{aligned}
\end{equation}
%
We can integrate this first term by parts
%
\begin{equation}\label{eqn:relativisticElectrodynamicsL13:690}
\begin{aligned}
\int d^4 x F^{ij} \PD{x^i}{} \delta(A_j)
&=
\int d^4 x \PD{x^i}{} \left( F^{ij} \delta(A_j) \right)
-\int d^4 x \left( \PD{x^i}{} F^{ij} \right) \delta(A_j).
\end{aligned}
\end{equation}
%
The first term is a four dimensional divergence, with the contraction of the four gradient \(\partial_i\) with a four vector \(B^i = F^{ij} \delta(A_j)\).

Prof. Poppitz chose \(dx^0 d^3 \Bx\) split of \(d^4 x\) to illustrate that this can be viewed as regular old spatial three vector divergences.  It is probably more rigorous to mandate that the four volume element is oriented
\begin{equation*}
 d^4 x = (1/4!)\epsilon_{ijkl} dx^i dx^j dx^k dx^l,
\end{equation*}
and then utilize the 4D version of the divergence theorem (or its Stokes Theorem equivalent).  The completely antisymmetric tensor should do most of the work required to express the oriented boundary volume.
%
%\begin{align*}
%\int_V d^4 x F^{ij} \PD{x^i}{} \delta(A_j)
%&=
%\inv{4!} \int_V \sum_{ijkl}\epsilon_{ijkl} dx^i dx^j dx^k dx^l F^{ij} \PD{x^i}{} \delta(A_j) \\
%&=
%\inv{4!} \int_V \sum_{ijkl} \epsilon_{ijkl} dx^j dx^k dx^l F^{ij} \left( dx^i \PD{x^i}{} \delta(A_j) \right) \\
%&=
%\inv{4!} \sum_{ijkl} \epsilon_{ijkl} \int_{\partial V} dx^j dx^k dx^l (\partial_i F^{ij}) \delta(A_j)
%-
%\int_V d^4 x \left( \partial_i F^{ij} \right) \delta(A_j).
%\end{align*}
%
Because we have specified that \(A^i\) is zero on the boundary, so is \(F^{ij}\), so these boundary terms are killed off.  We are left with
%
\begin{equation}\label{eqn:relativisticElectrodynamicsL13:710}
\begin{aligned}
\delta S[A^i(\Bx, t)]
&= -\inv{4 \pi c} \int d^4 x \delta (A_j) \partial_i F^{ij} - \inv{c^2} \int d^4 x j^i \delta(A_i) \\
&=
\int d^4 x \delta A_j(x)
\left(
-\inv{4 \pi c} \partial_i F^{ij}(x) - \inv{c^2} j^i
\right)  \\
&= 0.
\end{aligned}
\end{equation}
%
This gives us
\boxedEquation{eqn:relativisticElectrodynamicsL13:210}{
\partial_i F^{ij} = \frac{4 \pi}{c} j^j.
}
\section{Unpacking these.}
Recall that the Bianchi identity
\index{Bianchi identity}
%
\begin{equation}\label{eqn:relativisticElectrodynamicsL13:230}
\epsilon^{ijkl} \partial_j F_{kl} = 0,
\end{equation}
gave us
%
\begin{equation}\label{eqn:relativisticElectrodynamicsL13:250}
\begin{aligned}
\spacegrad \cdot \BB &= 0 \\
\spacegrad \cross \BE &= -\inv{c} \PD{t}{\BB}.
\end{aligned}
\end{equation}
%
How about the EOM that we have found by varying the action?  One of those equations is
%
\begin{equation}\label{eqn:relativisticElectrodynamicsL13:270}
\partial_\alpha F^{\alpha 0} = \frac{4 \pi}{c} j^0 = 4 \pi \rho,
\end{equation}
%
since \(j^0 = c \rho\).

Because
%
\begin{equation}\label{eqn:relativisticElectrodynamicsL13:290}
F^{\alpha 0} = (\BE)^\alpha,
\end{equation}
we have
\begin{equation}\label{eqn:relativisticElectrodynamicsL13:310}
\spacegrad \cdot \BE = 4 \pi \rho.
\end{equation}
%
The messier one to deal with is
%
\begin{equation}\label{eqn:relativisticElectrodynamicsL13:330}
\partial_i F^{i\alpha} = \frac{4 \pi}{c} j^\alpha.
\end{equation}
%
Splitting out the spatial and time indices for the four gradient we have
%
\begin{equation}\label{eqn:relativisticElectrodynamicsL13:730}
\begin{aligned}
\partial_i F^{i\alpha}
&= \partial_\beta F^{\beta \alpha} + \partial_0 F^{0 \alpha} \\
&= \partial_\beta F^{\beta \alpha} - \inv{c} \PD{t}{(\BE)^\alpha}.
\end{aligned}
\end{equation}
The spatial index tensor element is
\begin{equation}\label{eqn:relativisticElectrodynamicsL13:750}
\begin{aligned}
F^{\beta \alpha}
&=
\partial^\beta A^\alpha - \partial^\alpha A^\beta  \\
&=
- \PD{x^\beta}{A^\alpha} + \PD{x^\alpha}{A^\beta} \\
&=
\epsilon^{\alpha\beta\gamma} B^\gamma,
\end{aligned}
\end{equation}
%
so the sum becomes
%
\begin{equation}\label{eqn:relativisticElectrodynamicsL13:770}
\begin{aligned}
\partial_i F^{i\alpha}
&= \partial_\beta ( \epsilon^{\alpha\beta\gamma} B^\gamma) - \inv{c} \PD{t}{(\BE)^\alpha} \\
&=
\epsilon^{\beta\gamma\alpha} \partial_\beta B^\gamma - \inv{c} \PD{t}{(\BE)^\alpha} \\
&=
(\spacegrad \cross \BB)^\alpha - \inv{c} \PD{t}{(\BE)^\alpha}.
\end{aligned}
\end{equation}
This gives us
\begin{equation}\label{eqn:relativisticElectrodynamicsL13:350}
\frac{4 \pi}{c} j^\alpha
= (\spacegrad \cross \BB)^\alpha - \inv{c} \PD{t}{(\BE)^\alpha},
\end{equation}
%
or in vector form
%
\begin{equation}\label{eqn:relativisticElectrodynamicsL13:370}
\spacegrad \cross \BB - \inv{c} \PD{t}{\BE} = \frac{4 \pi}{c} \Bj.
\end{equation}
%
Summarizing what we know so far, we have
\boxedEquation{eqn:relativisticElectrodynamicsL13:390}{
\begin{aligned}
\partial_i F^{ij} &= \frac{4 \pi}{c} j^j \\
\epsilon^{ijkl} \partial_j F_{kl} &= 0,
\end{aligned}
}
or in vector form
\boxedEquation{eqn:relativisticElectrodynamicsL13:410}{
\begin{aligned}
\spacegrad \cdot \BE &= 4 \pi \rho \\
\spacegrad \cross \BB -\inv{c} \PD{t}{\BE} &= \frac{4 \pi}{c} \Bj \\
\spacegrad \cdot \BB &= 0 \\
\spacegrad \cross \BE +\inv{c} \PD{t}{\BB} &= 0.
\end{aligned}
}
\section{Speed of light.}
\index{Speed of light}
%
\paragraph{Claim}: ``\(c\)'' is the speed of EM waves in vacuum.
%
Study equations in vacuum (no sources, so \(j^i = 0\)) for \(A^i = (\phi, \BA)\).
%
\begin{equation}\label{eqn:relativisticElectrodynamicsL13:430}
\begin{aligned}
\spacegrad \cdot \BE &= 0 \\
\spacegrad \cross \BB &= \inv{c} \PD{t}{\BE}.
\end{aligned}
\end{equation}
%
where
%
\begin{equation}\label{eqn:relativisticElectrodynamicsL13:450}
\begin{aligned}
\BE &= - \spacegrad \phi - \inv{c} \PD{t}{\BA} \\
\BB &= \spacegrad \cross \BA.
\end{aligned}
\end{equation}
%
In terms of potentials
%
\begin{equation}\label{eqn:relativisticElectrodynamicsL13:790}
\begin{aligned}
0 &= \spacegrad \cross (\spacegrad \cross \BA) \\
&= \spacegrad \cross \BB \\
&= \inv{c} \PD{t}{\BE} \\
&= \inv{c} \PD{t}{} \left( - \spacegrad \phi - \inv{c} \PD{t}{\BA} \right) \\
&= -\inv{c} \PD{t}{} \spacegrad \phi - \inv{c^2} \frac{\partial^2 \BA}{\partial t^2}.
\end{aligned}
\end{equation}
%
Since we also have
%
\begin{equation}\label{eqn:relativisticElectrodynamicsL13:600}
\spacegrad \cross (\spacegrad \cross \BA) = \spacegrad (\spacegrad \cdot \BA) - \spacegrad^2 \BA,
\end{equation}
%
some rearrangement gives
%
\begin{equation}\label{eqn:relativisticElectrodynamicsL13:600b}
\spacegrad (\spacegrad \cdot \BA) = \spacegrad^2 \BA  -\inv{c} \PD{t}{} \spacegrad \phi - \inv{c^2} \frac{\partial^2 \BA}{\partial t^2}.
\end{equation}
%
The remaining equation \(\spacegrad \cdot \BE = 0\), in terms of potentials is
%
\begin{equation}\label{eqn:relativisticElectrodynamicsL13:491}
\begin{aligned}
\spacegrad \cdot \BE = - \spacegrad^2 \phi - \inv{c} \PD{t}{\spacegrad \cdot \BA}.
\end{aligned}
\end{equation}
%
We can make a gauge transformation that completely eliminates \eqnref{eqn:relativisticElectrodynamicsL13:491}, and reduces \eqnref{eqn:relativisticElectrodynamicsL13:600b} to a wave equation.
%
\begin{equation}\label{eqn:relativisticElectrodynamicsL13:490}
\begin{aligned}
(\phi, \BA) \rightarrow (\phi', \BA').
\end{aligned}
\end{equation}
%
with
%
\begin{equation}\label{eqn:relativisticElectrodynamicsL13:510}
\begin{aligned}
\phi &= \phi' - \inv{c} \PD{t}{\chi} \\
\BA &= \BA' + \spacegrad \chi.
\end{aligned}
\end{equation}
%
Can choose \(\chi(\Bx, t)\) to make \(\phi' = 0\) (\(\forall \phi \exists \chi, \phi' = 0\))
%
\begin{equation}\label{eqn:relativisticElectrodynamicsL13:530}
\begin{aligned}
\inv{c} \PD{t}{} \chi(\Bx, t) = \phi(\Bx, t).
\end{aligned}
\end{equation}
%
\begin{equation}\label{eqn:relativisticElectrodynamicsL13:550}
\begin{aligned}
\chi(\Bx, t) = c \int_{-\infty}^t dt' \phi(\Bx, t').
\end{aligned}
\end{equation}
%
Can also find a transformation that also allows \(\spacegrad \cdot \BA = 0\)
%
\paragraph{Q:} What would that second transformation be explicitly?
\paragraph{A:} To be revisited next lecture, when this is covered in full detail.
%
This is the Coulomb gauge
%
\begin{equation}\label{eqn:relativisticElectrodynamicsL13:570}
\begin{aligned}
\phi &= 0 \\
\spacegrad \cdot \BA &= 0.
\end{aligned}
\end{equation}
%
From \eqnref{eqn:relativisticElectrodynamicsL13:600b}, we then have
%
\begin{equation}\label{eqn:relativisticElectrodynamicsL13:610}
\inv{c^2} \frac{\partial^2 \BA'}{\partial t^2} -\spacegrad^2 \BA' = 0.
\end{equation}
%
which is the wave equation for the propagation of the vector potential \(\BA'(\Bx, t)\) through space at velocity \(c\), confirming that \(c\) is the speed of electromagnetic propagation (the speed of light).
