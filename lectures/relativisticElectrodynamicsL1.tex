%
% Copyright � 2012 Peeter Joot.  All Rights Reserved.
% Licenced as described in the file LICENSE under the root directory of this GIT repository.
%

%\chapter{Speed of light and simultaneity}
\index{relativity principle}
\index{speed of light}
\index{simultaneity}
\label{chap:relativisticElectrodynamicsL1}
%\blogpage{http://sites.google.com/site/peeterjoot/math2011/relativisticElectrodynamicsL1.pdf}
%\date{Jan 11, 2011}

\paragraph{Reading}

No reading from \citep{landau1980classical} appears to have been assigned, but relevant stuff can be found in chapter 1.
Covering \popcite{RelEM1-11.pdf}{lecture notes RelEM1-11.pdf}
%, we have reading: space, time and Gallilean relativity (1-6); speed of light and Einsteins relativity principle (7-10); relativity of simultaneity (11).

\section{Distance as a clock}

The title of this course is an oxymoron since ELECTRODYNAMICS == RELATIVITY.  In classical and quantum physics (non-gravitational) we start by postulating the existence of space and time.  These are, in non-gravitational physics, the arena where everything takes place.  The space that we work with is the three dimensional Euclidean space \R{3}.  One way of describing it is using three coordinates

\begin{equation}\label{eqn:relativisticElectrodynamicsL1:10}
\text{\R{3}} = \{ x, y, z ; x,y,z \in [-\infty,\infty] \}.
\end{equation}

We define a distance between \(P\) and \(P'\) as

\begin{equation}\label{eqn:relativisticElectrodynamicsL1:20}
\Abs{P P' } = \sqrt{ (x-x')^2 + (y-y')^2 + (z-z')^2 }
\end{equation}

\textunderline{time} is a parameter with respect to which positions of \textunderline{free} particles particles change at a constant rate.

Mathematically, we describe the motion of free particles by giving \((x(t), y(t), z(t))\) : coordinates as functions of t,

\begin{equation}\label{eqn:relativisticElectrodynamicsL1:30}
\frac{d^2 x_i(t)}{dt^2} = 0, \qquad i = 1,2,3
\end{equation}

Here \(x,y,z\) are the free particle coordinates in an ``internal frame'', the frame where \(\ddot{\Br} = 0\) holds for a free particle (\(\ddot{\Br} = d^2 \Br/dt^2 \)) for a free particle with tragectory such as \(x = v_0 t, y = z = 0\).

\section{The principle of relativity}

\makedefinition{Principle of relativity (Galileo or Einstein)}{dfn:L1:1}{
``Laws of nature are identical in all inertial frames''.

Equivalently, ``Identical experiments in two inertial frames yield identical results''.
}

\paragraph{What do we mean by laws of nature?}  Equations that describe dynamics.

Now we need to get more specific.  Identical equations means that the equations have the same form in two inertial frames provided, you express them (the equations) via the coordinates \(\Br,t\) in the given inertial frame.

FIXME: DRAW x,y,z COORDINATE SYSTEM with origin \(O\).  And another with origin \(O'\) where the origin is moving with velocity \(v\) in the y direction.

The Galilean relativity principle states that ``equations of motion are invariant under Galilean transformations''.  What do we mean by transformations?  If we have a point \(P(t)\) in space with coordinates in both frames that are related.  It is pretty clear that the coordinates \(x = x'\) and \(z = z'\).  What about the \(y'\) coordinate?  For that we have \(y' = y - v t\), so that the origins overlap (\(O = O'\)) at \(t=0\).

In Galilean relativity, time is absolute.  i.e. It is the same in all inertial frames.  It is now a no-brainer to find the velocities of the particle.  Taking derivatives we take time derivatives of

\begin{align}\label{eqn:relativisticElectrodynamicsL1:40}
x' &= x \\
y' &= y - v t \\
z' &= z,
\end{align}

for

\begin{align}\label{eqn:relativisticElectrodynamicsL1:50}
v_x' &= v_x \\
v_y' &= v_y - v \\
v_z' &= v_z.
\end{align}

In vector notation we have

\begin{align}\label{eqn:relativisticElectrodynamicsL1:60}
\Br' &= \Br - \Bv_0 t \\
\Bv' &= \Bv - \Bv_0
\end{align}

The principle of relativity says that the dynamical equations are invariant under such transformations.

Take Newton's law for example applied to two bodies, labeled by their masses \(M_1\) and \(M_2\).

These bodies may be interacting.  For example, with Newtonian gravitation

\begin{equation}\label{eqn:relativisticElectrodynamicsL1:70}
V(\Br_1 - \Br_2) = -G_N \frac{M_1 M_2}{\Abs{\Br_1 - \Br_2}},
\end{equation}

or the Van Der Waals, interaction

\begin{equation}\label{eqn:relativisticElectrodynamicsL1:80}
V(\Br_1 - \Br_2) = - \lr{\text{const}} \inv{\Abs{\Br_1 - \Br_2}^6},
\end{equation}

Our interaction is via a gradient \(\PDi{\Br}{f(\Br)} = ( \PDi{x}{f}, \PDi{y}{f}, \PDi{z}{f} )\)

\begin{align}\label{eqn:relativisticElectrodynamicsL1:90}
M_1 \ddot{\Br}_1 &= -\PD{\Br_1}{} V(\Br_1 - \Br_2) \\
M_2 \ddot{\Br}_2 &= -\PD{\Br_2}{} V(\Br_1 - \Br_2)
\end{align}

In the unprimed frame, these are ``the laws of physics''.  Consider a primed frame \(O' : \Br_i' = \Br_i - \Bv_0 t\) (for \(i=1,2\)).  Taking derivatives we have \(\Bv_i' = \Bv_i' + \Bv_0\), and \(\dot{\Bv}_i' = \dot{\Bv}_i'\).

We note that the distance between the two particles is unchanged in the primed coordinate system

\begin{dmath}\label{eqn:relativisticElectrodynamicsL1:100}
\Br_1' - \Br_2' = \Br_1 - \Bv_0 t -( \Br_2 - \Bv_0 t ) = \Br_1 - \Br_2
\end{dmath}

Similarly

\begin{equation}\label{eqn:relativisticElectrodynamicsL1:110}
\PD{\Br_i}{} = \PD{(\Br_i' + \Bv_0 t)}{} = \PD{\Br_i'}{}
\end{equation}

Observe that the interaction \eqnref{eqn:relativisticElectrodynamicsL1:90} is unchanged by this change in coordinates.

\section{Enter electromagnetism}

%This interaction invariance had no known exceptions until the advent of Maxwellian electrodynamics.  In electrodynamics we wiggle or shake a charge and generate fields, and the generation of such fields produce electodynamic waves.
%This disturbance propaga

If the only interactions are \(1/r\) gravity and \(1/r\) Coulomb, Galilean relativity holds.  Electromagnetism came along and Maxwell's prediction that electromagnetic waves exist and propagate with speed

\begin{equation}\label{eqn:relativisticElectrodynamicsL1:200}
c \approx 3 \times 10^8 m/s
\end{equation}

(Note that in SI units \(c = 1/\sqrt{ \epsilon_0 \mu_0 }\)).

It was proposed that the speed of light was the speed in a medium (the ``aether'') through which electrodynamic waves propagate.  The idea was that the oscillations of this medium constitute electromagnetic waves.  Then ``c'' would be the speed of light with respect to that medium.  This medium would fill all space.

PICTURE: of gradient field, with aether velocity at different points.  Superimposed on this is a picture of the Earth's orbit, so that the velocity of the aether could be measured at different points of the earth's orbit by measuring the speed of light at different points in the orbit.

PICTURE: of interferometer.

We can study this effect by rotating this platform to measure at different points of the day and the year.

We note that the speed of the earth is approximately \(v_{+} = 150 \times 10^6 km/ 10^7 s \approx 15 km/s\).

Aside: It was not clear to me where these numbers came from.  \href{http://www.wolframalpha.com/input/?i=speed+of+the+earth}{Wolfram alpha says} that the Earth's orbital speed is approximately \(32 km/s\), although that is still within an order of magnitude of the number used in class.

The shift of fringes would then be \(v_{+} \approx (v_{+}/c)^2 \approx 10^{-8}\).  What Einstein did was to elevate the principle of relativity to one that applies to electromagnetism, but replacing the transformation relating frames to the Lorentz transformation, a transformation observed by Lorentz and Poincare that leave Maxwell's equations invariant.  Einstein did this by postulating that the speed of light is a constant in all frames, and we will see how this is the case.

\makequestion{Is not this true only outside of matter?}{question:relativisticElectrodynamicsL1:1}{

In matter we have electromagnetic wave propagation at speeds less than \(c\).

\paragraph{A:}  (paraphrasing)

We can consider the in-matter case to be a special case, treating collections of discreet particles as continuous approximations.  It is only as a side effect of these approximations that one produces the in-matter Maxwell's equation, and we will consider the ``vacuum'' Maxwell equation as always true, provided the points of interest do not fall exactly on any specific particle.

Yes we have speed of light different in media.  Example, speed of light in water is \(3/4\) vacuum speed due to high index of refraction.  Also note that we can have effects like an electron moving in water can constantly emit light.  This is called Cerenkov radiation.
}


