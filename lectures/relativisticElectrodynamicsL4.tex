%
% Copyright � 2012 Peeter Joot.  All Rights Reserved.
% Licenced as described in the file LICENSE under the root directory of this GIT repository.
%

%\chapter{Spacetime geometry, Lorentz transformations, Minkowski diagrams}
\index{Lorentz transformation}
\index{Minkowski diagram}
\label{chap:relativisticElectrodynamicsL4}
%\blogpage{http://sites.google.com/site/peeterjoot/math2011/relativisticElectrodynamicsL4.pdf}
%\date{Jan 18, 2011}
%
\paragraph{Reading}
%
Still covering chapter 1 material from the text \citep{landau1980classical},
\popcite{RelEM12-26.pdf}{lecture notes RelEM12-26.pdf}, and
%: invariance of finite intervals (25-26).
\popcite{RelEM27-44.pdf}{lecture notes RelEM27-44.pdf}.
%: analogy with rotations and derivation of Lorentz transformations (27-32); Minkowski space diagram of boosted frame (32.1); using the diagram to find length contraction (32.2) ; nonrelativistic limit of boosts (33).
%
\section{More spacetime geometry}
%
PICTURE: ct,x curvy worldline with tangent vector \(\Bv\).

In an inertial frame moving with \(\Bv\), whose origin coincides with momentary position of this moving observer \(ds^2 = c^2 {dt'}^2 = c^2 dt^2 - \Br^2\)

``proper time'' is
%
\begin{equation}\label{eqn:relativisticElectrodynamicsL4:10}
dt' = dt \sqrt{ 1 - \inv{c^2} \left( \frac{d\Br}{dt} \right)^2 } = dt \sqrt{ 1 - \frac{\Bv^2}{c^2}}
\end{equation}
%
We see that \(dt' < dt\) if \(v > 0\), so that \(\sqrt{1-\Bv^2/c^2} < 1\).

In a manifestly invariant way we define the proper time as
%
\begin{equation}\label{eqn:relativisticElectrodynamicsL4:20}
d\tau \equiv \frac{ds}{c}
\end{equation}
%
So that between worldpoints \(a\) and \(b\) the proper time is a line integral over the worldline
%
\begin{equation}\label{eqn:relativisticElectrodynamicsL4:30}
d\tau \equiv \inv{c} \int_a^b ds.
\end{equation}
%
PICTURE: We are splitting up the worldline into many small pieces and summing them up.

%HOLE IN LECTURE NOTES: ON PROPER TIME for ``length'' of straight vs. curved worldlines: TO BE REVISITED.
%Prof. Poppitz promised to revisit this again next time ... his notes are confusing him, and he would like to move on.
%
\section{Finite interval invariance}
%
Tomorrow we are going to complete the proof about invariance.  We have shown that light like intervals are invariant, and that infinitesimal intervals are invariant.  We need to put these pieces together for finite intervals.
%
\section{Deriving the Lorentz transformation}
\index{Lorentz transformation}

Let us find the coordinate transforms that leave \(s_{12}^2\) invariant.  This generalizes Galileo's transformations.

We would like to generalize rotations, which leave spatial distance invariant.  Such a transformation also leaves the spacetime interval invariant.

In Euclidean space we can generate an arbitrary rotation by composition of rotation around any of the \(xy, yz, zx\) axis.

For 4D Euclidean space we would form any rotation by composition of any of the 6 independent rotations for the 6 available planes.  For example with \(x,y,z,w\) axis we can rotate in any of the \(xy, xz, xw, yz, yw, zw\) planes.

For spacetime we can ``rotate'' in \(x,t\), \(y,t\), \(z,t\) ``planes''.  Physically this is motion space (boosting a position).
%
\paragraph{Consider a \texorpdfstring{\(x,t\)}{x,t} transformation}
%
The trick (that is in the notes) is to rewrite the time as an analytical continuation of the time coordinate, as follows
%
\begin{equation}\label{eqn:relativisticElectrodynamicsL4:40}
ds^2 = c^2 dt^2 - dx^2
\end{equation}
%
and write
%
\begin{equation}\label{eqn:relativisticElectrodynamicsL4:50}
t \rightarrow i \tau,
\end{equation}
so that the interval becomes
\begin{equation}\label{eqn:relativisticElectrodynamicsL4:60}
ds^2 = - (c^2 d\tau^2 + dx^2)
\end{equation}
%
Now we have a structure that is familiar, and we can rotate as we normally do.  Prof does not want to go through the details of this ``trickery'' in class, but says to see the notes.  The end result is that we can transform as follows

\begin{align}\label{eqn:relativisticElectrodynamicsL4:70}
x' &= x \cosh \psi + ct \sinh \psi \\
ct' &= x \sinh \psi + ct \cosh \psi
\end{align}

which is analogous to a spatial rotation
\begin{align}\label{eqn:relativisticElectrodynamicsL4:70b}
x' &= x \cos \alpha + y \sin \alpha \\
y' &= -x \sin \alpha + y \cos \alpha
\end{align}


There are some differences in sign as well, but the important feature to recall is that \(\cosh^2 x - \sinh^2 x = (1/4)( e^{2x} + e^{-2x} + 2 - e^{2x} - e^{-2x} + 2 ) = 1\).  We call these hyperbolic rotations, something that is simply a mathematical transformation.  Now we want to relate this to something physical.
%
\paragraph{Q: What is \(\psi\)?}
%
The origin of \(O\) has coordinates \((t, \BO)\) in the \(O\) frame.

PICTURE (pg 32): \(O'\) frame translating along \(x\) axis with speed \(v_x\).  We have
%
\begin{equation}\label{eqn:relativisticElectrodynamicsL4:80}
\frac{x'}{c t'} = \frac{v_x}{c}
\end{equation}
%
However, using \eqnref{eqn:relativisticElectrodynamicsL4:70} we have for the (spatial) origin

\begin{align}\label{eqn:relativisticElectrodynamicsL4:90}
x' &= ct \sinh \psi \\
ct' &= ct \cosh \psi,
\end{align}
so that
\begin{equation}\label{eqn:relativisticElectrodynamicsL4:100}
\frac{x'}{c t'} = \tanh \psi = \frac{v_x}{c}.
\end{equation}
%
Using

\begin{align}\label{eqn:relativisticElectrodynamicsL4:110}
\cosh \psi &= \inv{\sqrt{1 - \tanh^2 \psi}} \\
\sinh \psi &= \frac{\tanh \psi}{\sqrt{1 - \tanh^2 \psi}}
\end{align}

Performing all the gory substitutions one gets
\begin{align}\label{eqn:relativisticElectrodynamicsL4:120}
x' &=
\inv{\sqrt{1 - v_x^2/c^2}} x
+
\frac{v_x/c}{\sqrt{1 - v_x^2/c^2}} c t \\
y' &= y \\
z' &= z \\
ct' &=
\frac{v_x/c}{\sqrt{1 - v_x^2/c^2}} x
+
\inv{\sqrt{1 - v_x^2/c^2}} c t
\end{align}

PICTURE: Let us go to the more conventional case, where \(O\) is at rest and \(O'\) is moving with velocity \(v_x\).

We achieve this by simply changing the sign of \(v_x\) in \eqnref{eqn:relativisticElectrodynamicsL4:120} above.  This gives us

\begin{align}\label{eqn:relativisticElectrodynamicsL4:120b}
x' &=
\inv{\sqrt{1 - v_x^2/c^2}} x
-
\frac{v_x/c}{\sqrt{1 - v_x^2/c^2}} c t \\
y' &= y \\
z' &= z \\
ct' &=
-\frac{v_x/c}{\sqrt{1 - v_x^2/c^2}} x
+
\inv{\sqrt{1 - v_x^2/c^2}} c t
\end{align}

We want some shorthand to make this easier to write and introduce
%
\begin{equation}\label{eqn:relativisticElectrodynamicsL4:130}
\gamma = \inv{\sqrt{1 - v_x^2/c^2}},
\end{equation}
%
so that \eqnref{eqn:relativisticElectrodynamicsL4:120b} becomes

\begin{align}\label{eqn:relativisticElectrodynamicsL4:140}
x' &=  \gamma \left( x - \frac{v_x}{c} ct \right) \\
ct' &=  \gamma \left( ct - \frac{v_x}{c} x \right)
\end{align}

We started the class by saying these would generalize the Galilean transformations.  Observe that if we take \(c \rightarrow \infty\), we have \(\gamma \rightarrow 1\) and

\begin{align}\label{eqn:relativisticElectrodynamicsL4:150}
x' &= x - v_x t + O((v_x/c)^2) \\
t' &= t  + O(v_x/c)
\end{align}

This is how to remember the signs.  We want things to match up with the non-relativistic limit.
%
\paragraph{Q: How do lines of constant \(x'\) and \(ct'\) look like on the \(x,ct\) spacetime diagram?}
%
Our starting point (again) is
\begin{align}\label{eqn:relativisticElectrodynamicsL4:140b}
x' &=  \gamma \left( x - \frac{v_x}{c} ct \right) \\
ct' &=  \gamma \left( ct - \frac{v_x}{c} x \right).
\end{align}

What are the points with \(x' = 0\).  Those are the points where \(x = (v_x/c) c t\).  This is the \(ct' axis\).  That is the straight worldline

PICTURE: worldline of \(O'\) origin.

What are the points with \(ct' = 0\).  Those are the points where \(c t = x v_x/c\).  This is the \(x' axis\).

Lines that are parallel to the \(x'\) axis are lines of constant \(x'\), and lines parallel to \(ct'\) axis are lines of constant \(t'\), but the light cone is the same for both.
%
\paragraph{What is this good for?}
%
We have time to pick from either length contraction or non-causality (how to kill your grandfather).  How about length contraction.  We can use the diagram to read the \(x\) or \(ct\) coordinates, or examine causality, but it is hard to read off \(t'\) or \(x'\) coordinates.
%
%\section{Causality}
