%
% Copyright � 2012 Peeter Joot.  All Rights Reserved.
% Licenced as described in the file LICENSE under the root directory of this GIT repository.
%

%\chapter{Spacetime, events, worldlines, proper time, invariance}
\index{spacetime}
\index{events}
\index{worldlines}
\index{proper time}
\index{invariance}
\label{chap:relativisticElectrodynamicsL2}
%\blogpage{http://sites.google.com/site/peeterjoot/math2011/relativisticElectrodynamicsL2.pdf}
%\date{Jan 12, 2011}
%
\paragraph{Reading}
%
No reading from \citep{landau1980classical} appears to have been assigned, but relevant stuff can be found in chapter 1.

Covering \popcite{RelEM12-26.pdf}{lecture notes RelEM12-26.pdf}.
%, we have reading: spacetime, spacetime points, worldlines, interval (12-14); invariance of infinitesimal intervals (15-17).

\section{Einstein's relativity principle}
\index{relativity principle}

\begin{enumerate}

\item Replace Galilean transformations between coordinates in differential inertial frames with Lorentz transforms between \((\Bx, t)\).  Postulate that these constitute the symmetries of physics.  Recall that Galilean transformations are symmetries of the laws of non-relativistic physics.

\item Speed of light \(c\) is the same in all inertial frames.  Phrased in this form, relativity leads to ``relativity of simultaneity''.

PICTURE: Three people on a platform, at positions \(1,3,2\), all with equidistant separation.  This stationary frame is labeled \(O\).  1 and 2 flash light signals at the same time and in frame \(O\) the reception of the light signal by 3 is observed as arriving at 3 simultaneously.

Now introduce a moving frame with origin \(O'\) moving along the positive x axis.  To a stationary observer in \(O'\) the three guys are seen to be moving in the \(-x\) direction.  The middle guy (3) is eventually going to be seen to receive the light signal by this \(O'\) observer, but less time is required for the light to get from 1 to 3, and more time is required for the light to get from 2 to 1 (3 is moving away from the light according to the \(O'\) observer).  Because the speed of light is perceived as constant for all observers, the perception is then that the light must arrive at 3 at different times.

This is very non-intuitive since we are implicitly trained by our surroundings that Galilean transformations govern mechanical behavior.

In \(O\), 1 and 2 send light signals simultaneously while in \(O'\) 1 sends light later than 2.  The conclusion, rather surprisingly compared to intuition, is that simultaneity is relative.

\end{enumerate}
%
\makequestion{On symmetries.}{question:relativisticElectrodynamicsL2:1}{
\paragraph{Q:}
A comment made that the symmetries impose the dynamics, and the symmetries provided the form of the Lagrangian in classical physics.  My question to this comment was

``When we have transformations that leave the Lagrangian unchanged (a symmetry), we have a conserved current.  I have done various exercises to compute those currents for various types of transformations (translation, spacetime translation, rotation, boosts, ...), but can not think of a way that the Lagrangian itself is defined these sorts symmetries.  Can you elaborate on what you mean by this?''
%
\paragraph{A:}
%
Ah, you see, what I meant by that is the following. For a free particle \(\LL\) should depend on \(\Bx\), \(\dot{\Bx}\) and \(t\).  Homogeneity of space and time do not allow to have \(x\) and \(t\) dependence and isotropy of space only permits dependence on \(\Abs{\dot{\Bx}}\). Finally, Gallilean relativity only allows \(\LL = \dot{\Bx}^2\) (times a constant). (See \citep{landau1960classical} vol 1 or \href{http://www.physics.utoronto.ca/~poppitz/epoppitz/PHY354_files/CMpp13.1-27.pdf}{my notes on PHY354 website}, p. 23-27).

So what was used is:

\begin{itemize}
\item Having only dependence on \(x\) and \(dx/dt\).
\item Spacetime homogeneity/isotropy.
\item Gallilean relativity.
\end{itemize}

Similar story holds in relativity, as we will see.


}


