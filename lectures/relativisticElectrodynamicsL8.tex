%
% Copyright � 2012 Peeter Joot.  All Rights Reserved.
% Licenced as described in the file LICENSE under the root directory of this GIT repository.
%

%\chapter{Relativistic dynamics}
\label{chap:relativisticElectrodynamicsL8}
%\blogpage{http://sites.google.com/site/peeterjoot/math2011/relativisticElectrodynamicsL8.pdf}
%\date{Feb 1, 2011}
%
\paragraph{Reading}
%
Covering chapter 2 material from the text \citep{landau1980classical}, and
\popcite{RelEMpp52-56.pdf}{lecture notes RelEMpp52-56.pdf},
%: equation of motion, symmetries, and conserved quantities (energy-momentum 4 vector) from relativistic particle action.
and \popcite{RelEMpp56.1-73.pdf}{lecture notes RelEMpp56.1-73.pdf}.
%: comments on mass, energy, momentum, and massless particles (56.1-58); particles in external fields: Lorentz scalar field (59-62); reminder of a vector field under spatial rotations (63) and a Lorentz vector field (64-65) [Tuesday, Feb. 1]; the action for a relativistic particle in an external 4-vector field (65-66); the equation of motion of a relativistic particle in an external electromagnetic (4-vector) field (67,68,73) [Wednesday, Feb. 2]; mathematical interlude: (69-72): on 3x3 antisymmetric matrices, 3-vectors, and totally antisymmetric 3-index tensor - please read by yourselves, preferably by Wed., Feb. 2 class! (this is important, we will also soon need the 4-dimensional generalization)

\section{Finishing previous arguments on action and proper velocity}
\index{action}
\index{proper velocity}

For a free particle, our action is
%
\begin{equation}\label{eqn:relativisticElectrodynamicsL8:680}
\begin{aligned}
S
&= - m c \int ds  \\
&= -m c^2 \int dt \sqrt{1 - \frac{\Bv^2}{c^2}}
\end{aligned}
\end{equation}

Our Lagrangian is
%
\begin{equation}\label{eqn:relativisticElectrodynamicsL8:}
\LL = -m c^2 \sqrt{1 - \frac{\Bv^2}{c^2}} .
\end{equation}

We can also make a non-relativistic velocity approximation
%
\begin{equation}\label{eqn:relativisticElectrodynamicsL8:700}
\begin{aligned}
\LL
&= -m c^2 \sqrt{1 - \frac{\Bv^2}{c^2}} \\
&= -m c^2 \left( 1 - \inv{2} \frac{\Bv^2}{c^2} \right) + O((\Bv^2/c^2)^2) \\
&\approx \mathLabelBox{-m c^2 }{constant} +
\mathLabelBox
[
   labelstyle={below of=m\themathLableNode, below of=m\themathLableNode}
]
{\inv{2} m \Bv^2 }{Classical Lagrangian for free particle}.
\end{aligned}
\end{equation}

It is good to know that we recover the familiar Newtonian case when our velocities are small enough.

Our job is to vary the action between a pair of spacetime points
%
\begin{equation}\label{eqn:relativisticElectrodynamicsL8:20}
(t_a, \Bx_a) \rightarrow (t_b, \Bx_b)
\end{equation}

The equations of motion that result from this variation, or from the Euler-Lagrange equations that one can obtain from this variation, are
%
\begin{equation}\label{eqn:relativisticElectrodynamicsL8:40}
\frac{d}{dt} ( \gamma \Bv) = 0
\end{equation}

We argued last time, by evaluating the derivatives of \eqnref{eqn:relativisticElectrodynamicsL8:40}, and taking dot and cross products with \(\Bv\) that we also have
%
\begin{equation}\label{eqn:relativisticElectrodynamicsL8:60}
\frac{d\Bv}{dt} = 0
\end{equation}

Observe that since \(d\Bv/dt = 0\), we also have \(d\gamma/dt = 0\)
%
\begin{equation}\label{eqn:relativisticElectrodynamicsL8:720}
\begin{aligned}
\frac{d\gamma}{dt}
&=
\frac{d}{dt} \inv{\sqrt{1 - \frac{\Bv^2}{c^2}}} \\
&=
\frac{d}{dt} \inv{\left(1 - \frac{\Bv^2}{c^2}\right)^{3/2}} (-1/2) (2) (-\Bv \cdot \dot{\Bv}) /c^2 \\
&= 0.
\end{aligned}
\end{equation}

We can therefore combine the pair of equations (after adjusting both to have dimensions of velocity)
%
\begin{equation}\label{eqn:relativisticElectrodynamicsL8:80}
\begin{aligned}
\frac{d}{dt} ( \gamma \Bv ) &= 0 \\
\frac{d}{dt} ( \gamma c ) &= 0,
\end{aligned}
\end{equation}

into
%
\begin{equation}\label{eqn:relativisticElectrodynamicsL8:100}
u^i = (u^0, \Bu).
\end{equation}

Here
%
\begin{equation}\label{eqn:relativisticElectrodynamicsL8:120}
\begin{aligned}
u^0 &= \gamma \\
\Bu &= \gamma \frac{\Bv}{c}.
\end{aligned}
\end{equation}

Since we have \(du^i/dt = 0\), pre-multiplying this by \(\gamma/c\) does not change the equation, and we have
%
\begin{equation}\label{eqn:relativisticElectrodynamicsL8:140}
0 = \inv{c \sqrt{1 - \frac{\Bv^2}{c^2}}} \frac{du^i}{dt}.
\end{equation}

This now puts things in a nice invariant form, with no bias towards any specific observer's time coordinates, and we have for the free particle
%
\begin{equation}\label{eqn:relativisticElectrodynamicsL8:160}
\frac{d u^i}{ds} = 0.
\end{equation}

\section{Symmetries of spacetime translation invariance}
\index{invariance!spacetime translation}

The symmetries of \(S\) imply conservation laws.  Our action has \(SO(1,3) \times T^4 = \) Lorentz x spacetime translation \(\equiv\) Poincar\'{e} group of symmetries.

Consider quantities conserved due to \(T^4\) factor
%
\begin{equation}\label{eqn:relativisticElectrodynamicsL8:180}
\begin{aligned}
\Bx &\rightarrow \Bx + \Ba \qquad \mbox {where \(\Ba\) is constant} \\
t &\rightarrow t + \text{constant}
\end{aligned}
\end{equation}

Observe that the Lagrangian is not a function of \(\Bx\), or \(t\) explicitly
%
\begin{equation}\label{eqn:relativisticElectrodynamicsL8:200}
\LL(\Bx, \Bv, t) = - m c \sqrt{1 - \frac{\Bv^2}{c^2}}
=
\LL(\Bv) .
\end{equation}

A consequence from this, utilizing the Euler-Lagrange equations is that we have a zero for the time derivative of the generalized momentum \(\PDi{\Bv}{\LL}\)
%
\begin{equation}\label{eqn:relativisticElectrodynamicsL8:220}
\frac{d}{dt} \PD{\Bv}{\LL} = \PD{\Bx}{\LL} = 0.
\end{equation}

Let us calculate that generalized momentum
\begin{equation}\label{eqn:relativisticElectrodynamicsL8:740}
\begin{aligned}
\PD{\Bv}{\LL}
&=
\PD{\Bv}{} \left(
-m c^2 \sqrt{1 - \frac{\Bv^2}{c^2}} \right) \\
&=
\PD{\Bv}{} \left( -m c^2 \frac{(1/2)(-2) \Bv/c^2}{ \sqrt{1 - \frac{\Bv^2}{c^2}} } \right) \\
&=
m \frac{\Bv}{ \sqrt{1 - \frac{\Bv^2}{c^2}} } \\
\end{aligned}
\end{equation}

So our generalized momentum is
%
\begin{equation}\label{eqn:relativisticElectrodynamicsL8:241}
\PD{\Bv}{\LL} = m \Bv \gamma.
\end{equation}

Evaluating the Euler-Lagrange equations above we find
%
\begin{equation}\label{eqn:relativisticElectrodynamicsL8:760}
\begin{aligned}
0 &= \frac{d}{dt} \left( m \gamma \Bv \right) \\
&= \frac{d}{dt} \left( m c u^{1,2,3} \right) \\
\end{aligned}
\end{equation}

Recall that \(u^0 = \gamma\), and that \(d\gamma/dt = 0\), so we also have
%
\begin{equation}\label{eqn:relativisticElectrodynamicsL8:242}
\frac{d}{dt} \left( m c u^i \right) = 0
\end{equation}

and again with multiplication by \(\gamma/c\) we have a Lorentz invariant relation, mostly a consequence of spacetime translation invariance
%
\begin{equation}\label{eqn:relativisticElectrodynamicsL8:243}
\frac{d}{ds} \left( m c u^i \right) = 0.
\end{equation}

We define this quantity, the invariant quantity (a four vector), as the relativistic momentum
%
\begin{equation}\label{eqn:relativisticElectrodynamicsL8:244}
p^i = m c u^i.
\end{equation}

A relativistic particle is characterizes by a conserved 4 vector quantity \(p^i\) with
%
\begin{equation}\label{eqn:relativisticElectrodynamicsL8:260}
\begin{aligned}
p^0 &= m c \gamma \\
\Bp &= m \gamma \Bv \\
p^i &= (p^0, \Bp)
\end{aligned}
\end{equation}

\section{Time translation invariance}
\index{invariance!time translation}
%
\begin{equation}\label{eqn:relativisticElectrodynamicsL8:280}
\LL(\Bx, \Bv, t) = \LL(\Bv)
\end{equation}

However, it helps to consider the more general case
%
\begin{equation}\label{eqn:relativisticElectrodynamicsL8:300}
\LL(\Bx, \Bv, t) = \LL(\Bx, \Bv)
\end{equation}

since we have no explicit time dependence.
%
\begin{equation}\label{eqn:relativisticElectrodynamicsL8:780}
\begin{aligned}
\frac{d}{dt} \LL(\Bv)
&= \PD{\Bx}{\LL} \cdot \dot{\Bx} + \PD{\Bv}{\LL} \cdot \dot{\Bv} \\
&= \left( \frac{d}{dt} \PD{\Bv}{\LL} \right) \cdot \Bv + \PD{\Bv}{\LL} \cdot \frac{d\Bv}{dt} \\
&= \frac{d}{dt} \left( \PD{\Bv}{\LL} \cdot \Bv\right)
\end{aligned}
\end{equation}

Regrouping, to pull all the derivative terms together provides the conservation identity
%
\begin{equation}\label{eqn:relativisticElectrodynamicsL8:319}
\frac{d}{dt} \left(
\PD{\Bv}{\LL} \cdot \Bv - \LL
\right) = 0.
\end{equation}

This quantity \(\PD{\Bv}{\LL} \cdot \Bv - \LL\) is usually identified as the Hamiltonian \(H\), the energy, but we will call it \(E\) here.

In our case, with the relativistic free particle Lagrangian
%
\begin{equation}\label{eqn:relativisticElectrodynamicsL8:320}
\LL = -m c^2 \sqrt{ 1 - \frac{\Bv^2}{c^2} },
\end{equation}

we have
%
\begin{equation}\label{eqn:relativisticElectrodynamicsL8:800}
\begin{aligned}
E
&= \PD{\Bv}{\LL} \cdot \Bv - \LL \\
&= \Bv \cdot \left( m \inv{\sqrt{1 - \frac{\Bv^2}{c^2}}} \Bv \right) + m c^2 \sqrt{1 - \frac{\Bv^2}{c^2}} \\
&= \frac{m \Bv^2}{\sqrt{ 1 - \frac{\Bv^2}{c^2}}} + mc^2 \sqrt{ 1 - \frac{\Bv^2}{c^2}} \\
&= \frac{\Bv^2 + m c^2 \left(1 - \frac{\Bv^2}{c^2} \right)}{\sqrt{1 - \frac{\Bv^2}{c^2}}} \\
&= \frac{m c^2 }{\sqrt{ 1 - \frac{\Bv^2}{c^2} }}
\end{aligned}
\end{equation}

So we define, for the energy, a conserved quantity under time translation, we have
\boxedEquation{eqn:relativisticElectrodynamicsL8:340}{
E = \gamma m c^2 = \frac{ m c^2 }{\sqrt{1 - \frac{\Bv^2}{c^2}}}.
}
It is only with the \(\Bv \rightarrow 0\) that we recover the famous tee-shirt expression
%
\begin{equation}\label{eqn:relativisticElectrodynamicsL8:360}
E = m c^2.
\end{equation}

Since we also know (from the spacetime translation) that \(p^0 = m c \gamma = E/c\), we get another conserved quantity for free since \((p^0, \Bp)\) then is also a symmetry (i.e. thus a conserved quantity)
%
\begin{equation}\label{eqn:relativisticElectrodynamicsL8:820}
\begin{aligned}
p^0 &= m \gamma c = \frac{E}{c} \\
\Bp &= m \gamma \Bv
\end{aligned}
\end{equation}
%
\begin{equation}\label{eqn:relativisticElectrodynamicsL8:380}
p^i = ( p^0, \Bp )
\end{equation}

Note that the only ``mass'' you ever want to talk about is ``m''.  This is a Lorentz scalar, and we will not use the old notions that mass changes with velocity or ``relativistic mass''.

\section{Some properties of the four momentum}
\index{four momentum}

We have
%
\begin{equation}\label{eqn:relativisticElectrodynamicsL8:840}
\begin{aligned}
p^i p_i
&= (p^0)^2 - \Bp^2 \\
&= m c^2 \gamma^2 - m^2 \gamma^2 \Bv^2 \\
&= m c^2 \gamma^2 \left( 1 - \frac{\Bv^2}{c^2} \right)  \\
&= m^2 c^2
\end{aligned}
\end{equation}

So we have
\boxedEquation{eqn:relativisticElectrodynamicsL8:400}{
p^i p_i = m^2 c^2.
}

We say that the 4-vector \(p^i\) represents a particle with mass \(m\).

Since four momentum is a conserved quantity we can use this conservation property to study relativistic collisions

PICTURE: two particles colliding with two particles resulting (particles trajectories as arrows)
%
\begin{equation}\label{eqn:relativisticElectrodynamicsL8:420}
\mathLabelBox{p_1^i + p_2^i}{four momentum before} =
\mathLabelBox
[
   labelstyle={below of=m\themathLableNode, below of=m\themathLableNode}
]
{p_3^i + p_4^i}{four momentum after}
\end{equation}
%
\begin{equation}\label{eqn:relativisticElectrodynamicsL8:440}
\begin{aligned}
\Bp &= \frac{m \Bv}{\sqrt{ 1 - \frac{\Bv^2}{c^2} }} \rightarrow 0 \qquad \mbox{when \(m \rightarrow 0\)} \\
E &= \frac{m c^2}{\sqrt{ 1 - \frac{\Bv^2}{c^2} }} \rightarrow 0 \qquad \mbox{when \(m \rightarrow 0\)}
\end{aligned}
\end{equation}

except when \(\Abs{\Bv} = c\), where if you take \(m \rightarrow 0\) and \(\Abs{\Bv} = c\) you can get anything (any values) in such a limit (limit does not exist).

However, because
%
\begin{equation}\label{eqn:relativisticElectrodynamicsL8:460}
\frac{E^2}{c^2} - \Bp^2 = m^2 c^2 = 0
\end{equation}

when \(m \rightarrow 0\), \(E\) and \(\Bp\) for a massless particle must obey \(E = c \Abs{\Bp}\).

Massless particles like photons (and gravitons if/when eventually measured) have lightlike 4 momentum vectors
%
\begin{equation}\label{eqn:relativisticElectrodynamicsL8:480}
p^i p_i = 0
\end{equation}

Gravity waves have not been seen yet, but the LIGO and LISA (extremely large infraferometers) experiments are expected to get some results on this in the near future.

\section{Where are we?}

In the notes there is a review (see that on one's own).  We will also want to eventually deal with the conservation laws in four vector form, since it will illustrate how the electric and magnetic fields have to be transformed.  We will get to that eventually.

\section{Interactions}
\index{interaction}

In classical mechanics we have
%
\begin{equation}\label{eqn:relativisticElectrodynamicsL8:500}
\LL_{\text{kinetic}} = \inv{2} m \Bv^2
\end{equation}
%
\begin{equation}\label{eqn:relativisticElectrodynamicsL8:520}
\LL = \inv{2} m \Bv^2  - U(\Br)
\end{equation}
%FIXME: what did he mean by this equality?
%\begin{equation}\label{eqn:relativisticElectrodynamicsL8:540}
%U_{\text{internal}} = U(\Br)
%\end{equation}
Here \(U(\Br)\) is an external potential.
%
\begin{equation}\label{eqn:relativisticElectrodynamicsL8:560}
S = S_{\text{free}} + S_{\text{interaction}}  = \int dt \inv{2} m \Bv^2 + \int dt (-U(\Br, t))
\end{equation}
The quantity \(U(\Br, t)\) is what we call a potential field.

What is the simplest invariant field we can have?  The simplest possibility is to have a relativistic particle which interacts with an external \textunderline{Lorentz scalar field}.  We would imagine that this is due to some other particle or some distribution of other fields.

Recall that the scalar field under rotations (reminder)

PICTURE: a point with coordinates in a fixed and a rotated coordinate system

That point is
%
\begin{equation}\label{eqn:relativisticElectrodynamicsL8:580}
P = (x, y) = (x', y')
\end{equation}

Similarly we can define a scalar quantity (like temperature or the Coulomb potential) is then assigned a value at each point
%
\begin{equation}\label{eqn:relativisticElectrodynamicsL8:600}
\phi(x, y) = \phi'(x', y')
\end{equation}

The value of this scalar in the \(x,y\) coordinates system at point \(P\) equals the value of this scalar in the \(x',y'\) coordinates system at the same point \(P\).

A Lorentz scalar field is like this, but for an event \(P = (ct, x) = (ct', x')\) is the same.

So, we would have
%
\begin{equation}\label{eqn:relativisticElectrodynamicsL8:620}
\phi(ct, x) = \phi'(ct', x')
\end{equation}

The value of this scalar in the \(x,ct\) coordinates system at event \(P\) equals the value of this scalar in the \(x',ct'\) coordinates system at the same event \(P\) in the primed frame.

Our action would then be
%
\begin{equation}\label{eqn:relativisticElectrodynamicsL8:640}
S = - m c \int ds + g \int ds \phi(x^i)
\end{equation}

Here \(g\) is a coupling constant, also called the ``charge'' of a particle under that scalar field.

Note that unfortunately nature has not provided us with scalar fields that are stable enough to observe in classical interactions

We do however have some scalar particles
%
\begin{equation}\label{eqn:relativisticElectrodynamicsL8:660}
\pi^0, \pi^\pm, k^0, k^\pm
\end{equation}

These are unstable and short ranged.

The LHC is looking for another unstable short lived scalar field (the Higgs).  So we have to unfortunately study a more complicated field, a vector field.  We will do that next time.
