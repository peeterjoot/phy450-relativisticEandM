%
% Copyright © 2012 Peeter Joot.  All Rights Reserved.
% Licenced as described in the file LICENSE under the root directory of this GIT repository.
%
\section{Energy momentum conservation}
\index{conservation!energy momentum}
\index{energy momentum tensor}

We have defined
%
\begin{equation}\label{eqn:relativisticElectrodynamicsL21b:320}
\begin{array}{l l l}
\calE &= \frac{\BE^2 + \BB^2}{8\pi} & \mbox{Energy density} \\
\frac{\BS}{c^2} &= \inv{4 \pi c} \BE \cross \BB & \mbox{Momentum density}
\end{array}
\end{equation}

(where \(\BS\) was defined as the energy flow).

Dimensional analysis arguments and analogy with classical mechanics were used to motivate these definitions, as opposed to starting with the field action to find these as a consequence of a symmetry.  We also saw that we had a conservation relationship that had the appearance of a four divergence of a four vector.  With
%
\begin{equation}\label{eqn:relativisticElectrodynamicsL21b:340}
P^i = (\calU/c, \BS/c^2),
\end{equation}

that was
%
\begin{equation}\label{eqn:relativisticElectrodynamicsL21b:360}
\partial_i P^i = - \BE \cdot \Bj/c^2
\end{equation}

The left had side has the appearance of a Lorentz scalar, since it contracts two four vectors, but the right hand side is the continuum equivalent to the energy term of the Lorentz force law and cannot be a Lorentz scalar.  The conclusion has to be that \(P^i\) is not a four vector, and it is natural to assume that these are components of a rank 2 four tensor instead (since we have got just one component of a rank 1 four tensor on the RHS).  We want to know find out how the EM energy and momentum densities transform.

\paragraph{Classical mechanics reminder}

Recall that in particle mechanics when we had a Lagrangian that had no explicit time dependence
%
\begin{equation}\label{eqn:relativisticElectrodynamicsL21b:380}
\LL(q, \dot{q}, \cancel{t}),
\end{equation}

that energy resulted from time translation invariance.  We found this by taking the full derivative of the Lagrangian, and employing the EOM for the system to find a conserved quantity
%
\begin{equation}\label{eqn:relativisticElectrodynamicsL21b:560}
\begin{aligned}
\ddt{} \LL(q, \qdot)
&=
\PD{q}{\LL} \PD{t}{q}
+\PD{\qdot}{\LL} \PD{t}{\qdot} \\
&=
\ddt{} \left( \PD{\qdot}{\LL} \right) \qdot
+\PD{\qdot}{\LL} \ddot{q} \\
&=
\ddt{} \left( \PD{\qdot}{\LL} \qdot \right)
\end{aligned}
\end{equation}

Taking differences we have
%
\begin{equation}\label{eqn:relativisticElectrodynamicsL21b:400}
\ddt{} \left( \PD{\qdot}{\LL} \qdot -\LL \right) = 0,
\end{equation}

and we labeled this conserved quantity the energy
%
\begin{equation}\label{eqn:relativisticElectrodynamicsL21b:420}
\calE = \PD{\qdot}{\LL} \qdot -\LL
\end{equation}

\paragraph{Our approach from the EM field action}

Our EM field action was
%
\begin{equation}\label{eqn:relativisticElectrodynamicsL21b:500}
S = -\inv{16 \pi c} \int d^4 x F_{i j} F^{i j}.
\end{equation}

The squared field tensor \(F_{i j} F^{i j}\) only depends on the fields \(A^i(\Bx, t)\) or its derivatives \(\partial_j A^i(\Bx, t)\), and not on the coordinates \(\Bx, t\) themselves.  This is very similar to the particle action with no explicit time dependence
%
\begin{equation}\label{eqn:relativisticElectrodynamicsL21b:520}
S = \int dt \left( \frac{m \dot{q}^2}{2} + V(q) \right).
\end{equation}

For the particle case we obtained our conservation relationship by taking time derivatives of the Lagrangian.  These are very similar with the action having no explicit dependence on space or time, only on the field, so what will we get if we take the coordinate partials of the EM Lagrangian density?

We will chew on this tomorrow and calculate
%
\begin{equation}\label{eqn:relativisticElectrodynamicsL21b:540}
\PD{x^k}{} \Bigl( F_{i j} F^{i j} \Bigr)
\end{equation}

in full gory details.  We will find that instead of finding a single conserved quantity \(C^A(\Bx, t)\), we instead find a quantity that only changes through escape from the boundary of a surface.
