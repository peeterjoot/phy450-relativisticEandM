%
% Copyright � 2012 Peeter Joot.  All Rights Reserved.
% Licenced as described in the file LICENSE under the root directory of this GIT repository.
%

%\chapter{PHY450H1S.  Relativistic Electrodynamics Lecture 23 (Taught by Prof. Erich Poppitz).  Energy Momentum Tensor}
%\chapter{Energy Momentum Tensor}
\label{chap:relativisticElectrodynamicsL23}
%\blogpage{http://sites.google.com/site/peeterjoot/math2011/relativisticElectrodynamicsL23.pdf}
%\date{Mar 29, 2011}
%
\paragraph{Reading}
%
%FIXME:
%Covering chapter X material from the text \citep{landau1980classical}.
Covering \popcite{RelEMpp166-180.pdf}{lecture notes RelEMpp166-180.pdf}.
% pp. 173-178:} the force on a surface element of a body (177-178); a plane wave example (179-180).
%Next week (last topic): attempt to go to the next order \((v/c)^3\) - radiation damping, the limitations of classical electrodynamics, and the relevant time/length/energy scales.
%
\section{Recap.}
%
Last time we found that spacetime translation invariance led to the four conservation relations
%
\begin{equation}\label{eqn:relativisticElectrodynamicsL23:10}
\partial_k T^{k m} = 0.
\end{equation}
%
where
%
\begin{equation}\label{eqn:relativisticElectrodynamicsL23:30}
T^{k m} = \inv{4 \pi} \left( F^{k j} F^{m l} g_{j l} + \inv{4} g^{k m} F_{i j} F^{i j} \right).
\end{equation}
%
last time we found for \(m = 0\)
%
\begin{equation}\label{eqn:relativisticElectrodynamicsL23:50}
\inv{c} \PD{t}{} T^{0 0} + \PD{x^\alpha}{} T^{\alpha 0} = 0.
\end{equation}
%
Here
%
\begin{equation}\label{eqn:relativisticElectrodynamicsL23:70}
\begin{array}{l l l}
T^{0 0} &= \inv{8 \pi} (\BE^2 + \BB^2) &= \mbox{energy density} \\
c T^{\alpha 0} &= \BS^\alpha &= \mbox{energy flux}
\end{array}.
\end{equation}
%
\section{Spatial components of \texorpdfstring{\(T^{k m.}\)}{Tkm}}
%
Now for \(m = 1,2,3\) we write
%
\begin{equation}\label{eqn:relativisticElectrodynamicsL23:90}
\partial_k T^{k \alpha} = 0.
\end{equation}
%
so we write
%
\begin{equation}\label{eqn:relativisticElectrodynamicsL23:110}
\PD{t}{} \frac{\BS^\alpha}{c^2} + \partial_\beta T^{\beta \alpha} = 0.
\end{equation}
%
Recall that we argued that
%
\begin{equation}\label{eqn:relativisticElectrodynamicsL23:130}
\frac{\BS}{c^2} = \text{momentum density}.
\end{equation}
%
(it also comes from Noether's theorem).
%
\begin{equation}\label{eqn:relativisticElectrodynamicsL23:150}
\PD{t}{} \left( \frac{T^{0\alpha}}{c} \right) + \PD{x^\beta}{} T^{\beta \alpha} = 0.
\end{equation}
%
or
%
\begin{equation}\label{eqn:relativisticElectrodynamicsL23:170}
\PD{t}{} \left( \frac{\BS^{\alpha}}{c^2} \right) + \PD{x^\beta}{} T^{\beta \alpha} = 0.
\end{equation}
%
Integrating over \(V\) we have
%
\begin{equation}\label{eqn:relativisticElectrodynamicsL23:490}
\begin{aligned}
\PD{t}{} \int_V d^3\Bx \left( \frac{\BS^{\alpha}}{c^2} \right)
&= -\int_V d^3\Bx \PD{x^\beta}{} T^{\beta \alpha} \\
&= -\int_V d^3\Bx \spacegrad \cdot (\Be_\beta T^{\beta \alpha}) \\
&= -\int_{\partial V} d^2 \sigma (\Bn \cdot \Be_\beta) T^{\beta \alpha} \\
&\equiv -\int_{\partial V} d^2 \sigma^\beta T^{\beta \alpha}.
\end{aligned}
\end{equation}
%
We write this as
%
\begin{equation}\label{eqn:relativisticElectrodynamicsL23:190}
\PD{t}{} \left( \text{momentum of EM fields in V} \right)^\alpha = - \int_{\partial V} d^2 \sigma^\beta T^{\beta \alpha}.
\end{equation}
%
and describe our spatial tensor components as
%
\begin{equation}\label{eqn:relativisticElectrodynamicsL23:210}
T^{\beta\alpha} = \text{flux of \(\alpha\)-th momentum through a unit area \(\perp \beta\)},
\end{equation}
%
where
%
\begin{equation}\label{eqn:relativisticElectrodynamicsL23:510}
\begin{aligned}
T^{\alpha\beta}
&= \inv{4\pi} \left( -F^{\alpha j} F^{\beta m} g_{m j} + \inv{4} g^{\alpha\beta} F^{ij}F_{ij} \right) \\
&= \inv{4\pi} \left(
-F^{\alpha 0} F^{\beta 0}
+F^{\alpha \sigma} F^{\beta \sigma}
- \inv{4} \delta^{\alpha\beta} 2 (\BB^2 - \BE^2) \right) \\
&= \inv{4\pi} \left(
-E^\alpha E^\beta
+\sum_\sigma
(\epsilon^{\mu\alpha\sigma} B^\mu)
(\epsilon^{\nu\beta\sigma} B^\nu)
- \inv{2} \delta^{\alpha\beta} (\BB^2 - \BE^2) \right) \\
&= \inv{4\pi} \biglr{
-E^\alpha E^\beta
+\sum_{\mu,\nu}
\mathLabelBox
[
   labelstyle={xshift=2cm},
   linestyle={out=270,in=90, latex-}
]
{\delta^{\mu\alpha}_{[\nu\beta]}
B^\mu B^\nu
}{$(\delta^{\alpha \beta} \delta^{\mu \nu} - \delta^{\alpha \nu} \delta^{\beta \mu} )
B^\mu B^\nu
=
\delta^{\alpha\beta}\BB^2 - B^\alpha B^\beta
$}
+ \inv{2} \delta^{\alpha\beta} (\BE^2 - \BB^2) } \\
&=
- \inv{4\pi} \left(
E^\alpha E^\beta
+B^\alpha B^\beta
+ \delta^{\alpha\beta}
\left(
-\frac{\BE^2}{2} + \frac{\BB^2}{2} - \BB^2 \right)
\right) \\
&=
- \inv{4\pi} \left(
E^\alpha E^\beta
+B^\alpha B^\beta
- \frac{\delta^{\alpha\beta} }{2}
\left(
\BE^2 + \BB^2
\right)
\right).
\end{aligned}
\end{equation}
%
We define
%
\begin{equation}\label{eqn:relativisticElectrodynamicsL23:230}
\sigma^{\alpha\beta}
=
-T^{\alpha\beta}
=
 \inv{4\pi} \left(
E^\alpha E^\beta
+B^\alpha B^\beta
- \frac{\delta^{\alpha\beta} }{2}
\left(
\BE^2 + \BB^2
\right)
\right).
\end{equation}
%
This is the Maxwell stress tensor.  Maxwell apparently derived this without any use of four vectors or symmetry arguments.  I had be curious what his arguments were and how he related this to the Lorentz force?

FIXME: latex: hard to layout this in gigantic matrix form without it wrapping.  Can an equation be rotated in latex?
%In gigantic matrix form, where symmetric opposites are omitted, we have for \(\Norm{T^{ij}}\)
%
\begin{equation}\label{eqn:relativisticElectrodynamicsL23:250}
\begin{aligned}
&T^{0j} =
\begin{bmatrix}
\inv{8 \pi}(\BE^2 + \BB^2) & \inv{4\pi} (\BE \cross \BB)^x & \inv{4\pi} (\BE \cross \BB)^y & \inv{4\pi} (\BE \cross \BB)^z
\end{bmatrix}
\\
&T^{1j} =
\begin{bmatrix}
.
& -\inv{4 \pi}\left(E_x^2 + B_x^2 - \inv{2}(\BE^2 + \BB^2)\right)
& -\inv{4 \pi}\left(E_x E_y + B_x B_y \right)
& -\inv{4 \pi}\left(E_x E_z + B_x B_z \right)
\end{bmatrix} \\
&T^{2j} =
\begin{bmatrix}
. & . & -\inv{4 \pi}\left(E_y^2 + B_y^2 - \inv{2}(\BE^2 + \BB^2)\right) & -\inv{4 \pi}\left(E_y E_z + B_y B_z \right)
\end{bmatrix}
\\
&T^{3j} =
\begin{bmatrix}
. & . & . & -\inv{4 \pi}\left(E_z^2 + B_z^2 - \inv{2}(\BE^2 + \BB^2)\right)
\end{bmatrix}.
\end{aligned}
\end{equation}
%
In words this matrix is
\begin{equation}\label{eqn:relativisticElectrodynamicsL23:270}
\begin{bmatrix}
\text{energy density} & \inv{c} (\text{energy flux in \(\xcap\)}) & \inv{c} (\text{energy flux in \(\ycap\)}) &
%\inv{c} (\text{energy flux in \(\zcap\)}) \\
\cdots \\
c \times (\text{momentum density})^x
& (\text{momentum})^x \text{flux in \(\xcap\)}
& (\text{momentum})^x \text{flux in \(\ycap\)}
&
%(\text{momentum})^x \text{flux in \(\zcap\)}
\cdots
\\
c \times (\text{momentum density})^y
& (\text{momentum})^y \text{flux in \(\xcap\)}
& (\text{momentum})^y \text{flux in \(\ycap\)}
&
%(\text{momentum})^y \text{flux in \(\zcap\)} \\
\cdots
\\
c \times (\text{momentum density})^z
& (\text{momentum})^z \text{flux in \(\xcap\)}
& (\text{momentum})^z \text{flux in \(\ycap\)}
&
%(\text{momentum})^z \text{flux in \(\zcap\)} \\
\cdots
\\
\end{bmatrix}.
\end{equation}
%
\section{On the geometry.}
%
PICTURE: rectangular area with normal \(\alphacap\), and area \(d^2 \sigma^\alpha \perp \alphacap\).

\(T^{\alpha\beta}\) is the amount of \(P^\beta\) that goes through unit area \(\perp \alphacap\) in unit time.  \(d^2 \sigma^\alpha T^{\alpha \beta}\) (no sum) is the amount of \(P^\beta\) through \(d^2 \sigma^\alpha\) in unit time.

For a general surface element

PICTURE: normal \(\Bn\) decomposed into perpendicular components \(\hat{\Balpha}\), \(\hat{\Bbeta}\), with respective area elements \(d^2 \sigma^\alpha\) and \(d^2 \sigma^\beta\).

PICTURE: triangulated area element decomposed into three perpendicular areas with their respective normals.

We have
%
\begin{equation}\label{eqn:relativisticElectrodynamicsL23:290}
\int d^3 \Bx \PD{x^\alpha}{} T^{\alpha\beta} = \int d^3 \Bx \spacegrad \cdot ( \Be_\beta T^{\alpha\beta} ) = \int d^2 \sigma (\Bn \cdot \Be_\beta) T^{\alpha \beta}.
\end{equation}
%
Write
%
\begin{equation}\label{eqn:relativisticElectrodynamicsL23:310}
d^2 \Bsigma = d^2 \sigma \Bn = \sum_\alpha d^2 \sigma n^\alpha \Be_\alpha,
\end{equation}
%
where \(\Bn = (n^1, n^2, n^3)\).  The amount of \(\beta\) momentum that goes through \(d^2 \sigma\) in unit time is
%
\begin{equation}\label{eqn:relativisticElectrodynamicsL23:330}
\sum_\alpha d^2\sigma^\alpha T^{\alpha\beta}.
\end{equation}
%
If this is greater than zero, this is a flow in the \(\Bn\) direction, whereas if less than zero the momentum flows in the \(-\Bn\) direction.

If \(d^2 \Bsigma\) is at the surface of the body, the rate of flow of \((\text{momentum})^\beta\) through \(d^2 \Bsigma\) is the \((\text{force})^\beta\) that acts on this element.

PICTURE: arbitrary surface depicted with an inwards normal \(\Bn\).

For this surface with the inwards normal we can write
%
\begin{equation}\label{eqn:relativisticElectrodynamicsL23:350}
df^\beta = \sum_\alpha d^2\sigma^\alpha T^{\alpha\beta}.
\end{equation}
%
The \((\text{force})^\beta\) acting on the \(d^2\Bsigma\) surface element.  With an outwards normal we can write this in terms of the Maxwell stress tensor, which has an inverted sign
%
\begin{equation}\label{eqn:relativisticElectrodynamicsL23:370}
df^\beta = \sum_\alpha d^2\sigma^\alpha \sigma^{\alpha\beta}.
\end{equation}
%
To find the force on the body we want
%
\begin{equation}\label{eqn:relativisticElectrodynamicsL23:390}
F^\beta = \oint_{\text{surface of body with inwards normal orientation}} d^2 \sigma^\alpha T^{\alpha\beta}.
\end{equation}
%
We can calculate the EM force on any body.  We need to know \(T^{\alpha\beta}\) on the surface, so we need the EM field on this boundary.
%
\makeexample{Wall absorbing all radiation hitting it.}{example:relativisticElectrodynamicsL23:1}{
%
With propagation direction \(\Bp\) along the \(\xcap\) direction, and mutually perpendicular \(\BE\) and \(\BB\).
%
\begin{equation}\label{eqn:relativisticElectrodynamicsL23:410}
c T^{xx} = \text{amount of \(P^x\) going in \(\xcap\) unit area \(\perp \xcap\) in unit time}.
\end{equation}
%
\begin{equation}\label{eqn:relativisticElectrodynamicsL23:430}
\begin{aligned}
df^x &= T^{x x} d^2 \sigma^x \\
df^y &= T^{y y} d^2 \sigma^y.
\end{aligned}
\end{equation}
%
with \(c p = \omega\), our fields are
%
\begin{equation}\label{eqn:relativisticElectrodynamicsL23:450}
\begin{aligned}
E_y &= p \beta \sin( c p t - p x ) \\
B_z &= p \beta \sin( c p t - p x ).
\end{aligned}
\end{equation}
%
\begin{equation}\label{eqn:relativisticElectrodynamicsL23:530}
\begin{aligned}
T^{x x}
&= -\inv{4 \pi} \left(
\cancel{(E^x)^2}
+\cancel{(B^x)^2}
-\inv{2} (\BE^2 + \BB^2) \right) \\
&= \inv{8 \pi} \left(
(E^y)^2
+(B^z)^2 \right) \\
&= \frac{p^2 \beta^2}{8 \pi} \sin^2 ( c p t - p x ).
\end{aligned}
\end{equation}
%
\begin{equation}\label{eqn:relativisticElectrodynamicsL23:470}
T^{y x} = - \inv{4 \pi} \left(
\cancel{ E^x } E^y + \cancel{ B^x B^y } \right) = 0.
\end{equation}
%
The off diagonal \(T^{\alpha \beta}\) components vanish since we have no non-zero pair of \(E_\alpha E_\beta\) or \(B_\alpha B_\beta\).  Our other two diagonal terms are also zero
%
\begin{equation}\label{eqn:relativisticElectrodynamicsL23:550}
\begin{aligned}
T^{y y}
&= -\inv{4 \pi} \left(
(E^y)^2
+\cancel{(B^y)^2}
-\inv{2} (\BE^2 + \BB^2) \right) \\
&= - \inv{4 \pi} p^2 \beta^2 \sin^2 ( c p t - p x ) \left( 1 - \inv{2} - \inv{2} \right) \\
&= 0.
\end{aligned}
\end{equation}
%
\begin{equation}\label{eqn:relativisticElectrodynamicsL23:570}
\begin{aligned}
T^{y y}
&= -\inv{4 \pi} \left(
+\cancel{(E^z)^2}
(B^z)^2
-\inv{2} (\BE^2 + \BB^2) \right) \\
&= - \inv{4 \pi} p^2 \beta^2 \sin^2 ( c p t - p x ) \left( 1 - \inv{2} - \inv{2} \right)  \\
&= 0.
\end{aligned}
\end{equation}
%
For non-perpendicular reflection we have the same deal.

PICTURE: reflection off of a wall, with reflection coefficient \(R\).
}
