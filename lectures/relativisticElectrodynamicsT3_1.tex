%
% Copyright © 2012 Peeter Joot.  All Rights Reserved.
% Licenced as described in the file LICENSE under the root directory of this GIT repository.
%

%\chapter{Relativistic motion in constant uniform electric or magnetic fields}
\index{electric field!uniform}
\index{magnetic field!uniform}
\label{chap:relativisticElectrodynamicsT3}
%\blogpage{http://sites.google.com/site/peeterjoot/math2011/relativisticElectrodynamicsT3.pdf}
%\date{Feb 3, 2011}
%
\makeproblem{Motion in an constant uniform Electric field}{pr:relativisticElectrodynamicsT3_1:1}{
Given
\begin{equation}\label{eqn:relativisticElectrodynamicsT3:}
\BE = E \xcap,
\end{equation}
we want to solve the problem
\begin{equation}\label{eqn:relativisticElectrodynamicsT3:20}
\BF = \frac{d\Bp}{dt} =
e \left( \BE + \frac{\Bv}{c} \cross \BB \right) = e \BE.
\end{equation}
Unlike second year classical physics, we will use relativistic momentum, so for only a constant electric field, our Lorentz force equation to solve becomes
%
\begin{equation}\label{eqn:relativisticElectrodynamicsT3:40}
\frac{d\Bp}{dt} = \frac{d (m \gamma \Bv)}{dt} = e \BE.
\end{equation}
} % makeproblem
%
\makeanswer{pr:relativisticElectrodynamicsT3_1:1}{
In components this is
%
\begin{equation}\label{eqn:relativisticElectrodynamicsT3:60}
\begin{aligned}
\pdot_x &= e E \\
\pdot_y &= \text{constant}.
\end{aligned}
\end{equation}
%
Integrating the \(x\) component we have
\begin{equation}\label{eqn:relativisticElectrodynamicsT3:65}
e E t + p_x(0)
=
\frac{m \xdot}{\sqrt{1 - (\xdot^2 + \ydot^2)/c^2}}.
\end{equation}
%
If we let \(p_x(0) = 0\), square and rearrange a bit we have
%
\begin{equation}\label{eqn:relativisticElectrodynamicsT3:70}
\frac{m^2}{(e E t)^2} \xdot^2 = 1 - \frac{\xdot^2 + \ydot^2}{c^2}.
\end{equation}
For
\begin{equation}\label{eqn:relativisticElectrodynamicsT3:80}
\xdot^2 = \frac{c^2 - \ydot^2}{1 + (\frac{mc}{eEt})^2}.
\end{equation}
%
Now for the \(y\) components, with \(p_y(0) = p_0\), our equation to solve is
%
\begin{equation}\label{eqn:relativisticElectrodynamicsT3:82}
\frac{m \ydot}{\sqrt{1 - (\xdot^2 + \ydot^2)/c^2}} = p_0.
\end{equation}
Squaring this one we have
\begin{equation}\label{eqn:relativisticElectrodynamicsT3:84}
\frac{c^2 m^2}{p_0^2} \ydot^2 = c^2 - \xdot^2 - \ydot^2,
\end{equation}
and
\begin{equation}\label{eqn:relativisticElectrodynamicsT3:86}
\ydot^2 = \frac{ c^2 - \xdot^2}{1 + \frac{m^2 c^2}{p_0^2}}.
\end{equation}
Observe that our energy is
%
\begin{equation}\label{eqn:relativisticElectrodynamicsT3:100}
\calE^2 = p^2 c^2 + m^2 c^4,
\end{equation}
and for \(t=0\)
\begin{equation}\label{eqn:relativisticElectrodynamicsT3:100b}
\calE_0^2 = p_0^2 c^2 + m^2 c^4.
\end{equation}
%
We can then write
%
\begin{equation}\label{eqn:relativisticElectrodynamicsT3:120}
\ydot^2 = \frac{ c^2 p_0^2 (c^2 - \xdot^2)}{ \calE_0^2 }.
\end{equation}
%
Some messy substitution, using \eqnref{eqn:relativisticElectrodynamicsT3:80}, yields
\boxedEquation{eqn:relativisticElectrodynamicsT3:140}{
\begin{aligned}
\xdot &= \frac{c^2 e E t}{\sqrt{ \calE_0^2 + (e c E t)^2 }} \\
\ydot &= \frac{c^2 p_0 }{\sqrt{ \calE_0^2 + (e c E t)^2 }}.
\end{aligned}
}

Solving for \(x\) we have
%
\begin{equation}\label{eqn:relativisticElectrodynamicsT3:220}
x(t) = c^2 e E \int \frac{dt' t'}{\sqrt{ \calE_0^2 + (e c E t')^2 }}.
\end{equation}
%
Can solve with hyperbolic substitution or
%
\begin{equation}\label{eqn:relativisticElectrodynamicsT3:240}
x(t) = c^2 e E \int \frac{dt' t'}{\sqrt{ \calE_0^2 + (e c E t')^2 }}.
\end{equation}
%
\begin{equation}\label{eqn:relativisticElectrodynamicsT3:260}
d(u^2) = 2 u du \implies u du = \inv{2} d(u^2).
\end{equation}
%
\begin{equation}\label{eqn:relativisticElectrodynamicsT3:280}
x(t) = \frac{c^2 e E}{2 \calE_0} \int \frac{d (u^2)}{\sqrt{ 1 + \left(\frac{e c E}{\calE_0}\right)^2 u^2 }}.
\end{equation}
%
Now we have something of the form
%
\begin{equation}\label{eqn:relativisticElectrodynamicsT3:290}
\int \frac{d v}{\sqrt{1 + a v}} = \frac{2}{a} \sqrt{1 + a v},
\end{equation}
%
so our final solution for \(x(t)\) is
%
\begin{equation}\label{eqn:relativisticElectrodynamicsT3:300}
x(t) = \inv{e E} \sqrt{ \calE_0^2 + (e c E t)^2 }.
\end{equation}
%
or
%
\begin{equation}\label{eqn:relativisticElectrodynamicsT3:320}
x^2 - c^2 t^2 = \frac{\calE_0^2}{ e^2 E^2 } = a^{-2}.
\end{equation}
%
Now for \(y(t)\) we have
%
\begin{equation}\label{eqn:relativisticElectrodynamicsT3:340}
y(t) = c^2 p_0 \int \frac{dt}{ \sqrt{\calE_0^2 + (e c E t)^2 }}.
\end{equation}
%
\begin{equation}\label{eqn:relativisticElectrodynamicsT3:360}
t = \frac{\calE_0}{ e c E} \sinh(u).
\end{equation}
%
\begin{equation}\label{eqn:relativisticElectrodynamicsT3:380}
dt = \frac{\calE_0}{ e c E} \cosh(u) du.
\end{equation}
%
\begin{equation}\label{eqn:relativisticElectrodynamicsT3:970}
\begin{aligned}
y(t)
&= \frac{c^2 p_0}{\calE_0} \int \frac{dt}{\sqrt{1 + (\frac{e c E}{\calE_0})^2 t^2 }} \\
&= \frac{c^2 p_0}{\calE_0}
\frac{\calE_0}{ e c E}
\int \frac{ du \cosh u }{\sqrt{1 + \sinh^2 u }} \\
&= \frac{c p_0}{ e E} u.
\end{aligned}
\end{equation}
%
A final bit of substitution, including a sort of odd seeming parametrization of \(x\) in terms of \(y\) in terms of \(t\), we have
\boxedEquation{eqn:relativisticElectrodynamicsT3:400}{
\begin{aligned}
y(t) &= \frac{c p_0}{ e E} \sinh^{-1} \left( \frac{e c E t}{\calE_0} \right) \\
x(y) &= \frac{\calE_0}{c E \cosh \left( \frac{y e E }{ c p_0} \right) }.
\end{aligned}
}
%
\paragraph{Checks}
%
FIXME: check the checks.
%
\begin{equation}\label{eqn:relativisticElectrodynamicsT3:420}
v \rightarrow c, t \rightarrow \infty.
\end{equation}
%
\begin{equation}\label{eqn:relativisticElectrodynamicsT3:440}
v << c, t \rightarrow 0.
\end{equation}
%
\begin{equation}\label{eqn:relativisticElectrodynamicsT3:990}
\begin{aligned}
m v_x &= e E t + ... \\
x &\sim t^2.
\end{aligned}
\end{equation}
%
\begin{equation}\label{eqn:relativisticElectrodynamicsT3:460}
m v_y = p_0 \rightarrow y \sim t.
\end{equation}
%
\begin{equation}\label{eqn:relativisticElectrodynamicsT3:480}
x(y) \sim y^2.
\end{equation}
%
(a parabola)
%
\paragraph{An alternate way}
%
There is also a tricky way (as in the text), with
%
\begin{equation}\label{eqn:relativisticElectrodynamicsT3:160}
\begin{aligned}
\Bp &= m \gamma \Bv  \\
\calE &= \gamma m c^2.
\end{aligned}
\end{equation}
%
We can solve this for \(\Bp\)
%
\begin{equation}\label{eqn:relativisticElectrodynamicsT3:1010}
\begin{aligned}
m \gamma &= \frac{\Bp \cdot \Bv}{\Bv^2} = \frac{\calE}{c^2} \\
\Bp \cross \Bv &= 0.
\end{aligned}
\end{equation}
%
With the cross product zero, \(\Bp\) has only a component in the direction of \(\Bv\), and we can invert to yield
%
\begin{equation}\label{eqn:relativisticElectrodynamicsT3:180}
\Bp = \frac{\calE \Bv}{c^2}.
\end{equation}
%
This implies
%
\begin{equation}\label{eqn:relativisticElectrodynamicsT3:200}
\xdot = \frac{c^2 p_x}{\calE},
\end{equation}
%
and one can work from there as well.

} % makeanswer
