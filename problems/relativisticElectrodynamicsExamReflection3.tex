%
% Copyright � 2012 Peeter Joot.  All Rights Reserved.
% Licenced as described in the file LICENSE under the root directory of this GIT repository.
%

%\label{chap:relativisticElectrodynamicsExamReflection}
%\blogpage{http://sites.google.com/site/peeterjoot/math2011/relativisticElectrodynamicsExamReflection.pdf}
%\date{April 13, 2011}

\makeproblem{Pion decay}{pr:relativisticElectrodynamicsExamReflection:3}{

FIXME: What was the exact question?  Looks like calculating the muon energy was desired, but this write up is confused, with discussion of multiple problems.

} % makeproblem

\makeanswer{pr:relativisticElectrodynamicsExamReflection:3}{


The problem above is very much like a midterm problem we had, so there was no justifiable excuse for messing up on it.  That midterm problem was to consider the split of a pion at rest into a neutrino (massless) and a muon, and to calculate the energy of the muon.  That one also follows the same pattern, a calculation of four momentum conservation, say

\begin{equation}\label{eqn:relativisticElectrodynamicsExamReflection:720}
(m_\pi c, 0) = \Hbar \frac{\omega}{c}(1, \kcap) + ( \calE_\mu/c, \Bp_\mu ).
\end{equation}

Here \(\omega\) is the frequency of the massless neutrino.  The massless nature is encoded by a four momentum that squares to zero, which follows from \((1, \kcap) \cdot (1, \kcap) = 1^2 - \kcap \cdot \kcap = 0\).

When I did this problem on the midterm, I perversely put in a scattering angle, instead of recognizing that the particles must scatter at 180 degree directions since spatial momentum components must also be preserved.  This and the combination of trying to work in spatial quantities led to a mess and I did not get the end result in anything that could be considered tidy.

The simple way to do this is to just rearrange to put the null vector on one side, and then square.  This gives us

\begin{equation}\label{eqn:relativisticElectrodynamicsExamReflection:1370}
\begin{aligned}
0 &=
\left(\Hbar \frac{\omega}{c}(1, \kcap) \right) \cdot
\left(\Hbar \frac{\omega}{c}(1, \kcap) \right) \\
&=
\left( (m_\pi c, 0) - ( \calE_\mu/c, \Bp_\mu ) \right) \cdot \left( (m_\pi c, 0) - ( \calE_\mu/c, \Bp_\mu ) \right) \\
&=
{m_\pi}^2 c^2 + {m_\nu}^2 c^2 - 2 (m_\pi c, 0) \cdot ( \calE_\mu/c, \Bp_\mu ) \\
&=
{m_\pi}^2 c^2 + {m_\nu}^2 c^2 - 2 m_\pi \calE_\mu
\end{aligned}
\end{equation}

A final re-arrangement gives us the muon energy

\begin{equation}\label{eqn:relativisticElectrodynamicsExamReflection:740}
\calE_\mu = \inv{2} \frac{ {m_\pi}^2 + {m_\nu}^2 }{m_\pi} c^2
\end{equation}

} % makeanswer
