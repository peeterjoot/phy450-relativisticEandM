%
% Copyright � 2012 Peeter Joot.  All Rights Reserved.
% Licenced as described in the file LICENSE under the root directory of this GIT repository.
%

\label{chap:relElectroDynProblemSet6}
%\blogpage{http://sites.google.com/site/peeterjoot/math2011/relElectroDynProblemSet6.pdf}
%\date{Mar 25, 2011}

%
\makeproblem{Monochromatic stress energy tensor}{pr:relElectroDynProblemSet6:2}{
%
\makesubproblem{}{pr:relElectroDynProblemSet6:2:a}
%
Show that the energy momentum tensor of a plane monochromatic wave with 4-vector
%
\begin{equation}\label{eqn:relativisticElectrodynamicsPS6:10}
k^i = \left( \frac{\omega}{c}, \Bk \right),
\end{equation}
%
and energy density \(\calE\) can be written as
%
\begin{equation}\label{eqn:relativisticElectrodynamicsPS6:20}
T^{i j} = \frac{\calE c^2}{\omega^2} k^i k^j.
\end{equation}
%
\makesubproblem{}{pr:relElectroDynProblemSet6:2:b}
%
Can one conclude now that \(\frac{\calE c^2}{\omega^2}\) for a plane wave is a Lorentz scalar?

} % makeproblem
%
\makeanswer{pr:relElectroDynProblemSet6:2}{
%
\makeSubAnswer{Determining the stress energy tensor.}{pr:relElectroDynProblemSet6:2:a}
%
In the Coulomb gauge we used Fourier methods to find that the potential had the form
%
\begin{equation}\label{eqn:relElectroDynProblemSet6:30}
\begin{aligned}
\phi &= 0 \\
\BA &= \Bbeta \cos(\omega t - \Bk \cdot \Bx) \\
c^2 \Bk^2 &= \omega^2 \\
\Bbeta \cdot \Bk &= 0.
\end{aligned}
\end{equation}
%
For this problem it appears that working in the Lorentz gauge is required, and we want solutions of the form
%
\begin{equation}\label{eqn:relElectroDynProblemSet6:50}
A^m = D^m \cos( k_a x^a ).
\end{equation}
%
First, observe that the Lorentz gauge condition \(\partial_m A^m = 0\) requires
%
\begin{equation}\label{eqn:relElectroDynProblemSet6:70}
-D^m k_m \sin( k_a x^a ) = 0.
\end{equation}
%
Application of the wave equation operator
%
\begin{equation}\label{eqn:relElectroDynProblemSet6:90}
\partial_b \partial^b A^m = 0,
\end{equation}
gives us
\begin{equation}\label{eqn:relElectroDynProblemSet6:110}
-D^m k_b k^b \cos( k_a x^a ) = 0,
\end{equation}
providing the lightlike constraint on \(k\).  All told our four potential with constraints is
%
\begin{equation}\label{eqn:relElectroDynProblemSet6:130}
\begin{aligned}
A^m &= D^m \cos( k_a x^a ) \\
k^a k_a &= 0 \\
D^m k_m &= 0.
\end{aligned}
\end{equation}
%
We could also arrive at this point using 4D Fourier methods, which would be fun, but a bit more time consuming, and a little overkill given that the problem only requires us to tackle the linear monochromatic case.

On to the problem.  We now need our electromagnetic tensor components.
%
\begin{equation}\label{eqn:relElectroDynProblemSet6:210}
\begin{aligned}
F^{ i j}
&= \partial^i A^j - \partial^j A^i \\
&=
D^j \partial^i \cos( k^a x_a )
-D^i \partial^j \cos( k^a x_a ) \\
&=
\sin( k^a x_a ) ( D^i k^j - D^j k^i )
\end{aligned}
\end{equation}
%
Our stress energy tensor is
%
\begin{equation}\label{eqn:relElectroDynProblemSet6:230}
\begin{aligned}
T^{i j}
&= \inv{4 \pi}
\left(
- F^{i a} F_{b a} g^{b j} + \inv{4} g^{i j} F_{ab} F^{ ab}
\right) \\
&= \inv{4 \pi}
\left(
- F^{a i} F_{a b} g^{b j} + \inv{4} g^{i j} F_{ab} F^{ ab}
\right)
\end{aligned}
\end{equation}
%
Let us now expand the product of tensors
%
\begin{equation}\label{eqn:relElectroDynProblemSet6:250}
\begin{aligned}
F_{a b} F^{a i}
&=
\sin^2( k^a x_a)
( D_a k_b - D_b k_a )
( D^a k^i - D^i k^a ) \\
&=
\sin^2( k^a x_a) (
D_a k_b D^a k^i
-D_a k_b D^i k^a
- D_b k_a D^a k^i
+ D_b k_a D^i k^a ) \\
&=
\sin^2( k^a x_a) (
D_a D^a k_b k^i
- \cancel{D_a k^a} k_b D^i
- D_b \cancel{k_a D^a} k^i
+ D_b D^i \cancel{k_a k^a} ) \\
&=
\sin^2( k^a x_a) D_a D^a k_b k^i
\end{aligned}
\end{equation}
%
We see from this that our action term is zero
%
\begin{equation}\label{eqn:relElectroDynProblemSet6:150}
F_{a b} F^{a b} = \sin^2( k^a x_a) D_a D^a \cancel{k_b k^b},
\end{equation}
%
so the stress energy tensor is reduced to
%
\begin{equation}\label{eqn:relElectroDynProblemSet6:270}
\begin{aligned}
T^{i j}
&= -\inv{4 \pi} \sin^2( k^a x_a) D_a D^a k_b k^i g^{j b} \\
&= -\inv{4 \pi} \sin^2( k^a x_a) D_a D^a k^j k^i \\
\end{aligned}
\end{equation}
%
The energy density term of the stress energy tensor encapsulates most of these terms
%
\begin{equation}\label{eqn:relElectroDynProblemSet6:170}
T^{0 0}
= -\inv{4 \pi} \sin^2( k^a x_a) D_a D^a \frac{\omega^2}{c^2} = \calE,
\end{equation}
%
so we can write
%
\begin{equation}\label{eqn:relElectroDynProblemSet6:190}
T^{i j}
= \calE \frac{c^2}{\omega^2} k^i k^j,
\end{equation}
%
which completes the first part of this problem.
%
\makeSubAnswer{On the question of the Lorentz scalar.}{pr:relElectroDynProblemSet6:2:b}
%
Yes, one can now conclude that \(\frac{\calE c^2}{\omega^2}\) for a plane wave is a Lorentz scalar.

Observe that the \(k^i k^j\) transforms as a rank 2 tensor, as does \(T^{i j}\).  Because the product \(\calE c^2/\omega^2\) and \(k^i k^j\) must transform as a rank 2 tensor, this can only mean that the \(\calE c^2/\omega^2\) portion transforms as a Lorentz scalar.

} % makeanswer
%
%%%\section{Problem 3.  Force from an incoming wave}
%%%\subsection{Statement}
%%%
%%%(Problem from the book.)  Find the force acting on a wall which reflects, with reflection coefficient \(R\), and incoming electromagnetic wave; a general incidence angle is assumed, and is, of course, equal to the angle of reflection.
%%%
%%%In this problem, if you decide use the stress tensor and look at the solution given in the text \citep{landau1980classical}, the stress tensor is argued to be  \(T^{\alpha \beta} = T^{\alpha \beta}(\text{incoming wave}) + T^{\alpha \beta}(\text{reflected wave})\).  This actually holds only for the components of \(T\) where one index, i.e., \(\alpha\) is the direction perpendicular to the wall (i.e. for the components of the stress tensor relevant for calculating the pressure and shear). To be completely happy with the use of the stress tensor, you may want to derive this fact, starting from the expressions for the electric and magnetic field (3-vectors) \(\BE = \BE_1 + \BE_2\), \(\BB = \BB_1 + \BB_2\), where \((\BE_1, \BE_2)\) and \((\BB_1, \BB_2)\) correspond to the (incoming, reflected) wave, noticing that \(\Abs{\BE_2} = \sqrt{R}\Abs{\BE_1}\), just like you did for Problem 1.5 (HW6).  Also, see Problem 3 of HW5.
%%%
%%%\subsection{Solution}
%%%
%%%FIXME: TODO: The solution for this (ungraded) problem was covered in the tutorial (before I got to trying it).  I would also like to try this independently (perhaps using arbitrary orientation for the reflected wave to spice things up since the simple case has been done for us.)
%%%
