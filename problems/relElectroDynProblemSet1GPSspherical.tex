%
% Copyright © 2012 Peeter Joot.  All Rights Reserved.
% Licenced as described in the file LICENSE under the root directory of this GIT repository.
%

For \cref{pr:relElectroDynProblemSet2:2} I'd initially gotten very carried away playing with algebra and geometry of circular intersection.
I did end up with a numerical computation (handed in on paper) where I attempted to force a bit of SR into the mix in the end.
I know that I really ought to have been tackling the problem without considering 3D geometries and only using SR concepts but had too much fun playing with things, and ran out of time.
I paid for that with a really poor mark, and later reworked part (a) as a basic SR problem.

Here is the play I did the first time around.

\paragraph{Discussion of the non-toy model}

It is fairly easy to find interesting info about the mechanisms that real GPS works using.  NASA has a nice \href{http://www.nasm.si.edu/gps/work.html}{How does GPS work} page \citep{nasaGPS}, and How stuff works has a nice \href{http://electronics.howstuffworks.com/gadgets/travel/gps.htm}{How GPS Receivers Work} article \citep{howStuffWorksGPS}.  Reading these one finds that the GPS clocks are actually kept synchronized.  The typical GPS receiver obviously has a clock, since we have countdown timers for time until arrival, is that clock accurate enough compared to the satellite atomic clocks to be used for the GPS location algorithm?  What is done in fact is to use the local receiver time to seed the iterative algorithms, allowing the local time to be calculated eventually with an accuracy that actually approaches that of the satellite's atomic clocks.  Some of the sources of error, like reflection of the signals, delaying them, and interference by atmosphere are also discussed in these articles.  Also interesting is that there is a table lookup of the satellites position implemented in the GPS receivers.  This table lookup is used to seed the iterative algorithms, and can be used to reduce calculation error.

Our basic GPS problem is to calculate the intersection of a number of ``spherical'' hypersurfaces.  This is made more interesting by the fact that this is both a non-linear and an over-specified problem.  Let us consider the geometric problem to get an idea of how to set up this problem.  Suppose that we have a set of \(k\) satellites, located at position \(\Bp_i, i \in [1,k]\), and we know that these are located with distance \(d_i\) from our position \(\Bx\).

Our problem is then to find the simultaneous solution to the following set of equations

\begin{equation}\label{eqn:relativisticElectrodynamicsA1:300}
\begin{aligned}
(\Bx - \Bp_1)^2 &= d_1^2 \\
(\Bx - \Bp_2)^2 &= d_2^2 \\
\vdots \\
(\Bx - \Bk_2)^2 &= d_k^2.
\end{aligned}
\end{equation}

Observe that even if we reduce this to a one dimensional problem in a single variable \(x\), we still have a non-linear system

\begin{equation}\label{eqn:relativisticElectrodynamicsA1:300b}
\begin{aligned}
0 &= (x - p_1)^2 - d_1^2 \\
0 &= (x - p_2)^2 - d_2^2 \\
\vdots \\
0 &= (x - k_2)^2 - d_k^2.
\end{aligned}
\end{equation}

We also need to be aware of the fact that each of the positions \(p_i\), and the respective distances \(d_i\) will in reality both have associated errors, so there is not likely any specific single value of \(x\) that ``solves'' this problem, unless it is setup in a contrived and perfect fashion.  This intrinsic error, and the \(k\) equations, one unknown nature of the problem (or three unknowns for spatial, or four for spatial and time position) suggests a least squares approach, but it will have to be one that also incorporates iteration.

We can setup our problem in matrix form, where we are looking for a solution to 

\begin{equation}\label{eqn:relativisticElectrodynamicsA1:310}
F(x) = 
\begin{bmatrix}
F_i(x)
\end{bmatrix}
=
\begin{bmatrix}
\Abs{\Bx - \Bp_1} - c \Abs{ t - t_1 } \\
\Abs{\Bx - \Bp_2} - c \Abs{ t - t_2 } \\
\vdots \\
\Abs{\Bx - \Bp_k} - c \Abs{ t - t_k } \\
\end{bmatrix}
= 0.
\end{equation}

We seek the spacetime event vector \(x = (c t, \Bx)\) for the spatial location and the exact local time at the location of the GPS receiver.  Given any approximation of the solution, we can refine the solution using Newton's root finding method by taking partials, forming the Jacobian matrix for our function \(F(x)\).  That is

\begin{equation}\label{eqn:relativisticElectrodynamicsA1:320}
F(x_0 + \Delta x) \approx F(x_0) + \evalbar{\PD{x^j}{F_i(x)}}{x_0} \Delta x^j = F(x_0) + J(x_0) \Delta x = 0
\end{equation}

This leaves us with the our least squares problem, requiring the generalized inverse to the matrix equation

\begin{equation}\label{eqn:relativisticElectrodynamicsA1:330}
x_1 = x_0 - J^\dagger (x_0) F(x_0),
\end{equation}

where 

\begin{equation}\label{eqn:relativisticElectrodynamicsA1:340}
J^\dagger = (J^\T J)^{-1} J^\T.
\end{equation}

This is a solution in the least squares sense that given \(b = J x\), the norm \(\Abs{ J \overbar{x} - b}\) is minimized by \(\overbar{x} = J^\dagger b\).

This iterative method of solution, in the context of finding fitting circles and ellipses can be found discussed in detail in \citep{gander1994least}.


\paragraph{Goofing around with the geometry of it all}

For our toy model we have two satellites \(A\) and \(B\) both moving in the positive x-axis direction at velocity \(V_x\) at height \(h\).  As seen above, we do not require the velocity of the satellites to setup the problem, and could express the problem to solve as the numerical solution of the set of equations

\begin{equation}\label{eqn:relativisticElectrodynamicsA1:350}
F(x) = 
\begin{bmatrix}
\sqrt{ h^2 + (x - x_A')^2 } - c \Abs{t - t_A'} \\
\sqrt{ h^2 + (x - x_B')^2 } - c \Abs{t - t_B'} \\
\end{bmatrix} = 0.
\end{equation}

If we assume that our GPS receiver's clock is synchronized sufficiently with satellites \(A\) and \(B\), this single variable problem admits a closed form for one iteration of the least squares process.  However, since we are asked for a result that includes a \(V_x/c\) term, we can augment our matrix equation by two additional rows, with a secondary set of data points introduced at an offset time interval.  That is

\begin{equation}\label{eqn:relativisticElectrodynamicsA1:360}
F(x) = 
\begin{bmatrix}
\sqrt{ h^2 + (x - x_A')^2 } - \Abs{c t - c t_A'} \\
\sqrt{ h^2 + (x - x_B')^2 } - \Abs{c t - c t_B'} \\
\sqrt{ h^2 + (x - x_A' - (V_x/c) c \delta t)^2 } - \Abs{c t - c t_A' - c \delta t} \\
\sqrt{ h^2 + (x - x_B' - (V_x/c) c \delta t)^2 } - \Abs{c t - c t_B' - c \delta t} \\
\end{bmatrix} = 0.
\end{equation}

With \(t\), \(\delta t\), \(x_A'\), and \(x_B'\) given, and an initial seed value for the iterative procedure assumed to be the midpoint \(x_0 = (x_A' + x_B')/2\), we can calculate a first approximation to the receiver position \(x_1 = x_0 + \Delta x\) using the Newton's procedure outlined above.

For this system our Jacobian elements are all differentials of the following form

\begin{equation}\label{eqn:relativisticElectrodynamicsA1:370}
\PD{x}{} \sqrt{ h^2 + (x - p)^2 } - \Delta = \frac{x-p}{\sqrt{h^2 + (x-p)^2}},
\end{equation}

so, our Jacobian is

\begin{equation}\label{eqn:relativisticElectrodynamicsA1:380}
J = 
\begin{bmatrix}
\frac{x - x_A'}{\sqrt{ h^2 + (x - x_A')^2 } } \\
\frac{x - x_B'}{\sqrt{ h^2 + (x - x_B')^2 } } \\
\frac{x - x_A' - (V_x/c) c \delta t}{\sqrt{ h^2 + (x - x_A' - (V_x/c) c \delta t)^2 } } \\
\frac{x - x_B' - (V_x/c) c \delta t}{\sqrt{ h^2 + (x - x_B' - (V_x/c) c \delta t)^2 } } \\
\end{bmatrix}
\end{equation}

Our deviation from the midpoint to first order in \(V_x/c\) is

\begin{equation}\label{eqn:relativisticElectrodynamicsA1:390}
\Delta x = - \evalbar{ \frac{J^\T F}{J^T J} }{x = (x_A' + x_B')/2}
\end{equation}

To tidy this up, let

\begin{equation}\label{eqn:relativisticElectrodynamicsA1:391}
s = \inv{2} (x_B' - x_A')
\end{equation}

\begin{equation}\label{eqn:relativisticElectrodynamicsA1:392}
D = (V_x/c) c \delta t
\end{equation}

\begin{equation}\label{eqn:relElectroDynProblemSet1:1500}
\begin{aligned}
\evalbar{J^\T J}{(x_A' + x_B')/2} 
&= 
\frac{s^2}{h^2 + s^2}
+\frac{(-s)^2}{h^2 + s^2}
+\frac{(s - D)^2}{h^2 + (s -D)^2}
+\frac{(-s - D)^2}{h^2 + (-s -D)^2} \\
&= 
\frac{2 s^2}{h^2 + s^2}
+\frac{(s - D)^2}{h^2 + (s -D)^2}
+\frac{(s + D)^2}{h^2 + (s +D)^2}
\end{aligned}
\end{equation}

and

\begin{equation}\label{eqn:relElectroDynProblemSet1:1520}
\begin{aligned}
\evalbar{J^\T F}{(x_A' + x_B')/2} 
&=
\begin{bmatrix}
\frac{s}{\sqrt{ h^2 + (s)^2 } } &
\frac{-s}{\sqrt{ h^2 + (-s)^2 } } &
\frac{s - D}{\sqrt{ h^2 + (s - D)^2 } } &
\frac{-s - D}{\sqrt{ h^2 + (-s - D)^2 } } 
\end{bmatrix} \\
&\qquad \begin{bmatrix}
\sqrt{ h^2 + (s)^2 } - \Abs{c t - c t_A'} \\
\sqrt{ h^2 + (-s)^2 } - \Abs{c t - c t_B'} \\
\sqrt{ h^2 + (s - D)^2 } - \Abs{c t - c t_A' - c \delta t} \\
\sqrt{ h^2 + (-s - D)^2 } - \Abs{c t - c t_B' - c \delta t} \\
\end{bmatrix} \\
&=
\frac{s}{\sqrt{s^2 + D^2}} \left( \Abs{ ct - c t_B' } - \Abs{ c t - c t_A'} \right) -2 D \\
&- \frac{(s - D)\Abs{c t - c t_A' - c \delta t} }{ \sqrt{ h^2 + (s - D)^2 } } 
+ \frac{(s + D)\Abs{c t - c t_B' - c \delta t} }{ \sqrt{ h^2 + (s + D)^2 } } 
\end{aligned}
\end{equation}

The final beastly ugly result, utilizing the helper variables of \eqnref{eqn:relativisticElectrodynamicsA1:392}, and \eqnref{eqn:relativisticElectrodynamicsA1:391}, we have for the deviation from the midpoint (after one iteration of this least squares Newton's method):

\begin{equation}\label{eqn:relativisticElectrodynamicsA1:400}
\Delta x =
\frac{
   -\frac{s}{\sqrt{s^2 + D^2}} \left( \Abs{ ct - c t_B' } - \Abs{ c t - c t_A'} \right) +2 D
   + \frac{(s - D)\Abs{c t - c t_A' - c \delta t} }{ \sqrt{ h^2 + (s - D)^2 } } 
   - \frac{(s + D)\Abs{c t - c t_B' - c \delta t} }{ \sqrt{ h^2 + (s + D)^2 } } }
{
   \frac{2 s^2}{h^2 + s^2}
   +\frac{(s - D)^2}{h^2 + (s -D)^2}
   +\frac{(s + D)^2}{h^2 + (s +D)^2}
}
\end{equation}

In practice it does not make much sense to compute this.  You will want to use a computer, and assuming the availability of a pre-canned SVD routine to compute the generalized inverse, the toy model would not be any easier to solve than the real thing.


%@inproceedings{ ion:gnss04,
%%http://www.gpstk.org/pub/Documentation/UsersGuide/gpstk-user-reference.pdf

%\subsection{Solution. Part 2}
%
%Now, let us bring some special relativity into the mix.  Specifically, lets look at how much the satellites have to correct their clocks to match Earth time, given their velocity relative to any point on the surface.
%
%See hand written notes for this part of the problem.  Given all the numbers, this would be cumbersome to try to type up.
%
%\paragraph{Disclaimer.  Was this really what was asked for?} While the calculation of part I was a fun, it was a strictly geometrical problem, and does not involve relativity.  It does not have to since GPS clocks are kept synchronized.  It seems very plausible to imagine that a completely different approach was to have been tried.

