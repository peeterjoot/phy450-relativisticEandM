%
% Copyright © 2012 Peeter Joot.  All Rights Reserved.
% Licenced as described in the file LICENSE under the root directory of this GIT repository.
%
\makeproblem{Toy GPS model}{pr:relElectroDynProblemSet1:3}{
%
A toy model of a GPS system has satellites moving in a straight line with constant velocity \(V_x\) and at a constant height \(h\) (measured, e.g., along the y-axis) above ``ground'' (the x-axis).  The satellites broadcast the time in their rest frame as well as their location at a time of broadcast.  Imagine a person on the ground receives simultaneously broadcasts from two satellites, \(A\) and \(B\), reporting their locations \(x_A'\) and \(x_B'\) as well as times of broadcast (which happen to be equal), \(t_A' = t_B'\).	
%
\makesubproblem{}{pr:relElectroDynProblemSet1:3:a}
%
Find a condition determining your position in \(x\).  Evaluate it to find your deviation from the midpoint between the satellites to first order in \(V_x/c\).
%
\makesubproblem{}{pr:relElectroDynProblemSet1:3:b}
%
For some real numbers, note that in reality there are 24 satellites, moving with \(V ~4 \text{km}/s\), a distance \(R \approx 2.7 \times 10^4 \text{km}\).  Use these numbers and the result from the previous problem (assuming a flat Earth, to be sure...) to get an idea whether (special) relativistic effects are important for the typical modern GPS accuracy of order 10 m (or less)?

} % makeproblem
%
\makeanswer{pr:relElectroDynProblemSet1:3}{
%
\makeSubAnswer{}{pr:relElectroDynProblemSet1:3:a}
%
We are looking for a worldpoints \((ct', x', y')\) in satellite frame on the light cone emanating from the satellite worldpoints \((ct_A', x_A', y_A')\), and \((ct_B', x_B', y_B')\).  These are
%
\begin{equation}\label{eqn:relElectroDynProblemSet1:1200}
\begin{aligned}
c^2 ( t_A' - t')^2 &= (x_A' - x')^2 + (y_A' - y')^2 \\
c^2 ( t_B' - t')^2 &= (x_B' - x')^2 + (y_B' - y')^2,
\end{aligned}
\end{equation}
%
where the worldpoints \((ct', x', y')\) are related to the stationary frame by
%
\begin{equation}\label{eqn:relElectroDynProblemSet1:1220}
\begin{bmatrix}
ct' \\
x' \\
y'
\end{bmatrix}
\begin{bmatrix}
\gamma & -\beta \gamma & 0 \\
-\beta \gamma & \gamma & 0 \\
0 & 0 & 1
\end{bmatrix}
\begin{bmatrix}
ct \\
x \\
y
\end{bmatrix}.
\end{equation}
%
The problem has been artificially simplified by stating that \(t_A' = t_B'\), and we can eliminate the \(y'\) terms since we want \(y_A' - y' = h = y_B' - y'\) at the point where the signal is received.

Suppose that in the observer frame the light signals are received with event coordinates \((c t_0, x_0, 0)\).  In the satellites rest frame these are
%
\begin{equation}\label{eqn:relElectroDynProblemSet1:1240}
\begin{aligned}
ct' &= \gamma ( c t_0 - \beta x_0 ) \\
x' &= \gamma ( x_0 - \beta c t_0 ).
\end{aligned}
\end{equation}
%
We can make these substitutions above, yielding
%
\begin{equation}\label{eqn:relElectroDynProblemSet1:1260}
\begin{aligned}
( c t_A' - \gamma c t_0 + \gamma \beta x_0)^2 &= (x_A' - \gamma x_0 + \gamma \beta c t_0 )^2 + h^2 \\
( c t_A' - \gamma c t_0 + \gamma \beta x_0)^2 &= (x_B' - \gamma x_0 + \gamma \beta c t_0 )^2 + h^2.
\end{aligned}
\end{equation}
%
Observe that the \(t_A' = t_B'\) condition allows us to equate the pair of RHS terms and thus have
%
\begin{equation}\label{eqn:relElectroDynProblemSet1:1280}
x_A' - \gamma x_0 + \gamma \beta c t_0 = \pm (x_B' - \gamma x_0 + \gamma \beta c t_0 ).
\end{equation}
%
If we pick the positive root, then we have \(x_A' = x_B'\), a perfectly valid mathematical solution, but not one that can be used for triangularization.  Taking the negative root instead and rearranging we have
%
\begin{equation}\label{eqn:relElectroDynProblemSet1:1301}
\gamma \beta c t_0 = \gamma x_0 - \inv{2}(x_A' + x_B').
\end{equation}
%
As a sanity check observe that if \(\beta = 0\) we have \(x_0 = \inv{2}(x_A' + x_B') = x_m'\), the midpoint in the satellite (also the observer frame for \(\beta = 0\)).  This is what we would expect if a simultaneous signal is received that emanated at the same time when both sources are at rest at the same height.

When \(\beta \ne 0\) we have
%
\begin{equation}\label{eqn:relElectroDynProblemSet1:1302}
\gamma c t_0 = \inv{\beta}(\gamma x_0 - x_m'),
\end{equation}
%
allowing us to eliminate \(\gamma c t_0\) terms from the equations we wish to solve
%
\begin{equation}\label{eqn:relElectroDynProblemSet1:1262}
\begin{aligned}
%( c t_A' - \inv{\beta}( \gamma x_0 - x_m') + \gamma \beta x_0)^2 &= (x_A' - \gamma x_0 + \gamma x_0 - x_m')^2 + h^2 \\
%( c t_A' - \inv{\beta}( \gamma x_0 - x_m') + \gamma \beta x_0)^2 &= (x_B' - \gamma x_0 + \gamma x_0 - x_m')^2 + h^2
\left( c t_A' - \inv{\beta}( \gamma x_0 - x_m') + \gamma \beta x_0 \right)^2 &= (x_A' - x_m')^2 + h^2 \\
\left( c t_A' - \inv{\beta}( \gamma x_0 - x_m') + \gamma \beta x_0 \right)^2 &= (x_B' - x_m')^2 + h^2.
\end{aligned}
\end{equation}
%
We can group the \(\gamma x_0\) terms on the LHS nicely
%
\begin{equation}\label{eqn:relElectroDynProblemSet1:1540}
\begin{aligned}
- \inv{\beta} \gamma x_0 \gamma \beta x_0
&=
\gamma x_0 ( - \inv{\beta} + \beta ) \\
&=
\inv{\beta} \gamma x_0 ( - 1 + \beta^2 ) \\
&=
-\inv{\beta} x_0,
\end{aligned}
\end{equation}
leaving
\begin{equation}\label{eqn:relElectroDynProblemSet1:1400}
\left( c t_A' -\inv{\beta} x_0 + \inv{\beta} x_m' \right)^2 = (x_A' - x_m')^2 + h^2 = (x_B' - x_m')^2 + h^2.
\end{equation}
%
The value \(\Abs{x_A' - x_m'} = \Abs{x_B' - x_m'} = \Abs{x_A' - x_B'}/2\) is half the separation \(L'\) of the satellites in their rest frame, so we have
%
\begin{equation}\label{eqn:relElectroDynProblemSet1:1420}
c t_A' -\inv{\beta} x_0 + \inv{\beta} x_m' = \pm \sqrt{{L'}^2/4 + h^2},
\end{equation}
or
\begin{equation}\label{eqn:relElectroDynProblemSet1:1440}
x_0 = x_m' + \beta c t_A' \mp \beta \sqrt{{L'}^2/4 + h^2}.
\end{equation}
%
Utilizing the inverse transformation we have for a x-axis spatial coordinate in the observer frame
%
\begin{equation}\label{eqn:relElectroDynProblemSet1:1460}
x = \gamma ( x' + \beta c t'),
\end{equation}
%
allowing the \(t_A'\) term to be eliminated in favour of the position that the midpoint between the satellites would have been observed at time \(t_A'\).  This gives us
%
\begin{equation}\label{eqn:relElectroDynProblemSet1:1480}
x_0 = \inv{\gamma} x_m \mp \beta \sqrt{{L'}^2/4 + h^2}.
\end{equation}
%
FIXME: Which sign is correct for this problem?  I had guess the negative sign.  Fixing that is probably the toughest part of this problem!
%
\makeSubAnswer{}{pr:relElectroDynProblemSet1:3:b}
%
FIXME: Had hand written notes for this part of the problem, with how I'd attempted it first (considering the actual geometric problem in 3D.)

} % makeanswer

