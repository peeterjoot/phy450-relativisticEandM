%
% Copyright � 2012 Peeter Joot.  All Rights Reserved.
% Licenced as described in the file LICENSE under the root directory of this GIT repository.
%

\label{chap:antisymmetricTensorTx}
%\blogpage{http://sites.google.com/site/peeterjoot/math2011/antisymmetricTensorTx.pdf}
%\date{Jan 14, 2011}
%
\section{Motivation}
%
I have an ancient copy of the course text \citep{landau1951classical} from the library right now (mine is on order still) for my PHY450H1S course (relativistic electrodynamics).  Given the transformation rule for a first rank tensor
%
\begin{equation}\label{eqn:antisymmetricTensorTx:5}
A_{i} = \alpha_{im} A'_{m},
\end{equation}
%
they list the transformation rule for a second rank tensor as
%
\begin{equation}\label{eqn:antisymmetricTensorTx:10}
A_{ik} = \alpha_{im} \alpha_{kl} A'_{ml}.
\end{equation}
%
This is not motivated in any way.  Let us compare to transformation of a bivector expressed in the Dirac basis, transformed by outermorphism.  That is specifically a transformation of a antisymmetric tensor (once expressed in components anyways), but should provide some intuition.

It is also worthwhile to note that there are some old fashioned notational quirks in this text (at least the old version that I have currently borrowed).  Specifically, they uses Latin indices four vectors with Greek indices for three vectors, completely opposite to what appears to be the current conventions.  They also do not use upper and lower indices to keep track of bookkeeping.  I will use the conventions I am used to for now.
%
\section{Notation and use of Geometric Algebra herein}
%
I will use conventions from \citep{doran2003gap} using the Dirac basis, with a preference for index upper coordinates, and express a vector as
%
\begin{equation}\label{eqn:antisymmetricTensorTx:20}
x = x^\alpha \gamma_\alpha = x_\alpha \gamma^\alpha,
\end{equation}
%
Here the basis pairs \(\{\gamma_\mu\}\) and \(\{\gamma^\mu\}\) are reciprocal frames with \(\gamma^\mu \cdot \gamma_\nu = {\delta^\mu}_\nu\).  I will have no need for any specific metric convention here.

The dot and wedge products used will be defined in terms of their Clifford Algebra formulation
%
\begin{equation}\label{eqn:antisymmetricTensorTx:25}
\begin{aligned}
a \cdot b &= \inv{2} (a b + b a) \\
a \wedge b &= \inv{2} (a b - b a).
\end{aligned}
\end{equation}
%
The dot product between two bivectors \(A\), \(B\) will also be used, defined as the scalar part of the product \(AB\).  In particular the identity for extraction of that scalar component from the dot product of two wedge products will be required
%
\begin{equation}\label{eqn:antisymmetricTensorTx:26}
(a \wedge b ) \cdot (c \wedge d)
= ( a (b \cdot c) - b ( a \cdot c) ) \cdot d
= (a \cdot d) (b \cdot c) - (b \cdot d) ( a \cdot c)
\end{equation}
%
\section{Transformation of the coordinates}
%
Let us assume our transformation is linear, and we will denote its action on vectors as follows
%
\begin{equation}\label{eqn:antisymmetricTensorTx:30}
x' = L(x) = x^\alpha L( \gamma_\alpha).
\end{equation}
%
Extracting coordinates for the transformed coordinates (assuming a non-moving frame where the unit vectors on both sides are the same), we have after dotting with \(\gamma^\mu\)
%
\begin{equation}\label{eqn:antisymmetricTensorTx:40}
{x'}^\mu = \left( {x'}^\alpha \gamma_\alpha \right) \cdot \gamma^\mu
= x^\alpha \left( L( \gamma_\alpha) \cdot \gamma^\mu \right)
\end{equation}
%
Now introduce a coordinate representation for the transformation \(L\)
%
\begin{equation}\label{eqn:antisymmetricTensorTx:50}
L( \gamma_\alpha) \cdot \gamma^\mu  = {L_\alpha}^\mu,
\end{equation}
%
so our transformation rule for the four vector coordinates becomes
%
\begin{equation}\label{eqn:antisymmetricTensorTx:55}
{x'}^\mu = x^\alpha {L_\alpha}^\mu.
\end{equation}
%
We are now ready to look at the transformation of a bivector (a quantity having a rank two antisymmetric tensor representation in coordinates), and see how the coordinates transform.

Let us transform by outermorphism of the transformed vector factors the bivector
%
\begin{equation}\label{eqn:antisymmetricTensorTx:60}
c = a \wedge b \rightarrow a' \wedge b'.
\end{equation}
%
First we will need the coordinate representation of the bivector before transformation.  We dot with \(\gamma^\nu \wedge \gamma^\mu\) to pick up the desired term
%
\begin{equation}\label{eqn:antisymmetricTensorTx:670}
\begin{aligned}
(a \wedge b) \cdot (\gamma_\nu \wedge \gamma_\mu)
&=
a^\alpha b^\beta (\gamma_\alpha \wedge \gamma_\beta) \cdot (\gamma^\nu \wedge \gamma^\mu) \\
&=
a^\alpha b^\beta ( \gamma_\alpha {\delta_\beta}^\nu -\gamma_\beta {\delta_\alpha}^\nu ) \cdot \gamma^\mu \\
&=
a^\alpha b^\beta ( {\delta_\alpha}^\mu {\delta_\beta}^\nu -{\delta_\beta}^\mu {\delta_\alpha}^\nu ) \\
&=
a^\mu b^\nu
-a^\nu b^\mu.
\end{aligned}
\end{equation}
%
If we introduce a rank two tensor now, say
%
\begin{equation}\label{eqn:antisymmetricTensorTx:70}
T^{\mu\nu} = a^\mu b^\nu -a^\nu b^\mu,
\end{equation}
we recover our bivector with
\begin{equation}\label{eqn:antisymmetricTensorTx:80}
a \wedge b = \inv{2} T^{\alpha \beta} \gamma_\alpha \wedge \gamma_\beta.
\end{equation}
%
Now let us look at the coordinate representation of the transformed bivector.  It will also be helpful to make use of the identity that can be observed above from the initial coordinate extraction
%
\begin{equation}\label{eqn:antisymmetricTensorTx:90}
(\gamma_\alpha \wedge \gamma_\beta) \cdot (\gamma^\nu \wedge \gamma^\mu) = {\delta_\alpha}^\mu {\delta_\beta}^\nu -{\delta_\beta}^\mu {\delta_\alpha}^\nu
\end{equation}
%
In coordinates our transformed bivector is
%
\begin{equation}\label{eqn:antisymmetricTensorTx:100}
a' \wedge b' =
a^\sigma {L_\sigma}^\alpha
b^\pi {L_\pi}^\beta
\gamma_\alpha \wedge \gamma_\beta,
\end{equation}
%
and we can proceed with the coordinate extraction by taking dot products with \(\gamma^\nu \wedge \gamma^\mu\) as before.  This gives us
%
\begin{equation}\label{eqn:antisymmetricTensorTx:690}
\begin{aligned}
( a' \wedge b' ) \cdot (\gamma^\nu \wedge \gamma^\mu)
&=
a^\sigma {L_\sigma}^\alpha
b^\pi {L_\pi}^\beta
\gamma_\alpha \wedge \gamma_\beta \\
&=
a^\sigma {L_\sigma}^\alpha
b^\pi {L_\pi}^\beta
( {\delta_\alpha}^\mu {\delta_\beta}^\nu -{\delta_\beta}^\mu {\delta_\alpha}^\nu  ) \\
&=
a^\sigma {L_\sigma}^\mu
b^\pi {L_\pi}^\nu
-
a^\sigma {L_\sigma}^\nu
b^\pi {L_\pi}^\mu \\
&=
a^\sigma {L_\sigma}^\mu
b^\pi {L_\pi}^\nu
-
a^\pi {L_\pi}^\nu
b^\sigma {L_\sigma}^\mu \\
&=
(a^\sigma b^\pi - a^\pi b^\sigma) {L_\sigma}^\mu {L_\pi}^\nu
\\
&=
T^{\sigma \pi} {L_\sigma}^\mu {L_\pi}^\nu.
\end{aligned}
\end{equation}
%
We are able to conclude that the bivector coordinates transform as
%
\begin{equation}\label{eqn:antisymmetricTensorTx:200}
T^{\mu \nu} \rightarrow T^{\sigma \pi} {L_\sigma}^\mu {L_\pi}^\nu.
\end{equation}
%
Except for the lowering index differences this verifies the rule \eqnref{eqn:antisymmetricTensorTx:10} from the text.

It would be reasonable seeming to impose such a tensor transformation rule on any antisymmetric rank 2 tensor, and in the text this is also imposed as the rule for transformation of symmetric rank 2 tensors.  Do we have a simple example of a rank 2 symmetric tensor that can be expressed geometrically?  The only one that comes to mind off the top of my head is the electrodynamic stress tensor, which is not exactly simple to work with.
%
\section{Lorentz transformation of the metric tensors}
%
Following up on the previous thought, it is not hard to come up with an example of a symmetric tensor a whole lot simpler than the electrodynamic stress tensor.  The metric tensor is probably the simplest symmetric tensor, and we get that by considering the dot product of two vectors.  Taking the dot product of vectors \(a\) and \(b\) for example we have
%
\begin{equation}\label{eqn:antisymmetricTensorTx:300}
a \cdot b
= a^\mu b^\nu \gamma_\mu \cdot \gamma_\nu
\end{equation}
%
From this, the metric tensors are defined as
%
\begin{equation}\label{eqn:antisymmetricTensorTx:310}
\begin{aligned}
g_{\mu\nu} &= \gamma_\mu \cdot \gamma_\nu \\
g^{\mu\nu} &= \gamma^\mu \cdot \gamma^\nu
\end{aligned}
\end{equation}
%
These are both symmetric and diagonal, and in fact equal (regardless of whether one picks a \(+,-,-,-\) or \(-,+,+,+\) signature for the space).

Let us look at the transformation of the dot product, utilizing the transformation of the four vectors being dotted to do so.  By definition, when both vectors are equal, we have the (squared) spacetime interval, which based on the speed of light being constant, has been found to be an invariant under transformation.
%
\begin{equation}\label{eqn:antisymmetricTensorTx:320}
a' \cdot b'
=
a^\mu b^\nu L(\gamma_\mu) \cdot L(\gamma_\nu)
\end{equation}
%
We note that, like any other vector, the image \(L(\gamma_\mu)\) of the Lorentz transform of the vector \(\gamma_\mu\) can be written as
%
\begin{equation}\label{eqn:antisymmetricTensorTx:330}
L(\gamma_\mu) = \left( L(\gamma_\mu) \cdot \gamma^\nu \right) \gamma_\nu.
\end{equation}
%
Similarly we can write any vector in terms of the reciprocal frame
\begin{equation}\label{eqn:antisymmetricTensorTx:340}
\gamma_\nu = (\gamma_\nu \cdot \gamma_\mu) \gamma^\mu.
\end{equation}
%
The dot product factor is a component of the metric tensor
%
\begin{equation}\label{eqn:antisymmetricTensorTx:350}
g_{\nu \mu} = \gamma_\nu \cdot \gamma_\mu,
\end{equation}
so we see that the dot product transforms as
%
\begin{equation}\label{eqn:antisymmetricTensorTx:320b}
a' \cdot b'
= a^\mu b^\nu
( L(\gamma_\mu) \cdot \gamma^\alpha )
( L(\gamma_\nu) \cdot \gamma^\beta )
\gamma_\alpha
\cdot
\gamma_\beta
= a^\mu b^\nu
{L_\mu}^\alpha
{L_\nu}^\beta
g_{\alpha \beta}
\end{equation}
%
In particular, for \(a = b\) where we have the invariant interval defined by the condition \(a^2 = {a'}^2\), we must have
%
\begin{equation}\label{eqn:antisymmetricTensorTx:360}
a^\mu a^\nu g_{\mu \nu}
= a^\mu a^\nu
{L_\mu}^\alpha
{L_\nu}^\beta
g_{\alpha \beta}
\end{equation}
%
This implies that the symmetric metric tensor transforms as
%
\begin{equation}\label{eqn:antisymmetricTensorTx:370}
g_{\mu\nu}
=
{L_\mu}^\alpha
{L_\nu}^\beta
g_{\alpha \beta}
\end{equation}
%
Recall from \eqnref{eqn:antisymmetricTensorTx:200} that the coordinates representation of a bivector, an antisymmetric quantity transformed as
%
\begin{equation}\label{eqn:antisymmetricTensorTx:200b}
T^{\mu \nu} \rightarrow T^{\sigma \pi} {L_\sigma}^\mu {L_\pi}^\nu.
\end{equation}
%
This is a very similar transformation, but differs from the bivector case where our free indices were upper indices.  Suppose that we define an alternate set of coordinates for the Lorentz transformation.  Let
%
\begin{equation}\label{eqn:antisymmetricTensorTx:380}
{L^\mu}_\nu = L(\gamma^\mu) \cdot \gamma_\nu.
\end{equation}
%
This can be related to the previous coordinate matrix by
\begin{equation}\label{eqn:antisymmetricTensorTx:390}
{L^\mu}_\nu = g^{\mu \alpha } g_{\nu \beta } {L_\alpha}^\beta.
\end{equation}
%
If we examine how the coordinates of \(x^2\) transform in their lower index representation we find
%
\begin{equation}\label{eqn:antisymmetricTensorTx:400}
{x'}^2 = x_\mu x_\nu {L^\mu}_\alpha {L^\nu}_\beta g^{\alpha \beta} = x^2 = x_\mu x_\nu g^{\mu \nu},
\end{equation}
%
and therefore find that the (upper index) metric tensor transforms as
%
\begin{equation}\label{eqn:antisymmetricTensorTx:410}
g^{\mu \nu} \rightarrow
g^{\alpha \beta}
{L^\mu}_\alpha {L^\nu}_\beta .
\end{equation}
%
Compared to \(\eqnref{eqn:antisymmetricTensorTx:200b}\) we have almost the same structure of transformation.  Are these the same?  Does the notation I picked here introduce an apparent difference that does not actually exist?  We really want to know if we have the identity
%
\begin{equation}\label{eqn:antisymmetricTensorTx:420}
L(\gamma_\mu) \cdot \gamma^\nu
\questionEquals
L(\gamma^\nu) \cdot \gamma_\mu,
\end{equation}
%
If that were to be the case, then given the notation selected it would mean that \({L_\mu}^\nu = {L^\nu}_\mu\).  If that were true it would justify a notational simplification \({L_\mu}^\nu = {L^\nu}_\mu = L^\nu_\mu\).
%
\section{The inverse Lorentz transformation}
%
To answer this question, let us consider a specific example, an x-axis boost of rapidity \(\alpha\).  For that our Lorentz transformation takes the following form
%
\begin{equation}\label{eqn:antisymmetricTensorTx:430}
L(x) = e^{-\sigma_1 \alpha/2} x e^{\sigma_1 \alpha/2},
\end{equation}
%
where \(\sigma_k = \gamma_k \gamma_0\).  Since \(\sigma_1\) anticommutes with \(\gamma_0\) and \(\gamma_1\), but commutes with \(\gamma_2\) and \(\gamma_3\), we have
%
\begin{equation}\label{eqn:antisymmetricTensorTx:440}
L(x) = (x^0 \gamma_0 + x^1 \gamma_1) e^{\sigma_1 \alpha} + x^2 \gamma_2 + x^3 \gamma_3,
\end{equation}
and after expansion this is
\begin{equation}\label{eqn:antisymmetricTensorTx:441}
L(x) =
\gamma_0 ( x^0 \cosh \alpha - x^1 \sinh \alpha )
+\gamma_1 ( x^1 \cosh \alpha - x^0 \sinh \alpha )
+\gamma_2
+\gamma_3.
\end{equation}
%
Note that this is the first time a specific metric preference has been imposed, and \(+,-,-,-\) has been used.

Observe that for the basis vectors themselves we have
%
\begin{equation}\label{eqn:antisymmetricTensorTx:450}
\begin{bmatrix}
L(\gamma_0) \\
L(\gamma_1) \\
L(\gamma_2) \\
L(\gamma_3)
\end{bmatrix}
=
\begin{bmatrix}
\gamma_0 \cosh \alpha - \gamma_1 \sinh \alpha \\
-\gamma_0 \sinh \alpha + \gamma_1 \cosh \alpha \\
\gamma_2 \\
\gamma_3
\end{bmatrix}
\end{equation}
%
Forming a matrix with \(\mu\) indexing over rows and \(\nu\) indexing over columns we have
%
\begin{equation}\label{eqn:antisymmetricTensorTx:460}
{L_\mu}^\nu =
\begin{bmatrix}
\cosh \alpha &- \sinh \alpha & 0 & 0 \\
-\sinh \alpha & \cosh \alpha & 0 & 0 \\
0 & 0 & 1 & 0 \\
0 & 0 & 0 & 1
\end{bmatrix}
\end{equation}
%
Performing the same expansion for \({L^\nu}_\mu\), again with \(\mu\) indexing over rows, we have
%
\begin{equation}\label{eqn:antisymmetricTensorTx:470}
{L^\nu}_\mu =
\begin{bmatrix}
\cosh \alpha & \sinh \alpha & 0 & 0 \\
\sinh \alpha & \cosh \alpha & 0 & 0 \\
0 & 0 & 1 & 0 \\
0 & 0 & 0 & 1
\end{bmatrix}.
\end{equation}
%
This answers the question.  We cannot assume that \({L_\mu}^\nu = {L^\nu}_\mu\).  In fact, in this particular case, we have \({L^\nu}_\mu = ({L_\mu}^\nu)^{-1}\).  Is that a general condition?  Note that for the general case, we have to consider compounded transformations, where each can be a boost or rotation.

With my text still not here I have obtained a newer version of the course text from a different UofT library.  In this newer version \citep{landau1971classical} (still not the 4th edition) it is at least updated with the ``modern'' upper and lower index formalism.

In this version they define a four-dimensional second rank tensor as the set of sixteen quantities
%
\begin{equation}\label{eqn:antisymmetricTensorTx:471}
A^{\mu\nu},
\end{equation}
%
provided these transform under coordinate transformations like the products of components of two four vectors.  They also provide raising and lowering rules that distinguish the quantities \({A^{\mu}}_\nu\), and \({A_{\mu}}^\nu\) by relating these to the raising and lowering operations so that, for example, \({A_0}^1 = A^{01}\), \({A^0}_1 = -A^{01}\).  This is consistent with the notation I have used fairly blunderingly that seemed natural.  This also highlights the difference between \({L_\mu}^\nu\), and \({L^\nu}_\mu\).  We can relate both of these back to the index upper tensor representation
%
\begin{equation}\label{eqn:antisymmetricTensorTx:472}
\begin{aligned}
{L_\alpha}^\nu &= g_{\mu \alpha} L^{\mu \nu} \\
{L^\mu}_\alpha &= g_{\nu \alpha} L^{\mu \nu}
\end{aligned}
\end{equation}
%
This shows precisely how the two objects relate back to the original tensor \(L^{\mu \nu}\), and why we cannot just write \(L_\alpha^\nu\) or \(L^\mu_\alpha\) respectively.

Note that in the third edition they still (somewhat surprisingly to me) continue to latin indices for \(0,1,2,3\) and greek for \(1,2,3\) as in the original 1951 version.
%
\section{Duality in tensor form}
%
Let us consider the subject of duality to antisymmetric forms.  Within a geometric algebra context our duality is provided by multiplication by the pseudoscalar for the space.

For instance in \R{3} the dual to a bivector is the familiar cross product
%
\begin{equation}\label{eqn:antisymmetricTensorTx:500}
\Ba \cross \Bb = -I (\Ba \wedge \Bb),
\end{equation}
%
where \(I = \Be_1 \Be_2 \Be_3\).  In our spacetime context we use the pseudoscalar \(I = \gamma_0 \gamma_1 \gamma_2 \gamma_3\).  Let us compute the coordinate representation of our vector, bivector, and trivector duals, which should compare with the tensor representation of the text.

In the text we have a statement that given an antisymmetric tensor \(T^{\mu \nu}\), its dual is
%
\begin{equation}\label{eqn:antisymmetricTensorTx:510}
\inv{2} {e^{\mu \nu}}_{\alpha \beta} T^{\alpha \beta}
\end{equation}
%
(I have adjusted the notation for the antisymmetric pseudotensor \(\epsilon\) to retain free upper indices).

How does this compare the to Geometric Algebra bivector dual in spacetime?  Let
%
\begin{equation}\label{eqn:antisymmetricTensorTx:520}
T = \inv{2} T^{\mu \nu} \gamma_\mu \wedge \gamma_\nu
%= \inv{2} \sum_{\mu < \nu} (T^{\mu \nu} - T^{\nu\mu} ) \gamma_\mu \wedge \gamma_\nu
= \sum_{\mu < \nu} T^{\mu \nu} \gamma_\mu \wedge \gamma_\nu.
\end{equation}
%
We dot with \(\gamma^\nu \wedge \gamma^\mu\) to extract the (tensor) coordinate representation
%
\begin{equation}\label{eqn:antisymmetricTensorTx:710}
\begin{aligned}
T \cdot (\gamma^\nu \wedge \gamma^\mu)
&=
\inv{2} T^{\alpha \beta} (\gamma_\alpha \wedge \gamma_\beta ) \cdot (\gamma^\nu \wedge \gamma^\mu) \\
&=
\inv{2} T^{\alpha \beta} ( {\delta_\beta}^\nu {\delta_\alpha}^\mu -{\delta_\alpha}^\nu {\delta_\beta}^\mu ) \\
&=
\inv{2} (T^{\mu \nu} - T^{\nu \mu}) \\
&=
T^{\mu \nu}.
\end{aligned}
\end{equation}
%
The index manipulation gets a little hairy, but one can expand the dot products \((I T) \cdot (\gamma^\nu \wedge \gamma^\mu)\) to find that this dual has coordinates have the value,
%
\begin{equation}\label{eqn:antisymmetricTensorTx:530}
(I T) \cdot (\gamma^\nu \wedge \gamma^\mu) = C {e^{\mu \nu}}_{\alpha \beta} T^{\alpha \beta},
\end{equation}
%
where \(C\) is a constant multiplier that I messed up computing the actual value for.

It is also possible to verify that \((IT) \cdot T = 0\).  Thus we can describe the duality of \(T^{\mu \nu}\) and \({e^{\mu \nu}}_{\alpha \beta} T^{\alpha \beta}\) as the geometrical condition \(T = a b\), \(IT = c d\), where \(a, b, c, d\) are all mutually perpendicular.

Given a vector \(x = x^\mu \gamma_\mu = x_\mu \gamma^\mu\) it is also possible to confirm that the coordinate representation of the Geometric Algebra vector dual has the form
%
\begin{equation}\label{eqn:antisymmetricTensorTx:540}
I x \sim e^{\sigma \pi \nu \mu} \gamma_\sigma \gamma_\sigma \gamma_\pi x_\nu
\end{equation}
%
The coordinates of this product are a multiple of \(\epsilon^{\sigma \pi \nu \mu } x_\mu\), which has the form specified in the text.
%
\section{Stokes Theorem}
%
%In \citep{gabook:stokesNoTensor} I worked through the Geometric Algebra expression for Stokes Theorem.  For a \(k-1\) grade blade, the final result of that work was
I once worked through the Geometric Algebra expression for Stokes Theorem.  For a \(k-1\) grade blade, the final result of that work was
%
\begin{equation}\label{eqn:antisymmetricTensorTx:600}
\int
( \grad \wedge F ) \cdot d^k x
=
\inv{(k-1)!} \epsilon^{ r s \cdots t u } \int da_u \PD{a_{u}}{F} \cdot
(dx_r \wedge dx_s \wedge \cdots \wedge dx_t)
\end{equation}
%
Let us expand this in coordinates to attempt to get the equivalent expression for an antisymmetric tensor of rank \(k-1\).

Starting with the RHS of \eqnref{eqn:antisymmetricTensorTx:600} we have
%
\begin{equation}\label{eqn:antisymmetricTensorTx:610}
\begin{aligned}
F &= \inv{(k-1)!}
F_{\mu_1 \mu_2 \cdots \mu_{k-1} }
\gamma^{\mu_1} \wedge \gamma^{ \mu_2 } \wedge \cdots \wedge \gamma^{\mu_{k-1}}
\\
dx_r \wedge dx_s \wedge \cdots \wedge dx_t &=
\PD{a_r}{x^{\nu_1}}
\PD{a_s}{x^{\nu_2}}
\cdots
\PD{a_t}{x^{\nu_{k-1}}}
\gamma_{\nu_1} \wedge \gamma_{ \nu_2 } \wedge \cdots \wedge \gamma_{\nu_{k-1}}
da_r da_s \cdots da_t
\end{aligned}
\end{equation}
%
We need to expand the dot product of the wedges, for which we have
\begin{equation}\label{eqn:antisymmetricTensorTx:620}
\begin{aligned}
&\left(
\gamma^{\mu_1} \wedge \gamma^{ \mu_2 } \wedge \cdots \wedge \gamma^{\mu_{k-1}}
\right)
\cdot
\left(
\gamma_{\nu_1} \wedge \gamma_{ \nu_2 } \wedge \cdots \wedge \gamma_{\nu_{k-1}}
\right) \\
&=
{\delta^{\mu_{k-1}}}_{\nu_1}  {\delta^{ \mu_{k-2} }}_{\nu_2}  \cdots  {\delta^{\mu_{1}} }_{\nu_{k-1}}
\epsilon^{\nu_1 \nu_2 \cdots \nu_{k-1}}.
\end{aligned}
\end{equation}
%
Putting all the LHS bits together we have
\begin{equation}\label{eqn:antisymmetricTensorTx:730}
\begin{aligned}
&\inv{((k-1)!)^2} \epsilon^{ r s \cdots t u } \int da_u \PD{a_{u}}{} F_{\mu_1 \mu_2 \cdots \mu_{k-1} } \\
&\qquad {\delta^{\mu_{k-1}}}_{\nu_1}  {\delta^{ \mu_{k-2} }}_{\nu_2}  \cdots  {\delta^{\mu_{1}} }_{\nu_{k-1}}
\epsilon^{\nu_1 \nu_2 \cdots \nu_{k-1}}
\PD{a_r}{x^{\nu_1}}
\PD{a_s}{x^{\nu_2}}
\cdots
\PD{a_t}{x^{\nu_{k-1}}}
da_r da_s \cdots da_t \\
&=\inv{((k-1)!)^2} \epsilon^{ r s \cdots t u } \int da_u \PD{a_{u}}{} F_{\mu_1 \mu_2 \cdots \mu_{k-1} } \\
&\qquad \epsilon^{\mu_{k-1} \mu_{k-2} \cdots \mu_{1}}
\PD{a_r}{x^{\mu_{k-1}}}
\PD{a_s}{x^{\mu_{k-2}}}
\cdots
\PD{a_t}{x^{\mu_1}}
da_r da_s \cdots da_t \\
&=
\inv{((k-1)!)^2} \epsilon^{ r s \cdots t u } \int da_u \PD{a_{u}}{} F_{\mu_1 \mu_2 \cdots \mu_{k-1} }
\Abs{\frac{\partial(
x^{\mu_{k-1}},x^{\mu_{k-2}},\cdots,x^{\mu_1}
)}{\partial(a_r, a_s, \cdots, a_t)}}
da_r da_s \cdots da_t.
\end{aligned}
\end{equation}
%
Now, for the LHS of \eqnref{eqn:antisymmetricTensorTx:600} we have
%
\begin{equation}\label{eqn:antisymmetricTensorTx:750}
\begin{aligned}
\grad \wedge F
&=
\gamma^\mu \wedge \partial_\mu F \\
&=
\inv{(k-1)!}\PD{x^{\mu_k}}{} F_{\mu_1 \mu_2 \cdots \mu_{k-1}}
\gamma^{\mu_k} \wedge
\gamma^{\mu_1} \wedge \gamma^{ \mu_2 } \wedge \cdots \wedge \gamma^{\mu_{k-1}},
\end{aligned}
\end{equation}
and the volume element of
\begin{equation}\label{eqn:antisymmetricTensorTx:770}
\begin{aligned}
d^k x
&=
\PD{a_1}{x^{\nu_1}}
\PD{a_2}{x^{\nu_2}}
\cdots
\PD{a_k}{x^{\nu_{k}}}
\gamma_{\nu_1} \wedge \gamma_{ \nu_2 } \wedge \cdots \wedge \gamma_{\nu_k}
da_1 da_2 \cdots da_k.
\end{aligned}
\end{equation}
%
Our dot product is
\begin{equation}\label{eqn:antisymmetricTensorTx:630}
\begin{aligned}
\left(\gamma^{\mu_k} \wedge
\gamma^{\mu_1} \wedge \gamma^{ \mu_2 } \wedge \cdots \wedge \gamma^{\mu_{k-1}} \right) & \cdot
\left( \gamma_{\nu_1} \wedge \gamma_{ \nu_2 } \wedge \cdots \wedge \gamma_{\nu_k} \right) \\
&=
{\delta^{\mu_{k-1}}}_{\nu_1}  {\delta^{ \mu_{k-2} }}_{\nu_2}  \cdots
{\delta^{\mu_{1}} }_{\nu_{k-1}}
{\delta^{\mu_{k}} }_{\nu_{k}}
\epsilon^{\nu_1 \nu_2 \cdots \nu_{k}}
\end{aligned}
\end{equation}
%
The LHS of our k-form now evaluates to
%
\begin{equation}\label{eqn:antisymmetricTensorTx:790}
\begin{aligned}
(\gamma^\mu \wedge \partial_\mu F) \cdot d^k x
&=
\inv{(k-1)!}\PD{x^{\mu_k}}{} F_{\mu_1 \mu_2 \cdots \mu_{k-1}} \\
&{\delta^{\mu_{k-1}}}_{\nu_1}  {\delta^{ \mu_{k-2} }}_{\nu_2}  \cdots
{\delta^{\mu_{1}} }_{\nu_{k-1}}
{\delta^{\mu_{k}} }_{\nu_{k}}
\epsilon^{\nu_1 \nu_2 \cdots \nu_{k}}
\PD{a_1}{x^{\nu_1}}
\PD{a_2}{x^{\nu_2}}
\cdots
\PD{a_k}{x^{\nu_{k}}}
da_1 da_2 \cdots da_k \\
&=
\inv{(k-1)!}\PD{x^{\mu_k}}{} F_{\mu_1 \mu_2 \cdots \mu_{k-1}} \\
&\epsilon^{\mu_{k-1} \mu_{k-2} \cdots \mu_1 \mu_{k}}
\PD{a_1}{x^{\mu_{k-1}}}
\PD{a_2}{x^{\mu_{k-2}}}
\cdots
\PD{a_{k-1}}{x^{\mu_{1}}}
\PD{a_k}{x^{\mu_{k}}}
da_1 da_2 \cdots da_k \\
&=
\inv{(k-1)!}\PD{x^{\mu_k}}{} F_{\mu_1 \mu_2 \cdots \mu_{k-1}}
\Abs{\frac{\partial(
x^{\mu_{k-1}},
x^{\mu_{k-2}},
\cdots
x^{\mu_{1}},
x^{\mu_{k}}
)}{\partial(a_1, a_2, \cdots, a_{k-1}, a_k)}
}
da_1 da_2 \cdots da_k.
\end{aligned}
\end{equation}
%
Presuming no mistakes were made anywhere along the way (including in the original Geometric Algebra expression), we have arrived at Stokes Theorem for rank \(k-1\) antisymmetric tensors \(F\)
\boxedEquation{eqn:antisymmetricTensorTx:650}{
\begin{aligned}
&\int
\PD{x^{\mu_k}}{} F_{\mu_1 \mu_2 \cdots \mu_{k-1}}
\Abs{\frac{\partial(
x^{\mu_{k-1}},
x^{\mu_{k-2}},
\cdots
x^{\mu_{1}},
x^{\mu_{k}}
)}{\partial(a_1, a_2, \cdots, a_{k-1}, a_k)}
}
da_1 da_2 \cdots da_k \\
&=
\inv{(k-1)!} \epsilon^{ r s \cdots t u } \int da_u \PD{a_{u}}{} F_{\nu_1 \nu_2 \cdots \nu_{k-1} }
\Abs{\frac{\partial(
x^{\nu_{k-1}},x^{\nu_{k-2}},\cdots,x^{\nu_1}
)}{\partial(a_r, a_s, \cdots, a_t)}}
da_r da_s \cdots da_t.
\end{aligned}
}

The next task is to validate this, expanding it out for some specific ranks and hypervolume element types, and to compare the results with the familiar 3d expressions.
