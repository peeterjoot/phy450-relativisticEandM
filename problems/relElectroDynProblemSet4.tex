%
% Copyright � 2012 Peeter Joot.  All Rights Reserved.
% Licenced as described in the file LICENSE under the root directory of this GIT repository.
%

\label{chap:relElectroDynProblemSet4}
%\blogpage{http://sites.google.com/site/peeterjoot/math2011/relElectroDynProblemSet4.pdf}
%\date{Mar 3, 2011}
%
\makeproblem{Energy, momentum, etc., of EM waves}{pr:relElectroDynProblemSet4:1}{
%
\makesubproblem{Energy and momentum density}{pr:relElectroDynProblemSet4:1a}
%
Calculate the energy density, energy flux, and momentum density of a plane monochromatic linearly polarized electromagnetic wave.
\makesubproblem{}{pr:relElectroDynProblemSet4:1b}
%
Calculate the values of these quantities averaged over a period.
\makesubproblem{}{pr:relElectroDynProblemSet4:1c}
%
Imagine that a plane monochromatic linearly polarized wave incident on a surface (let the angle between the wave vector and the normal to the surface be \(\theta\)) is completely reflected.  Find the pressure that the EM wave exerts on the surface.
\makesubproblem{}{pr:relElectroDynProblemSet4:1d}
%
To plug in some numbers, note that the intensity of sunlight hitting the Earth is about \(1300 W/m^2\) ( the intensity is the average power per unit area transported by the wave).  If sunlight strikes a perfect absorber, what is the pressure exerted?  What if it strikes a perfect reflector?  What fraction of the atmospheric pressure does this amount to?

} % makeproblem
%
\makeanswer{pr:relElectroDynProblemSet4:1}{
%
\makeSubAnswer{}{pr:relElectroDynProblemSet4:1a}
%
Because it does not add too much complexity, I am going to calculate these using the more general elliptically polarized wave solutions.  Our vector potential (in the Coulomb gauge \(\phi = 0\), \(\spacegrad \cdot \BA = 0\)) has the form
%
\begin{equation}\label{eqn:relElectroDynProblemSet4:10}
\BA = \Real \Bbeta e^{i (\omega t - \Bk \cdot \Bx) }.
\end{equation}
%
The elliptical polarization case only differs from the linear by allowing \(\Bbeta\) to be complex, rather than purely real or purely imaginary.  Observe that the Coulomb gauge condition \(\spacegrad \cdot \BA\) implies
%
\begin{equation}\label{eqn:relElectroDynProblemSet4:30}
\Bbeta \cdot \Bk = 0,
\end{equation}
%
a fact that will kill of terms in a number of places in the following manipulations.

Also observe that for this to be a solution to the wave equation operator
%
\begin{equation}\label{eqn:relElectroDynProblemSet4:50}
\inv{c^2} \PDSq{t}{} - \Delta,
\end{equation}
%
the frequency and wave vector must be related by the condition
%
\begin{equation}\label{eqn:relElectroDynProblemSet4:70}
\frac{\omega}{c} = \Abs{\Bk} = k.
\end{equation}
%
For the time and spatial phase let us write
%
\begin{equation}\label{eqn:relElectroDynProblemSet4:90}
\theta = \omega t - \Bk \cdot \Bx.
\end{equation}
%
In the Coulomb gauge, our electric and magnetic fields are
%
\begin{equation}\label{eqn:relElectroDynProblemSet4:110}
\begin{aligned}
\BE &= -\inv{c}\PD{t}{\BA} = \Real \frac{-i\omega}{c} \Bbeta e^{i\theta} \\
\BB &= \spacegrad \cross \BA = \Real i \Bbeta \cross \Bk e^{i\theta}.
\end{aligned}
\end{equation}
%
Similar to \S 48 of the text \citep{landau1980classical}, let us split \(\Bbeta\) into a phase and perpendicular vector components so that
%
\begin{equation}\label{eqn:relElectroDynProblemSet4:130}
\Bbeta = \Bb e^{-i\alpha}.
\end{equation}
%
where \(\Bb\) has a real square
%
\begin{equation}\label{eqn:relElectroDynProblemSet4:150}
\Bb^2 = \Abs{\Bbeta}^2.
\end{equation}
%
This allows a split into two perpendicular real vectors
%
\begin{equation}\label{eqn:relElectroDynProblemSet4:170}
\Bb = \Bb_1 + i \Bb_2,
\end{equation}
where \(\Bb_1 \cdot \Bb_2 = 0\) since \(\Bb^2 = \Bb_1^2 - \Bb_2^2 + 2 \Bb_1 \cdot \Bb_2\) is real.

Our electric and magnetic fields are now reduced to
\begin{equation}\label{eqn:relElectroDynProblemSet4:190}
\begin{aligned}
\BE &= \Real \left( \frac{-i\omega}{c} \Bb e^{i(\theta - \alpha)} \right) \\
\BB &= \Real \left( i \Bb \cross \Bk e^{i(\theta - \alpha)} \right),
\end{aligned}
\end{equation}
or explicitly in terms of \(\Bb_1\) and \(\Bb_2\)
%
\begin{equation}\label{eqn:relElectroDynProblemSet4:210}
\begin{aligned}
\BE &= \frac{\omega}{c} ( \Bb_1 \sin(\theta-\alpha) + \Bb_2 \cos(\theta-\alpha)) \\
\BB &= ( \Bk \cross \Bb_1 ) \sin(\theta-\alpha) + (\Bk \cross \Bb_2) \cos(\theta-\alpha).
\end{aligned}
\end{equation}
%
The special case of interest for this problem, since it only strictly asked for linear polarization, is where \(\alpha = 0\) and one of \(\Bb_1\) or \(\Bb_2\) is zero (i.e. \(\Bbeta\) is strictly real or strictly imaginary).  The case with \(\Bbeta\) strictly real, as done in class, is
%
\begin{equation}\label{eqn:relElectroDynProblemSet4:230}
\begin{aligned}
\BE &= \frac{\omega}{c} \Bb_1 \sin(\theta-\alpha) \\
\BB &= ( \Bk \cross \Bb_1 ) \sin(\theta-\alpha).
\end{aligned}
\end{equation}
%
Now lets calculate the energy density and Poynting vectors.  We will need a few intermediate results.
%
\begin{equation}\label{eqn:relElectroDynProblemSet4:380}
\begin{aligned}
(\Real \Bd e^{i\phi})^2
&= \inv{4} ( \Bd e^{i\phi} + \Bd^\conj e^{-i\phi})^2 \\
&= \inv{4} ( \Bd^2 e^{2 i \phi} + (\Bd^\conj)^2 e^{-2 i \phi} + 2 \Abs{\Bd}^2 ) \\
&= \inv{2} \left( \Abs{\Bd}^2 + \Real ( \Bd e^{i \phi} )^2 \right),
\end{aligned}
\end{equation}
%
and
%
\begin{equation}\label{eqn:relElectroDynProblemSet4:400}
\begin{aligned}
(\Real \Bd e^{i\phi}) \cross (\Real \Be e^{i\phi})
&= \inv{4}
( \Bd e^{i\phi} + \Bd^\conj e^{-i\phi}) \cross ( \Be e^{i\phi} + \Be^\conj e^{-i\phi}) \\
&= \inv{2} \Real \left( \Bd \cross \Be^\conj + (\Bd \cross \Be) e^{2 i \phi} \right).
\end{aligned}
\end{equation}
%
Let us use arrowed vectors for the phasor parts
%
\begin{equation}\label{eqn:relElectroDynProblemSet4:250}
\begin{aligned}
\vec{E} &= \frac{-i\omega}{c} \Bb e^{i(\theta - \alpha)} \\
\vec{B} &= i \Bb \cross \Bk e^{i(\theta - \alpha)},
\end{aligned}
\end{equation}
%
where we can recover our vector quantities by taking real parts \(\BE = \Real \vec{E}\), \(\BB = \Real \vec{B}\).  Our energy density in terms of these phasors is then
%
\begin{equation}\label{eqn:relElectroDynProblemSet4:270}
\calE
= \inv{8\pi} (\BE^2 + \BB^2)
= \inv{16\pi} \left( \Abs{\vec{E}}^2 + \Abs{\vec{B}}^2 + \Real ({\vec{E}}^2 + {\vec{B}}^2) \right).
\end{equation}
This is
\begin{equation}\label{eqn:relElectroDynProblemSet4:420}
\begin{aligned}
\calE
&=
\inv{16\pi}
\left(
\frac{\omega^2}{c^2} \Abs{\Bb}^2 + \Abs{\Bb \cross \Bk}^2
-\Real \left(
\frac{\omega^2}{c^2} \Bb^2 + (\Bb \cross \Bk)^2
\right)
e^{2 i(\theta - \alpha)}
\right).
\end{aligned}
\end{equation}
%
Note that \(\omega^2/c^2 = \Bk^2\), and \(\Abs{\Bb \cross \Bk} = \Abs{\Bb}^2 \Bk^2\) (since \(\Bb \cdot \Bk = 0\)).  Also \((\Bb \cross \Bk)^2 = \Bb^2 \Bk^2\), so we have
\boxedEquation{eqn:relElectroDynProblemSet4:290}{
\calE
=
\frac{ \Bk^2 }{8\pi}
\left(
\Abs{\Bb}^2
-\Real \Bb^2 e^{2 i(\theta - \alpha)}
\right).
}

Now, for the Poynting vector.  We have
%
\begin{equation}\label{eqn:relElectroDynProblemSet4:310}
S = \frac{c}{4 \pi} \BE \cross \BB = \frac{c}{8 \pi} \Real \left( \vec{E} \cross \vec{B}^\conj + \vec{E} \cross \vec{B} \right).
\end{equation}
This is
\begin{equation}\label{eqn:relElectroDynProblemSet4:440}
\begin{aligned}
S
&= \frac{c}{8 \pi} \Real \left( -k \Bb \cross (\Bb^\conj \cross \Bk) + k \Bb \cross (\Bb \cross \Bk ) e^{2 i(\theta - \alpha)} \right).
\end{aligned}
\end{equation}
%
Reducing the terms we get \(\Bb \cross (\Bb^\conj \cross \Bk) = -\Bk \Abs{\Bb}^2\), and \(\Bb \cross (\Bb \cross \Bk) = -\Bk \Bb^2\), leaving
\boxedEquation{eqn:relElectroDynProblemSet4:330}{
S
= \frac{c \hat{\Bk} \Bk^2 }{8 \pi} \left( \Abs{\Bb}^2 - \Real \Bb^2 e^{2 i(\theta - \alpha)} \right) = c \hat{\Bk} \calE.
}

Now, the text in \S 47 defines the energy flux as the Poynting vector, and the momentum density as \(\BS/c^2\), so we just divide \eqnref{eqn:relElectroDynProblemSet4:330} by \(c^2\) for the momentum density and we are done.  For the linearly polarized case (all that was actually asked for, but less cool to calculate), where \(\Bb\) is real, we have
%
\begin{equation}\label{eqn:relElectroDynProblemSet4:350}
\begin{aligned}
\mbox{Energy density} &= \calE = \frac{ \Bk^2 \Bb^2 }{8\pi} ( 1 - \cos( 2 (\omega t - \Bk \cdot \Bx)) ) \\
\mbox{Energy flux} &= \BS = c \hat{\Bk} \calE \\
\mbox{Momentum density} &= \inv{c^2} \BS = \frac{\hat{\Bk}}{c} \calE.
\end{aligned}
\end{equation}
%
\makeSubAnswer{}{pr:relElectroDynProblemSet4:1b}
%
We want to average over one period, the time \(T\) such that \(\omega T = 2 \pi\), so the average is
%
\begin{equation}\label{eqn:relativisticElectrodynamicsL16:360}
\expectation{f} = \frac{\omega}{2\pi} \int_0^{2\pi/\omega} f dt.
\end{equation}
%
It is clear that this will just kill off the sinusoidal terms, leaving
%
\begin{equation}\label{eqn:relElectroDynProblemSet4:350b}
\begin{aligned}
\mbox{Average Energy density} &= \expectation{\calE} = \frac{ \Bk^2 \Abs{\Bb}^2 }{8\pi} \\
\mbox{Average Energy flux} &= \expectation{\BS} = c \hat{\Bk} \calE \\
\mbox{Average Momentum density} &= \inv{c^2} \expectation{\BS} = \frac{\hat{\Bk}}{c} \calE.
\end{aligned}
\end{equation}
%
\makeSubAnswer{}{pr:relElectroDynProblemSet4:1c}
%
The magnitude of the momentum of light is related to its energy by
%
\begin{equation}\label{eqn:relElectroDynProblemSet4:351}
\Bp = \frac{\calE}{c}.
\end{equation}
%
and can thus loosely identify the magnitude of the force as
%
\begin{equation}\label{eqn:relElectroDynProblemSet4:460}
\begin{aligned}
\ddt{\Bp}
&= \inv{c} \PD{t}{} \int \frac{\BE^2 + \BB^2}{8 \pi} d^3 \Bx \\
&= \int d^2 \Bsigma \cdot \frac{\BS}{c}.
\end{aligned}
\end{equation}
%
With pressure as the force per area, we could identify
%
\begin{equation}\label{eqn:relElectroDynProblemSet4:252}
\frac{\BS}{c}.
\end{equation}
%
as the instantaneous (directed) pressure on a surface.  What is that for linearly polarized light?  We have from above for the linear polarized case (where \(\Abs{\Bb}^2 = \Bb^2\))
%
\begin{equation}\label{eqn:relElectroDynProblemSet4:353}
\BS = \frac{c \kcap \Bk^2 \Bb^2 }{8 \pi} ( 1 - \cos( 2 (\omega t - \Bk \cdot \Bx) ) ).
\end{equation}
%
If we look at the magnitude of the average pressure from the radiation, we have
%
\begin{equation}\label{eqn:relElectroDynProblemSet4:354}
\Abs{\frac{\expectation{\BS}}{c}} = \frac{\Bk^2 \Bb^2 }{8 \pi}.
\end{equation}
%
\makeSubAnswer{}{pr:relElectroDynProblemSet4:1d}
%
With atmospheric pressure at \(101.3 k Pa\), and the pressure from the light at \(1300 W/ 3 x 10^8 m/s\), we have roughly \(4 x 10^-5 Pa\) of pressure from the sunlight being only \(\sim 10^-{10}\) of the total atmospheric pressure.  Wow.  Very tiny!

Would it make any difference if the surface is a perfect absorber or a reflector?  Consider a ball hitting a wall.  If it manages to embed itself in the wall, the wall will have to move a bit to conserve momentum.  However, if the ball bounces off twice the momentum has been transferred to the wall.  The numbers above would be for perfect absorbtion, so double them for a perfect reflector.
}

