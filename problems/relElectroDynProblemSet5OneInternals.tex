%
% Copyright � 2012 Peeter Joot.  All Rights Reserved.
% Licenced as described in the file LICENSE under the root directory of this GIT repository.
%

%
\makeproblem{Sinusoidal current density on an infinite flat conducting sheet}{pr:relElectroDynProblemSet5:1}{
%
An infinitely thin flat conducting surface lying in the \(x-z\) plane carries a surface current density:
%
\begin{equation}\label{eqn:relElectroDynProblemSet5P1:10}
\Bkappa = \Be_3 \theta(t) \kappa_0 \sin\omega t
\end{equation}
%
Here \(\Be_3\) is a unit vector in the \(z\) direction, \(\kappa_0\) is the peak value of the current density, and \(\theta(t)\) is the theta function: \(\theta(t < 0) - 0, \theta(t > 0) = 1\).

%
\makesubproblem{}{pr:relElectroDynProblemSet5:1:a}
%
Write down the equations determining the electromagnetic potentials.  Specify which gauge you choose to work in.
\makesubproblem{}{pr:relElectroDynProblemSet5:1:b}
%
Find the electromagnetic potentials outside the plane.
\makesubproblem{}{pr:relElectroDynProblemSet5:1:c}
%
Find the electric and magnetic fields outside the plane.
\makesubproblem{}{pr:relElectroDynProblemSet5:1:d}
%
Give a physical interpretation of the results of the previous section.  Do they agree with your qualitative expectations?
\makesubproblem{}{pr:relElectroDynProblemSet5:1:e}
%
Find the direction and magnitude of the energy flux outside the plane.
\makesubproblem{}{pr:relElectroDynProblemSet5:1:f}
%
Consider a point at some distance from the plane.  Sketch the intensity of the electromagnetic field near this point as a function of time.  Explain physically.
\makesubproblem{}{pr:relElectroDynProblemSet5:1:g}
%
Consider now a point near the plane.  Are the electric and magnetic fields you found continuous across the conducting plane?  Explain.
} % makeproblem
%
\makeanswer{pr:relElectroDynProblemSet5:1}{
%
\makeSubAnswer{Determining the electromagnetic potentials.}{pr:relElectroDynProblemSet5:1:a}
%\makeSubAnswer{}{pr:relElectroDynProblemSet5:1:b}
%
Augmenting the surface current density with a delta function we can form the current density for the system
%
\begin{equation}\label{eqn:relElectroDynProblemSet5P1:30}
\BJ = \delta(y) \Bkappa = \Be_3 \theta(t) \delta(y) \kappa_0 \sin\omega t.
\end{equation}
%
With only a current distribution specified use of the Coulomb gauge allows for setting the scalar potential on the surface equal to zero, so that we have
%
\begin{equation}\label{eqn:relElectroDynProblemSet5P1:50}
\begin{aligned}
\square \BA &= \frac{4 \pi \BJ}{c} \\
\BE &= - \inv{c} \PD{t}{\BA} \\
\BB &= \spacegrad \cross \BB
\end{aligned}
\end{equation}
%
Utilizing our Green's function
%
\begin{equation}\label{eqn:relElectroDynProblemSet5P1:70}
G(\Bx, t) = \frac{\delta(t - \Abs{\Bx}/c)}{4 \pi \Abs{\Bx}} = \delta^3(\Bx)\delta(t),
\end{equation}
%
we can invert our vector potential equation, solving for \(\BA\)
%
\begin{equation}\label{eqn:relElectroDynProblemSet5OneInternals:282}
\begin{aligned}
\BA(\Bx, t)
&= \int d^3 \Bx' dt' \square_{\Bx', t'} G(\Bx - \Bx', t - t') \BA(\Bx', t') \\
&= \int d^3 \Bx' dt' G(\Bx - \Bx', t - t') \frac{4 \pi \BJ(\Bx', t')}{c} \\
&= \int d^3 \Bx' dt'
\frac{\delta(t - t' - \Abs{\Bx -\Bx'}/c)}{4 \pi \Abs{\Bx - \Bx'}}
\frac{4 \pi \BJ(\Bx', t')}{c} \\
&= \int d^3 \Bx'
\frac{\BJ(\Bx', t - \Abs{\Bx - \Bx'}/c}{c \Abs{\Bx - \Bx'}} \\
&= \inv{c} \int dx' dy' dz'
\Be_3 \theta(t - \Abs{\Bx -\Bx'}/c) \delta(y) \kappa_0  \\
&\qquad \sin(\omega(t - \Abs{\Bx -\Bx'}/c))
\inv{\Abs{\Bx - \Bx'}} \\
&= \frac{
\Be_3 \kappa_0
}{c} \int dx' dz'
\theta(t - \Abs{\Bx -(x', 0, z')}/c)  \\
&\qquad \sin(\omega(t - \Abs{\Bx -(x', 0, z')}/c))
\inv{\Abs{\Bx - (x', 0, z')}} \\
&= \frac{
\Be_3 \kappa_0
}{c} \int dx' dz'
\theta\left(t - \inv{c} \sqrt{(x-x')^2 + y^2 + (z-z')^2}\right)  \\
&\qquad \frac{\sin\left(\omega\left(t - \inv{c} \sqrt{(x-x')^2 + y^2 + (z-z')^2}\right)\right)}{\sqrt{(x-x')^2 + y^2 + (z-z')^2}}
\end{aligned}
\end{equation}
%
Now a switch to polar coordinates makes sense.  Let us use
%
\begin{equation}\label{eqn:relElectroDynProblemSet5P1:90}
\begin{aligned}
x' - x &= r \cos\alpha \\
z' - z &= r \sin\alpha
\end{aligned}
\end{equation}
%
This gives us
%
\begin{equation}\label{eqn:relElectroDynProblemSet5OneInternals:302}
\begin{aligned}
\BA(\Bx, t)
&= \frac{
\Be_3 \kappa_0
}{c} \int_{r=0}^\infty \int_{\alpha=0}^{2\pi} r dr d\alpha
\theta\left(t - \inv{c} \sqrt{r^2 + y^2}\right)
\frac{\sin\left(\omega\left(t - \inv{c} \sqrt{r^2 + y^2 }\right)\right)}{\sqrt{r^2 + y^2}} \\
&= \frac{
2 \pi \Be_3 \kappa_0
}{c} \int_{r=0}^\infty r dr
\theta\left(t - \inv{c} \sqrt{r^2 + y^2}\right)
\frac{\sin\left(\omega\left(t - \inv{c} \sqrt{r^2 + y^2 }\right)\right)}{\sqrt{r^2 + y^2}}.
\end{aligned}
\end{equation}
%
Since the theta function imposes a
%
\begin{equation}\label{eqn:relElectroDynProblemSet5P1:110}
t - \inv{c} \sqrt{r^2 + y^2 } > 0
\end{equation}
%
constraint, equivalent to
%
\begin{equation}\label{eqn:relElectroDynProblemSet5P1:130}
c^2 t^2 > r^2 + y^2,
\end{equation}
%
we can reduce the upper range of the integral
%
\begin{equation}\label{eqn:relElectroDynProblemSet5OneInternals:322}
\begin{aligned}
\BA(\Bx, t)
&= \frac{
2 \pi \Be_3 \kappa_0
}{c}
\int_{r=0}^{\sqrt{c^2 t^2 - y^2}} r dr
\frac{\sin\left(\omega\left(t - \inv{c} \sqrt{r^2 + y^2 }\right)\right)}{\sqrt{r^2 + y^2}}
\theta\left(t - \inv{c} \sqrt{r^2 + y^2}\right) \\
&= \frac{
2 \pi \Be_3 \kappa_0
}{c}
\int_{r=0}^{\sqrt{c^2 t^2 - y^2}} \frac{ \omega r}{c} \frac{\omega dr }{c} \\
&\qquad \frac{\sin\left(\omega\left(t - \inv{c} \sqrt{r^2 + y^2 }\right)\right)}{\frac{\omega}{c} \sqrt{r^2 + y^2}} \frac{c}{\omega}
\theta\left(t - \inv{\omega} \sqrt{\frac{\omega^2 r^2}{c^2} + k^2 y^2}\right)  \\
&=
\frac{ 2 \pi \Be_3 \kappa_0 }{\omega}
\int_{u=0}^{\sqrt{\omega^2 t^2 - k^2 y^2}} u du  \\
&\qquad \frac{\sin\left( \omega t - \sqrt{u^2 + k^2 y^2 } \right)}{\sqrt{u^2 + k^2 y^2}}
\theta\left(t - \inv{\omega} \sqrt{u^2 + k^2 y^2}\right).
\end{aligned}
\end{equation}
%
Here \(k = \omega/c\), and \(u = k r\).  One more change of variables
%
\begin{equation}\label{eqn:relElectroDynProblemSet5P1:131}
\begin{aligned}
v^2 &= u^2 + k^2 y^2 \\
v dv &= u du,
\end{aligned}
\end{equation}
%
gives us
%
\begin{equation}\label{eqn:relElectroDynProblemSet5OneInternals:342}
\begin{aligned}
u du \frac{\sin\left( \omega t - \sqrt{u^2 + k^2 y^2 } \right)}{\sqrt{u^2 + k^2 y^2}}
&= v dv \frac{\sin\left( \omega t - \Abs{v} \right)}{\Abs{v}} \\
&= dv \frac{d}{dv} \cos(\omega t - \Abs{v})
\end{aligned}
\end{equation}
%
Omitting the integration limits temporarily we want to evaluate
%
\begin{equation}\label{eqn:relElectroDynProblemSet5OneInternals:362}
\begin{aligned}
&\int dv \theta( t - \Abs{v}/\omega ) \frac{d}{dv} \cos(\omega t - \Abs{v}) \\
&=
\int dv
\frac{d}{dv} \left( \cos(\omega t - \Abs{v}) \theta( t - \Abs{v}/\omega ) \right)
-
\int dv \cos(\omega t - \Abs{v} ) \frac{d}{dv} \theta( t - \Abs{v}/\omega ) \\
&=
\cos(\omega t - \Abs{v})
\theta( t - \Abs{v}/\omega )
-
\int dv \cos(\omega t - \Abs{v} ) \delta( t - \Abs{v}/\omega ) \left(-\frac{\sgn{v}}{\omega}\right)
\end{aligned}
\end{equation}
%
This last integral only takes a value at \(v = \Abs{v} = \sqrt{u^2 + k^2 y^2} = \omega t\), and recalling that \(\delta(a x) = \delta(x)/\Abs{a}\) we have
%
\begin{equation}\label{eqn:relElectroDynProblemSet5P1:132}
-
\int dv \cos(\omega t - \Abs{v} ) \delta( t - \Abs{v}/\omega ) \left(-\frac{\sgn{v}}{\omega}\right)
=
\cos(0) = 1.
\end{equation}
%
However because we are integrating over a definite range, this entire term therefore vanishes.  We are left with
%
\begin{equation}\label{eqn:relElectroDynProblemSet5OneInternals:382}
\begin{aligned}
\BA(\Bx, t)
&=
\frac{ 2 \pi \Be_3 \kappa_0 }{\omega}
\left. \cos(\omega t - \sqrt{u^2 + k^2 y^2})
\theta\left(t - \frac{\sqrt{u^2 + k^2 y^2}}{\omega} \right)
\right\vert_{u=0}^{\sqrt{\omega^2 t^2 - k^2 y^2}} \\
&=
\frac{ 2 \pi \Be_3 \kappa_0 }{\omega}
\Bigl(
\cos(\omega t - \omega \Abs{t}) \theta(t - \Abs{t})
- \cos(\omega t - k \Abs{y} ) \theta( t - \Abs{y}/c )
\Bigr)
\end{aligned}
\end{equation}
%
For \(t \ge 0\), \(\theta(t - \Abs{t}) = \theta(0) = 1/2\), but is zero for \(t < 0\), so we have
%
\begin{equation}\label{eqn:relElectroDynProblemSet5P1:170b}
\BA(\Bx, t)
= \frac{ 2 \pi \kappa_0 }{\omega} \Be_3 \left( \inv{2} - \cos(\omega(t - \Abs{y}/c)) \theta(t - \Abs{y}/c) \right).
\end{equation}
%
However, since we take either spatial or time derivatives of the vector potential to get the fields, the constant term will not effect the result, so it is equivalent to write just
%
\begin{equation}\label{eqn:relElectroDynProblemSet5P1:170}
\BA(\Bx, t)
= -\frac{ 2 \pi \kappa_0 }{\omega} \Be_3 \cos(\omega(t - \Abs{y}/c)) \theta(t - \Abs{y}/c).
\end{equation}
%
\makeSubAnswer{Find the electric and magnetic fields outside the plane}{pr:relElectroDynProblemSet5:1:c}
%
Our electric field can be calculated by inspection.  For \(t > \Abs{y}/c\) we have
%
\begin{equation}\label{eqn:relElectroDynProblemSet5P1:190}
\BE = -\inv{c}\PD{t}{\BA}
= -\frac{
2 \pi \kappa_0 \omega
}{c^2} \Be_3 \sin(\omega(t - \Abs{y}/c)).
\end{equation}
%
For the magnetic field we have, also for \(t > \Abs{y}/c\) we have
%
\begin{equation}\label{eqn:relElectroDynProblemSet5OneInternals:402}
\begin{aligned}
\BB
&= \spacegrad \cross \BA \\
&= -\frac{ 2 \pi \kappa_0 }{c} \Be_3 \cross \spacegrad (1 -\cos(\omega(t - \Abs{y}/c)))) \\
&= \frac{
2 \pi \kappa_0
}{c}
(-\sin(\omega(t - \Abs{y}/c))
\Be_3 \cross \spacegrad \omega(t - \Abs{y}/c) \\
&= \frac{
2 \pi \kappa_0 \omega
}{c^2}
\sin(\omega(t - \Abs{y}/c))
\Be_3 \cross
 \spacegrad \Abs{y} \\
&= \frac{
2 \pi \kappa_0 \omega
}{c^2}
\sin(\omega(t - \Abs{y}/c))
\Be_3 \cross \Be_2,
\end{aligned}
\end{equation}
%
which gives us
%
\begin{equation}\label{eqn:relElectroDynProblemSet5P1:210}
\begin{aligned}
\BE &= -\frac{ 2 \pi \kappa_0 \omega }{c^2} \Be_3 \sin(\omega(t - \Abs{y}/c)) \theta(t - \Abs{y}/c) \\
\BB &= -\frac{ 2 \pi \kappa_0 \omega }{c^2} \Be_1 \sin(\omega(t - \Abs{y}/c)) \theta(t - \Abs{y}/c)
\end{aligned}
\end{equation}
%
\makeSubAnswer{Give a physical interpretation of the results of the previous section}{pr:relElectroDynProblemSet5:1:d}
%
It was expected that the lack of boundary on the conducting sheet would make the potential away from the plane only depend on the \(y\) components of the spatial distance, and this is precisely what we find performing the grunt work of the integration.

Given that we had a sinusoidal forcing function for our wave equation, it seems logical that we also find our non-homogeneous solution to the wave equation has sinusoidal dependence.  We find that the sinusoidal current results in sinusoidal potentials and fields very much like one has in the electric circuits problem that we solve with phasors in engineering applications.

We find that the electric and magnetic fields are oriented parallel to the plane containing the surface current density, with the electric field in the direction of the current, and the magnetic field perpendicular to that, but having energy propagate outwards from the plane.

We see that the step function for the current results in a transient response, which is intuitively pleasing.  The application of the current does not result in an instantanious field in all space, but instead there is time required for the field to propagate to the point of measurement.  The time required for the field to propagate is the time for light to reach that point \(t = \Abs{y}/c\).
%
\makeSubAnswer{Find the direction and magnitude of the energy flux outside the plane}{pr:relElectroDynProblemSet5:1:e}
%
Our energy flux, the Poynting vector, is
%
\begin{equation}\label{eqn:relElectroDynProblemSet5P1:220}
\BS
= \frac{c}{4\pi}
\left( \frac{
2 \pi \kappa_0 \omega
}{c^2} \right)^2 \sin^2(\omega(t - \Abs{y}/c) \Be_3 \cross \Be_1.
\end{equation}
This is
\begin{equation}\label{eqn:relElectroDynProblemSet5P1:240}
\BS
=
\frac{ \pi \kappa_0^2 \omega^2 }{c^3} \sin^2(\omega(t - \Abs{y}/c) \Be_2
=
\frac{ \pi \kappa_0^2 \omega^2 }{2 c^3} (1 - \cos( 2 \omega(t - \Abs{y}/c) ) ) \Be_2.
\end{equation}
%
This energy flux is directed outwards along the \(y\) axis, with magnitude oscillating around an average value of
%
\begin{equation}\label{eqn:relElectroDynProblemSet5P1:260}
\Abs{\expectation{S}} = \frac{ \pi \kappa_0^2 \omega^2 }{2 c^3}.
\end{equation}
%
\makeSubAnswer{Sketch the intensity of the electromagnetic field far from the plane}{pr:relElectroDynProblemSet5:1:f}
%
I am assuming here that this question does not refer to the flux intensity \(\expectation{\BS}\), since that is constant, and boring to sketch.

The time varying portion of either the electric or magnetic field is proportional to
%
\begin{equation}\label{eqn:relElectroDynProblemSet5P1:261}
\sin( \omega t - \omega \Abs{y}/c )
\end{equation}
%
We have a sinusoid as a function of time, of period \(T = 2 \pi/\omega\) where the phase is adjusted at each position by the factor \(\omega \Abs{y}/c\).  Every increase of \(\Delta y = 2 \pi c/\omega\) shifts the waveform back.

A sketch is attached.
%
\makeSubAnswer{Continuity across the plane?}{pr:relElectroDynProblemSet5:1:g}
%
It is sufficient to consider either the electric or magnetic field for the continuity question since the continuity is dictated by the sinusoidal term for both fields.

The point in time only changes the phase, so let us consider the electric field at \(t=0\), and an infinitesimal distance \(y = \pm \epsilon c/\omega\).  At either point we have
%
\begin{equation}\label{eqn:relElectroDynProblemSet5P1:262}
\BE(0, \pm \epsilon c/\omega, 0, 0) = \frac{ 2 \pi \kappa_0 \omega }{c^2} \Be_3 \epsilon
\end{equation}
%
In the limit as \(\epsilon \rightarrow 0\) the field strength matches on either side of the plane (and happens to equal zero for this \(t= 0\) case).

We have a discontinuity in the spatial derivative of either field near the plate, but not for the fields themselves.  A plot illustrates this nicely

\imageFigure{../figures/phy450-relativisticEandM/phy450ps5P1Q7}{\(\sin(t - \Abs{y})\)}{fig:phy450ps5P1Q7}{0.4}

FIXME: this plot was from before I had reintroduced the \(\theta\) function I had dropped.  It is not right, and does not display the transient response that I expected but did not see in the calculation I had submitted originally.
%
\paragraph{Grading notes}
%
This was the graded question (I lost 1.5 marks).  I got my units wrong when I integrated to find \(\BA\), resulting in an additional \(\omega/c\) in every result from that point on.  I should have done a dimensional analysis check.  I also dropped the \(\theta\) function thinking that incorporating that into the integral bounds was enough.  Without this we do not have the \(t > \Abs{y}/c\) propagation rate for the fields, and they counterinutively (and erroneously) appear at all points in space.  I have fixed up the units and reworked the bits utilizing the \(\theta\) functions now, and believe it to be correct.  I had had trouble with the interpretation part of the question initially since my result did not make sense to me.
} % makeanswer
