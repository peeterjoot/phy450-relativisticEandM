%
% Copyright � 2012 Peeter Joot.  All Rights Reserved.
% Licenced as described in the file LICENSE under the root directory of this GIT repository.
%

\label{chap:relElectroDynProblemSet1}
%\blogpage{http://sites.google.com/site/peeterjoot/math2011/relElectroDynProblemSet1.pdf}
%\date{Jan 22, 2011}
%
%\section{Disclaimer}
%
%This problem set is as yet ungraded.  It is also incomplete since I ended up hand-writing some parts.
%
%\section{Problem 1}
%\subsection{Statement}
%
%Some Minkowski diagram exercises.
%
%\subsection{Solution}
%
%These will be hand written.
%%%I will hand write my solutions for these.  To prep for it, let us suppose that \((ct', x')\) and \((ct, x)\) are related by
%%%
%%%\begin{equation}\label{eqn:relativisticElectrodynamicsA1:1}
%%%\begin{bmatrix}
%%%x' \\
%%%ct'
%%%\end{bmatrix}
%%%\begin{bmatrix}
%%%\cosh \alpha & \sinh\alpha \\
%%%\sinh \alpha & \cosh\alpha
%%%\end{bmatrix}
%%%\begin{bmatrix}
%%%x \\
%%%ct
%%%\end{bmatrix}.
%%%\end{equation}
%%%
%%%Some rearrangement, after eliminating \(x\) yields a hyperbolic family of curves, one for each value of \(ct\)
%%%
%%%\begin{equation}\label{eqn:relativisticElectrodynamicsA1:2}
%%%\frac{(ct'/ct)^2}{1 -\beta^2} - \frac{(x'/ct)^2}{(1-\beta^2)/\beta^2} = 1
%%%\end{equation}
%%%
%%%Considering the \(ct = 0\) curves for \(ct\ne 0\) we get real solutions for \(x\) thus, fixing the orientation of the hyperbolas, so we see these hyperbolas cross the x-axis, and not the \(ct\)-axis.


