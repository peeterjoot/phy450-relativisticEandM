%
% Copyright � 2012 Peeter Joot.  All Rights Reserved.
% Licenced as described in the file LICENSE under the root directory of this GIT repository.
%

\label{chap:complexPlanarBoost}
%\blogpage{http://sites.google.com/site/peeterjoot/math2011/complexPlanarBoost.pdf}
%\date{Mar 8, 2011}
%
\section{Motivation}
%
In the electrodynamics midterm we had a question on circular motion.  This screamed for use of complex numbers to describe the spatial parts of the spacetime trajectories.

Let us play with this a bit.
%
\section{Our invariant}
%
Suppose we describe our spacetime point as a paired time and complex number
%
\begin{equation}\label{eqn:complexPlanarBoost:10}
X = (ct, z).
\end{equation}

Our spacetime invariant interval in this form is thus
%
\begin{equation}\label{eqn:complexPlanarBoost:30}
X^2 \equiv (ct)^2 - \Abs{z}^2.
\end{equation}

Not much different than the usual coordinate representation of the spatial coordinates, except that we have a \(\Abs{z}^2\) replacing the usual \(\Bx^2\).

Taking the spacetime distance between \(X\) and another point, say \(\tilde{X} = ( c \tilde{t}, \tilde{z})\) motivates the inner product between two points in this representation
%
\begin{equation}\label{eqn:complexPlanarBoost:270}
\begin{aligned}
(X - \tilde{X})^2
&= (ct - c \tilde{t})^2 - \Abs{z - \tilde{z}}^2 \\
&= (ct - c \tilde{t})^2 - (z - \tilde{z})(z^\conj - \tilde{z}^\conj) \\
&= (ct)^2 - 2 (ct) (c \tilde{t}) + (c \tilde{t})^2
- \Abs{z}^2 - \Abs{\tilde{z}}^2 + (z \tilde{z}^\conj + z^\conj \tilde{z}) \\
&= X^2 + \tilde{X}^2 - 2 \left( (ct) (c \tilde{t}) - \inv{2}(z \tilde{z}^\conj + z^\conj \tilde{z}) \right) \\
\end{aligned}
\end{equation}

It is clear that it makes sense to define
%
\begin{equation}\label{eqn:complexPlanarBoost:50}
X \cdot \tilde{X} = (ct) (c \tilde{t}) - \Real (z \tilde{z}^\conj),
\end{equation}

consistent with our original starting point
%
\begin{equation}\label{eqn:complexPlanarBoost:70}
X^2 = X \cdot X.
\end{equation}

Let us also introduce a complex inner product
%
\begin{equation}\label{eqn:complexPlanarBoost:90}
\innerprod{z}{\tilde{z}} \equiv \inv{2} \left( z \tilde{z}^\conj + z^\conj \tilde{z}) \right) = \Real (z \tilde{z}^\conj).
\end{equation}

Our dot product can now be written
%
\begin{equation}\label{eqn:complexPlanarBoost:110}
X \cdot \tilde{X} = (ct) (c \tilde{t}) - \innerprod{z}{\tilde{z}}.
\end{equation}
%
\section{Change of basis}
%
Our standard basis for our spatial components is \(\{1, i\}\), but we are free to pick any other basis should we choose.  In particular, if we rotate our basis counterclockwise by \(\phi\), our new basis, still orthonormal, is \(\{ e^{i\phi}, i e^{i\phi} \}\).

In any orthonormal basis the coordinates of a point with respect to that basis are real, so just as we can write
%
\begin{equation}\label{eqn:complexPlanarBoost:130}
z = \innerprod{1}{z} + i \innerprod{i}{z},
\end{equation}

we can extract the coordinates in the rotated frame, also simply by taking inner products
%
\begin{equation}\label{eqn:complexPlanarBoost:150}
z = e^{i \phi} \innerprod{e^{i \phi}}{z} + i e^{i\phi} \innerprod{i e^{i\phi} }{z}.
\end{equation}

The values \(\innerprod{e^{i \phi}}{z}\), and \(\innerprod{i e^{i\phi} }{z}\) are the (real) coordinates of the point \(z\) in this rotated basis.

This is enough that we can write the Lorentz boost immediately for a velocity \(\vec{v} = c \beta e^{i\phi}\) at an arbitrary angle \(\phi\) in the plane
%
\begin{equation}\label{eqn:complexPlanarBoost:170}
\begin{bmatrix}
ct' \\
\innerprod{e^{i\phi}}{z'} \\
\innerprod{i e^{i\phi}}{z'}
\end{bmatrix}
=
\begin{bmatrix}
\gamma & -\gamma \beta & 0 \\
-\gamma \beta & \gamma & 0 \\
0 & 0 & 1
\end{bmatrix}
\begin{bmatrix}
ct \\
\innerprod{e^{i\phi}}{z} \\
\innerprod{i e^{i\phi}}{z}
\end{bmatrix}
\end{equation}

Let us translate this to \(ct, x, y\) coordinates as a check.  For the spatial component parallel to the boost direction we have
%
\begin{equation}\label{eqn:complexPlanarBoost:290}
\begin{aligned}
\innerprod{e^{i\phi}}{x + iy}
&=
\Real ( e^{-i\phi} (x + i y) ) \\
&=
\Real ( (\cos\phi - i \sin\phi)(x + i y) ) \\
&=
x \cos\phi + y \sin\phi,
\end{aligned}
\end{equation}

and the perpendicular components are
%
\begin{equation}\label{eqn:complexPlanarBoost:310}
\begin{aligned}
\innerprod{ i e^{i\phi}}{x + iy}
&=
\Real ( -i e^{-i\phi} (x + i y) ) \\
&=
\Real ( (-i \cos\phi - \sin\phi)(x + i y) ) \\
&=
-x \sin\phi + y \cos\phi.
\end{aligned}
\end{equation}

Grouping the two gives
%
\begin{equation}\label{eqn:complexPlanarBoost:200}
\begin{bmatrix}
\innerprod{e^{i\phi}}{x + iy}  \\
\innerprod{i e^{i\phi}}{x + iy}
\end{bmatrix}
=
\begin{bmatrix}
\cos\phi & \sin\phi \\
-\sin\phi & \cos\phi
\end{bmatrix}
\begin{bmatrix}
x \\
y
\end{bmatrix}
= R_{-\phi}
\begin{bmatrix}
x \\
y
\end{bmatrix}
\end{equation}

The boost equation in terms of the cartesian coordinates is thus
%
\begin{equation}\label{eqn:complexPlanarBoost:220}
\begin{bmatrix}
1 & 0 \\
0 & R_{-\phi}
\end{bmatrix}
\begin{bmatrix}
c t' \\
x' \\
y'
\end{bmatrix}
=
\begin{bmatrix}
\gamma & -\gamma \beta & 0 \\
-\gamma \beta & \gamma & 0 \\
0 & 0 & 1
\end{bmatrix}
\begin{bmatrix}
1 & 0 \\
0 & R_{-\phi}
\end{bmatrix}
\begin{bmatrix}
c t \\
x \\
y
\end{bmatrix}.
\end{equation}

Writing
%
\begin{equation}\label{eqn:complexPlanarBoost:240}
\begin{bmatrix}
c t' \\
x' \\
y'
\end{bmatrix}
=
\Norm{{\wedge^\mu}_\nu}
\begin{bmatrix}
c t \\
x \\
y
\end{bmatrix},
\end{equation}

the boost matrix \(\Norm{{\wedge^\mu}_\nu}\) is found to be (after a bit of work)
%
\begin{equation}\label{eqn:complexPlanarBoost:330}
\begin{aligned}
\Norm{{\wedge^\mu}_\nu} &=
\begin{bmatrix}
1 & 0 \\
0 & R_{\phi}
\end{bmatrix}
\begin{bmatrix}
\gamma & -\gamma \beta & 0 \\
-\gamma \beta & \gamma & 0 \\
0 & 0 & 1
\end{bmatrix}
\begin{bmatrix}
1 & 0 \\
0 & R_{-\phi}
\end{bmatrix} \\
&=
\begin{bmatrix}
\gamma & - \gamma \beta \cos\phi & -\gamma \beta \sin\phi \\
-\gamma \beta \cos\phi & \gamma \cos^2\phi + \sin^2 \phi & (\gamma -1) \sin\phi \cos\phi \\
-\gamma \beta \sin\phi & (\gamma -1) \sin\phi \cos\phi & \gamma \sin^2\phi + \cos^2\phi \\
\end{bmatrix} \\
\end{aligned}
\end{equation}

A final bit of regrouping gives
%
\begin{equation}\label{eqn:complexPlanarBoost:250}
\Norm{{\wedge^\mu}_\nu}
=
\begin{bmatrix}
\gamma & - \gamma \beta \cos\phi & -\gamma \beta \sin\phi \\
-\gamma \beta \cos\phi & 1 + ( \gamma -1) \cos^2\phi & (\gamma -1) \sin\phi \cos\phi \\
-\gamma \beta \sin\phi & (\gamma -1) \sin\phi \cos\phi & 1 + (\gamma -1) \sin^2\phi \\
\end{bmatrix}.
\end{equation}

This is consistent with the result stated in \citep{wiki:LorentzBoost}, finishing the game for the day.
