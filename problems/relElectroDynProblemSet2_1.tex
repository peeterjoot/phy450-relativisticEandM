%
% Copyright � 2012 Peeter Joot.  All Rights Reserved.
% Licenced as described in the file LICENSE under the root directory of this GIT repository.
%

%
%
%\chapter{Problem Set 2}
\label{chap:relElectroDynProblemSet2}
%\blogpage{http://sites.google.com/site/peeterjoot/math2011/relElectroDynProblemSet2.pdf}
%\date{Feb 1, 2011}
%
\makeproblem{Particle collision}{pr:relElectroDynProblemSet2:1}{
%
A particle of rest mass \(m\) whose energy is three times its rest energy collides with an identical particle at rest.  Suppose they stick together.  Use conservation laws to find the mass of the resulting particle and its velocity.  Is its mass greater or smaller than \(2m\)?  Comment.

} % makeproblem
%
\makeanswer{pr:relElectroDynProblemSet2:1}{
%

The energy of the initially moving particle before collision is
%
\begin{equation}\label{eqn:relElectroDynProblemSet2:10}
\calE = \frac{m c^2 }{\InvGamma} = 3 m c^2.
\end{equation}
%
Solving for the velocity we have
%
\begin{equation}\label{eqn:relElectroDynProblemSet2:30}
\Abs{\frac{\Bv}{c}} = \frac{2 \sqrt{2}}{3}.
\end{equation}
%
Our four velocity is
%
\begin{equation}\label{eqn:relElectroDynProblemSet2:50}
u^i
= \gamma \left( 1, \frac{\Bv}{c} \right) = ( 3, 2 \sqrt{2} ).
\end{equation}
%
Designate the four momentum for this particle as
%
\begin{equation}\label{eqn:relElectroDynProblemSet2:70}
p_{(1)}^i = m c ( 3, 2 \sqrt{2} ).
\end{equation}
%
For the second particle we have
%
\begin{equation}\label{eqn:relElectroDynProblemSet2:90}
p_{(2)}^i = m c ( 1, 0 ).
\end{equation}
%
Our initial and final four momentum will be equal, and our resulting velocity can only be in the direction of the initial particle.  This leaves us with
%
\begin{equation}\label{eqn:relElectroDynProblemSet2:1110}
\begin{aligned}
p_{(f)}^i
&= M c \inv{\sqrt{1 - \frac{\Bv_f^2}{c^2}}} \left( 1, \frac{\Bv_f}{c} \right) \\
&= m c ( 1, 0 ) + m c ( 3, 2 \sqrt{2} )  \\
&= m c ( 4, 2 \sqrt{2} ) \\
&= 4 m c \left( 1, \inv{\sqrt{2}} \right)
\end{aligned}
\end{equation}
%
Our final velocity is \(v_f = c/\sqrt{2}\).

We have \(M \gamma = 4\) for the final particle, but we have
%
\begin{equation}\label{eqn:relElectroDynProblemSet2:110}
\gamma = \frac{1}{\sqrt{1 - 1/2}} = \sqrt{2},
\end{equation}
%
so our final mass is
%
\begin{equation}\label{eqn:relElectroDynProblemSet2:130}
M = \frac{4}{\sqrt{2}} = 2 \sqrt{2} > 2.
\end{equation}
%
Relativistically, we have conservation of four-momentum, not conservation of mass, so a composite body will not necessarily have a mass measurement that is the sum of the parts.  One possible way to reconcile this statement with intuition is to define mass in terms of the four momentum
%
\begin{equation}\label{eqn:relElectroDynProblemSet2:130b}
m^2 = \frac{p^i p_i}{c^2},
\end{equation}
%
and think of it as a derived quantity, not fundamental.

}

