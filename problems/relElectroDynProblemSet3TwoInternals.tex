%
% Copyright � 2012 Peeter Joot.  All Rights Reserved.
% Licenced as described in the file LICENSE under the root directory of this GIT repository.
%

%
\makeproblem{Transformation properties of \( \BE, \BB \) again.}{pr:relElectroDynProblemSet3:2}{
%
\makesubproblem{}{pr:relElectroDynProblemSet3:2a}
%
Use the form of \(F^{i j}\) from page 82 in the class notes, the transformation law for \(\Norm{ F^{i j} }\) given further down that same page, and the explicit form of the \(SO(1,3)\) matrix \(\hat{O}\) (say, corresponding to motion in the positive \(x_1\) direction with speed \(v\)) to derive the transformation law of the fields \(\BE\) and \(\BB\).  Use the transformation law to find the electromagnetic field of a charged particle moving with constant speed \(v\) in the positive \(x_1\) direction and check that the result agrees with the one that you obtained in Homework 2.
%
\makesubproblem{}{pr:relElectroDynProblemSet3:2b}
%
A particle is moving with velocity \(\Bv\) in perpendicular \(\BE\) and \(\BB\) fields, all given in some particular ``stationary'' frame of reference.

\begin{enumerate}
\item Show that there exists a frame where the problem of finding the particle trajectory can be reduced to having either only an electric or only a magnetic field.
\item Explain what determines which case takes place.
\item Find the velocity \(\Bv_0\) of that frame relative to the ``stationary'' frame.
\end{enumerate}

} % makeproblem
%
\makeanswer{pr:relElectroDynProblemSet3:2}{
\makeSubAnswer{}{pr:relElectroDynProblemSet3:2a}
%
Given a transformation of coordinates
%
\begin{equation}\label{eqn:relElectroDynProblemSet3:1220}
{x'}^i \rightarrow {O^i}_j x^j.
\end{equation}
%
our rank 2 tensor \(F^{i j}\) transforms as
%
\begin{equation}\label{eqn:relElectroDynProblemSet3:1240}
F^{i j} \rightarrow
{O^i}_a
F^{a b}
{O^j}_b.
\end{equation}
%
Introducing matrices
%
\begin{equation}\label{eqn:relElectroDynProblemSet3:1260}
\begin{aligned}
\hat{O} &= \Norm{{O^i}_j} \\
\hat{F} &=
\Norm{F^{ij}} =
\begin{bmatrix}
0 & -E_x & -E_y & -E_z \\
E_x & 0 & -B_z & B_y \\
E_y & B_z & 0 & -B_x \\
E_z & -B_y & B_x & 0
\end{bmatrix}.
\end{aligned}
\end{equation}
%
and noting that \(\hat{O}^\T = \Norm{{O^j}_i}\), we can express the electromagnetic strength tensor transformation as
%
\begin{equation}\label{eqn:relElectroDynProblemSet3:1280}
\hat{F} \rightarrow \hat{O} \hat{F} \hat{O}^\T.
\end{equation}
%
The class notes use \({x'}^i \rightarrow O^{ij} x^j\), which violates our conventions on mixed upper and lower indices, but the end result \eqnref{eqn:relElectroDynProblemSet3:1280} is the same.
%
\begin{equation}\label{eqn:relElectroDynProblemSet3:1300}
\Norm{{O^i}_j} =
\begin{bmatrix}
\cosh\alpha & -\sinh\alpha & 0 & 0 \\
-\sinh\alpha & \cosh\alpha & 0 & 0 \\
0 & 0 & 1 & 0 \\
0 & 0 & 0 & 1
\end{bmatrix}.
\end{equation}
%
Writing
%
\begin{equation}\label{eqn:relElectroDynProblemSet3:1840}
\begin{aligned}
C &= \cosh\alpha = \gamma \\
S &= -\sinh\alpha = -\gamma \beta,
\end{aligned}
\end{equation}
%
we can compute the transformed field strength tensor
%
\begin{equation}\label{eqn:relElectroDynProblemSet3TwoInternals:1900}
\begin{aligned}
\hat{F}' &=
\begin{bmatrix}
C & S & 0 & 0 \\
S & C & 0 & 0 \\
0 & 0 & 1 & 0 \\
0 & 0 & 0 & 1
\end{bmatrix}
\begin{bmatrix}
0 & -E_x & -E_y & -E_z \\
E_x & 0 & -B_z & B_y \\
E_y & B_z & 0 & -B_x \\
E_z & -B_y & B_x & 0
\end{bmatrix}
\begin{bmatrix}
C & S & 0 & 0 \\
S & C & 0 & 0 \\
0 & 0 & 1 & 0 \\
0 & 0 & 0 & 1
\end{bmatrix} \\
&=
\begin{bmatrix}
C & S & 0 & 0 \\
S & C & 0 & 0 \\
0 & 0 & 1 & 0 \\
0 & 0 & 0 & 1
\end{bmatrix}
\begin{bmatrix}
- S E_x        & -C E_x        & -E_y  & -E_z \\
C E_x          & S E_x         & -B_z  & B_y \\
C E_y + S B_z  & S E_y + C B_z & 0     & -B_x \\
C E_z - S B_y  & S E_z - C B_y & B_x   & 0
\end{bmatrix} \\
&=
\begin{bmatrix}
0 & -E_x & -C E_y - S B_z & - C E_z + S B_y \\
E_x & 0 & -S E_y - C B_z & - S E_z + C B_y \\
C E_y + S B_z & S E_y + C B_z & 0 & -B_x \\
C E_z - S B_y & S E_z - C B_y & B_x & 0
\end{bmatrix} \\
&=
\begin{bmatrix}
0 & -E_x & -\gamma(E_y - \beta B_z) & - \gamma(E_z + \beta B_y) \\
E_x & 0 & - \gamma (-\beta E_y + B_z) & \gamma( \beta E_z + B_y) \\
\gamma (E_y - \beta B_z) & \gamma(-\beta E_y + B_z) & 0 & -B_x \\
\gamma (E_z + \beta B_y) & -\gamma(\beta E_z + B_y) & B_x & 0
\end{bmatrix}.
\end{aligned}
\end{equation}
%
As a check we have the antisymmetry that is expected.  There is also a regularity to the end result that is aesthetically pleasing, hinting that things are hopefully error free.  In coordinates for \(\BE\) and \(\BB\) this is
%
\begin{equation}\label{eqn:relElectroDynProblemSet3:1320}
\begin{aligned}
E_x &\rightarrow E_x \\
E_y &\rightarrow \gamma ( E_y - \beta B_z ) \\
E_z &\rightarrow \gamma ( E_z + \beta B_y ) \\
B_z &\rightarrow B_x \\
B_y &\rightarrow \gamma ( B_y + \beta E_z ) \\
B_z &\rightarrow \gamma ( B_z - \beta E_y ).
\end{aligned}
\end{equation}
%
Writing \(\Bbeta = \Be_1 \beta\), we have
%
\begin{equation}\label{eqn:relElectroDynProblemSet3:1340}
\Bbeta \cross \BB =
\begin{vmatrix}
\Be_1 & \Be_2 & \Be_3 \\
\beta & 0 & 0 \\
B_x & B_y & B_z
\end{vmatrix}
= \Be_2 (-\beta B_z) + \Be_3( \beta B_y ),
\end{equation}
%
which puts us enroute to a tidier vector form
%
\begin{equation}\label{eqn:relElectroDynProblemSet3:1360}
\begin{aligned}
E_x &\rightarrow E_x \\
E_y &\rightarrow \gamma ( E_y + (\Bbeta \cross \BB)_y ) \\
E_z &\rightarrow \gamma ( E_z + (\Bbeta \cross \BB)_z ) \\
B_z &\rightarrow B_x \\
B_y &\rightarrow \gamma ( B_y - (\Bbeta \cross \BE)_y ) \\
B_z &\rightarrow \gamma ( B_z - (\Bbeta \cross \BE)_z ).
\end{aligned}
\end{equation}
%
For a vector \(\BA\), write \(\BA_\parallel = (\BA \cdot \vcap)\vcap\), \(\BA_\perp = \BA - \BA_\parallel\), allowing a compact description of the field transformation
%
\begin{equation}\label{eqn:relElectroDynProblemSet3:1380}
\begin{aligned}
\BE &\rightarrow \BE_\parallel + \gamma \BE_\perp + \gamma (\Bbeta \cross \BB)_\perp \\
\BB &\rightarrow \BB_\parallel + \gamma \BB_\perp - \gamma (\Bbeta \cross \BE)_\perp.
\end{aligned}
\end{equation}
%
Now, we want to consider the field of a moving particle.  In the particle's (unprimed) rest frame the field due to its potential \(\phi = q/r\) is
%
\begin{equation}\label{eqn:relElectroDynProblemSet3:1400}
\begin{aligned}
\BE &= \frac{q}{r^2} \rcap \\
\BB &= 0.
\end{aligned}
\end{equation}
%
Coordinates for a ``stationary'' observer, who sees this particle moving along the x-axis at speed \(v\) are related by a boost in the \(-v\) direction
%
\begin{equation}\label{eqn:relElectroDynProblemSet3:1420}
\begin{bmatrix}
ct' \\
x' \\
y' \\
z'
\end{bmatrix}
\begin{bmatrix}
\gamma & \gamma (v/c) & 0 & 0 \\
\gamma (v/c) & \gamma & 0 & 0 \\
0 & 0 & 1 & 0 \\
0 & 0 & 0 & 1
\end{bmatrix}
\begin{bmatrix}
ct \\
x \\
y \\
z
\end{bmatrix}.
\end{equation}
Therefore the fields in the observer frame will be
\begin{equation}\label{eqn:relElectroDynProblemSet3:1440}
\begin{aligned}
\BE' &= \BE_\parallel + \gamma \BE_\perp - \gamma \frac{v}{c}(\Be_1 \cross \BB)_\perp = \BE_\parallel + \gamma \BE_\perp \\
\BB' &= \BB_\parallel + \gamma \BB_\perp + \gamma \frac{v}{c}(\Be_1 \cross \BE)_\perp = \gamma \frac{v}{c}(\Be_1 \cross \BE)_\perp.
\end{aligned}
\end{equation}
%
More explicitly with \(\BE = \frac{q}{r^3}(x, y, z)\) this is
%
\begin{equation}\label{eqn:relElectroDynProblemSet3:1460}
\begin{aligned}
\BE' &= \frac{q}{r^3}(x, \gamma y, \gamma z) \\
\BB' &= \gamma \frac{q v}{c r^3} ( 0, -z, y ).
\end{aligned}
\end{equation}
%
Comparing to Problem 3 in Problem set 2, I see that this matches the result obtained by separately transforming the gradient, the time partial, and the scalar potential.  Actually, if I am being honest, I see that I made a sign error in all the coordinates of \(\BE'\) when I initially did (this ungraded problem) in problem set 2.  That sign error should have been obvious by considering the \(v=0\) case which would have mysteriously resulted in inversion of all the coordinates of the observed electric field.
%
\makeSubAnswer{}{pr:relElectroDynProblemSet3:2b}
%
\paragraph{Part 1 and 2:} Existence of the transformation.
%
In the single particle Lorentz trajectory problem we wish to solve
%
\begin{equation}\label{eqn:relElectroDynProblemSet3:1500}
m c \frac{du^i}{ds} = \frac{e}{c} F^{i j} u_j,
\end{equation}
%
which in matrix form we can write as
%
\begin{equation}\label{eqn:relElectroDynProblemSet3:1520}
\frac{d U}{ds} = \frac{e}{m c^2} \hat{F} \hat{G} U.
\end{equation}
%
where we write our column vector proper velocity as \(U = \Norm{u^i}\).  Under transformation of coordinates \({u'}^i = {O^i}_j x^j\), with \(\hat{O} = \Norm{{O^i}_j}\), this becomes
%
\begin{equation}\label{eqn:relElectroDynProblemSet3:1540}
\hat{O} \frac{d U}{ds} = \frac{e}{m c^2} \hat{O} \hat{F} \hat{O}^\T \hat{G} \hat{O} U.
\end{equation}
%
Suppose we can find eigenvectors for the matrix \(\hat{O} \hat{F} \hat{O}^\T \hat{G}\).  That is for some eigenvalue \(\lambda\), we can find an eigenvector \(\Sigma\)
%
\begin{equation}\label{eqn:relElectroDynProblemSet3:1560}
\hat{O} \hat{F} \hat{O}^\T \hat{G} \Sigma = \lambda \Sigma.
\end{equation}
Rearranging we have
\begin{equation}\label{eqn:relElectroDynProblemSet3:1580}
(\hat{O} \hat{F} \hat{O}^\T \hat{G} - \lambda I) \Sigma = 0,
\end{equation}
and conclude that \(\Sigma\) lies in the null space of the matrix \(\hat{O} \hat{F} \hat{O}^\T \hat{G} - \lambda I\) and that this difference of matrices must have a zero determinant
%
\begin{equation}\label{eqn:relElectroDynProblemSet3:1600}
\Det (\hat{O} \hat{F} \hat{O}^\T \hat{G} - \lambda I) = -\Det (\hat{O} \hat{F} \hat{O}^\T - \lambda \hat{G}) = 0.
\end{equation}
%
Since \(\hat{G} = \hat{O} \hat{G} \hat{O}^\T\) for any Lorentz transformation \(\hat{O}\) in \(SO(1,3)\), and \(\Det ABC = \Det A \Det B \Det C\) we have
%
\begin{equation}\label{eqn:relElectroDynProblemSet3:1820}
\Det (\hat{O} \hat{F} \hat{O}^\T - \lambda G)
=
\Det (\hat{F} - \lambda \hat{G}).
\end{equation}
%
In problem 1.6, we called this our characteristic equation \(P(\lambda) = \Det (\hat{F} - \lambda \hat{G})\).  Observe that the characteristic equation is Lorentz invariant for any \(\lambda\), which requires that the eigenvalues \(\lambda\) are also Lorentz invariants.

In problem 1.6 of this problem set we computed that this characteristic equation expands to
%
\begin{equation}\label{eqn:relElectroDynProblemSet3:1620}
P(\lambda) = \Det (\hat{F} - \lambda \hat{G}) = (\BE \cdot \BB)^2 + \lambda^2 (\BB^2 - \BE^2) + \lambda^4.
\end{equation}
%
The eigenvalues for the system, also each necessarily Lorentz invariants, are
%
\begin{equation}\label{eqn:relElectroDynProblemSet3:1080b}
\lambda = \pm \inv{\sqrt{2}} \sqrt{ \BE^2 - \BB^2 \pm \sqrt{ (\BE^2 - \BB^2)^2 - 4 (\BE \cdot \BB)^2 }}.
\end{equation}
%
Observe that in the specific case where \(\BE \cdot \BB = 0\), as in this problem, we must have \(\BE' \cdot \BB'\) in all frames, and the two non-zero eigenvalues of our characteristic polynomial are simply
%
\begin{equation}\label{eqn:relElectroDynProblemSet3:1640}
\lambda = \pm \sqrt{\BE^2 - \BB^2}.
\end{equation}
%
These and \(\BE \cdot \BB = 0\) are the invariants for this system.   If we have \(\BE^2 > \BB^2\) in one frame, we must also have \({\BE'}^2 > {\BB'}^2\) in another frame, still maintaining perpendicular fields.  In particular if \(\BB' = 0\) we maintain real eigenvalues.  Similarly if \(\BB^2 > \BE^2\) in some frame, we must always have imaginary eigenvalues, and this is also true in the \(\BE' = 0\) case.

While the problem can be posed as a pure diagonalization problem (and even solved numerically this way for the general constant fields case), we can also work symbolically, thinking of the trajectories problem as simply seeking a transformation of frames that reduce the scope of the problem to one that is more tractable.  That does not have to be the linear transformation that diagonalizes the system.  Instead we are free to transform to a frame where one of the two fields \(\BE'\) or \(\BB'\) is zero, provided the invariants discussed are maintained.
%
\paragraph{Part 3:} Finding the boost velocity that wipes out one of the fields.
%
Let us now consider a Lorentz boost \(\hat{O}\), and seek to solve for the boost velocity that wipes out one of the fields, given the invariants that must be maintained for the system

To make things concrete, suppose that our perpendicular fields are given by \(\BE = E \Be_2\) and \(\BB = B \Be_3\).

Let also assume that we can find the velocity \(\Bv_0\) for which one or more of the transformed fields is zero.  Suppose that velocity is
%
\begin{equation}\label{eqn:relElectroDynProblemSet3:1480}
\Bv_0 = v_0 (\alpha_1, \alpha_2, \alpha_3) = v_0 \vcap_0,
\end{equation}
%
where \(\alpha_i\) are the direction cosines of \(\Bv_0\) so that \(\sum_i \alpha_i^2 = 1\).  We will want to compute the components of \(\BE\) and \(\BB\) parallel and perpendicular to this velocity.

Those are
%
\begin{equation}\label{eqn:relElectroDynProblemSet3TwoInternals:1920}
\begin{aligned}
\BE_\parallel
&= E \Be_2 \cdot (\alpha_1, \alpha_2, \alpha_3) (\alpha_1, \alpha_2, \alpha_3) \\
&= E \alpha_2 (\alpha_1, \alpha_2, \alpha_3).
\end{aligned}
\end{equation}
%
\begin{equation}\label{eqn:relElectroDynProblemSet3TwoInternals:1940}
\begin{aligned}
\BE_\perp
&= E \Be_2 - \BE_\parallel \\
&= E (-\alpha_1 \alpha_2, 1 - \alpha_2^2, -\alpha_2 \alpha_3) \\
&= E (-\alpha_1 \alpha_2, \alpha_1^2 + \alpha_3^2, -\alpha_2 \alpha_3).
\end{aligned}
\end{equation}
%
For the magnetic field we have
%
\begin{equation}\label{eqn:relElectroDynProblemSet3TwoInternals:1960}
\begin{aligned}
\BB_\parallel
&= B \alpha_3 (\alpha_1, \alpha_2, \alpha_3),
\end{aligned}
\end{equation}
%
and
%
\begin{equation}\label{eqn:relElectroDynProblemSet3TwoInternals:1980}
\begin{aligned}
\BB_\perp
&= B \Be_3 - \BB_\parallel \\
&= B (-\alpha_1 \alpha_3, -\alpha_2 \alpha_3, \alpha_1^2 + \alpha_2^2).
\end{aligned}
\end{equation}
%
Now, observe that \((\Bbeta \cross \BB)_\parallel \sim ((\Bv_0 \cross \BB) \cdot \Bv_0) \Bv_0\), but this is just zero.  So we have \((\Bbeta \cross \BB)_\parallel = \Bbeta \cross \BB\).  So our cross products terms are just
%
\begin{equation}\label{eqn:relElectroDynProblemSet3TwoInternals:2000}
\begin{aligned}
\vcap_0 \cross \BB &=
        \begin{vmatrix}
        \Be_1 & \Be_2 & \Be_3 \\
        \alpha_1 & \alpha_2 & \alpha_3 \\
        0 & 0 & B
        \end{vmatrix}
= B (\alpha_2, -\alpha_1, 0) \\
\vcap_0 \cross \BE &=
        \begin{vmatrix}
        \Be_1 & \Be_2 & \Be_3 \\
        \alpha_1 & \alpha_2 & \alpha_3 \\
        0 & E & 0
        \end{vmatrix}
= E (-\alpha_3, 0, \alpha_1).
\end{aligned}
\end{equation}
%
We can now express how the fields transform, given this arbitrary boost velocity.  From \eqnref{eqn:relElectroDynProblemSet3:1380}, this is
%
\begin{equation}\label{eqn:relElectroDynProblemSet3:1380b}
\begin{aligned}
\BE &\rightarrow
E \alpha_2 (\alpha_1, \alpha_2, \alpha_3)
+ \gamma E (-\alpha_1 \alpha_2, \alpha_1^2 + \alpha_3^2, -\alpha_2 \alpha_3)
+ \gamma \frac{v_0^2}{c^2}
B (\alpha_2, -\alpha_1, 0) \\
\BB &\rightarrow
B \alpha_3 (\alpha_1, \alpha_2, \alpha_3)
+ \gamma B (-\alpha_1 \alpha_3, -\alpha_2 \alpha_3, \alpha_1^2 + \alpha_2^2)
- \gamma \frac{v_0^2}{c^2} E (-\alpha_3, 0, \alpha_1).
\end{aligned}
\end{equation}
%
\paragraph{Zero Electric field case}
%
Let us tackle the two cases separately.  First when \(\Abs{\BB} > \Abs{\BE}\), we can transform to a frame where \(\BE'=0\).  In coordinates from \eqnref{eqn:relElectroDynProblemSet3:1380b} this supplies us three sets of equations.  These are
%
\begin{equation}\label{eqn:relElectroDynProblemSet3:1660}
\begin{aligned}
0 &= E \alpha_2 \alpha_1 (1 - \gamma) + \gamma \frac{v_0^2}{c^2} B \alpha_2  \\
0 &= E \alpha_2^2 + \gamma E (\alpha_1^2 + \alpha_3^2) - \gamma \frac{v_0^2}{c^2} B \alpha_1  \\
0 &= E \alpha_2 \alpha_3 (1 - \gamma).
\end{aligned}
\end{equation}
%
With an assumed solution the \(\Be_3\) coordinate equation implies that one of \(\alpha_2\) or \(\alpha_3\) is zero.  Perhaps there are solutions with \(\alpha_3 = 0\) too, but inspection shows that \(\alpha_2 = 0\) nicely kills off the first equation.  Since \(\alpha_1^2 + \alpha_2^2 + \alpha_3^2 = 1\), that also implies that we are left with
%
\begin{equation}\label{eqn:relElectroDynProblemSet3:1680}
0 = E - \frac{v_0^2}{c^2} B \alpha_1.
\end{equation}
%
Or
%
\begin{equation}\label{eqn:relElectroDynProblemSet3:1700}
\begin{aligned}
\alpha_1 &= \frac{E}{B} \frac{c^2}{v_0^2} \\
\alpha_2 &= 0 \\
\alpha_3 &= \sqrt{
1 - \frac{E^2}{B^2} \frac{c^4}{v_0^4}
}.
\end{aligned}
\end{equation}
%
Our velocity was \(\Bv_0 = v_0 (\alpha_1, \alpha_2, \alpha_3)\) solving the problem for the \(\Abs{\BB}^2 > \Abs{\BE}^2\) case up to an adjustable constant \(v_0\).  That constant comes with constraints however, since we must also have our cosine \(\alpha_1 \le 1\).  Expressed another way, the magnitude of the boost velocity is constrained by the relation
%
\begin{equation}\label{eqn:relElectroDynProblemSet3:1720}
\frac{\Bv_0^2}{c^2} \ge \Abs{\frac{E}{B}}.
\end{equation}
%
It appears we may also pick the equality case, so one velocity (not unique) that should transform away the electric field is
\boxedEquation{eqn:relElectroDynProblemSet3:1860}{
\Bv_0 = c \sqrt{\Abs{\frac{E}{B}}} \Be_1 = \pm c \sqrt{\Abs{\frac{E}{B}}} \frac{\BE \cross \BB}{\Abs{\BE} \Abs{\BB}}.
}
This particular boost direction is perpendicular to both fields.  Observe that this highlights the invariance condition \(\Abs{\frac{E}{B}} < 1\) since we see this is required for a physically realizable velocity.  Boosting in this direction will reduce our problem to one that has only the magnetic field component.
%
\paragraph{Zero Magnetic field case}
%
Now, let us consider the case where we transform the magnetic field away, the case when our characteristic polynomial has strictly real eigenvalues \(\lambda = \pm \sqrt{\BE^2 - \BB^2}\).  In this case, if we write out our equations for the transformed magnetic field and require these to separately equal zero, we have
%
\begin{equation}\label{eqn:relElectroDynProblemSet3:1740}
\begin{aligned}
0 &= B \alpha_3 \alpha_1 ( 1 - \gamma ) + \gamma \frac{v_0^2}{c^2} E \alpha_3 \\
0 &= B \alpha_2 \alpha_3 ( 1 - \gamma ) \\
0 &= B (\alpha_3^2 + \gamma (\alpha_1^2 + \alpha_2^2)) - \gamma \frac{v_0^2}{c^2} E \alpha_1.
\end{aligned}
\end{equation}
%
Similar to before we see that \(\alpha_3 = 0\) kills off the first and second equations, leaving just
%
\begin{equation}\label{eqn:relElectroDynProblemSet3:1760}
0 = B - \frac{v_0^2}{c^2} E \alpha_1.
\end{equation}
%
We now have a solution for the family of direction vectors that kill the magnetic field off
%
\begin{equation}\label{eqn:relElectroDynProblemSet3:1780}
\begin{aligned}
\alpha_1 &= \frac{B}{E} \frac{c^2}{v_0^2} \\
\alpha_2 &= \sqrt{ 1 - \frac{B^2}{E^2} \frac{c^4}{v_0^4} } \\
\alpha_3 &= 0.
\end{aligned}
\end{equation}
%
In addition to the initial constraint that \(\Abs{\frac{B}{E}} < 1\), we have as before, constraints on the allowable values of \(v_0\)
%
\begin{equation}\label{eqn:relElectroDynProblemSet3:1800}
\frac{\Bv_0^2}{c^2} \ge \Abs{\frac{B}{E}}.
\end{equation}
%
Like before we can pick the equality \(\alpha_1^2 = 1\), yielding a boost direction of
\boxedEquation{eqn:relElectroDynProblemSet3:1880}{
\Bv_0 = c \sqrt{\Abs{\frac{B}{E}}} \Be_1 = \pm c \sqrt{\Abs{\frac{B}{E}}} \frac{\BE \cross \BB}{\Abs{\BE} \Abs{\BB}}.
}

Again, we see that the invariance condition \(\Abs{\BB} < \Abs{\BE}\) is required for a physically realizable velocity if that velocity is entirely perpendicular to the fields.
%
\paragraph{Notes on grading of my solution}
%
I lost two marks on this problem.  One for \eqnref{eqn:relElectroDynProblemSet3:1460} where he wanted primes on the variables
%
\begin{equation}\label{eqn:relElectroDynProblemSet3:1460b}
\begin{aligned}
\BE' &= \frac{q}{{r'}^3}(x', \gamma y', \gamma z') \\
\BB' &= \gamma \frac{q v}{c {r'}^3} ( 0, -z', y' ),
\end{aligned}
\end{equation}
%
however, I do not think that is correct.  Compare to problem set 2, problem 3, where this exactly matches the expected result, yet is only correct when the variables are the unprimed ones?

FIXME: Talk to Simon to see what he means.

Also, immediately before \eqnref{eqn:relElectroDynProblemSet3:1860} he underlined ``one velocity (not unique)'', and put an X beside it.

FIXME: is all that logic before \eqnref{eqn:relElectroDynProblemSet3:1860} wrong? (because that shows the boost velocity is not unique).  If I try the very simplest boost applied to the \(\BE = E \Be_2\) and \(\BB = B \Be_3\) I find a very different result (with no square root).  I think I am guilty of trying to be too general and not going back and checking for the simplest case.  Even so, where are my errors?
}
