%
% Copyright © 2012 Peeter Joot.  All Rights Reserved.
% Licenced as described in the file LICENSE under the root directory of this GIT repository.
%
\makeproblem{Transformation of velocities.}{pr:relElectroDynProblemSet1:2}{
%
From the Lorentz transformations of space and time coordinates.

%
\makesubproblem{Derive the transformation of velocities.}{pr:relElectroDynProblemSet1:2:a}
%
With a particle moving with \(\Bv\) in the unprimed (stationary) frame, find its velocity \(\Bv'\) in the primed frame.  The primed frame is moving with some \(\BV\) with respect to the unprimed one.  Make sure to finally derive the general ``addition of velocities'' equation in terms of vectors and dot products, as given in \citep{landau1971classical}.
%
\makesubproblem{Velocities relative to \( c \).}{pr:relElectroDynProblemSet1:2:b}
%
Then, use the addition of velocities rule to show that:

\begin{enumerate}
\item
%a)
if \(v < c\) in one frame, then \(v' < c\) in any other frame.
\item
%b.)
If \(v = c\) in one frame, then \(v' = c\) in any other frame, and
\item
%c.)
if \(v> c\) in one frame, than \(v' > c\) in any other frame.
\end{enumerate}
} % makeproblem
%
\makeanswer{pr:relElectroDynProblemSet1:2}{
\makeSubAnswer{}{pr:relElectroDynProblemSet1:2:a}
We need a vector form of the Lorentz transform to start with.  Let us write \(\Bsigma\) for a unit vector colinear with the primed frame velocity \(\BV\), so that \(\BV = (\BV \cdot \Bsigma) \Bsigma\).  When our boost was in the \(x\) direction, our Lorentz transformation was in terms of \(x = \Bx \cdot \xcap\).  The component in the direction of the boost is now \(\Bx \cdot \Bsigma\), and we have

\begin{subequations}
\begin{equation}\label{eqn:relativisticElectrodynamicsA1:200}
\begin{aligned}
c t' &= \gamma \left( ct - (\Bx \cdot \Bsigma) \frac{\BV \cdot \Bsigma}{c} \right) \\
\Bx' \cdot \Bsigma &= \gamma \left( \Bx \cdot \Bsigma - \frac{\BV \cdot \Bsigma}{c} c t \right) \\
\Bx' \wedge \Bsigma &= \Bx \wedge \Bsigma .
\end{aligned}
\end{equation}
\end{subequations}

We can add the vector components using \(\Bx = (\Bx \cdot \Bsigma) \Bsigma + (\Bx \wedge \Bsigma) \Bsigma\), leaving
\begin{subequations}
\begin{equation}\label{eqn:relativisticElectrodynamicsA1:210}
\begin{aligned}
c t' &= \gamma \left( ct - (\Bx \cdot \Bsigma) \frac{\BV \cdot \Bsigma}{c} \right) \\
\Bx' &= (\Bx \wedge \Bsigma) \Bsigma + \gamma \left( (\Bx \cdot \Bsigma) \Bsigma - \frac{\BV}{c} c t \right).
\end{aligned}
\end{equation}
\end{subequations}

Writing \((\Bx \wedge \Bsigma) \Bsigma = \Bx - (\Bx \cdot \Bsigma)\Bsigma\) we have for the spatial component transformation
%
\begin{equation}\label{eqn:relativisticElectrodynamicsA1:220}
\Bx' = \Bx + (\Bx \cdot \Bsigma) \Bsigma (\gamma - 1) - \gamma \frac{\BV}{c} c t.
\end{equation}
%
Now we are set to take derivatives to calculate the velocities.  This gives us
\begin{subequations}
\begin{equation}\label{eqn:relativisticElectrodynamicsA1:230}
\begin{aligned}
\frac{dt'}{dt} &= \gamma \left( 1 - \left( \frac{d\Bx}{dt} \cdot \Bsigma \right) \frac{\BV \cdot \Bsigma}{c^2} \right) \\
\frac{d\Bx'}{dt'} \frac{d t'}{dt} &= \frac{d\Bx}{dt} + \left(\frac{d\Bx}{dt} \cdot \Bsigma\right) \Bsigma (\gamma - 1) - \gamma \frac{\BV}{c} c .
\end{aligned}
\end{equation}
\end{subequations}

Dividing this pair of equations, and using \(\Bv = d\Bx/dt\), and \(\Bv' = d\Bx'/dt'\), this is
%
\begin{equation}\label{eqn:relativisticElectrodynamicsA1:240}
\Bv' = \frac{\gamma^{-1} \Bv + (\Bv \cdot \Bsigma) \Bsigma (1 - \gamma^{-1}) - \BV}{ 1 - \left( \Bv \cdot \Bsigma \right) (\BV \cdot \Bsigma)/c^2 }.
\end{equation}
%
Since \(\BV\) and our direction vector \(\Bsigma\) are colinear, we have \((\Bv \cdot \Bsigma) (\BV \cdot \Bsigma) = \Bv \cdot \Bsigma\), and can simplify this last expression slightly
\boxedEquation{eqn:relativisticElectrodynamicsA1:250}{
\Bv' = \frac{\gamma^{-1} \Bv + (\Bv \cdot \Bsigma) \Bsigma (1 - \gamma^{-1}) - \BV}{ 1 - \Bv \cdot \BV/c^2 }.
}

Finally, if we are to compare to the text, we note that the inverse expression requires replacement of \(\BV\) with \(-\BV\) and switching \(\Bv\) with \(\Bv'\).  That gives us
%
\begin{equation}\label{eqn:relativisticElectrodynamicsA1:250i}
\Bv = \frac{\gamma^{-1} \Bv' + (\Bv' \cdot \Bsigma) \Bsigma (1 - \gamma^{-1}) + \BV}{ 1 + \Bv' \cdot \BV/c^2 }.
\end{equation}
%
The expression in the text is also a small velocity approximation.  For \(\Abs{\BV} \ll c\), we have \(\gamma^{-1} \approx 1\), and \((1 + \Bv' \cdot \BV/c^2)^{-1} \approx 1 - \Bv' \cdot \BV/c^2\).  This gives us
%
\begin{equation}\label{eqn:relativisticElectrodynamicsA1:250a}
\Bv \approx (\Bv' + \BV)( 1 - \Bv' \cdot \BV/c^2 ) \approx \BV + \Bv' - \Bv' (\Bv' \cdot \BV)/c^2.
\end{equation}
%
One additional approximation was made dropping the \(\BV (\Bv' \cdot \BV)/c^2\) term which is quadratic in \(\BV/c\), which leave us with equation \(5.3\) in the text as desired.
%
\makeSubAnswer{}{pr:relElectroDynProblemSet1:2:b}
%

In \eqnref{eqn:relativisticElectrodynamicsA1:250i}, let us write \(\Bv' = u \Bu\), where \(\Bu\) is a unit vector, \(V = \BV \cdot \Bsigma\), and \(\alpha = \Bu \cdot \Bsigma\) for the direction cosine between the primed frame's direction of motion and the particle's velocity direction (also in the unprimed frame).  The stationary frame's particle velocity is then
%
\begin{equation}\label{eqn:relativisticElectrodynamicsA1:260}
\Bv = \frac{\gamma^{-1} u \Bu + u \alpha \Bsigma (1 - \gamma^{-1}) + V \Bsigma}{ 1 + \alpha u V/c^2 }.
\end{equation}
%
As a check, note that for \(1 = \alpha = \Bu \cdot \Bsigma = \cos(0)\), we recover the familiar addition of velocities formula
%
\begin{equation}\label{eqn:relativisticElectrodynamicsA1:260b}
\Bv = \Bu \frac{u + V}{ 1 + u V/c^2 }.
\end{equation}
%
We want to put \eqnref{eqn:relativisticElectrodynamicsA1:260} into a form that renders it more tractable for general angles too.  Factoring out the \(\gamma^{-1}\) term appears to do the job, yielding
%
\begin{equation}\label{eqn:relativisticElectrodynamicsA1:260c}
\Bv = \frac{u \gamma^{-1} (\Bu -\alpha \Bsigma) + (u \alpha + V) \Bsigma}{ 1 + \alpha u V/c^2 }.
\end{equation}
%
After a bit of reduction and rearranging we can dot this with itself to calculate
%
\begin{equation}\label{eqn:relativisticElectrodynamicsA1:270}
\Bv^2 = \frac{V^2(1 - \alpha^2)(1 - u^2/c^2) + (u + \alpha V)^2}{ (1 + \alpha u V/c^2)^2 }.
\end{equation}
%
Note that for \(u = c\), we have \(\Bv^2 = c^2\), regardless of the direction of \(\BV\) with respect to the motion of the particle in the unprimed frame.  This should not be surprising since this light like invariance is exactly what the Lorentz transformation is designed to maintain.  It is however slightly comforting to know that the algebra appears to be still be kosher after all this.  This also answers part (b) of this question, since we have tackled the \(v = c\) case in the primed frame, and seen that the speed remains \(v = c\) in the unprimed frame (and thus any frame moving at constant speed relative to another).

Observe that since \(1 - \alpha^2 = \sin^2\theta\), and \(u \le c\), this is positive definite as expected.  If one allowed \(u > c\) in some frame, our speed could go imaginary!

For the \(u < c\) and \(u > c\) cases, let \(x = u/c\) and \(y = V/c\).  This allows \eqnref{eqn:relativisticElectrodynamicsA1:270} to be casted in a simpler form
%
\begin{equation}\label{eqn:relativisticElectrodynamicsA1:270e}
\Bv^2 = c^2 \frac{y^2 (1 - \alpha^2)(1 - x^2) + (x + \alpha y)^2}{ (1 + \alpha x y)^2 }.
\end{equation}
%
We wish to verify that (a) given any \(x \in (-1,1)\), we have \(\Bv^2 < c^2\) for all \(y \in (-1,1)\), \(\alpha \in (-1,1)\), and (c) given any \(\Abs{x} > 1\), we have \(\Bv^2 > c^2\) for all \(y \in (-1,1)\), \(\alpha \in (-1,1)\).

Considering (a) first, this requires a demonstration that
%
\begin{equation}\label{eqn:relativisticElectrodynamicsA1:280}
y^2 (1 - \alpha^2)(1 - x^2) + (x + \alpha y)^2 < (1 + \alpha x y)^2 .
\end{equation}
%
Expanding out the products and canceling terms, we want to show that for (a) that if \(\Abs{x},\Abs{y} < 1\) we have
%
\begin{equation}\label{eqn:relativisticElectrodynamicsA1:290a}
x^2 (1 - y^2) + y^2 < 1,
\end{equation}
%
and for (c) that if \(\Abs{x} > 1\), we have for any \(\Abs{y} < 1\)
%
\begin{equation}\label{eqn:relativisticElectrodynamicsA1:290c}
x^2 (1 - y^2) + y^2 > 1.
\end{equation}
%
Observe that the frame velocity orientation direction cosines have completely dropped out, leaving just the (relative to \(c\)) velocity terms.

% https://www.wolframalpha.com/input/?i=graph+x%5E2%281+-y%5E2%29+%2B+y%5E2
To get an initial feel for this function \(f(x,y) = x^2 (1 - y^2) + y^2\), notice that
in \cref{fig:getAFeeling:getAFeelingFig1},
%\href{https://goo.gl/5AnNF}{when graphed}
we have a bowl with a minimum (zero) at the origin, and what appears to be a uniform value of one on the boundary (case (b)).  Then provided \(\Abs{y} < 1\) it appears that the function \(f\) increases monotonically to a value greater than one (case (c)).  While looking at a plot is not any sort of rigorous proof, let us move on to some of the other problems for now, and return to this last loose thread later if time permits.
\imageFigure{../figures/phy450-relativisticEandM/getAFeelingFig1}{Plot of \(x^2 (1 - y^2) + y^2\).}{fig:getAFeeling:getAFeelingFig1}{0.3}
} % makeanswer
