%
% Copyright � 2012 Peeter Joot.  All Rights Reserved.
% Licenced as described in the file LICENSE under the root directory of this GIT repository.
%

\label{chap:relElectroDynProblemSet5}
%\blogpage{http://sites.google.com/site/peeterjoot/math2011/relElectroDynProblemSet5.pdf}
%\date{Mar 14, 2011}

%
\makeproblem{Fields generated by an arbitrarily moving charge}{pr:relElectroDynProblemSet5:2}{
%
Show that for a particle moving on a worldline parametrized by \((ct, \Bx_c(t))\), the retarded time \(t_r\) with respect to an arbitrary space time point \((ct, \Bx)\), defined in class as:
%
\begin{equation}\label{eqn:relElectroDynProblemSet5:10}
\Abs{\Bx - \Bx_c(t_r)} = c(t - t_r)
\end{equation}
%
obeys
%
\begin{equation}\label{eqn:relElectroDynProblemSet5:30}
\spacegrad t_r = -\frac{\Bx - \Bx_c(t_r)}{
c \Abs{\Bx - \Bx_c(t_r)} - \Bv_c(t_r) \cdot (\Bx - \Bx_c(t_r))
}
\end{equation}
%
and
%
\begin{equation}\label{eqn:relElectroDynProblemSet5:50}
\PD{t}{t_r} = \frac{c \Abs{\Bx - \Bx_c(t_r)}}{
c \Abs{\Bx - \Bx_c(t_r)} - \Bv_c(t_r) \cdot (\Bx - \Bx_c(t_r))
}
\end{equation}
%
\makesubproblem{}{pr:relElectroDynProblemSet5:2:a}
%
Then, use these to derive the expressions for \(\BE\) and \(\BB\) given in the book (and in the class notes).
%
\makesubproblem{}{pr:relElectroDynProblemSet5:2:b}
%
Finally, re-derive the already familiar expressions for the EM fields of a particle moving with uniform velocity.

} % makeproblem
%
\makeanswer{pr:relElectroDynProblemSet5:2}{
%
\paragraph{Gradient and time derivatives of the retarded time function}
\index{retarded time}

Let us use notation something like our text \citep{landau1980classical}, where the solution to this problem is outlined in \S 63, and write
%
\begin{equation}\label{eqn:relElectroDynProblemSet5:70}
\begin{aligned}
\BR(t_r) &= \Bx - \Bx_c(t_r) \\
R &= \Abs{\BR}
\end{aligned}
\end{equation}
%
where
%
\begin{equation}\label{eqn:relElectroDynProblemSet5:90}
\PD{t_r}{\BR} = - \Bv_c.
\end{equation}
%
From \(R^2 = \BR \cdot \BR\) we also have
%
\begin{equation}\label{eqn:relElectroDynProblemSet5:110}
2 R \PD{t_r}{R} = 2 \BR \cdot \PD{t_r}{\BR},
\end{equation}
so if we write
\begin{equation}\label{eqn:relElectroDynProblemSet5:130}
\Rcap = \frac{\BR}{R},
\end{equation}
we have
%
\begin{equation}\label{eqn:relElectroDynProblemSet5:150}
R'(t_r) = -\Rcap \cdot \Bv_c.
\end{equation}
%
Proceeding in the manner of the text, we have
%
\begin{equation}\label{eqn:relElectroDynProblemSet5:170}
\PD{t}{R} = \PD{t_r}{R} \PD{t}{t_r} = -\Rcap \cdot \Bv_c \PD{t}{t_r}.
\end{equation}
%
From \eqnref{eqn:relElectroDynProblemSet5:10} we also have
%
\begin{equation}\label{eqn:relElectroDynProblemSet5:200}
R = \Abs{\Bx - \Bx_c(t_r)} = c(t - t_r),
\end{equation}
so
%
\begin{equation}\label{eqn:relElectroDynProblemSet5:220}
\PD{t}{R} = c\left(1 - \PD{t}{t_r}\right).
\end{equation}
%
This and \eqnref{eqn:relElectroDynProblemSet5:170} gives us
\boxedEquation{eqn:relElectroDynProblemSet5:240}{
\PD{t}{t_r} = \inv{ 1 -\Rcap \cdot \frac{\Bv_c}{c} }.
}

For the gradient we operate on the implicit equation \eqnref{eqn:relElectroDynProblemSet5:200} again.  This gives us
%
\begin{equation}\label{eqn:relElectroDynProblemSet5:300}
\spacegrad R = \spacegrad (c t - c t_r) = - c \spacegrad t_r.
\end{equation}
%
However, we can also use the spatial definition of \(R = \Abs{\Bx - \Bx_c(t')}\).  Note that this distance \(R = R(t_r)\) is a function of space and time, since \(t_r = t_r(\Bx, t)\) is implicitly a function of the spatial and time positions at which the retarded time is to be measured.
%
\begin{equation}\label{eqn:relElectroDynProblemSet5:1060}
\begin{aligned}
\spacegrad R
&=
\spacegrad \sqrt{(\Bx - \Bx_c(t_r))^2} \\
&=
\inv{2R} \spacegrad (\Bx - \Bx_c(t_r))^2 \\
&=
\inv{R} (x^\beta - x_c^\beta) \Be_\alpha \partial_\alpha (x^\beta - x_c^\beta(t_r)) \\
&=
\inv{R} (\BR)_\beta \Be_\alpha ({\delta_\alpha}^\beta - \partial_\alpha x_c^\beta(t_r)).
\end{aligned}
\end{equation}
%
We have only this bit \(\partial_\alpha x_c^\beta(t_r)\) to expand, but that is just going to require a chain rule expansion.  This is easier to see in a more generic form
%
\begin{equation}\label{eqn:relElectroDynProblemSet5:320}
\PD{x^\alpha}{f(g)} = \PD{g}{f} \PD{x^\alpha}{g},
\end{equation}
%
so we have
%
\begin{equation}\label{eqn:relElectroDynProblemSet5:340}
\PD{x^\alpha}{x_c^\beta(t_r)} = \PD{t_r}{x_c^\beta(t_r)} \PD{x^\alpha}{t_r},
\end{equation}
%
which gets us close to where we want to be
%
\begin{equation}\label{eqn:relElectroDynProblemSet5:1080}
\begin{aligned}
\spacegrad R
&=
\inv{R} \left(\BR - (\BR)_\beta \PD{t_r}{x_c^\beta(t_r)} \Be_\alpha \PD{x^\alpha}{t_r} \right) \\
&=
\inv{R} \left(\BR - \BR \cdot \PD{t_r}{\Bx_c^\beta(t_r)} \spacegrad t_r \right)
\end{aligned}
\end{equation}
%
Putting the pieces together we have only minor algebra left since we can now equate the two expansions of \(\spacegrad R\)
%
\begin{equation}\label{eqn:relElectroDynProblemSet5:360}
- c \spacegrad t_r = \Rcap - \Rcap \cdot \Bv_c(t_r) \spacegrad t_r.
\end{equation}
%
This is given in the text, but these in between steps are left for us and for our homework assignments!  From this point we can rearrange to find the desired result
\boxedEquation{eqn:relElectroDynProblemSet5:380}{
\spacegrad t_r = -\inv{c} \frac{\Rcap }{ 1 - \Rcap \cdot \frac{\Bv_c}{c} } = - \frac{\Rcap}{c} \PD{t}{t_r}.
}
%
\makeSubAnswer{}{pr:relElectroDynProblemSet5:2:a}
%
\paragraph{Computing the EM fields from the Lienard-Wiechert potentials}
\index{Lienard-Wiechert potentials}
Now we are ready to derive the values of \(\BE\) and \(\BB\) that arise from the Lienard-Wiechert potentials.  We have for the electric field.
We will evaluate
%
\begin{equation}\label{eqn:relElectroDynProblemSet5:1100}
\begin{aligned}
\BE &= -\inv{c} \PD{t}{\BA} - \spacegrad \phi \\
\BB &= \spacegrad \cross \BB.
\end{aligned}
\end{equation}
%
For the electric field we will use the chain rule on the vector potential
%
\begin{equation}\label{eqn:relElectroDynProblemSet5:400}
\PD{t}{\BA} = \PD{t}{t_r} \PD{t_r}{\BA}.
\end{equation}
%
Similarly for the gradient of the scalar potential we have
%
\begin{equation}\label{eqn:relElectroDynProblemSet5:1120}
\begin{aligned}
\spacegrad \phi
&=
\Be_\alpha \PD{x^\alpha}{\phi} \\
&=
\Be_\alpha \PD{t_r}{\phi} \PD{x^\alpha}{t_r} \\
&=
\PD{t_r}{\phi} \spacegrad t_r \\
&=
- \PD{t_r}{\phi} \frac{\Rcap}{c} \PD{t}{t_r}.
\end{aligned}
\end{equation}
%
Our electric field is thus
\begin{equation}\label{eqn:relElectroDynProblemSet5:420}
\BE = - \PD{t}{t_r} \left( \inv{c} \PD{t_r}{\BA} - \frac{\Rcap}{c} \PD{t_r}{\phi} \right).
\end{equation}
%
For the magnetic field we have
%
\begin{equation}\label{eqn:relElectroDynProblemSet5:1140}
\begin{aligned}
\spacegrad \cross \BA
&=
\Be_\alpha \cross \PD{x^\alpha}{\BA} \\
&=
\Be_\alpha \cross
\PD{t_r}{\BA} \PD{x^\alpha}{t_r}.
\end{aligned}
\end{equation}
%
The magnetic field will therefore be found by evaluating
%
\begin{equation}\label{eqn:relElectroDynProblemSet5:440}
\BB = (\spacegrad t_r) \cross \PD{t_r}{\BA} = - \PD{t}{t_r} \frac{\Rcap}{c} \cross \PD{t_r}{\BA}
\end{equation}
%
Let us compare this to \(\Rcap \cross \BE\)
%
\begin{equation}\label{eqn:relElectroDynProblemSet5:1160}
\begin{aligned}
\Rcap \cross \BE
&= \Rcap \cross \left(
- \PD{t}{t_r} \left( \inv{c} \PD{t_r}{\BA} - \frac{\Rcap}{c} \PD{t_r}{\phi} \right) \right) \\
&= \Rcap \cross \left( - \PD{t}{t_r} \inv{c} \PD{t_r}{\BA} \right)
\end{aligned}
\end{equation}
%
This equals \eqnref{eqn:relElectroDynProblemSet5:440}, verifying that we have
%
\begin{equation}\label{eqn:relElectroDynProblemSet5:450}
\BB = \Rcap \cross \BE,
\end{equation}
%
something that we can determine even without fully evaluating \(\BE\).

We are now left to evaluate the retarded time derivatives found in \eqnref{eqn:relElectroDynProblemSet5:420}.  Our potentials are
%
\begin{equation}\label{eqn:relElectroDynProblemSet5:470}
\begin{aligned}
\phi(\Bx, t) &= \frac{e}{R(t_r)} \PD{t}{t_r} \\
\BA(\Bx, t) &= \frac{e \Bv_c(t_r)}{c R(t_r)} \PD{t}{t_r}
\end{aligned}
\end{equation}
%
It is clear that the quantity \(\PDi{t}{t_r}\) is going to show up all over the place, so let us label it \(\gamma_{t_r}\).  This is justified by comparing to a particle's boosted rest frame worldline
%
\begin{equation}\label{eqn:relElectroDynProblemSet5:260}
\begin{bmatrix}
c t' \\
x'
\end{bmatrix}
=
\gamma
\begin{bmatrix}
1 & -\beta \\
-\beta & 1
\end{bmatrix}
\begin{bmatrix}
c t \\
0
\end{bmatrix}
=
\begin{bmatrix}
\gamma c t \\
-\gamma \beta c t
\end{bmatrix},
\end{equation}
%
where we have \(\PDi{t}{t'} = \gamma\), so for the remainder of this part of this problem we will write
%
\begin{equation}\label{eqn:relElectroDynProblemSet5:280}
\gamma_{t_r} \equiv \PD{t}{t_r} = \inv{ 1 -\Rcap \cdot \frac{\Bv_c}{c} }.
\end{equation}
%
Using primes to denote partial derivatives with respect to the retarded time \(t_r\) we have
%
\begin{equation}\label{eqn:relElectroDynProblemSet5:490}
\begin{aligned}
\phi' &= e \left( -\frac{R'}{R^2} \gamma_{t_r} + \frac{\gamma_{t_r}'}{R} \right) \\
\BA' &= e \frac{\Bv_c}{c} \left( -\frac{R'}{R^2} \gamma_{t_r} + \frac{\gamma_{t_r}'}{R} \right)
+ e \frac{\Ba_c}{c} \frac{\gamma_{t_r}}{R},
\end{aligned}
\end{equation}
%
so the electric field is
%
\begin{equation}\label{eqn:relElectroDynProblemSet5:1180}
\begin{aligned}
\BE
&= - \gamma_{t_r} \left( \inv{c} \BA' - \frac{\Rcap}{c} \phi' \right) \\
&= - \frac{e \gamma_{t_r}}{c} \left(
\frac{\Bv_c}{c} \left( -\frac{R'}{R^2} \gamma_{t_r} + \frac{\gamma_{t_r}'}{R} \right)
+ \frac{\Ba_c}{c} \frac{\gamma_{t_r}}{R}
- \Rcap \left( -\frac{R'}{R^2} \gamma_{t_r} + \frac{\gamma_{t_r}'}{R} \right)
\right) \\
&= - \frac{e \gamma_{t_r}}{c} \left(
\frac{\Bv_c}{c} \left( \frac{c}{R^2} \gamma_{t_r} + \frac{\gamma_{t_r}'}{R} \right)
+ \frac{\Ba_c}{c} \frac{\gamma_{t_r}}{R}
- \Rcap \left( \frac{c}{R^2} \gamma_{t_r} + \frac{\gamma_{t_r}'}{R} \right)
\right) \\
&= - \frac{e \gamma_{t_r}}{c R} \left( \gamma_{t_r} \left(
\frac{\Ba_c}{c}
+\frac{\Bv_c}{R}
- \frac{\Rcap c}{R}
\right)
+ \gamma_{t_r}' \left( \frac{\Bv_c}{c} - \Rcap \right)
\right).
\end{aligned}
\end{equation}
%
Here is where things get slightly messy.
%
\begin{equation}\label{eqn:relElectroDynProblemSet5:1200}
\begin{aligned}
\gamma_{t_r}'
&= \PD{t_r}{} \inv{1 - \frac{\Bv_c}{c} \cdot \Rcap } \\
&= -\gamma_{t_r}^2 \PD{t_r}{} \left( 1 - \frac{\Bv_c}{c} \cdot \Rcap \right) \\
&= \gamma_{t_r}^2 \left( \frac{\Ba_c}{c} \cdot \Rcap + \frac{\Bv_c}{c} \cdot \Rcap' \right),
\end{aligned}
\end{equation}
and messier
\begin{equation}\label{eqn:relElectroDynProblemSet5:1220}
\begin{aligned}
\Rcap'
&= \PD{t_r}{} \frac{ \BR }{ R } \\
&= \frac{ \BR' }{ R } - \frac{\BR R'}{R^2} \\
&= -\frac{ \Bv_c }{ R } - \frac{\Rcap (-c)}{R} \\
&= \inv{R} \left( -\Bv_c + c \Rcap \right),
\end{aligned}
\end{equation}
then a bit unmessier
%
\begin{equation}\label{eqn:relElectroDynProblemSet5:1240}
\begin{aligned}
\gamma_{t_r}'
&= \gamma_{t_r}^2 \left( \frac{\Ba_c}{c} \cdot \Rcap + \frac{\Bv_c}{c} \cdot \Rcap' \right) \\
&= \gamma_{t_r}^2 \left( \frac{\Ba_c}{c} \cdot \Rcap + \frac{\Bv_c}{c R} \cdot (-\Bv_c + c \Rcap) \right) \\
&= \gamma_{t_r}^2 \left( \Rcap \cdot \left( \frac{\Ba_c}{c} + \frac{\Bv_c}{R} \right) - \frac{\Bv_c^2}{c R} \right).
\end{aligned}
\end{equation}
%
Now we are set to plug this back into our electric field expression and start grouping terms
%
\begin{equation}\label{eqn:relElectroDynProblemSet5:1260}
\begin{aligned}
\BE
&= - \frac{e \gamma_{t_r}^2}{c R} \left(
\frac{\Ba_c}{c}
+\frac{\Bv_c}{R}
- \frac{\Rcap c}{R}
+ \gamma_{t_r} \left( \Rcap \cdot \left( \frac{\Ba_c}{c} + \frac{\Bv_c}{R} \right) - \frac{\Bv_c^2}{c R} \right)
\left( \frac{\Bv_c}{c} - \Rcap \right)
\right) \\
&= - \frac{e \gamma_{t_r}^3}{c R} \left(
\left(
\frac{\Ba_c}{c}
+\frac{\Bv_c}{R}
- \frac{\Rcap c}{R}
\right) \left(
1 -\Rcap \cdot \frac{\Bv_c}{c}
\right)
+ \left( \Rcap \cdot \left( \frac{\Ba_c}{c} + \frac{\Bv_c}{R} \right) - \frac{\Bv_c^2}{c R} \right)
\left( \frac{\Bv_c}{c} - \Rcap \right)
\right) \\
&=
- \frac{e \gamma_{t_r}^3}{c^2 R} \left(
\Ba_c
\left(
1 -\Rcap \cdot \frac{\Bv_c}{c}
\right)
+\Rcap \cdot \Ba_c \left( \frac{\Bv_c}{c} - \Rcap \right)
\right) \\
&\quad - \frac{e \gamma_{t_r}^3}{c R} \left(
\left(
\frac{\Bv_c}{R}
- \frac{\Rcap c}{R}
\right)
\left(
1 -\Rcap \cdot \frac{\Bv_c}{c}
\right)
+ \left( \Rcap \cdot \left( \frac{\Bv_c}{R} \right) - \frac{\Bv_c^2}{c R} \right)
\left( \frac{\Bv_c}{c} - \Rcap \right)
\right).
\end{aligned}
\end{equation}
%
Using
%
\begin{equation}\label{eqn:relElectroDynProblemSet5:510}
\Ba \cross (\Bb \cross \Bc) = \Bb (\Ba \cdot \Bc) - \Bc (\Ba \cdot \Bb)
\end{equation}
%
We can verify that
%
\begin{equation}\label{eqn:relElectroDynProblemSet5:1280}
\begin{aligned}
- \left(
\Ba_c
\left(
1 -\Rcap \cdot \frac{\Bv_c}{c}
\right)
+\Rcap \cdot \Ba_c \left( \frac{\Bv_c}{c} - \Rcap \right)
\right)
&=
-\Ba_c + \Ba \Rcap \cdot \frac{\Bv}{c} - \Rcap \cdot \Ba_c \frac{\Bv_c}{c} + \Rcap \cdot \Ba_c \Rcap \\
&= \Rcap \cross \left( \left(\Rcap - \frac{\Bv_c}{c} \right) \cross \Ba_c \right),
\end{aligned}
\end{equation}
%
which gets us closer to the desired end result
%
\begin{equation}\label{eqn:relElectroDynProblemSet5:530}
\begin{aligned}
\BE
&=
\frac{e \gamma_{t_r}^3}{c^2 R} \Rcap \cross \left( \left(\Rcap - \frac{\Bv_c}{c} \right) \cross \Ba_c \right) \\
&- \frac{e \gamma_{t_r}^3}{c R^2} \left(
\left(
\Bv_c
- \Rcap c
\right)
\left(
1 -\Rcap \cdot \frac{\Bv_c}{c}
\right)
+ \left( \Rcap \cdot \Bv_c - \frac{\Bv_c^2}{c} \right)
\left( \frac{\Bv_c}{c} - \Rcap \right)
\right).
\end{aligned}
\end{equation}
%
It is also easy to show that the remaining bit reduces nicely, since all the dot product terms conveniently cancel
%
\begin{equation}\label{eqn:relElectroDynProblemSet5:550}
- \left(
\left(
\Bv_c
- \Rcap c
\right)
\left(
1 -\Rcap \cdot \frac{\Bv_c}{c}
\right)
+ \left( \Rcap \cdot \Bv_c - \frac{\Bv_c^2}{c} \right)
\left( \frac{\Bv_c}{c} - \Rcap \right)
\right)
=
c
\left( 1 - \frac{\Bv_c^2}{c^2} \right)
\left( \Rcap - \frac{\Bv}{c} \right)
\end{equation}
%
This completes the exercise, leaving us with
\boxedEquation{eqn:relElectroDynProblemSet5:570}{
\begin{aligned}
\BE
&=
\frac{e \gamma_{t_r}^3}{c^2 R} \Rcap \cross \left( \left(\Rcap - \frac{\Bv_c}{c} \right) \cross \Ba_c \right)
+\frac{e \gamma_{t_r}^3}{R^2}
\left( 1 - \frac{\Bv_c^2}{c^2} \right)
\left( \Rcap - \frac{\Bv_c}{c} \right) \\
\BB &= \Rcap \cross \BE.
\end{aligned}
}

Looking back to \eqnref{eqn:relElectroDynProblemSet5:280} where \(\gamma_{t_r}\) was defined, we see that this compares to (63.8-9) in the text.
%
\makeSubAnswer{}{pr:relElectroDynProblemSet5:2:b}
%
\paragraph{EM fields from a uniformly moving source}
%
For a uniform source moving in space at constant velocity
%
\begin{equation}\label{eqn:relElectroDynProblemSet5:590}
\Bx_c(t) = \Bv t,
\end{equation}
%
our retarded time measured from the spacetime point \((ct, \Bx)\) is defined implicitly by
%
\begin{equation}\label{eqn:relElectroDynProblemSet5:610}
R = \Abs{\Bx - \Bx_c(t_r) } = c (t - t_r).
\end{equation}
%
Squaring this we have
%
\begin{equation}\label{eqn:relElectroDynProblemSet5:630}
\Bx^2 + \Bv^2 t_r^2 - 2 t_r \Bx \cdot \Bv = c^2 t^2 + c^2 t_r^2 - 2 c t t_r,
\end{equation}
or
\begin{equation}\label{eqn:relElectroDynProblemSet5:650}
( c^2 -\Bv^2) t_r^2 + 2 t_r ( - c t + \Bx \cdot \Bv ) = \Bx^2 - c^2 t^2.
\end{equation}
%
Rearranging to complete the square we have
%
\begin{equation}\label{eqn:relElectroDynProblemSet5:1300}
\begin{aligned}
&\left( \sqrt{c^2 - \Bv^2} t_r - \frac{ t c^2 - \Bx \cdot \Bv }{\sqrt{c^2 - \Bv^2}} \right)^2  \\
&= \Bx^2 - c^2 t^2 +\frac{ (t c^2 - \Bx \cdot \Bv)^2 }{c^2 - \Bv^2} \\
&= \frac{ (\Bx^2 - c^2 t^2)( c^2 - \Bv^2) + (t c^2 - \Bx \cdot \Bv)^2}{ c^2 - \Bv^2} \\
&= \frac{ \Bx^2 c^2 - \Bx^2 \Bv^2 - \cancel{c^4 t^2} + c^2 t^2 \Bv^2 + \cancel{t^2 c^4} + (\Bx \cdot \Bv)^2 - 2 t c^2 (\Bx \cdot \Bv) }{ c^2 - \Bv^2 } \\
&= \frac{ c^2 ( \Bx^2 + t^2 \Bv^2 -2 t (\Bx \cdot \Bv)) - \Bx^2 \Bv^2 + (\Bx \cdot \Bv)^2 }{ c^2 - \Bv^2 } \\
&= \frac{ c^2 ( \Bx - \Bv t)^2 - (\Bx \cross \Bv)^2 }{ c^2 - \Bv^2 }.
\end{aligned}
\end{equation}
%
Taking roots (and keeping the negative so that we have \(t_r = t - \Abs{\Bx}/c\) for the \(\Bv = 0\) case, we have
%
\begin{equation}\label{eqn:relElectroDynProblemSet5:670}
\sqrt{1 - \frac{\Bv^2}{c^2}} c t_r
=
\inv{\sqrt{1 - \frac{\Bv^2}{c^2}}} \left(
c t - \Bx \cdot \frac{\Bv}{c} - \sqrt{ \left( \Bx - \Bv t \right)^2 - \left(\Bx \cross \frac{\Bv}{c}\right)^2 }
\right),
\end{equation}
%
or with \(\Bbeta = \Bv/c\), this is
%
\begin{equation}\label{eqn:relElectroDynProblemSet5:690}
c t_r = \inv{1 - \Bbeta^2} \left( c t - \Bx \cdot \Bbeta - \sqrt{ \left( \Bx - \Bv t \right)^2 - \left(\Bx \cross \Bbeta\right)^2 } \right).
\end{equation}
%
What is our retarded distance \(R = c t - c t_r\)?  We get
%
\begin{equation}\label{eqn:relElectroDynProblemSet5:710}
R = \frac{\Bbeta \cdot (\Bx - \Bv t) + \sqrt{ (\Bx - \Bv t)^2 - (\Bx \cross \Bbeta)^2 }}{ 1 - \Bbeta^2 }.
\end{equation}
%
For the vector distance we get (with \(\Bbeta \cdot (\Bx \wedge \Bbeta) = (\Bbeta \cdot \Bx) \Bbeta - \Bx \Bbeta^2\))
%
\begin{equation}\label{eqn:relElectroDynProblemSet5:730}
\BR = \frac{\Bx -\Bv t + \Bbeta \cdot (\Bx \wedge \Bbeta) + \Bbeta \sqrt{ (\Bx - \Bv t)^2 - (\Bx \cross \Bbeta)^2 }}{ 1 - \Bbeta^2 }.
\end{equation}
%
For the unit vector \(\Rcap = \BR/R\) we have
%
\begin{equation}\label{eqn:relElectroDynProblemSet5:750}
\Rcap = \frac{\Bx - \Bv t + \Bbeta \cdot (\Bx \wedge \Bbeta) + \Bbeta \sqrt{ (\Bx - \Bv t)^2 - (\Bx \cross \Bbeta)^2 }}{
\Bbeta \cdot (\Bx - \Bv t) + \sqrt{ (\Bx - \Bv t)^2 - (\Bx \cross \Bbeta)^2 }
}.
\end{equation}
%
The acceleration term in the electric field is zero, so we are left with just
%
\begin{equation}\label{eqn:relElectroDynProblemSet5:770}
\BE
=
\frac{e \gamma_{t_r}^3}{R^2}
\left( 1 - \frac{\Bv_c^2}{c^2} \right)
\left( \Rcap - \frac{\Bv_c}{c} \right).
\end{equation}
%
Leading to \(\gamma_{t_r}\), we have
%
\begin{equation}\label{eqn:relElectroDynProblemSet5:940}
\Rcap \cdot \Bbeta = \frac{\Bbeta \cdot (\Bx - \Bv t + R^\conj \Bbeta)}{\Bbeta \cdot (\Bx - \Bv t) + R^\conj},
\end{equation}
%
where, following \S 38 of the text we write
%
\begin{equation}\label{eqn:relElectroDynProblemSet5:960}
R^\conj = \sqrt{(\Bx - \Bv t)^2 - (\Bx \cross \Bbeta)^2 }
\end{equation}
%
This gives us
%
\begin{equation}\label{eqn:relElectroDynProblemSet5:790}
\gamma_{t_r} =
\frac{\Bbeta \cdot (\Bx - \Bv t) + R^\conj}{R^\conj(1 - \Bbeta^2)}.
\end{equation}
%
Observe that this equals one when \(\Bbeta = 0\) as expected.

We can also compute
%
\begin{equation}\label{eqn:relElectroDynProblemSet5:1320}
\begin{aligned}
\Rcap - \Bbeta &=
\frac{\Bx + \Bbeta \cdot (\Bx \wedge \Bbeta) - \Bv t + \Bbeta \sqrt{ (\Bx - \Bv t)^2 - (\Bx \cross \Bbeta)^2 }}{
\Bbeta \cdot (\Bx - \Bv t) + \sqrt{ (\Bx - \Bv t)^2 - (\Bx \cross \Bbeta)^2 }
} - \Bbeta \\
&=
\frac{(\Bx - \Bv t)(1 -\Bbeta^2)}{
\Bbeta \cdot (\Bx - \Bv t) + \sqrt{ (\Bx - \Bv t)^2 - (\Bx \cross \Bbeta)^2 }
}.
\end{aligned}
\end{equation}
%
Our long and messy expression for the field is therefore
%
\begin{equation}\label{eqn:relElectroDynProblemSet5:1340}
\begin{aligned}
\BE &=
e \gamma_{t_r}^3 \inv{R^2} (1 - \Bbeta^2)(\Rcap - \Bbeta) \\
&=
e
\left(
\frac{\Bbeta \cdot (\Bx - \Bv t) + R^\conj}{R^\conj(1 - \Bbeta^2)}
\right)^3
\frac{(1 - \Bbeta^2)^2 }{ (\Bbeta \cdot (\Bx - \Bv t) + R^\conj)^2 }
(1 -\Bbeta^2)
\frac{(\Bx - \Bv t)(1 -\Bbeta^2)}{
\Bbeta \cdot (\Bx - \Bv t) + R^\conj
}.
\end{aligned}
\end{equation}
This gives us our final result
\begin{equation}\label{eqn:relElectroDynProblemSet5:980}
\BE =
e \inv{(R^\conj)^3}
(1 -\Bbeta^2)
(\Bx - \Bv t).
\end{equation}
%
As a small test we observe that we get the expected result
%
\begin{equation}\label{eqn:relElectroDynProblemSet5:820}
\BE = e \frac{\Bx}{\Abs{\Bx}^3},
\end{equation}
for the \(\Bbeta = 0\) case.

When \(\Bv = V \Be_1\) this also recovers equation (38.6) from the text as desired, and if we switch to primed coordinates
%
\begin{equation}\label{eqn:relElectroDynProblemSet5:840}
\begin{aligned}
x' &= \gamma( x - v t) \\
y' &= y \\
z' &= z \\
(1 - \beta^2) {r'}^2 &= (x - v t)^2 + (y^2 + z^2)(1 - \beta^2),
\end{aligned}
\end{equation}
%
we recover the field equation derived twice before in previous problem sets
%
\begin{equation}\label{eqn:relElectroDynProblemSet5:1000}
\BE = \frac{e}{(r')^3} ( x', \gamma y', \gamma z')
\end{equation}
%
\paragraph{EM fields from a uniformly moving source along x axis}
%
Initially I had errors in the vector treatment above, so tried with the simpler case using uniform velocity \(v\) along the \(x\) axis instead.  Comparison of the two showed where my errors were in the vector algebra, and that is now also fixed up.

Performing all the algebra to solve for \(t_r\) in
%
\begin{equation}\label{eqn:relElectroDynProblemSet5:860}
\Abs{\Bx - v t_r \Be_1} = c(t - t_r),
\end{equation}
%
I get
%
\begin{equation}\label{eqn:relElectroDynProblemSet5:880}
c t_r = \frac{c t - x \beta - \sqrt{ (x- v t)^2 + (y^2 + z^2)(1-\beta^2) } }{ 1 - \beta^2 } = - \gamma (\beta x' + r' )
\end{equation}
%
This matches the vector expression from \eqnref{eqn:relElectroDynProblemSet5:690} with the special case of \(\Bv = v \Be_1\) so we at least started off on the right foot.

For the retarded distance \(R = ct - c t_r\) we get
%
\begin{equation}\label{eqn:relElectroDynProblemSet5:900}
R = \frac{ \beta( x - v t) + \sqrt{ (x- v t)^2 + (y^2 + z^2)(1-\beta^2) } }{ 1 - \beta^2 } = \gamma( \beta x' + r' )
\end{equation}
%
This also matches \eqnref{eqn:relElectroDynProblemSet5:710}, so things still seem okay with the vector approach.  What is our vector retarded distance
%
\begin{equation}\label{eqn:relElectroDynProblemSet5:1360}
\begin{aligned}
\BR
&= \Bx - \beta c t_r \Be_1 \\
&= (x - \beta c t_r, y, z) \\
&= \left( \frac{ x - v t + \beta \sqrt{ (x- v t)^2 + (y^2 + z^2)(1-\beta^2) } }{ 1 - \beta^2 }, y, z \right) \\
&= \left( \gamma (x' + \beta r'), y', z' \right),
\end{aligned}
\end{equation}
so
\begin{equation}\label{eqn:relElectroDynProblemSet5:1380}
\begin{aligned}
\Rcap
&= \inv{ \gamma (\beta x' + r') } \left( \gamma(x' + \beta r'), y', z' \right) \\
&= \inv{ \beta x' + r' } \left( x' + \beta r', \frac{y'}{\gamma}, \frac{z'}{\gamma} \right)
\end{aligned}
\end{equation}
%
\begin{equation}\label{eqn:relElectroDynProblemSet5:1400}
\begin{aligned}
\Rcap -\Bbeta
&= \inv{ \gamma (\beta x' + r') } \left( \gamma(x' + \beta r'), y', z' \right) - (\beta, 0, 0) \\
&= \inv{ \beta x' + r' } \left( x'(1- \beta^2), \frac{y'}{\gamma}, \frac{z'}{\gamma} \right) \\
&= \inv{\gamma (\beta x' + r')} ( x - v t, y, z).
\end{aligned}
\end{equation}
%
For \(\PDi{t}{t_r}\), using \(\Rcap\) calculated above, or from \eqnref{eqn:relElectroDynProblemSet5:880} calculating directly I get
%
\begin{equation}\label{eqn:relElectroDynProblemSet5:920}
\PD{t}{t_r} = \frac{r' + \beta x'}{r'(1 - \beta^2)} = \frac{\gamma( r' + \beta x') }{R^\conj},
\end{equation}
%
where, as in \S 38 of the text, we write
%
\begin{equation}\label{eqn:relElectroDynProblemSet5:1020}
R^\conj = \sqrt{ (x - v t)^2 + (y^2 + z^2)(1-\beta^2) }.
\end{equation}
%
Putting all the pieces together I get
%
\begin{equation}\label{eqn:relElectroDynProblemSet5:1420}
\begin{aligned}
\BE
&= e (1 -\beta^2) \frac{(x - v t, y, z)}{\cancel{\gamma( \beta x' + r'})} \left( \frac{\cancel{\gamma( r' + \beta x')}}{R^\conj} \right)^3 \inv{\cancel{ \gamma^2 (\beta x' + r')^2 } },
\end{aligned}
\end{equation}
%
so we have
%
\begin{equation}\label{eqn:relElectroDynProblemSet5:1040}
\BE
= e \frac{1 -\beta^2}{(R^\conj)^3} (x - v t, y, z).
\end{equation}
%
This matches equation (38.6) in the text.
} % makeanswer
%
%\section{Problem 3}
%
%FIXME: TODO.
