% ****************************************************************************************************
% classicthesis-config.tex
% formerly known as loadpackages.sty, classicthesis-ldpkg.sty, and classicthesis-preamble.sty
% Use it at the beginning of your ClassicThesis.tex, or as a LaTeX Preamble
% in your ClassicThesis.{tex,lyx} with % ****************************************************************************************************
% classicthesis-config.tex
% formerly known as loadpackages.sty, classicthesis-ldpkg.sty, and classicthesis-preamble.sty
% Use it at the beginning of your ClassicThesis.tex, or as a LaTeX Preamble
% in your ClassicThesis.{tex,lyx} with % ****************************************************************************************************
% classicthesis-config.tex
% formerly known as loadpackages.sty, classicthesis-ldpkg.sty, and classicthesis-preamble.sty
% Use it at the beginning of your ClassicThesis.tex, or as a LaTeX Preamble
% in your ClassicThesis.{tex,lyx} with % ****************************************************************************************************
% classicthesis-config.tex
% formerly known as loadpackages.sty, classicthesis-ldpkg.sty, and classicthesis-preamble.sty
% Use it at the beginning of your ClassicThesis.tex, or as a LaTeX Preamble
% in your ClassicThesis.{tex,lyx} with \input{classicthesis-config}
% ****************************************************************************************************
% If you like the classicthesis, then I would appreciate a postcard.
% My address can be found in the file ClassicThesis.pdf. A collection
% of the postcards I received so far is available online at
% http://postcards.miede.de
% ****************************************************************************************************

% ****************************************************************************************************
% 1. Configure classicthesis for your needs here, e.g., remove "drafting" below
% in order to deactivate the time-stamp on the pages
% ****************************************************************************************************
\PassOptionsToPackage{eulerchapternumbers,listings,%
				 pdfspacing,%
				 subfig,beramono,parts}{classicthesis}										
% ********************************************************************
% Available options for classicthesis.sty
% (see ClassicThesis.pdf for more information):
% drafting
% parts nochapters linedheaders
% eulerchapternumbers beramono eulermath pdfspacing minionprospacing
% tocaligned dottedtoc manychapters
% listings floatperchapter subfig
% ********************************************************************

% ********************************************************************
%
% http://tex.stackexchange.com/questions/3676/too-many-math-alphabets-error
%
% adding \usepackage{txfonts} % \ointclockwise
%
% caused too many alphabets error.  se workaround:

\newcommand{\bmmax}{0}
\newcommand{\hmmax}{0}

% ********************************************************************
% https://tex.stackexchange.com/a/348976/15
\BeforeTOCHead[toc]{\cleardoublepage\pdfbookmark{\contentsname}{toc}}
\BeforeTOCHead[lof]{\cleardoublepage\pdfbookmark{\listfigurename}{lof}}

% ********************************************************************
% Triggers for this config
% ********************************************************************
\usepackage{ifthen}
\usepackage{censor}
\newboolean{enable-backrefs} % enable backrefs in the bibliography
\setboolean{enable-backrefs}{true} % true false

%% \newboolean{redacted}
%% \usepackage{redacted}
%%
%% \newcommand{\withproblemsets}[1]{%
%% \ifthenelse{\boolean{redacted}}%
%% {%
%% \textcolor{Maroon}{%
%% PROBLEM SET RELATED MATERIAL REDACTED IN THIS DOCUMENT.%
%% %
%% PLEASE FEEL FREE TO EMAIL ME FOR THE FULL VERSION IF YOU AREN'T TAKING THE COURSE.%
%% }%
%% %
%% \input{censor1pg.tex}%
%% %
%% \textcolor{Maroon}{%
%% END-REDACTION%
%% }%
%% }%
%% {#1}%
%% }
%%
%% \newcommand{\withproblemsetsParagraph}[1]{%
%% \ifthenelse{\boolean{redacted}}%
%% {%
%% \textcolor{Maroon}{%
%% PROBLEM SET RELATED MATERIAL REDACTED IN THIS DOCUMENT.%
%% %
%% PLEASE FEEL FREE TO EMAIL ME FOR THE FULL VERSION IF YOU AREN'T TAKING THE COURSE.%
%% }%
%% %
%% \input{censor1p.tex}%
%% %\input{censor1p.tex}%
%% %\input{censor1p.tex}%
%% %
%% \textcolor{Maroon}{%
%% END-REDACTION%
%% }%
%% }%
%% {#1}%
%% }
%%
%% \newcommand{\withproblemsetsMessage}[1]{%
%% \ifthenelse{\boolean{redacted}}%
%% {%
%% #1%
%% }%
%% {\relax}%
%% }

% ****************************************************************************************************


% ****************************************************************************************************


% ****************************************************************************************************
% 2. Personal data and user ad-hoc commands
% ****************************************************************************************************
\newcommand{\myTitle}{Relativistic Electrodynamics\xspace}
\newcommand{\mySubtitle}{Notes and problems from UofT PHY450H1S 2011\xspace}
\newcommand{\myDegree}{}
\newcommand{\myName}{Peeter Joot\xspace}
%\newcommand{\myName}{Peeter Joot\ peeterjoot@protonmail.com\xspace}
\newcommand{\myProf}{}
\newcommand{\myOtherProf}{}
\newcommand{\mySupervisor}{}
\newcommand{\myFaculty}{}
\newcommand{\myDepartment}{}
\newcommand{\myUni}{}
\newcommand{\myLocation}{}
%----------------------
   %\newcommand{\myTime}{June 2012}
   \input{./.revinfo/gitCommitDateAsMyTime.tex}
%----------------------
% v.9 includes index.
%\newcommand{\myVersion}{version v.9\xspace}
% now posted to:
% https://peeterjoot.com/archives/math2011/phy450.pdf

% ********************************************************************
% Setup, finetuning, and useful commands
% ********************************************************************
\newcounter{dummy} % necessary for correct hyperlinks (to index, bib, etc.)
\newlength{\abcd} % for ab..z string length calculation
\providecommand{\mLyX}{L\kern-.1667em\lower.25em\hbox{Y}\kern-.125emX\@}
%\newcommand{\ie}{i.\,e.}
%\newcommand{\Ie}{I.\,e.}
%\newcommand{\eg}{e.\,g.}
%\newcommand{\Eg}{E.\,g.}
% ****************************************************************************************************


% ****************************************************************************************************
% 3. Loading some handy packages
% ****************************************************************************************************
% ********************************************************************
% Packages with options that might require adjustments
% ********************************************************************
\PassOptionsToPackage{latin9}{inputenc}	% latin9 (ISO-8859-9) = latin1+"Euro sign"
 \usepackage{inputenc}				

%\PassOptionsToPackage{ngerman,american}{babel}   % change this to your language(s)
% Spanish languages need extra options in order to work with this template
%\PassOptionsToPackage{spanish,es-lcroman}{babel}
 \usepackage{babel}					

\PassOptionsToPackage{square,numbers}{natbib}
 \usepackage{natbib}				

\PassOptionsToPackage{fleqn}{amsmath}		% math environments and more by the AMS
 \usepackage{amsmath}

% ********************************************************************
% General useful packages
% ********************************************************************
\PassOptionsToPackage{T1}{fontenc} % T2A for cyrillics
	\usepackage{fontenc}
\usepackage{xspace} % to get the spacing after macros right
\usepackage{mparhack} % get marginpar right
\usepackage{fixltx2e} % fixes some LaTeX stuff
\PassOptionsToPackage{printonlyused,smaller}{acronym}
	\usepackage{acronym} % nice macros for handling all acronyms in the thesis
%\renewcommand*{\acsfont}[1]{\textssc{#1}} % for MinionPro
%\renewcommand{\bflabel}[1]{{#1}\hfill} % fix the list of acronyms
% ****************************************************************************************************


% ****************************************************************************************************
% 4. Setup floats: tables, (sub)figures, and captions
% ****************************************************************************************************
\usepackage{tabularx} % better tables
	\setlength{\extrarowheight}{3pt} % increase table row height
\newcommand{\tableheadline}[1]{\multicolumn{1}{c}{\spacedlowsmallcaps{#1}}}
\newcommand{\myfloatalign}{\centering} % to be used with each float for alignment
\usepackage{caption}
\captionsetup{format=hang,font=small}
\usepackage{subfig}
% ****************************************************************************************************


% ****************************************************************************************************
% 5. Setup code listings
% ****************************************************************************************************
\usepackage{listings}
\let\origlstlisting=\lstlisting
\let\endoriglstlisting=\endlstlisting
\renewenvironment{lstlisting}
    {\mathcode`\-=\hyphenmathcode
     \everymath{}\mathsurround=0pt\origlstlisting}
    {\endoriglstlisting}
%\lstset{emph={trueIndex,root},emphstyle=\color{BlueViolet}}%\underbar} % for special keywords
\lstset{language=[LaTeX]Tex,%C++,
    keywordstyle=\color{RoyalBlue},%\bfseries,
    basicstyle=\small\ttfamily,
    %identifierstyle=\color{NavyBlue},
    commentstyle=\color{Green}\ttfamily,
    stringstyle=\rmfamily,
    numbers=none,%left,%
    numberstyle=\scriptsize,%\tiny
    stepnumber=5,
    numbersep=8pt,
    showstringspaces=false,
    breaklines=true,
    frameround=ftff,
    frame=single,
    belowcaptionskip=.75\baselineskip
    %frame=L
}
% ****************************************************************************************************    		


% ****************************************************************************************************
% 6. PDFLaTeX, hyperreferences and citation backreferences
% ****************************************************************************************************
% ********************************************************************
% Using PDFLaTeX
% ********************************************************************
\PassOptionsToPackage{pdftex,hyperfootnotes=false,pdfpagelabels,hyperindex=true}{hyperref}
	\usepackage{hyperref}  % backref linktocpage pagebackref
%----------------------
\pdfcompresslevel=9
% http://tex.stackexchange.com/questions/52317/pdftex-warning-version-allowed
%\pdfminorversion=5
\pdfminorversion=7
\pdfobjcompresslevel=3
%----------------------
\pdfadjustspacing=1
\PassOptionsToPackage{pdftex}{graphicx}
	\usepackage{graphicx}

% ********************************************************************
% Setup the style of the backrefs from the bibliography
% (translate the options to any language you use)
% ********************************************************************
\newcommand{\backrefnotcitedstring}{\relax}%(Not cited.)
\newcommand{\backrefcitedsinglestring}[1]{(Cited on page~#1.)}
\newcommand{\backrefcitedmultistring}[1]{(Cited on pages~#1.)}
\ifthenelse{\boolean{enable-backrefs}}%
{%
		\PassOptionsToPackage{hyperpageref}{backref}
		\usepackage{backref} % to be loaded after hyperref package
		   \renewcommand{\backreftwosep}{ and~} % separate 2 pages
		   \renewcommand{\backreflastsep}{, and~} % separate last of longer list
		   \renewcommand*{\backref}[1]{}  % disable standard
		   \renewcommand*{\backrefalt}[4]{% detailed backref
		      \ifcase #1 %
		         \backrefnotcitedstring%
		      \or%
		         \backrefcitedsinglestring{#2}%
		      \else%
		         \backrefcitedmultistring{#2}%
		      \fi}%
}{\relax}

% ********************************************************************
% Hyperreferences
% ********************************************************************
\hypersetup{%
    %draft,	% = no hyperlinking at all (useful in b/w printouts)
    colorlinks=true, linktocpage=true, pdfstartpage=3, pdfstartview=FitV,%
    % uncomment the following line if you want to have black links (e.g., for printing)
    %colorlinks=false, linktocpage=false, pdfborder={0 0 0}, pdfstartpage=3, pdfstartview=FitV,%
    breaklinks=true, pdfpagemode=UseNone, pageanchor=true, pdfpagemode=UseOutlines,%
    plainpages=false, bookmarksnumbered, bookmarksopen=true, bookmarksopenlevel=1,%
    hypertexnames=true, pdfhighlight=/O,%nesting=true,%frenchlinks,%
    urlcolor=webbrown, linkcolor=RoyalBlue, citecolor=webgreen, %pagecolor=RoyalBlue,%
    %urlcolor=Black, linkcolor=Black, citecolor=Black, %pagecolor=Black,%
    pdftitle={\myTitle},%
    pdfauthor={\textcopyright\ \myName},%
    pdfsubject={Relativistic Electrodynamics},%
    pdfkeywords={Relativistic Electrodynamics} {PHY450H1S} {University of Toronto},%
    pdfcreator={pdfLaTeX},%
    pdfproducer={LaTeX with hyperref and classicthesis}%
}

% ********************************************************************
% Setup autoreferences
% ********************************************************************
% There are some issues regarding autorefnames
% http://www.ureader.de/msg/136221647.aspx
% http://www.tex.ac.uk/cgi-bin/texfaq2html?label=latexwords
% you have to redefine the makros for the
% language you use, e.g., american, ngerman
% (as chosen when loading babel/AtBeginDocument)
% ********************************************************************
\makeatletter
\@ifpackageloaded{babel}%
    {%
       \addto\extrasamerican{%
					\renewcommand*{\figureautorefname}{Figure}%
					\renewcommand*{\tableautorefname}{Table}%
					\renewcommand*{\partautorefname}{Part}%
					\renewcommand*{\chapterautorefname}{Chapter}%
					\renewcommand*{\sectionautorefname}{Section}%
					\renewcommand*{\subsectionautorefname}{Section}%
					\renewcommand*{\subsubsectionautorefname}{Section}% 	
				}%
       \addto\extrasngerman{%
					\renewcommand*{\paragraphautorefname}{Absatz}%
					\renewcommand*{\subparagraphautorefname}{Unterabsatz}%
					\renewcommand*{\footnoteautorefname}{Fu\"snote}%
					\renewcommand*{\FancyVerbLineautorefname}{Zeile}%
					\renewcommand*{\theoremautorefname}{Theorem}%
					\renewcommand*{\appendixautorefname}{Anhang}%
					\renewcommand*{\equationautorefname}{Gleichung}%
					\renewcommand*{\itemautorefname}{Punkt}%
				}%	
			% Fix to getting autorefs for subfigures right (thanks to Belinda Vogt for changing the definition)
			\providecommand{\subfigureautorefname}{\figureautorefname}%  			
    }{\relax}
\makeatother

% https://tex.stackexchange.com/questions/95488/list-of-figures-and-page-numbering
% https://tex.stackexchange.com/a/95616/15
\makeatletter
\newcommand{\emptypage}[1]{%
  \cleardoublepage
  \begingroup
  \let\ps@plain\ps@empty
  \pagestyle{empty}
  #1
  \cleardoublepage}
\makeatletter

% ****************************************************************************************************
% 7. Last calls before the bar closes
% ****************************************************************************************************
% ********************************************************************
% Development Stuff
% ********************************************************************
\listfiles
%\PassOptionsToPackage{l2tabu,orthodox,abort}{nag}
%	\usepackage{nag}
%\PassOptionsToPackage{warning, all}{onlyamsmath}
%	\usepackage{onlyamsmath}

% ********************************************************************
% Last, but not least...
% ********************************************************************
\usepackage[sixbynine]{../latex/classicthesis_mine/classicthesis}
% ****************************************************************************************************


% ****************************************************************************************************
% 8. Further adjustments (experimental)
% ****************************************************************************************************
% ********************************************************************
% Changing the text area
% ********************************************************************
%\linespread{1.05} % a bit more for Palatino
%\areaset[current]{312pt}{761pt} % 686 (factor 2.2) + 33 head + 42 head \the\footskip
%\setlength{\marginparwidth}{7em}%
%\setlength{\marginparsep}{2em}%

% ********************************************************************
% Using different fonts
% ********************************************************************
%\usepackage[oldstylenums]{kpfonts} % oldstyle notextcomp
%\usepackage[osf]{libertine}
%\usepackage{hfoldsty} % Computer Modern with osf
%\usepackage[light,condensed,math]{iwona}
%\renewcommand{\sfdefault}{iwona}
%\usepackage{lmodern} % <-- no osf support :-(
%\usepackage[urw-garamond]{mathdesign} <-- no osf support :-(
% ****************************************************************************************************

%----------------------------------------------------------------------------------------
% peeter's mods from from mylayout.sty
%----------------------------------------------------------------------------------------

% http://www.artofproblemsolving.com/Wiki/index.php/LaTeX:Layout#The_Easy_Way
% replaces manual settings (\parindent, \parskip, \topmargin, ...)
%\usepackage[margin=3.0cm]{geometry}
%\usepackage[left=3cm,right=4cm]{geometry}

\usepackage{parameters}

% this boolean is now crippled by hardcoding 6x9 in:
% 1) config.tex above:
%   \usepackage[sixbynine]{classicthesis} % customized ancient version of classicthesis.sty
% 2) GAelectrodynamic.tex: paper=6in:9in
\newboolean{use-lettersize}
\setboolean{use-lettersize}{false}
%\setboolean{use-lettersize}{true}

\newboolean{print-version}

\ifthenelse{\boolean{use-lettersize}}
{
   % http://www.artofproblemsolving.com/Wiki/index.php/LaTeX:Layout#The_Easy_Way
   % replaces manual settings (\parindent, \parskip, \topmargin, ...)
   %\usepackage[margin=3.0cm]{geometry}
   % default:
   \usepackage[left=3cm,right=4cm]{geometry}
   %\usepackage[left=2cm,right=1.9cm,top=2.1cm,bottom=1.9cm]{geometry}

   \setboolean{print-version}{false}
}
{
   %\usepackage[paperheight=9in,paperwidth=6in,left=2cm,right=1.7cm,top=1.7cm,bottom=2cm]{geometry}
   %\usepackage[paperheight=9in,paperwidth=6in,left=2cm,right=1.9cm,top=1.9cm,bottom=2.2cm]{geometry}

   % first print version: GAelectrodynamics.V0.1.14.6x9.pdf
   % margins on the bottom look a bit big.
   %\usepackage[paperheight=9in,paperwidth=6in,left=2cm,right=2cm,top=1.9cm,bottom=2.2cm]{geometry}
   \usepackage[paperheight=9in,paperwidth=6in,left=2cm,right=2cm,top=1.9cm,bottom=1.5cm]{geometry}

   % this is the size of the Landau and Lifshitz books (as measured with a ruler)
   %\usepackage[paperheight=9.61in,paperwidth=6.69in,left=2cm,right=1.9cm,top=1.9cm,bottom=1.8cm]{geometry}

   %\setboolean{print-version}{true}
   \setboolean{print-version}{false}
}

%\PassOptionsToPackage{answerdelayed}{exercise}
\usepackage{peeters_layout}
\usepackage{thisbook}

\myClassicThesisOverrides

% ****************************************************************************************************
% If you like the classicthesis, then I would appreciate a postcard.
% My address can be found in the file ClassicThesis.pdf. A collection
% of the postcards I received so far is available online at
% http://postcards.miede.de
% ****************************************************************************************************

% ****************************************************************************************************
% 1. Configure classicthesis for your needs here, e.g., remove "drafting" below
% in order to deactivate the time-stamp on the pages
% ****************************************************************************************************
\PassOptionsToPackage{eulerchapternumbers,listings,%
				 pdfspacing,%
				 subfig,beramono,parts}{classicthesis}										
% ********************************************************************
% Available options for classicthesis.sty
% (see ClassicThesis.pdf for more information):
% drafting
% parts nochapters linedheaders
% eulerchapternumbers beramono eulermath pdfspacing minionprospacing
% tocaligned dottedtoc manychapters
% listings floatperchapter subfig
% ********************************************************************

% ********************************************************************
%
% http://tex.stackexchange.com/questions/3676/too-many-math-alphabets-error
%
% adding \usepackage{txfonts} % \ointclockwise
%
% caused too many alphabets error.  se workaround:

\newcommand{\bmmax}{0}
\newcommand{\hmmax}{0}

% ********************************************************************
% https://tex.stackexchange.com/a/348976/15
\BeforeTOCHead[toc]{\cleardoublepage\pdfbookmark{\contentsname}{toc}}
\BeforeTOCHead[lof]{\cleardoublepage\pdfbookmark{\listfigurename}{lof}}

% ********************************************************************
% Triggers for this config
% ********************************************************************
\usepackage{ifthen}
\usepackage{censor}
\newboolean{enable-backrefs} % enable backrefs in the bibliography
\setboolean{enable-backrefs}{true} % true false

%% \newboolean{redacted}
%% \usepackage{redacted}
%%
%% \newcommand{\withproblemsets}[1]{%
%% \ifthenelse{\boolean{redacted}}%
%% {%
%% \textcolor{Maroon}{%
%% PROBLEM SET RELATED MATERIAL REDACTED IN THIS DOCUMENT.%
%% %
%% PLEASE FEEL FREE TO EMAIL ME FOR THE FULL VERSION IF YOU AREN'T TAKING THE COURSE.%
%% }%
%% %
%% \input{censor1pg.tex}%
%% %
%% \textcolor{Maroon}{%
%% END-REDACTION%
%% }%
%% }%
%% {#1}%
%% }
%%
%% \newcommand{\withproblemsetsParagraph}[1]{%
%% \ifthenelse{\boolean{redacted}}%
%% {%
%% \textcolor{Maroon}{%
%% PROBLEM SET RELATED MATERIAL REDACTED IN THIS DOCUMENT.%
%% %
%% PLEASE FEEL FREE TO EMAIL ME FOR THE FULL VERSION IF YOU AREN'T TAKING THE COURSE.%
%% }%
%% %
%% \input{censor1p.tex}%
%% %\input{censor1p.tex}%
%% %\input{censor1p.tex}%
%% %
%% \textcolor{Maroon}{%
%% END-REDACTION%
%% }%
%% }%
%% {#1}%
%% }
%%
%% \newcommand{\withproblemsetsMessage}[1]{%
%% \ifthenelse{\boolean{redacted}}%
%% {%
%% #1%
%% }%
%% {\relax}%
%% }

% ****************************************************************************************************


% ****************************************************************************************************


% ****************************************************************************************************
% 2. Personal data and user ad-hoc commands
% ****************************************************************************************************
\newcommand{\myTitle}{Relativistic Electrodynamics\xspace}
\newcommand{\mySubtitle}{Notes and problems from UofT PHY450H1S 2011\xspace}
\newcommand{\myDegree}{}
\newcommand{\myName}{Peeter Joot\xspace}
%\newcommand{\myName}{Peeter Joot\ peeterjoot@protonmail.com\xspace}
\newcommand{\myProf}{}
\newcommand{\myOtherProf}{}
\newcommand{\mySupervisor}{}
\newcommand{\myFaculty}{}
\newcommand{\myDepartment}{}
\newcommand{\myUni}{}
\newcommand{\myLocation}{}
%----------------------
   %\newcommand{\myTime}{June 2012}
   \input{./.revinfo/gitCommitDateAsMyTime.tex}
%----------------------
% v.9 includes index.
%\newcommand{\myVersion}{version v.9\xspace}
% now posted to:
% https://peeterjoot.com/archives/math2011/phy450.pdf

% ********************************************************************
% Setup, finetuning, and useful commands
% ********************************************************************
\newcounter{dummy} % necessary for correct hyperlinks (to index, bib, etc.)
\newlength{\abcd} % for ab..z string length calculation
\providecommand{\mLyX}{L\kern-.1667em\lower.25em\hbox{Y}\kern-.125emX\@}
%\newcommand{\ie}{i.\,e.}
%\newcommand{\Ie}{I.\,e.}
%\newcommand{\eg}{e.\,g.}
%\newcommand{\Eg}{E.\,g.}
% ****************************************************************************************************


% ****************************************************************************************************
% 3. Loading some handy packages
% ****************************************************************************************************
% ********************************************************************
% Packages with options that might require adjustments
% ********************************************************************
\PassOptionsToPackage{latin9}{inputenc}	% latin9 (ISO-8859-9) = latin1+"Euro sign"
 \usepackage{inputenc}				

%\PassOptionsToPackage{ngerman,american}{babel}   % change this to your language(s)
% Spanish languages need extra options in order to work with this template
%\PassOptionsToPackage{spanish,es-lcroman}{babel}
 \usepackage{babel}					

\PassOptionsToPackage{square,numbers}{natbib}
 \usepackage{natbib}				

\PassOptionsToPackage{fleqn}{amsmath}		% math environments and more by the AMS
 \usepackage{amsmath}

% ********************************************************************
% General useful packages
% ********************************************************************
\PassOptionsToPackage{T1}{fontenc} % T2A for cyrillics
	\usepackage{fontenc}
\usepackage{xspace} % to get the spacing after macros right
\usepackage{mparhack} % get marginpar right
\usepackage{fixltx2e} % fixes some LaTeX stuff
\PassOptionsToPackage{printonlyused,smaller}{acronym}
	\usepackage{acronym} % nice macros for handling all acronyms in the thesis
%\renewcommand*{\acsfont}[1]{\textssc{#1}} % for MinionPro
%\renewcommand{\bflabel}[1]{{#1}\hfill} % fix the list of acronyms
% ****************************************************************************************************


% ****************************************************************************************************
% 4. Setup floats: tables, (sub)figures, and captions
% ****************************************************************************************************
\usepackage{tabularx} % better tables
	\setlength{\extrarowheight}{3pt} % increase table row height
\newcommand{\tableheadline}[1]{\multicolumn{1}{c}{\spacedlowsmallcaps{#1}}}
\newcommand{\myfloatalign}{\centering} % to be used with each float for alignment
\usepackage{caption}
\captionsetup{format=hang,font=small}
\usepackage{subfig}
% ****************************************************************************************************


% ****************************************************************************************************
% 5. Setup code listings
% ****************************************************************************************************
\usepackage{listings}
\let\origlstlisting=\lstlisting
\let\endoriglstlisting=\endlstlisting
\renewenvironment{lstlisting}
    {\mathcode`\-=\hyphenmathcode
     \everymath{}\mathsurround=0pt\origlstlisting}
    {\endoriglstlisting}
%\lstset{emph={trueIndex,root},emphstyle=\color{BlueViolet}}%\underbar} % for special keywords
\lstset{language=[LaTeX]Tex,%C++,
    keywordstyle=\color{RoyalBlue},%\bfseries,
    basicstyle=\small\ttfamily,
    %identifierstyle=\color{NavyBlue},
    commentstyle=\color{Green}\ttfamily,
    stringstyle=\rmfamily,
    numbers=none,%left,%
    numberstyle=\scriptsize,%\tiny
    stepnumber=5,
    numbersep=8pt,
    showstringspaces=false,
    breaklines=true,
    frameround=ftff,
    frame=single,
    belowcaptionskip=.75\baselineskip
    %frame=L
}
% ****************************************************************************************************    		


% ****************************************************************************************************
% 6. PDFLaTeX, hyperreferences and citation backreferences
% ****************************************************************************************************
% ********************************************************************
% Using PDFLaTeX
% ********************************************************************
\PassOptionsToPackage{pdftex,hyperfootnotes=false,pdfpagelabels,hyperindex=true}{hyperref}
	\usepackage{hyperref}  % backref linktocpage pagebackref
%----------------------
\pdfcompresslevel=9
% http://tex.stackexchange.com/questions/52317/pdftex-warning-version-allowed
%\pdfminorversion=5
\pdfminorversion=7
\pdfobjcompresslevel=3
%----------------------
\pdfadjustspacing=1
\PassOptionsToPackage{pdftex}{graphicx}
	\usepackage{graphicx}

% ********************************************************************
% Setup the style of the backrefs from the bibliography
% (translate the options to any language you use)
% ********************************************************************
\newcommand{\backrefnotcitedstring}{\relax}%(Not cited.)
\newcommand{\backrefcitedsinglestring}[1]{(Cited on page~#1.)}
\newcommand{\backrefcitedmultistring}[1]{(Cited on pages~#1.)}
\ifthenelse{\boolean{enable-backrefs}}%
{%
		\PassOptionsToPackage{hyperpageref}{backref}
		\usepackage{backref} % to be loaded after hyperref package
		   \renewcommand{\backreftwosep}{ and~} % separate 2 pages
		   \renewcommand{\backreflastsep}{, and~} % separate last of longer list
		   \renewcommand*{\backref}[1]{}  % disable standard
		   \renewcommand*{\backrefalt}[4]{% detailed backref
		      \ifcase #1 %
		         \backrefnotcitedstring%
		      \or%
		         \backrefcitedsinglestring{#2}%
		      \else%
		         \backrefcitedmultistring{#2}%
		      \fi}%
}{\relax}

% ********************************************************************
% Hyperreferences
% ********************************************************************
\hypersetup{%
    %draft,	% = no hyperlinking at all (useful in b/w printouts)
    colorlinks=true, linktocpage=true, pdfstartpage=3, pdfstartview=FitV,%
    % uncomment the following line if you want to have black links (e.g., for printing)
    %colorlinks=false, linktocpage=false, pdfborder={0 0 0}, pdfstartpage=3, pdfstartview=FitV,%
    breaklinks=true, pdfpagemode=UseNone, pageanchor=true, pdfpagemode=UseOutlines,%
    plainpages=false, bookmarksnumbered, bookmarksopen=true, bookmarksopenlevel=1,%
    hypertexnames=true, pdfhighlight=/O,%nesting=true,%frenchlinks,%
    urlcolor=webbrown, linkcolor=RoyalBlue, citecolor=webgreen, %pagecolor=RoyalBlue,%
    %urlcolor=Black, linkcolor=Black, citecolor=Black, %pagecolor=Black,%
    pdftitle={\myTitle},%
    pdfauthor={\textcopyright\ \myName},%
    pdfsubject={Relativistic Electrodynamics},%
    pdfkeywords={Relativistic Electrodynamics} {PHY450H1S} {University of Toronto},%
    pdfcreator={pdfLaTeX},%
    pdfproducer={LaTeX with hyperref and classicthesis}%
}

% ********************************************************************
% Setup autoreferences
% ********************************************************************
% There are some issues regarding autorefnames
% http://www.ureader.de/msg/136221647.aspx
% http://www.tex.ac.uk/cgi-bin/texfaq2html?label=latexwords
% you have to redefine the makros for the
% language you use, e.g., american, ngerman
% (as chosen when loading babel/AtBeginDocument)
% ********************************************************************
\makeatletter
\@ifpackageloaded{babel}%
    {%
       \addto\extrasamerican{%
					\renewcommand*{\figureautorefname}{Figure}%
					\renewcommand*{\tableautorefname}{Table}%
					\renewcommand*{\partautorefname}{Part}%
					\renewcommand*{\chapterautorefname}{Chapter}%
					\renewcommand*{\sectionautorefname}{Section}%
					\renewcommand*{\subsectionautorefname}{Section}%
					\renewcommand*{\subsubsectionautorefname}{Section}% 	
				}%
       \addto\extrasngerman{%
					\renewcommand*{\paragraphautorefname}{Absatz}%
					\renewcommand*{\subparagraphautorefname}{Unterabsatz}%
					\renewcommand*{\footnoteautorefname}{Fu\"snote}%
					\renewcommand*{\FancyVerbLineautorefname}{Zeile}%
					\renewcommand*{\theoremautorefname}{Theorem}%
					\renewcommand*{\appendixautorefname}{Anhang}%
					\renewcommand*{\equationautorefname}{Gleichung}%
					\renewcommand*{\itemautorefname}{Punkt}%
				}%	
			% Fix to getting autorefs for subfigures right (thanks to Belinda Vogt for changing the definition)
			\providecommand{\subfigureautorefname}{\figureautorefname}%  			
    }{\relax}
\makeatother

% https://tex.stackexchange.com/questions/95488/list-of-figures-and-page-numbering
% https://tex.stackexchange.com/a/95616/15
\makeatletter
\newcommand{\emptypage}[1]{%
  \cleardoublepage
  \begingroup
  \let\ps@plain\ps@empty
  \pagestyle{empty}
  #1
  \cleardoublepage}
\makeatletter

% ****************************************************************************************************
% 7. Last calls before the bar closes
% ****************************************************************************************************
% ********************************************************************
% Development Stuff
% ********************************************************************
\listfiles
%\PassOptionsToPackage{l2tabu,orthodox,abort}{nag}
%	\usepackage{nag}
%\PassOptionsToPackage{warning, all}{onlyamsmath}
%	\usepackage{onlyamsmath}

% ********************************************************************
% Last, but not least...
% ********************************************************************
\usepackage[sixbynine]{../latex/classicthesis_mine/classicthesis}
% ****************************************************************************************************


% ****************************************************************************************************
% 8. Further adjustments (experimental)
% ****************************************************************************************************
% ********************************************************************
% Changing the text area
% ********************************************************************
%\linespread{1.05} % a bit more for Palatino
%\areaset[current]{312pt}{761pt} % 686 (factor 2.2) + 33 head + 42 head \the\footskip
%\setlength{\marginparwidth}{7em}%
%\setlength{\marginparsep}{2em}%

% ********************************************************************
% Using different fonts
% ********************************************************************
%\usepackage[oldstylenums]{kpfonts} % oldstyle notextcomp
%\usepackage[osf]{libertine}
%\usepackage{hfoldsty} % Computer Modern with osf
%\usepackage[light,condensed,math]{iwona}
%\renewcommand{\sfdefault}{iwona}
%\usepackage{lmodern} % <-- no osf support :-(
%\usepackage[urw-garamond]{mathdesign} <-- no osf support :-(
% ****************************************************************************************************

%----------------------------------------------------------------------------------------
% peeter's mods from from mylayout.sty
%----------------------------------------------------------------------------------------

% http://www.artofproblemsolving.com/Wiki/index.php/LaTeX:Layout#The_Easy_Way
% replaces manual settings (\parindent, \parskip, \topmargin, ...)
%\usepackage[margin=3.0cm]{geometry}
%\usepackage[left=3cm,right=4cm]{geometry}

\usepackage{parameters}

% this boolean is now crippled by hardcoding 6x9 in:
% 1) config.tex above:
%   \usepackage[sixbynine]{classicthesis} % customized ancient version of classicthesis.sty
% 2) GAelectrodynamic.tex: paper=6in:9in
\newboolean{use-lettersize}
\setboolean{use-lettersize}{false}
%\setboolean{use-lettersize}{true}

\newboolean{print-version}

\ifthenelse{\boolean{use-lettersize}}
{
   % http://www.artofproblemsolving.com/Wiki/index.php/LaTeX:Layout#The_Easy_Way
   % replaces manual settings (\parindent, \parskip, \topmargin, ...)
   %\usepackage[margin=3.0cm]{geometry}
   % default:
   \usepackage[left=3cm,right=4cm]{geometry}
   %\usepackage[left=2cm,right=1.9cm,top=2.1cm,bottom=1.9cm]{geometry}

   \setboolean{print-version}{false}
}
{
   %\usepackage[paperheight=9in,paperwidth=6in,left=2cm,right=1.7cm,top=1.7cm,bottom=2cm]{geometry}
   %\usepackage[paperheight=9in,paperwidth=6in,left=2cm,right=1.9cm,top=1.9cm,bottom=2.2cm]{geometry}

   % first print version: GAelectrodynamics.V0.1.14.6x9.pdf
   % margins on the bottom look a bit big.
   %\usepackage[paperheight=9in,paperwidth=6in,left=2cm,right=2cm,top=1.9cm,bottom=2.2cm]{geometry}
   \usepackage[paperheight=9in,paperwidth=6in,left=2cm,right=2cm,top=1.9cm,bottom=1.5cm]{geometry}

   % this is the size of the Landau and Lifshitz books (as measured with a ruler)
   %\usepackage[paperheight=9.61in,paperwidth=6.69in,left=2cm,right=1.9cm,top=1.9cm,bottom=1.8cm]{geometry}

   %\setboolean{print-version}{true}
   \setboolean{print-version}{false}
}

%\PassOptionsToPackage{answerdelayed}{exercise}
\usepackage{peeters_layout}
\usepackage{thisbook}

\myClassicThesisOverrides

% ****************************************************************************************************
% If you like the classicthesis, then I would appreciate a postcard.
% My address can be found in the file ClassicThesis.pdf. A collection
% of the postcards I received so far is available online at
% http://postcards.miede.de
% ****************************************************************************************************

% ****************************************************************************************************
% 1. Configure classicthesis for your needs here, e.g., remove "drafting" below
% in order to deactivate the time-stamp on the pages
% ****************************************************************************************************
\PassOptionsToPackage{eulerchapternumbers,listings,%
				 pdfspacing,%
				 subfig,beramono,parts}{classicthesis}										
% ********************************************************************
% Available options for classicthesis.sty
% (see ClassicThesis.pdf for more information):
% drafting
% parts nochapters linedheaders
% eulerchapternumbers beramono eulermath pdfspacing minionprospacing
% tocaligned dottedtoc manychapters
% listings floatperchapter subfig
% ********************************************************************

% ********************************************************************
%
% http://tex.stackexchange.com/questions/3676/too-many-math-alphabets-error
%
% adding \usepackage{txfonts} % \ointclockwise
%
% caused too many alphabets error.  se workaround:

\newcommand{\bmmax}{0}
\newcommand{\hmmax}{0}

% ********************************************************************
% https://tex.stackexchange.com/a/348976/15
\BeforeTOCHead[toc]{\cleardoublepage\pdfbookmark{\contentsname}{toc}}
\BeforeTOCHead[lof]{\cleardoublepage\pdfbookmark{\listfigurename}{lof}}

% ********************************************************************
% Triggers for this config
% ********************************************************************
\usepackage{ifthen}
\usepackage{censor}
\newboolean{enable-backrefs} % enable backrefs in the bibliography
\setboolean{enable-backrefs}{true} % true false

%% \newboolean{redacted}
%% \usepackage{redacted}
%%
%% \newcommand{\withproblemsets}[1]{%
%% \ifthenelse{\boolean{redacted}}%
%% {%
%% \textcolor{Maroon}{%
%% PROBLEM SET RELATED MATERIAL REDACTED IN THIS DOCUMENT.%
%% %
%% PLEASE FEEL FREE TO EMAIL ME FOR THE FULL VERSION IF YOU AREN'T TAKING THE COURSE.%
%% }%
%% %
%% \input{censor1pg.tex}%
%% %
%% \textcolor{Maroon}{%
%% END-REDACTION%
%% }%
%% }%
%% {#1}%
%% }
%%
%% \newcommand{\withproblemsetsParagraph}[1]{%
%% \ifthenelse{\boolean{redacted}}%
%% {%
%% \textcolor{Maroon}{%
%% PROBLEM SET RELATED MATERIAL REDACTED IN THIS DOCUMENT.%
%% %
%% PLEASE FEEL FREE TO EMAIL ME FOR THE FULL VERSION IF YOU AREN'T TAKING THE COURSE.%
%% }%
%% %
%% \input{censor1p.tex}%
%% %\input{censor1p.tex}%
%% %\input{censor1p.tex}%
%% %
%% \textcolor{Maroon}{%
%% END-REDACTION%
%% }%
%% }%
%% {#1}%
%% }
%%
%% \newcommand{\withproblemsetsMessage}[1]{%
%% \ifthenelse{\boolean{redacted}}%
%% {%
%% #1%
%% }%
%% {\relax}%
%% }

% ****************************************************************************************************


% ****************************************************************************************************


% ****************************************************************************************************
% 2. Personal data and user ad-hoc commands
% ****************************************************************************************************
\newcommand{\myTitle}{Relativistic Electrodynamics\xspace}
\newcommand{\mySubtitle}{Notes and problems from UofT PHY450H1S 2011\xspace}
\newcommand{\myDegree}{}
\newcommand{\myName}{Peeter Joot\xspace}
%\newcommand{\myName}{Peeter Joot\ peeterjoot@protonmail.com\xspace}
\newcommand{\myProf}{}
\newcommand{\myOtherProf}{}
\newcommand{\mySupervisor}{}
\newcommand{\myFaculty}{}
\newcommand{\myDepartment}{}
\newcommand{\myUni}{}
\newcommand{\myLocation}{}
%----------------------
   %\newcommand{\myTime}{June 2012}
   \input{./.revinfo/gitCommitDateAsMyTime.tex}
%----------------------
% v.9 includes index.
%\newcommand{\myVersion}{version v.9\xspace}
% now posted to:
% https://peeterjoot.com/archives/math2011/phy450.pdf

% ********************************************************************
% Setup, finetuning, and useful commands
% ********************************************************************
\newcounter{dummy} % necessary for correct hyperlinks (to index, bib, etc.)
\newlength{\abcd} % for ab..z string length calculation
\providecommand{\mLyX}{L\kern-.1667em\lower.25em\hbox{Y}\kern-.125emX\@}
%\newcommand{\ie}{i.\,e.}
%\newcommand{\Ie}{I.\,e.}
%\newcommand{\eg}{e.\,g.}
%\newcommand{\Eg}{E.\,g.}
% ****************************************************************************************************


% ****************************************************************************************************
% 3. Loading some handy packages
% ****************************************************************************************************
% ********************************************************************
% Packages with options that might require adjustments
% ********************************************************************
\PassOptionsToPackage{latin9}{inputenc}	% latin9 (ISO-8859-9) = latin1+"Euro sign"
 \usepackage{inputenc}				

%\PassOptionsToPackage{ngerman,american}{babel}   % change this to your language(s)
% Spanish languages need extra options in order to work with this template
%\PassOptionsToPackage{spanish,es-lcroman}{babel}
 \usepackage{babel}					

\PassOptionsToPackage{square,numbers}{natbib}
 \usepackage{natbib}				

\PassOptionsToPackage{fleqn}{amsmath}		% math environments and more by the AMS
 \usepackage{amsmath}

% ********************************************************************
% General useful packages
% ********************************************************************
\PassOptionsToPackage{T1}{fontenc} % T2A for cyrillics
	\usepackage{fontenc}
\usepackage{xspace} % to get the spacing after macros right
\usepackage{mparhack} % get marginpar right
\usepackage{fixltx2e} % fixes some LaTeX stuff
\PassOptionsToPackage{printonlyused,smaller}{acronym}
	\usepackage{acronym} % nice macros for handling all acronyms in the thesis
%\renewcommand*{\acsfont}[1]{\textssc{#1}} % for MinionPro
%\renewcommand{\bflabel}[1]{{#1}\hfill} % fix the list of acronyms
% ****************************************************************************************************


% ****************************************************************************************************
% 4. Setup floats: tables, (sub)figures, and captions
% ****************************************************************************************************
\usepackage{tabularx} % better tables
	\setlength{\extrarowheight}{3pt} % increase table row height
\newcommand{\tableheadline}[1]{\multicolumn{1}{c}{\spacedlowsmallcaps{#1}}}
\newcommand{\myfloatalign}{\centering} % to be used with each float for alignment
\usepackage{caption}
\captionsetup{format=hang,font=small}
\usepackage{subfig}
% ****************************************************************************************************


% ****************************************************************************************************
% 5. Setup code listings
% ****************************************************************************************************
\usepackage{listings}
\let\origlstlisting=\lstlisting
\let\endoriglstlisting=\endlstlisting
\renewenvironment{lstlisting}
    {\mathcode`\-=\hyphenmathcode
     \everymath{}\mathsurround=0pt\origlstlisting}
    {\endoriglstlisting}
%\lstset{emph={trueIndex,root},emphstyle=\color{BlueViolet}}%\underbar} % for special keywords
\lstset{language=[LaTeX]Tex,%C++,
    keywordstyle=\color{RoyalBlue},%\bfseries,
    basicstyle=\small\ttfamily,
    %identifierstyle=\color{NavyBlue},
    commentstyle=\color{Green}\ttfamily,
    stringstyle=\rmfamily,
    numbers=none,%left,%
    numberstyle=\scriptsize,%\tiny
    stepnumber=5,
    numbersep=8pt,
    showstringspaces=false,
    breaklines=true,
    frameround=ftff,
    frame=single,
    belowcaptionskip=.75\baselineskip
    %frame=L
}
% ****************************************************************************************************    		


% ****************************************************************************************************
% 6. PDFLaTeX, hyperreferences and citation backreferences
% ****************************************************************************************************
% ********************************************************************
% Using PDFLaTeX
% ********************************************************************
\PassOptionsToPackage{pdftex,hyperfootnotes=false,pdfpagelabels,hyperindex=true}{hyperref}
	\usepackage{hyperref}  % backref linktocpage pagebackref
%----------------------
\pdfcompresslevel=9
% http://tex.stackexchange.com/questions/52317/pdftex-warning-version-allowed
%\pdfminorversion=5
\pdfminorversion=7
\pdfobjcompresslevel=3
%----------------------
\pdfadjustspacing=1
\PassOptionsToPackage{pdftex}{graphicx}
	\usepackage{graphicx}

% ********************************************************************
% Setup the style of the backrefs from the bibliography
% (translate the options to any language you use)
% ********************************************************************
\newcommand{\backrefnotcitedstring}{\relax}%(Not cited.)
\newcommand{\backrefcitedsinglestring}[1]{(Cited on page~#1.)}
\newcommand{\backrefcitedmultistring}[1]{(Cited on pages~#1.)}
\ifthenelse{\boolean{enable-backrefs}}%
{%
		\PassOptionsToPackage{hyperpageref}{backref}
		\usepackage{backref} % to be loaded after hyperref package
		   \renewcommand{\backreftwosep}{ and~} % separate 2 pages
		   \renewcommand{\backreflastsep}{, and~} % separate last of longer list
		   \renewcommand*{\backref}[1]{}  % disable standard
		   \renewcommand*{\backrefalt}[4]{% detailed backref
		      \ifcase #1 %
		         \backrefnotcitedstring%
		      \or%
		         \backrefcitedsinglestring{#2}%
		      \else%
		         \backrefcitedmultistring{#2}%
		      \fi}%
}{\relax}

% ********************************************************************
% Hyperreferences
% ********************************************************************
\hypersetup{%
    %draft,	% = no hyperlinking at all (useful in b/w printouts)
    colorlinks=true, linktocpage=true, pdfstartpage=3, pdfstartview=FitV,%
    % uncomment the following line if you want to have black links (e.g., for printing)
    %colorlinks=false, linktocpage=false, pdfborder={0 0 0}, pdfstartpage=3, pdfstartview=FitV,%
    breaklinks=true, pdfpagemode=UseNone, pageanchor=true, pdfpagemode=UseOutlines,%
    plainpages=false, bookmarksnumbered, bookmarksopen=true, bookmarksopenlevel=1,%
    hypertexnames=true, pdfhighlight=/O,%nesting=true,%frenchlinks,%
    urlcolor=webbrown, linkcolor=RoyalBlue, citecolor=webgreen, %pagecolor=RoyalBlue,%
    %urlcolor=Black, linkcolor=Black, citecolor=Black, %pagecolor=Black,%
    pdftitle={\myTitle},%
    pdfauthor={\textcopyright\ \myName},%
    pdfsubject={Relativistic Electrodynamics},%
    pdfkeywords={Relativistic Electrodynamics} {PHY450H1S} {University of Toronto},%
    pdfcreator={pdfLaTeX},%
    pdfproducer={LaTeX with hyperref and classicthesis}%
}

% ********************************************************************
% Setup autoreferences
% ********************************************************************
% There are some issues regarding autorefnames
% http://www.ureader.de/msg/136221647.aspx
% http://www.tex.ac.uk/cgi-bin/texfaq2html?label=latexwords
% you have to redefine the makros for the
% language you use, e.g., american, ngerman
% (as chosen when loading babel/AtBeginDocument)
% ********************************************************************
\makeatletter
\@ifpackageloaded{babel}%
    {%
       \addto\extrasamerican{%
					\renewcommand*{\figureautorefname}{Figure}%
					\renewcommand*{\tableautorefname}{Table}%
					\renewcommand*{\partautorefname}{Part}%
					\renewcommand*{\chapterautorefname}{Chapter}%
					\renewcommand*{\sectionautorefname}{Section}%
					\renewcommand*{\subsectionautorefname}{Section}%
					\renewcommand*{\subsubsectionautorefname}{Section}% 	
				}%
       \addto\extrasngerman{%
					\renewcommand*{\paragraphautorefname}{Absatz}%
					\renewcommand*{\subparagraphautorefname}{Unterabsatz}%
					\renewcommand*{\footnoteautorefname}{Fu\"snote}%
					\renewcommand*{\FancyVerbLineautorefname}{Zeile}%
					\renewcommand*{\theoremautorefname}{Theorem}%
					\renewcommand*{\appendixautorefname}{Anhang}%
					\renewcommand*{\equationautorefname}{Gleichung}%
					\renewcommand*{\itemautorefname}{Punkt}%
				}%	
			% Fix to getting autorefs for subfigures right (thanks to Belinda Vogt for changing the definition)
			\providecommand{\subfigureautorefname}{\figureautorefname}%  			
    }{\relax}
\makeatother

% https://tex.stackexchange.com/questions/95488/list-of-figures-and-page-numbering
% https://tex.stackexchange.com/a/95616/15
\makeatletter
\newcommand{\emptypage}[1]{%
  \cleardoublepage
  \begingroup
  \let\ps@plain\ps@empty
  \pagestyle{empty}
  #1
  \cleardoublepage}
\makeatletter

% ****************************************************************************************************
% 7. Last calls before the bar closes
% ****************************************************************************************************
% ********************************************************************
% Development Stuff
% ********************************************************************
\listfiles
%\PassOptionsToPackage{l2tabu,orthodox,abort}{nag}
%	\usepackage{nag}
%\PassOptionsToPackage{warning, all}{onlyamsmath}
%	\usepackage{onlyamsmath}

% ********************************************************************
% Last, but not least...
% ********************************************************************
\usepackage[sixbynine]{../latex/classicthesis_mine/classicthesis}
% ****************************************************************************************************


% ****************************************************************************************************
% 8. Further adjustments (experimental)
% ****************************************************************************************************
% ********************************************************************
% Changing the text area
% ********************************************************************
%\linespread{1.05} % a bit more for Palatino
%\areaset[current]{312pt}{761pt} % 686 (factor 2.2) + 33 head + 42 head \the\footskip
%\setlength{\marginparwidth}{7em}%
%\setlength{\marginparsep}{2em}%

% ********************************************************************
% Using different fonts
% ********************************************************************
%\usepackage[oldstylenums]{kpfonts} % oldstyle notextcomp
%\usepackage[osf]{libertine}
%\usepackage{hfoldsty} % Computer Modern with osf
%\usepackage[light,condensed,math]{iwona}
%\renewcommand{\sfdefault}{iwona}
%\usepackage{lmodern} % <-- no osf support :-(
%\usepackage[urw-garamond]{mathdesign} <-- no osf support :-(
% ****************************************************************************************************

%----------------------------------------------------------------------------------------
% peeter's mods from from mylayout.sty
%----------------------------------------------------------------------------------------

% http://www.artofproblemsolving.com/Wiki/index.php/LaTeX:Layout#The_Easy_Way
% replaces manual settings (\parindent, \parskip, \topmargin, ...)
%\usepackage[margin=3.0cm]{geometry}
%\usepackage[left=3cm,right=4cm]{geometry}

\usepackage{parameters}

% this boolean is now crippled by hardcoding 6x9 in:
% 1) config.tex above:
%   \usepackage[sixbynine]{classicthesis} % customized ancient version of classicthesis.sty
% 2) GAelectrodynamic.tex: paper=6in:9in
\newboolean{use-lettersize}
\setboolean{use-lettersize}{false}
%\setboolean{use-lettersize}{true}

\newboolean{print-version}

\ifthenelse{\boolean{use-lettersize}}
{
   % http://www.artofproblemsolving.com/Wiki/index.php/LaTeX:Layout#The_Easy_Way
   % replaces manual settings (\parindent, \parskip, \topmargin, ...)
   %\usepackage[margin=3.0cm]{geometry}
   % default:
   \usepackage[left=3cm,right=4cm]{geometry}
   %\usepackage[left=2cm,right=1.9cm,top=2.1cm,bottom=1.9cm]{geometry}

   \setboolean{print-version}{false}
}
{
   %\usepackage[paperheight=9in,paperwidth=6in,left=2cm,right=1.7cm,top=1.7cm,bottom=2cm]{geometry}
   %\usepackage[paperheight=9in,paperwidth=6in,left=2cm,right=1.9cm,top=1.9cm,bottom=2.2cm]{geometry}

   % first print version: GAelectrodynamics.V0.1.14.6x9.pdf
   % margins on the bottom look a bit big.
   %\usepackage[paperheight=9in,paperwidth=6in,left=2cm,right=2cm,top=1.9cm,bottom=2.2cm]{geometry}
   \usepackage[paperheight=9in,paperwidth=6in,left=2cm,right=2cm,top=1.9cm,bottom=1.5cm]{geometry}

   % this is the size of the Landau and Lifshitz books (as measured with a ruler)
   %\usepackage[paperheight=9.61in,paperwidth=6.69in,left=2cm,right=1.9cm,top=1.9cm,bottom=1.8cm]{geometry}

   %\setboolean{print-version}{true}
   \setboolean{print-version}{false}
}

%\PassOptionsToPackage{answerdelayed}{exercise}
\usepackage{peeters_layout}
\usepackage{thisbook}

\myClassicThesisOverrides

% ****************************************************************************************************
% If you like the classicthesis, then I would appreciate a postcard.
% My address can be found in the file ClassicThesis.pdf. A collection
% of the postcards I received so far is available online at
% http://postcards.miede.de
% ****************************************************************************************************

% ****************************************************************************************************
% 1. Configure classicthesis for your needs here, e.g., remove "drafting" below
% in order to deactivate the time-stamp on the pages
% ****************************************************************************************************
\PassOptionsToPackage{eulerchapternumbers,listings,%
				 pdfspacing,%
				 subfig,beramono,parts}{classicthesis}										
% ********************************************************************
% Available options for classicthesis.sty
% (see ClassicThesis.pdf for more information):
% drafting
% parts nochapters linedheaders
% eulerchapternumbers beramono eulermath pdfspacing minionprospacing
% tocaligned dottedtoc manychapters
% listings floatperchapter subfig
% ********************************************************************

% ********************************************************************
%
% http://tex.stackexchange.com/questions/3676/too-many-math-alphabets-error
%
% adding \usepackage{txfonts} % \ointclockwise
%
% caused too many alphabets error.  se workaround:

\newcommand{\bmmax}{0}
\newcommand{\hmmax}{0}

% ********************************************************************
% Triggers for this config
% ********************************************************************
\usepackage{ifthen}
\usepackage{censor}
\newboolean{enable-backrefs} % enable backrefs in the bibliography
\setboolean{enable-backrefs}{true} % true false

%% \newboolean{redacted}
%% \usepackage{redacted}
%%
%% \newcommand{\withproblemsets}[1]{%
%% \ifthenelse{\boolean{redacted}}%
%% {%
%% \textcolor{Maroon}{%
%% PROBLEM SET RELATED MATERIAL REDACTED IN THIS DOCUMENT.%
%% %
%% PLEASE FEEL FREE TO EMAIL ME FOR THE FULL VERSION IF YOU AREN'T TAKING THE COURSE.%
%% }%
%% %
%% \input{censor1pg.tex}%
%% %
%% \textcolor{Maroon}{%
%% END-REDACTION%
%% }%
%% }%
%% {#1}%
%% }
%%
%% \newcommand{\withproblemsetsParagraph}[1]{%
%% \ifthenelse{\boolean{redacted}}%
%% {%
%% \textcolor{Maroon}{%
%% PROBLEM SET RELATED MATERIAL REDACTED IN THIS DOCUMENT.%
%% %
%% PLEASE FEEL FREE TO EMAIL ME FOR THE FULL VERSION IF YOU AREN'T TAKING THE COURSE.%
%% }%
%% %
%% \input{censor1p.tex}%
%% %\input{censor1p.tex}%
%% %\input{censor1p.tex}%
%% %
%% \textcolor{Maroon}{%
%% END-REDACTION%
%% }%
%% }%
%% {#1}%
%% }
%%
%% \newcommand{\withproblemsetsMessage}[1]{%
%% \ifthenelse{\boolean{redacted}}%
%% {%
%% #1%
%% }%
%% {\relax}%
%% }

% ****************************************************************************************************


% ****************************************************************************************************


% ****************************************************************************************************
% 2. Personal data and user ad-hoc commands
% ****************************************************************************************************
\newcommand{\myTitle}{Relativistic Electrodynamics\xspace}
\newcommand{\mySubtitle}{Notes and problems from UofT PHY450H1S 2011\xspace}
\newcommand{\myDegree}{}
\newcommand{\myName}{Peeter Joot\xspace}
%\newcommand{\myName}{Peeter Joot\ peeterjoot@protonmail.com\xspace}
\newcommand{\myProf}{}
\newcommand{\myOtherProf}{}
\newcommand{\mySupervisor}{}
\newcommand{\myFaculty}{}
\newcommand{\myDepartment}{}
\newcommand{\myUni}{}
\newcommand{\myLocation}{}
%----------------------
   %\newcommand{\myTime}{June 2012}
   \input{./.revinfo/gitCommitDateAsMyTime.tex}
%----------------------
% v.9 includes index.
%\newcommand{\myVersion}{version v.9\xspace}
% now posted to:
% http://peeterjoot.com/archives/math2011/phy450.pdf

% ********************************************************************
% Setup, finetuning, and useful commands
% ********************************************************************
\newcounter{dummy} % necessary for correct hyperlinks (to index, bib, etc.)
\newlength{\abcd} % for ab..z string length calculation
\providecommand{\mLyX}{L\kern-.1667em\lower.25em\hbox{Y}\kern-.125emX\@}
%\newcommand{\ie}{i.\,e.}
%\newcommand{\Ie}{I.\,e.}
%\newcommand{\eg}{e.\,g.}
%\newcommand{\Eg}{E.\,g.}
% ****************************************************************************************************


% ****************************************************************************************************
% 3. Loading some handy packages
% ****************************************************************************************************
% ********************************************************************
% Packages with options that might require adjustments
% ********************************************************************
\PassOptionsToPackage{latin9}{inputenc}	% latin9 (ISO-8859-9) = latin1+"Euro sign"
 \usepackage{inputenc}				

%\PassOptionsToPackage{ngerman,american}{babel}   % change this to your language(s)
% Spanish languages need extra options in order to work with this template
%\PassOptionsToPackage{spanish,es-lcroman}{babel}
 \usepackage{babel}					

\PassOptionsToPackage{square,numbers}{natbib}
 \usepackage{natbib}				

\PassOptionsToPackage{fleqn}{amsmath}		% math environments and more by the AMS
 \usepackage{amsmath}

% ********************************************************************
% General useful packages
% ********************************************************************
\PassOptionsToPackage{T1}{fontenc} % T2A for cyrillics
	\usepackage{fontenc}
\usepackage{xspace} % to get the spacing after macros right
\usepackage{mparhack} % get marginpar right
\usepackage{fixltx2e} % fixes some LaTeX stuff
\PassOptionsToPackage{printonlyused,smaller}{acronym}
	\usepackage{acronym} % nice macros for handling all acronyms in the thesis
%\renewcommand*{\acsfont}[1]{\textssc{#1}} % for MinionPro
%\renewcommand{\bflabel}[1]{{#1}\hfill} % fix the list of acronyms
% ****************************************************************************************************


% ****************************************************************************************************
% 4. Setup floats: tables, (sub)figures, and captions
% ****************************************************************************************************
\usepackage{tabularx} % better tables
	\setlength{\extrarowheight}{3pt} % increase table row height
\newcommand{\tableheadline}[1]{\multicolumn{1}{c}{\spacedlowsmallcaps{#1}}}
\newcommand{\myfloatalign}{\centering} % to be used with each float for alignment
\usepackage{caption}
\captionsetup{format=hang,font=small}
\usepackage{subfig}
% ****************************************************************************************************


% ****************************************************************************************************
% 5. Setup code listings
% ****************************************************************************************************
\usepackage{listings}
%\lstset{emph={trueIndex,root},emphstyle=\color{BlueViolet}}%\underbar} % for special keywords
\lstset{language=[LaTeX]Tex,%C++,
    keywordstyle=\color{RoyalBlue},%\bfseries,
    basicstyle=\small\ttfamily,
    %identifierstyle=\color{NavyBlue},
    commentstyle=\color{Green}\ttfamily,
    stringstyle=\rmfamily,
    numbers=none,%left,%
    numberstyle=\scriptsize,%\tiny
    stepnumber=5,
    numbersep=8pt,
    showstringspaces=false,
    breaklines=true,
    frameround=ftff,
    frame=single,
    belowcaptionskip=.75\baselineskip
    %frame=L
}
% ****************************************************************************************************    		


% ****************************************************************************************************
% 6. PDFLaTeX, hyperreferences and citation backreferences
% ****************************************************************************************************
% ********************************************************************
% Using PDFLaTeX
% ********************************************************************
\PassOptionsToPackage{pdftex,hyperfootnotes=false,pdfpagelabels,hyperindex=true}{hyperref}
	\usepackage{hyperref}  % backref linktocpage pagebackref
%----------------------
\pdfcompresslevel=9
% http://tex.stackexchange.com/questions/52317/pdftex-warning-version-allowed
%\pdfminorversion=5
\pdfminorversion=7
\pdfobjcompresslevel=3
%----------------------
\pdfadjustspacing=1
\PassOptionsToPackage{pdftex}{graphicx}
	\usepackage{graphicx}

% ********************************************************************
% Setup the style of the backrefs from the bibliography
% (translate the options to any language you use)
% ********************************************************************
\newcommand{\backrefnotcitedstring}{\relax}%(Not cited.)
\newcommand{\backrefcitedsinglestring}[1]{(Cited on page~#1.)}
\newcommand{\backrefcitedmultistring}[1]{(Cited on pages~#1.)}
\ifthenelse{\boolean{enable-backrefs}}%
{%
		\PassOptionsToPackage{hyperpageref}{backref}
		\usepackage{backref} % to be loaded after hyperref package
		   \renewcommand{\backreftwosep}{ and~} % separate 2 pages
		   \renewcommand{\backreflastsep}{, and~} % separate last of longer list
		   \renewcommand*{\backref}[1]{}  % disable standard
		   \renewcommand*{\backrefalt}[4]{% detailed backref
		      \ifcase #1 %
		         \backrefnotcitedstring%
		      \or%
		         \backrefcitedsinglestring{#2}%
		      \else%
		         \backrefcitedmultistring{#2}%
		      \fi}%
}{\relax}

% ********************************************************************
% Hyperreferences
% ********************************************************************
\hypersetup{%
    %draft,	% = no hyperlinking at all (useful in b/w printouts)
    colorlinks=true, linktocpage=true, pdfstartpage=3, pdfstartview=FitV,%
    % uncomment the following line if you want to have black links (e.g., for printing)
    %colorlinks=false, linktocpage=false, pdfborder={0 0 0}, pdfstartpage=3, pdfstartview=FitV,%
    breaklinks=true, pdfpagemode=UseNone, pageanchor=true, pdfpagemode=UseOutlines,%
    plainpages=false, bookmarksnumbered, bookmarksopen=true, bookmarksopenlevel=1,%
    hypertexnames=true, pdfhighlight=/O,%nesting=true,%frenchlinks,%
    urlcolor=webbrown, linkcolor=RoyalBlue, citecolor=webgreen, %pagecolor=RoyalBlue,%
    %urlcolor=Black, linkcolor=Black, citecolor=Black, %pagecolor=Black,%
    pdftitle={\myTitle},%
    pdfauthor={\textcopyright\ \myName},%
    pdfsubject={Relativistic Electrodynamics},%
    pdfkeywords={Relativistic Electrodynamics} {PHY450H1S} {University of Toronto},%
    pdfcreator={pdfLaTeX},%
    pdfproducer={LaTeX with hyperref and classicthesis}%
}

% ********************************************************************
% Setup autoreferences
% ********************************************************************
% There are some issues regarding autorefnames
% http://www.ureader.de/msg/136221647.aspx
% http://www.tex.ac.uk/cgi-bin/texfaq2html?label=latexwords
% you have to redefine the makros for the
% language you use, e.g., american, ngerman
% (as chosen when loading babel/AtBeginDocument)
% ********************************************************************
\makeatletter
\@ifpackageloaded{babel}%
    {%
       \addto\extrasamerican{%
					\renewcommand*{\figureautorefname}{Figure}%
					\renewcommand*{\tableautorefname}{Table}%
					\renewcommand*{\partautorefname}{Part}%
					\renewcommand*{\chapterautorefname}{Chapter}%
					\renewcommand*{\sectionautorefname}{Section}%
					\renewcommand*{\subsectionautorefname}{Section}%
					\renewcommand*{\subsubsectionautorefname}{Section}% 	
				}%
       \addto\extrasngerman{%
					\renewcommand*{\paragraphautorefname}{Absatz}%
					\renewcommand*{\subparagraphautorefname}{Unterabsatz}%
					\renewcommand*{\footnoteautorefname}{Fu\"snote}%
					\renewcommand*{\FancyVerbLineautorefname}{Zeile}%
					\renewcommand*{\theoremautorefname}{Theorem}%
					\renewcommand*{\appendixautorefname}{Anhang}%
					\renewcommand*{\equationautorefname}{Gleichung}%
					\renewcommand*{\itemautorefname}{Punkt}%
				}%	
			% Fix to getting autorefs for subfigures right (thanks to Belinda Vogt for changing the definition)
			\providecommand{\subfigureautorefname}{\figureautorefname}%  			
    }{\relax}
\makeatother

% https://tex.stackexchange.com/questions/95488/list-of-figures-and-page-numbering
% https://tex.stackexchange.com/a/95616/15
\makeatletter
\newcommand{\emptypage}[1]{%
  \cleardoublepage
  \begingroup
  \let\ps@plain\ps@empty
  \pagestyle{empty}
  #1
  \cleardoublepage}
\makeatletter

% ****************************************************************************************************
% 7. Last calls before the bar closes
% ****************************************************************************************************
% ********************************************************************
% Development Stuff
% ********************************************************************
\listfiles
%\PassOptionsToPackage{l2tabu,orthodox,abort}{nag}
%	\usepackage{nag}
%\PassOptionsToPackage{warning, all}{onlyamsmath}
%	\usepackage{onlyamsmath}

% ********************************************************************
% Last, but not least...
% ********************************************************************
\usepackage[sixbynine]{../latex/classicthesis_mine/classicthesis}
% ****************************************************************************************************


% ****************************************************************************************************
% 8. Further adjustments (experimental)
% ****************************************************************************************************
% ********************************************************************
% Changing the text area
% ********************************************************************
%\linespread{1.05} % a bit more for Palatino
%\areaset[current]{312pt}{761pt} % 686 (factor 2.2) + 33 head + 42 head \the\footskip
%\setlength{\marginparwidth}{7em}%
%\setlength{\marginparsep}{2em}%

% ********************************************************************
% Using different fonts
% ********************************************************************
%\usepackage[oldstylenums]{kpfonts} % oldstyle notextcomp
%\usepackage[osf]{libertine}
%\usepackage{hfoldsty} % Computer Modern with osf
%\usepackage[light,condensed,math]{iwona}
%\renewcommand{\sfdefault}{iwona}
%\usepackage{lmodern} % <-- no osf support :-(
%\usepackage[urw-garamond]{mathdesign} <-- no osf support :-(
% ****************************************************************************************************

%----------------------------------------------------------------------------------------
% peeter's mods from from mylayout.sty
%----------------------------------------------------------------------------------------

% http://www.artofproblemsolving.com/Wiki/index.php/LaTeX:Layout#The_Easy_Way
% replaces manual settings (\parindent, \parskip, \topmargin, ...)
%\usepackage[margin=3.0cm]{geometry}
%\usepackage[left=3cm,right=4cm]{geometry}

% this boolean is now crippled by hardcoding 6x9 in:
% 1) config.tex above:
%   \usepackage[sixbynine]{classicthesis} % customized ancient version of classicthesis.sty
% 2) GAelectrodynamic.tex: paper=6in:9in
\newboolean{use-lettersize}
\setboolean{use-lettersize}{false}
%\setboolean{use-lettersize}{true}

\newboolean{print-version}

\ifthenelse{\boolean{use-lettersize}}
{
   % http://www.artofproblemsolving.com/Wiki/index.php/LaTeX:Layout#The_Easy_Way
   % replaces manual settings (\parindent, \parskip, \topmargin, ...)
   %\usepackage[margin=3.0cm]{geometry}
   % default:
   \usepackage[left=3cm,right=4cm]{geometry}
   %\usepackage[left=2cm,right=1.9cm,top=2.1cm,bottom=1.9cm]{geometry}

   \setboolean{print-version}{false}
}
{
   %\usepackage[paperheight=9in,paperwidth=6in,left=2cm,right=1.7cm,top=1.7cm,bottom=2cm]{geometry}
   %\usepackage[paperheight=9in,paperwidth=6in,left=2cm,right=1.9cm,top=1.9cm,bottom=2.2cm]{geometry}

   % first print version: GAelectrodynamics.V0.1.14.6x9.pdf
   % margins on the bottom look a bit big.
   %\usepackage[paperheight=9in,paperwidth=6in,left=2cm,right=2cm,top=1.9cm,bottom=2.2cm]{geometry}
   \usepackage[paperheight=9in,paperwidth=6in,left=2cm,right=2cm,top=1.9cm,bottom=1.5cm]{geometry}

   % this is the size of the Landau and Lifshitz books (as measured with a ruler)
   %\usepackage[paperheight=9.61in,paperwidth=6.69in,left=2cm,right=1.9cm,top=1.9cm,bottom=1.8cm]{geometry}

   %\setboolean{print-version}{true}
   \setboolean{print-version}{false}
}

%\PassOptionsToPackage{answerdelayed}{exercise}
\usepackage{peeters_layout}
\usepackage{thisbook}

\myClassicThesisOverrides
